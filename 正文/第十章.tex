\section[第十章\q 布尔什维克党为实现国家社会主义工业化而斗争(1926—1929年)]{第十章\\ 布尔什维克党为实现国家社会主义工业化而斗争 \\{\zihao{3}(1926—1929年)}}

\subsection[一\q 社会主义工业化时期的困难和克服困难的斗争。托洛茨基派—季诺维也夫派反党联盟的形成。这个联盟的反苏行动。联盟的失败]{一\\ 社会主义工业化时期的困难和克服困难的斗争。\\托洛茨基派—季诺维也夫派反党联盟的形成。\\这个联盟的反苏行动。联盟的失败}

第十四次代表大会以后,党为实现苏维埃政权关于国家社会主义工业化的总方针而展开了斗争。

恢复时期的任务是首先活跃农业,从农业取得原料和粮食,并在农业的带动下恢复工业,恢复现有的工厂。

苏维埃政权较为容易地解决了这些任务。

但是恢复时期有三大缺点:

第一,当时的工厂都是老厂,技术陈旧落后,可能很快就不能生产了。任务是要用新技术来改造这些工厂。

第二,恢复时期的工业,基础非常狭窄;在当时的工厂中缺乏数十个、教百个机器制造厂。这些工厂是国家绝对必需的,是我们当时没有而需要建立的,因为没有这些工厂,工业就不能认为是真正的工业。任务是要建立这些工厂,用现代化技术把它们装备起来。

第三,恢复时期的工业主要是轻工业。这种工业已有发展而且已走上正轨,但是轻工业发展本身后来也因重工业薄弱而受到阻碍,更不用说国家的其他种种只有靠高度发展的重工业才能满足的需要了。任务是要在现在侧重发展重工业。

所有这些新任务,都是社会主义工业化政策所应解决的。

必须新建沙俄所没有的一系列工业部门,即建立新的机器制造厂、机床制造厂、汽车制造厂、化学工厂和冶金工厂,创立本国的发动机和电站设备的生产,增加金属和煤炭的开采量,因为这是社会主义在苏联胜利所必需的。

必须建立新的国防工业,即修建新的大炮制造厂、炮弹制造厂、飞机制造厂、坦克制造厂和机关枪制造厂,因为这是为了在资本主义包围下保卫苏联所必需的。

必须建立拖拉机制造厂和现代化农业机器制造厂,用它们的产品供给农业,使千百万小的个体农户有可能过渡到集体农庄的大生产,因为这是为了社会主义在农村胜利所必需的。

这一切都是工业化政策所应做到的,因为国家社会主义工业化的实质就在于此。

当然,这样巨大的基本建设没有数十亿投资是不行的。指靠外债没有可能,因为资本主义国家拒绝贷款。只好在投有外援的情况下靠自己的资金来从事建设。而当时我们的国家还不富裕。

这就是当时的主要困难之一。

资本主义国家建立自己的重工业,通常是靠从外面流入资金,即靠掠夺殖民地,靠战败国人民的赔教,靠外债。苏维埃国家根本不能靠掠夺殖民地或战败国人民这样的龌龊办法来取得工业化的资金。至于外债,苏联又被切断了来源,因为资本主义国家拒绝贷款给苏联。必须在国内找到资金。

而在苏联也就找到了这样的资金。在苏联找到了任何一个资本主义国家所没有的积累泉源。苏维埃国家掌握了十月社会主义革命从资本家地主手中夺来的一切工厂和一切土地,以及运输业、银行、国内外贸易。国营工厂、运输业、贸易和银行所得到的利润,现在已不是供寄生的贤本家阶级消费,而是用于进一步扩大工业了。

苏维埃政权废除了沙皇的外债;这些外债,单是利息一项,人民每年就要支付几亿金卢布,苏维埃政权消灭了地主土地所有制,免除了农民每年向地主交纳的约五亿金卢布的地租。农民摆脱了这一切重担,就能够帮助国家建设新的强大的工业。农民迫切需要获得拖拉机和农业机器。

所有这些收入的泉源都掌握在苏维埃国家手中。这些泉源能够为建设重工业提供几亿至几十亿卢布。只是必须以主人翁的态度来办事,在开支上厉行节约,实行生产合理化,降低生产成本,消灭非生产费用等等。

苏维埃政权也正是这样做的。

由于实行节约制度,基本建设资金的积累逐年增多。这就有可能着手兴建一些大型企业,如德涅泊水电站、土尔克斯坦—西伯利亚铁路、斯大林格勒拖拉机制造厂、几个机床制造厂、“阿模”汽车制造厂(即斯大林汽车制造厂)等等。

1926—1927年度工业的投资约十亿卢布,而过了三年就已经有五十亿左右了。

工业化的事业向前推进了。

资本主义国家认为苏联社会主义经济的巩固对资本主义制度的生存是个威胁。因此,各帝国主义政府采取了一切办法来对苏联施加新的压力、制造混乱、破坏或者至少是阻挠苏联的工业化。

1927年5月,英国执政的保守党人(“死硬派”)向“阿尔柯斯”(苏联对英贸易公司)进行了挑衅性的袭击。1927年5月26日,英国保守党政府宣布同苏联断绝外交关系和商务关系。

1927年6月7日,波兰籍俄国白卫分子在华沙刺杀了苏联大使沃依柯夫同志。

同时在苏联境内,英国特务和破坏分子在列宁格勒向党的俱乐部投掷炸弹,炸伤约三十人,其中有几人受重伤。

1927年夏,在柏林、北京、上海和天津,差不多同时发生了袭击苏联大使馆和商务代办处的事件。

这就给苏维埃政权造成了额外的困难。

但是苏联没有屈服于压力,很容易地击退了帝国主义者及其走狗的挑衅性袭击。

托洛茨基派和其他反对派的破坏活动,给党和苏维埃国家也带来了不少困难。难怪斯大林同志当时说,“正在建立一种从张伯伦到托洛茨基的统一战线之类的东西”\footnote{见《斯大林全集》第9卷第282页。——译者注}来反对苏维埃政权。尽管党的第十四次代表大会作了决定,反对派也声明对党忠诚,但是他们并没有放下武器。不仅如此,他们的破坏活动、分裂活动搞得更厉害了。

1926年夏,托洛茨基派和季诺维也夫派结成一个反党联盟,把所有已被击败的反对派集团的残兵败将纠集在这个联盟的周围,奠定了他们那个反列宁主义的地下党的基础,从而粗暴地破坏了党章和历次党代表大会关于禁止成立派别组织的决议。党中央警告说:这个类似有名的孟什维克八月联盟的反党联盟如果不解散,它的参加者就不会有好下场。但是他们不肯罢休。

同年秋,在党的第十五次代表会议前夕,他们在莫斯科、列宁格勒和其他几个城市的工厂党员大会上又搞袭击,企图强迫党再次进行争论。同时他们还提出自己的纲领要党员讨论,而这个纲领不过是通常的托洛茨基主义—孟什维主义的反列宁主义纲领的翻版罢了。党员群众给了反对派分子无情的回击,有些地方干脆把他们赶出了会场。中央再次警告联盟的参加者说,党不能再容忍他们搞破坏活动了。

反对派分子由托洛茨基、季诺维也夫、加米涅夫和索柯里尼柯夫签名向中央递交声明,谴责自己的派别活动,保证今后对党忠诚。然而这个联盟事实上仍继续存在,它的参加者并没有停止反党的地下活动。他们继续拼凑自己那个反列宁主义的党,建立秘密印刷所,在自己的同伙中征收党费,散发自己的纲领。

鉴于托洛茨基派和季诺维也夫派的这些行为,党的第十五次代表会议(1926年11月)和共产国际执委会扩大全会(1926年12月)把托洛茨基派—季诺维也夫派联盟的问题提出来讨论,并在自己的决议中痛斥联盟参加者是分裂主义者,指出他们在自己的纲领中已经滚到孟什维克的立场上去了。

但是,这并没有使联盟参加者醒悟过来。1927年,当英国保守党人同苏联断绝外交关系和商务关系的时候,他们又变本加厉地攻击党。他们炮制了一个新的反列宁主义的纲领,即所谓“八十三人纲领”,并在党员中散发,要求中央再在全党展开争论。

在所有的反对派纲领中,这个纲领算是最虚伪最骗人的了。

在口头上,即在纲领中,托洛茨基被和季诺维也夫派不反对遵守党的决议。并且表示对党忠诚,但事实上,他们极其粗暴地破坏党的决议,嘲笑对党和党中央的任何忠诚。

在口头上,即在纲领中,他们不反对党的统一,并表示反对分裂,但事实上,他们极其粗暴地破坏党的统一,实行分裂的路线,并且单独建立了自己的反列宁主义的秘密党,而这个党已具备了成为一个反苏反革命政党的一切条件。

在口头上,即在纲领中,他们赞成工业化政策,甚至责备中央实行工业化的速度不够快,但事实上,他们咒骂党关于社会主义在苏联胜利问题的决议,嘲笑社会主义工业化政策,要求把一系列工厂租让给外国人,把自己的主要希望寄托在外国资本主义在苏联的租让企业上。

在口头上,即在纲领中,他们赞成集体农庄运动,甚至责备中央进行集体化的速度不够快,但事实上,他们嘲笑吸收农民参加社会主义建设的政策,宣传说工人阶级同农民必然发生“无法解决的冲突”,并把自己的希望寄托在农村的“文明租佃人”富农身上。

这是反对派的一切虚伪纲领中最虚伪的纲领。

这个纲领原来就是为了欺骗党的。

中央拒绝立刻宣布进行争论,并向反对派分子说,进行争论只能根据党章的规定,即只能在党代表大会召开前两个月举行。

1927年10月,即在第十五次代表大会召开前两个月,党中央委员会宣布进行全党争论。争论会开起来了。争论的结果对于托洛茨基派—季诺维也夫派联盟是极其可悲的。投票赞成中央政策的党员有七十二万四千人。赞成托洛茨基派和季诺维也夫派联盟的只有四千人,即不到百分之一。反党联盟遭到了惨败。党以压倒多数一致否决了这个联盟的纲领。

这就是党的明确表达出来的意志,而联盟的参加者自己正是向党提出申诉的。

但是这次教训也没有使联盟参加者醒悟过来。他们不仅不服从党的意志,反而决定破坏党的意志。还在争论结束之前,他们看到自己不可避免要遭到可耻的失败,就决定采取更加尖锐的斗争形式来反对党和苏维埃政府。他们决定在莫斯科和列宁格勒举行公开的抗议示威。他们选定了11月7日即十月革命纪念日作为自己示威的日子,因为这一天苏联劳动者要举行全民的革命示威。这样,托洛茨基派和季诺维也夫派就是有意举行一个平行的示威了。不出所料,联盟的参加者能够带上街去的只是他们那一撮少得可怜的应声虫。应声虫和他们的头目被全民示威队伍冲垮和撵走了。

托洛茨基派和季诺维也夫派滚进了反苏泥潭,现在已经是不容置疑的了。在进行全党争论时,他们是向党控诉中央的,而在这里,在他们举行这个可怜的示威时,他们已走上向敌对阶级控诉党和苏维埃国家的道路了。既然他们立意破坏布尔什维克党,也就必然要滑到破坏苏维埃国家的道路上去,因为在苏维埃国家里,布尔什维克党和国家是分不开的。这样,托洛茨基派—季诺维也夫派联盟的头目们就是自外于党了,因为在布尔什维克党的队伍里不能再容忍有滚进反苏泥潭的人了。

1927年11月14日,中央委员会和中央监察委员会联席会议把托洛茨基和季诺维也夫两人开除出党。


\subsection[二\q 社会主义工业化的成就。农业的落后。党的第十五次代表大会。农业集体化的方针。托洛茨基派—季诺维也夫派联盟的被粉碎。政治上的两面派手腕]{二\\社会主义工业化的成就。农业的落后。\\党的第十五次代表大会。农业集体化的方针。\\托洛茨基派—季诺维也夫派联盟的被粉碎。\\政治上的两面派手腕}

到1927年底,可以看出社会主义工业化政策已取得了有决定意义的成就。新经济政策条件下实行的工业化,短时期内就有了重大的进展。工业和整个农业(包括林业和渔业),就其总产值来说,不仅达到战前水平,而且超过了这个水平。工业在国民经济中的比重已增加到百分之四十二,达到了战前的相应水平。

工业中的社会主义成分迅速增长,私营成分下降。社会主义成分从1924—1925年度的百分之八十一增加到1926—1927年度的百分之八十六,而私营成分的比重在同一时期内从百分之十九降到百分之十四。

这就是说,苏联的工业化具有鲜明的社会主义性质,苏联工业在沿着社会主义生产体系获得胜利的道路发展,工业方面“谁战胜谁”的问题已经提前获得了有利于社会主义的解决。

私商也被迅速排挤出商业。私商在零售方面所占的比重从1924—1925年度的百分之四十二降到1926—1927年度的百分之三十二。批发商业就更不用说了,这里私商所占的比重在同一时期内从百分之九降到了百分之五。

社会主义大工业增长得还要迅速,它的产值在1927年,即恢复时期后的第一年,比上年增加了百分之十八。这是创纪录的增长数,是最先进的资本主义国家的大工业所达不到的。

农业,特别是粮食生产,情况就不同了。虽然整个农业已超过战前水平,但它的主要部门(粮食生产)的总产值只等于战前水平的百分之九十一,而粮食产量的商品部分,即出售以供城市需要的部分,才勉强达到战前水平的百分之三十七,并且所有的材料都说明,粮食的商品产量有继续下降的危险。

这就是说,1918年开始的农村中那些大型商品经济单位变小再变小的过程还在继续着。变小和再变小的农民经济成为只能提供最低限度商品粮的半自然经济。1927年这个时期的粮食生产虽然只略低于战前的产量,但能出售以供城市的粮食,则仅仅等于战前的三分之一强。

毫无疑义,在粮食生产的这种情况下,苏联的军队和城市就会陷于经常挨饿的境地。

这是粮食生产的危机。随之而来的必定是畜牧业的危机。

为了摆脱这种状况,农业必须过渡到能使用拖拉机和农业机器、能使粮食生产的商品产量提高几倍的大生产。国家面临着两种可能:或者是过渡到资本主义的大生产,这就意味着农民群众破产,工人阶级和农民的联盟灭亡,富农的力量加强,社会主义在农村失败;或者是另一条道路,即把小农户联合成社会主义的大农庄,联合成为能使用拖拉机和其他现代化机器来迅速提高粮食生产及其商品产量的集体农庄。

当然,布尔什维克党和苏维埃国家只能走发展农业的第二条道路即集体农庄道路。

党在这方面遵循的是列宁的如下一些关于在农业中必须从小农经济过渡到大规模的劳动组合的集体经济的指示:

\begin{quotation}
(一)“靠小农经济是摆脱不了贫困的。”(《列宁全集》俄文第3版第24卷第540页)\footnote{见《列宁全集》第30卷第126页。——译者注}

(二)“如果我们仍然依靠小经济来生活,即使我们是自由土地上的自由公民,也不免要灭亡的。”《列宁全集》俄文第3版第20卷第417页)\footnote{见《列宁全集》第24卷第463页。——译者注}

(三)“农民经济进一步发展的条件是,必须稳固地保证它能进一步过渡,而进一步过渡就必然使利益最小的、最落后的、细小的、孤立的农民经济逐渐联合起来,组织成公共的大规模的农业经济。”(《列宁全集》俄文第3版第26卷第299页)\footnote{见《列宁全集》第32卷第275页。——译者注}

(四)“掌握国家政权的工人阶级,只有在事实上向农民表明了公共的、集体的、协作的、劳动组合的耕种制的优越性,只有用协作的、劳动组合的经济帮助了农民,才能真正向农民证明自己正确,才能真正可靠地把千百万农民群众吸引到自己方面来。”《列宁全集》俄文第3版第24卷第579页)\footnote{见《列宁全集》第2版第4卷第106页。——译者注}
\end{quotation}

党的第十五次代表大会于1927年12月2日开幕。出席这次大会的有八百九十八名有表决权的代表和七百七十一名有发言权的代表,代表着八十八万七千一百三十三名党员和三十四万八千九百五十七名预备党员。

斯大林同志在总结报告中指出了工业化的成就和社会主义工业的迅速高涨,同时向党提出了如下的任务:

\begin{quotation}
“扩大和巩固我们城乡国民经济一切部门中的社会主义经济命脉,采取消灭国民经济中的资本主义成分的方针。”\footnote{见《斯大林全集》第10卷第256页。——译者注}
\end{quotation}

斯大林同志拿农业和工业比较,指出农业特别是粮食生产由于分散和不能采用现代化技术而落后了,他着重指出,农业的这种不相称的状况将对整个国民经济造成威胁。

\begin{quotation}
斯大林同志问道:“出路究竟在哪里呢?”

斯大林同志回答说:“出路就在于把分散的小农户转变为以公共耕种制为基础的联合起来的大农庄,就在于转变到以高度的新技术为基础的集体耕种制。出路就在于逐步地然而一往直前地不用强迫手段而用示范和说服的方法把小的以至最小的农户联合为以公共的互助的集体的耕种制为基础、利用农业机器和拖拉机,采用集约耕作的科学方法的大农庄。别的出路是没有的。”\footnote{同上,第261页。——译者注}
\end{quotation}

第十五次代表大会通过了关于尽力开展农业集体化的决议。大会拟定了扩大和巩固集体农庄和国营农场网的计划,并明确指出了实现农业集体化的方法。

同时,大会进发出了如下的指示:

\begin{quotation}
“继续向富农展开进攻,并采取一系列新的措施来限制农村资本主义的发展和引导农民经济沿着社会主义的方向前进。”(《联共(布)决议汇编》俄文版第2册第260页)\footnote{见《苏联共产党决议汇编》第3分册第109页。——译者注}
\end{quotation}

最后,大会从加强国民经济的计划原则出发.并考虑到在国民经济全线组织社会主义对资本主义成分有计划地展开进攻,指示有关机关编制国民经济的第一个五年计划。

党的第十五次代表大会结束了社会主义建设问题以后,就来讨论消灭托洛茨基派—季诺维也夫派联盟的问题。

大会确认:“反对派在思想上已同列宁主义决裂,蜕化成了孟什维主义的集团,走上了向国际和国内资产阶级势力投降的道路,客观上变成了反对无产阶级专政制度的第三种势力的工具。”(《联共(布)决议汇编》俄文版第2册第232页)\footnote{见《苏联共产党决议汇编》第3分册第364页。——译者注}

大会认为,党和反对派之问的意见分歧已经发展为纲领上的分歧,托洛茨基反对派已走走上了进行反苏斗争的道路。因此,第十五次代表大会宣布,参加托洛茨基反对派和宣传其观点同留在布尔什维克党的队伍内不能相容。

大会批准了中央委员会和中央监察委员会联席会议关于开除托洛茨基和季诺维也夫两人出党的决定,并通过决定把托洛茨基派—季诺维也夫派联盟的所有骨干分子,如拉狄克、普列奥布拉任斯基、拉柯夫斯基、皮达可夫、谢烈布利雅柯夫、伊·斯米尔诺夫、加米涅夫、萨尔基斯、萨发罗夫、里弗施茨、穆吉万、斯米尔加和整个“民主集中派”集团(萨普龙诺夫、弗·斯米尔诺夫、鲍古斯拉夫斯基、德罗布尼斯等)开除出党。

思想上被打败和组织上被粉碎的托洛茨基派—季诺维也夫派联盟的参加者,丧失了自己在人民中的最后一点影响。

被开除出党的反列宁主义分子,在党的第十五次代表大会以后不久就递交声明要同托洛茨基主义决裂,并请求让他们回到党里来。当然,那时党还不可能知道托洛茨基、拉柯夫斯基、拉狄克、克列斯廷斯基、索柯里尼柯夫等人早已是人民的敌人和受外国间谍机关雇用的特务,还不可能知道加米涅夫、季诺维也夫、皮达可夫等人已经同资本主义国家内的苏联敌人建立联系,要同他们一起“合作”来反对苏联人民。但党已有充分的经验教训,知道这些屡次在最紧要的关头反对列宁和列宁党的人是什么坏事都干得出来的。因此,党对被开除者的声明还抱怀疑。为了检验(初步的检验)声明者的诚意,党提出如下几项要求作为恢复党籍的条件:

(一)公开谴责托洛茨基主义是反布尔什维主义的和反苏的思想体系;

(二)公开承认党的政策是唯一正确的政策;

(三)无条件地服从党和党的机关的决议;

(四)要经过一个考察期。党在考察期内对声明者进行考察;在考察期满后,根据考察结果,个别地提出每个被开除者的党籍恢复问题。

当时党是这样盘算的:被开除者公开承认这几条,在任何情况下对党都是有利的,因为这会破坏托洛茨基派—季诺维也夫派联盟队伍的统一,引起他们内部的瓦解,再次显示党的正确和强大,并使党有可能在声明者确有诚意的情况下让党的原工作人员回到党内来,而在他们没有诚意的情况下,则在大家面前揭露他们的面目,让大家看到他们已经不是犯错误的人,而是一些毫无原则的野心家、工人阶级的骗子和不可救药的两面派。

大多数被开除者接受了党所提出的入党条件,并在报刊上发表了相应的声明。

党怜惜他们,给了他们重新回到党和工人阶级队伍的机会,恢复了他们的党员资格。

但是后来发现,托洛茨基派—季诺维也夫派联盟的“骨干分子”(除少数外)的声明,是虚伪透顶的两面派的声明。

原来,这些老爷还在递交声明以前,就已不再是一个准备在人民面前坚持自己观点的政治派别了;他们已变成一伙毫无原则的野心家,甘愿在大家面前践踏自己观点的那一点残余,在大家面前颂扬自己所敌视的党的观点,像变色龙一样需要什么颜色就变什么颜色,只求自己能留在党内,留在工人阶级内,好有机会来危害工人阶级及其政党。

托洛茨基派—季诺维也夫派联盟的“骨干分子”,原来都是政治骗子、政治上的两面派。

政治上的两面派通常总是从欺骗做起,用欺骗人民、欺骗工人阶级、欺骗工人阶级党的手段来干自己的黑暗勾当。但是,绝不可把政治上的两面派仅仅看作一些骗子。政治上的两面派是一伙毫无原则的政治野心家;他们早已丧失了人民的信任,力图用各种方法来重新博得信任——不管是欺骗的方法、变色龙变色的方法、招摇撞骗的方法都可以,只要自己能保留政治活动家的称号就行。政治上的两面派是一伙毫无原则的政治野心家;他们为了在“适当时机”重新爬上政治舞台、骑在人民头上当“统治者”,面对什么人都依靠——哪怕是刑事犯、社会渣滓和人民的死敌。

托洛茨基派—季诺维也夫派联盟的“骨干分子”,原来就是这样一些政治上的两面派。


\subsection[三\q 对富农的进攻。布哈林—李可夫反党集团。第一个五年计划的采取。社会主义竞赛。群众性集体农庄运动的开始]{三\\对富农的进攻。布哈林—李可夫反党集团。\\第一个五年计划的采取。社会主义竞赛。\\群众性集体农庄运动的开始}

托洛茨基派—季诺维也夫派联盟煽动反对党的政策,反对社会主义建设,反对集体化。布哈林派也进行煽动,说集体农庄事业行不通,说不要触动富农,因为富农会自行“长入”社会主义,说资产阶级发财致富对社会主义没有危险。所有这些煽动,在国内资本主义分子中,首先是在富农中得到了强烈的反应。现在富农从报刊上的反应知道,他们并不是孤立的,他们有托洛茨基、季诺维也夫、加米涅夫、布哈林、李可夫等为他们辩护申诉。这种情况当然不能不煽起富农反对苏维埃政府政策的情绪。果然,富农的反抗愈来愈厉害了。大批大批的富农开始拒绝把他们囤积了不少的余粮卖给苏维埃国家。他们开始对集体农庄庄员和农村中党和苏维埃的工作人员采取恐怖手段,纵火焚烧集体农庄和国家粮站。

党懂得,只要富农的反抗还没有被打垮,只要农民还没有亲眼看到富农在公开的战斗中被打败,工人阶级和红军就要吃缺粮的苦头,而农民的集体农庄运动也不可能变成群众性的。

党遵照党的第十五次代表大会的指示,转而对富农实行坚决的进攻。党在进攻中实行这样的口号,牢固地依靠贫农,巩固同中农的联盟,坚决反对富农。为了对付富农拒绝按固定价格把余粮卖给国家,党和政府采取了一系列反对富农的非常措施,施行了在富农和投机分子拒绝按固定价格把余粮卖给国家时由法庭判处没收其余粮的刑法典第一百零七条,同时给了贫农一系列优待,如贫农可分到百分之二十五从富农那里没收的粮食。

非常措施发生了效力:贫农和中农加入了坚决反对富农的斗争,富农被孤立了,富农和投机分子的反抗被打垮了。到1928年底,苏维埃国家已拥有充足的粮食储备,而集体农庄运动也以更坚定的步伐向前迈进了。

就在这年,在顿巴斯的沙赫特区破获了一个庞大的资产阶级专家破坏组织。沙赫特破坏分子同从前的企业老板(俄国和外国的资本家)和外国军事间谍机关有紧密勾结。他们的目的是破坏社会主义工业的增长,促使在苏联恢复资本主义。破坏分子不合理地在井下进行开采以减少采煤量。他们毁坏机器和通风设备,设法使矿井崩塌,炸毁和焚烧矿井、工厂和电站。破坏分子故意阻挠工人物质生活状况的改善,违反苏维埃劳动保护法。

破坏分子被交付法庭审判了。他们受到了法庭应有的惩罚。

党中央委员会建议各级党组织从沙赫特案件中吸取教训。斯大林同志指示说:布尔什维克经济工作人员自己应该成为生产技术行家,以免今后再受旧资产阶级专家中的破坏分子的欺骗;必须加速从工人阶级队伍中培养新的技术干部。

根据中央的决议,改进了高等技术学校培养青年专家的工作。数以千计的党员、共青团员和忠于工人阶级事业的非党员被动员去学习。

在党没有转入对富农的进攻、仍在忙于消灭托洛茨基派—季诺维也夫派联盟的时候,布哈林—李可夫集团还比较沉得住气,还处在反党势力的后备地位,还不敢公开支持托洛茨基派,有时甚至还和党一起反对托洛茨基派。随着党转而对富农实行进攻、对富农采取非常措施,布哈林—李可夫集团就扔掉了假面具,公开跳出来反对党的政策。布哈林—李可夫集团的富农本性已经按捺不住了,于是这个集团的参加者就公开出来替富农辩护。他们要求取消非常措施,并吓唬头脑简单的人说,否则农业就会开始“退化”(下降、衰落、崩溃),而且硬说退化已经开始。他们看不见集体农庄和国营农场这些高级形式的农业的增长,一看见富农经济在衰落,就把富农经济的退化说成农业的退化。为了使自己能在理论上站住脚,他们炮制了可笑的“阶级斗争熄灭论”,胡说什么:社会主义在同资本主义成分的斗争中取得的成就愈多,阶级斗争就愈缓和;阶级斗争很快就会完全熄灭下去,阶级敌人不经反抗就会让出自己的一切阵地;因此用不着对富农实行进攻。这样,他们就恢复了他们所谓富农和平长入社会主义的陈腐的资产阶级理论,践踏了列宁主义的著名原理,即阶级敌人愈是失去立足的基地、社会主义愈是取得成就,阶级敌人的反抗就会采取愈加尖锐的形式,而阶级斗争只有在阶级敌人被消灭以后才会“熄灭”。

不难明白,党面前的这个布哈林—李可夫集团是个右倾机会主义集团;它和托洛茨基派—季诺维也夫派联盟的区别仅仅是在形式上,即仅仅在于:托洛茨基派和季诺维也夫派有某种可能用“不断革命”这种左的、空喊革命的词句来掩盖自己的投降主义实质,而布哈林—李可夫集团是在党转入对富农的进攻的时候跳出来反党的,因而已没有可能掩盖自己的投降主义面目,不得不去掉假面具而公开地、不加粉饰地替我国的反动势力首先是富农辩护。

党知道,布哈林—李可夫集团迟早会和托洛茨基派—季诺维也夫派联盟残余携起手来共同反党。

布哈林—李可夫集团在进行政治活动的同时,还进行了组织“工作”来收罗支持者。通过布哈林纠集了斯列普柯夫、马列茨基、爱恒瓦里德和哥登别尔格之流的资产阶级青年,通过托姆斯基纠集了官僚化了的工会领导人(美尔尼昌斯基、多加多夫等),通过李可夫纠集了腐化了的苏维埃领导人(阿·斯米尔诺夫、埃斯蒙特、弗·施米特等)。凡是政治上腐化的、不掩盖自己投降主义情绪的人,都欣然加入了这个集团。

当时布哈林—李可夫集团得到了莫斯科党组织领导人(乌格拉诺夫、柯托夫、乌哈诺夫、柳亭、雅果达、波朗斯基等)的支持。但一部分右倾分子仍然是隐蔽的,没有公开反对党的路线。在莫斯科的党报党刊上和党员大会上,当时有人大造舆论,说必须向富农让步,说不宜向富农征税,说工业化对人民负担太重,说建设重工业为时过早。乌格拉诺夫反对修建德涅泊水电站,要求把资金由重工业转到轻工业。乌格拉诺夫和其他右倾投降主义者硬说,莫斯科过去是而且将来仍然是出产印花布的莫斯科,莫斯科用不着修建机器制造厂。

莫斯科党组织揭露了乌格拉诺夫及其同伙,向他们提出了最后警告,并更加紧密地团结在党中央委员会周围。斯大林同志1928年在联共(布)莫斯科委员会全会上指出必须进行两条战线的斗争时,认为要集中火力反对右倾。斯大林同志说,右倾分子是富农在党内的代理人。

\begin{quotation}
斯大林同志说:“如果右倾在我们党内获得胜利,就会放纵资本主义势力,破坏无产阶级的革命阵地,增多资本主义在我国恢复的机会。”(《列宁主义问题》俄文第10版第234页)\footnote{见斯大林《列宁主义问题》第239页。——译者注}
\end{quotation}

1929年初查明,布哈林代表右倾投降主义者集团通过加米涅夫同托洛茨集团挂上了钩,并同他们订立协定共同反党。中央揭露了右倾投降主义者的这种犯罪活动,并警告说,这种勾当会使布哈林、李可夫、托姆斯基等人遭到可悲的下场。但右倾投降主义者不肯甘休。他们在中央又提出一个反党纲领(一项声明)。中央谴责了这个纲领。中央再次警告他们,叫他们不要忘记托洛茨基派—季诺维也夫派联盟的下场。但布哈林—李可夫集团仍不理睬,继续进行反党活动。李可夫、托姆斯基和布哈林向中央提出辞职声明,想借此来恐吓党。中央谴责了这种怠工的辞职政策。最后,1929年中央十一月全会确认,宣传右倾机会主义者的观点同留在党内不能相容,建议把右倾投降主义者的急先锋和领导者布哈林开除出中央政治局,而对李可夫、托姆斯基和右倾反对派其他参加者则提出了严重警告。

右倾投降主义者的头目们见势不妙。就递交声明,承认自己错误,承认党的政治路线正确。

右倾投降主义者决定暂时实行退却,以便保存自己的实力免遭粉碎。

党和右倾投降主义者斗争的第一阶段就到此结束。

党内再次出现的意见分歧,不能不引起苏联外部敌人的注意。他们以为党内“再次出现的纷争”是党削弱的表现,又企图把苏联卷入战争,破坏我国还没有巩固的工业化事业。1929年夏,帝国主义者挑起中苏冲突,唆使中国军阀强占中东铁路(中东铁路是属于苏联的),指使中国白军侵犯我国远东边界。但中国军阀的袭击在很短期间就被消灭了,被红军击败的军阀退却了,这次冲突以我国同满洲当局签订和平协定而宣告结束。

苏联的和平政策排除了一切干扰、克服了外敌的阴谋和党内的“纷争”而再次胜利了。

不久,原被英国保守党人中断的苏英外交系和商务关系恢复了。

党在顺利地打退内外敌人的进攻的同时,进行了大量的工作来开展重工业的建设,组织社会主义竞赛,建设国营农场和集体农庄,最后,为采取和实现国民经济第一个五年计划准备必要的条件。

1929年4月,召开了党的第十六次代表会议。这次会议的主要议题是第一个五年计划。会议否决了右倾投降主义者所维护的五年计划的“最低”方案,采取了五年计划的“理想”方案,要求无条件地加以执行。

这样,党采取了有名的社会主义建设的第一个五年计划。

按照五年计划规定,1928—1933年国民经济基本投资额为六百四十六亿卢布。其中工业(电气化在内)的投资为一百九十五亿卢布,运输业为一百亿卢布,农业为二百三十二亿卢布。

这是用现代化技术装备苏联工农业的宏伟计划。

\begin{quotation}
斯大林同志指出:“五年计划的基本任务就是要在我国创立一种不仅能把全部工业而且能把运输业和农业都按社会主义原则进行改造和改组的工业。”(《列宁主义问题》俄文第10版第485页)\footnote{见斯大林《列宁主义问题》第446页。——译者注}
\end{quotation}

这个计划虽然非常宏伟,但对布尔什维克来说并不是什么出乎意料和冲昏头脑的事情。它是由工业化和集体化的全部发展进程准备好了的。它是由在此以前已普及于工农群众、反映在社会主义竞赛中的劳动高潮准备好了的。

党的第十六次代表会议通过了关于开展社会主义竞赛的告全体劳动者书。

社会主义竞赛展示了劳动和新的劳动态度的卓越榜样。工人和集体农庄庄员在许多企业、集体农庄和国营农场提出了响应计划。他们做出了英勇工作的榜样。他们不仅完成而且超额完成了党和政府制定的社会主义建设计划。人们对劳动的看法改变了。在资本主义下,劳动是不自由的苦役,而现在它开始变成“光荣的事情,荣耀的事情,英勇豪迈的事情”(斯大林)\footnote{见《斯大林全集》第12卷第275页。——译者注}。

全国各地进行着新的大规模的工业建设。德涅泊水电站工程的建设开展起来了。在顿巴斯,开始了克拉马托尔斯克工厂和戈尔洛沃工厂的修建以及鲁干斯克机车制造厂的改建。新的矿井和高炉增多了。在乌拉尔兴建着乌拉尔机器制造厂、伯列兹尼基和索里卡姆斯克两个化学联合企业。马格尼托哥尔斯克钢铁厂动工了。在莫斯科和高尔基,两个大型的汽车制造厂的建设开展起来了。许多地方都在兴建大型的拖拉机制造厂和联合收割机制造厂,如顿河岸罗斯托夫在修建大型的农业机器制造厂。苏联第二个产煤基地库兹巴斯在扩建。一座宏伟的拖拉机制造厂经过十一个月就在草原地带的斯大林格勒矗立起来。在德涅泊水电站和斯大林格勒拖拉机制造厂的建设中,工人们打破了劳动生产率的世界纪录。

历史上还没有过这样大规模的新工业建设、这样的新建设热潮、这样的千百万工人阶级群众的劳动英雄主义。

这是工人阶级在社会主义竞赛基础上开展起来的真正的劳动高潮。

农民这次也不落后。农村中也开始了农民群众建立集体农庄的劳动高潮。农民群众开始确定地转向集体农庄方面。用拖拉机和其他机器装备起来的国营农场和机器拖拉机站,在这里起了巨大的作用。农民们一批批地来到国营农场和机器拖拉机站,观看拖拉机和其他农业机器的操作,看得眉飞色舞,马上决定“加入集体农庄”。农民过去分散成了小而又小的个体经济,没有什么像样的农具和牵引力,没有可能开垦大片的荒地,没有改善经济的前途,为贫困所压抑,孤苦零丁而无人过问。现在他们终于找到了出路,找到了走向美好生活的道路;这条道路就在于把小农户联合为大集体即集体农庄,就在于采用能够开垦任何“硬地”、任何荒地的拖拉机,就在于从国家方面获得机器、资金、人员和意见的帮助,就在于有了可能免除富农的盘剥——因为苏维埃政府不久前刚打败了富农,把他们打翻在地。使千百万农民群众拍手称快。

在这个基础上,群众性的集体农庄运动开始了,接着又全面铺开了:它到1929年底时急剧发展起来,那空前的增长速度连我国社会主义工业也未曾有过。

集体农庄的耕地面积1928年是一百三十九万公顷,1929年是四百二十六万二千公顷,到了1930年,集体农庄已经有可能提出耕种一千五百万公顷的计划了。

斯大林同志在《大转变的一年》一文(1929年)中讲到集体农庄的增长速度时说:“应当承认,这样快的发展速度连我国社会主义化的大工业也未曾有过,虽然选种工业的发展速度一般说来已经很快了。”\footnote{见《斯大林全集》第12卷第113页。——译者注}

过是集体农庄运动发展中的转变。

这是群众性集体农庄运动的开始。

斯大林同志在《大转变的一年》一文中问道:“目前集体农庄运动中的新现象是什么呢?”他回答说:

\begin{quotation}
“目前集体农庄运动中具有决定意义的新现象,就是农民已经不象从前那样一批一批地加入集体农庄,而是整村、整乡、整区、甚至整个专区地加入了。这是什么意思呢?这就是说,中农加入集体农庄了。这是农业发展中的根本转变的基础,而这个根本转变是苏维埃政权……最重要的成就。”\footnote{同上,第118页。——译者注}
\end{quotation}

这就是说,在全盘集体化基础上消灭富农阶级的任务正在成熟,或者说已经成熟了。


\subsection{简短的结论}

党在1926—1929年为国家社会主义工业化的斗争中克服了国内和国际的巨大困难。党和工人阶级的努力使国家社会主义工业化政策获得了胜利。

工业化最困难的任务之一,即为建设重工业积累资金的任务,基本上已经解决了。能够重新装备整个国民经济的重工业的基础奠定了。

采取了社会生义建设的第一个五年计划。大规模地开展了新工厂、国营农场和集体农庄的建设。

在沿着社会主义道路向前发展的同时,国内阶级斗争尖锐起来,党内斗争尖锐起来。这一斗争最重要的结果是:镇压了富农的反抗,揭露了托洛茨基派—季诺维也夫派投降主义者联盟是反苏联盟,揭露了右倾投降主义者是富农的代理人,驱逐了托洛茨基派出党,确认托洛茨基派的观点和右倾机会主义者的观点同联共(布)党籍不能相容。

托洛茨基派被布尔什维克党在思想上打败、在工人阶级中丧失了任何根基后,已不再是一个政治派别,而变成了一伙毫无原则的怀有野心的政治骗子,一帮政治上的两面派。

党在奠定了重工业的基础之后,就动员工人阶级和农民实现苏联社会主义改造的第一个五年计划。全国各地展开了千百万劳动群众的社会主义竞赛,掀起了蓬勃的劳动高潮,培养出了新的劳动纪律。

这一时期以大转变的一年告终,这个转变的标志就是社会主义在工业中获得了极大的成就,农业取得了第一批重大的成就,中农转向集体农庄方面,群众性的集体农庄运动开始。


