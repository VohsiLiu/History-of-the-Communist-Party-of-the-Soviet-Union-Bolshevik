\section[第十一章\q 布尔什维克党为实现农业集体化而斗争(1930—1934年)]{第十一章\\ 布尔什维克党为实现农业集体化而斗争 \\{\zihao{3}(1930—1934年)}}

\subsection[一\q 1930—1934年间的国际形势。资本主义国家的经济危机。日本强占中国东三省。德国法西斯分子上台。两个战争策源地]{一\\ 1930—1934年间的国际形势。\\资本主义国家的经济危机。日本强占中国东三省。\\德国法西斯分子上台。两个战争策源地}

苏联在国家社会主义工业化方面获得了重大的成就、迅速地发展着工业,而资本主义国家却在1929年底爆发了破坏力空前的世界经济危机,并在以后的三年中加深了这一危机。工业危机同农业危机交织在一起,又使资本主义国家的状况更加恶化。

在危机的三年(1930—1933年)内,苏联工业增长了一倍以上,在1933年达到了1929年的百分之二百零一,而美国工业在1933年底却降到1929年的百分之六十五,英国工业降到百分之八十六,德国工业降到百分之六十六,法国工业降到百分之七十七。

这种情况再次显示了社会主义经济制度对资本主义经济制度的优越性。这表明社会主义国家是世界上唯一没有经济危机的国家。

由于世界经济危机的结果,有二千四百万失业工人陷于饥饿、贫困,痛苦的境地。几千万农民一直受着农业危机的折磨。

世界经济危机使帝国主义国家之间、战胜国与战败国之间、帝国主义国家与殖民地和附属国之间、工人与资本家之间、农民与地主之间的矛盾,更加尖锐起来。

斯大林同志在党的第十六次代表大会上所作的总结报告指出,资产阶级将从两方面寻找摆脱经济危机的出路——一方面是建立法西斯专政,即建立资本主义极端反动分子、极端沙文主义分子,极端帝国主义分子的专政来镇压工人阶级;另一方面是挑起重新分割殖民地和势力范围的战争来掠夺防御能力薄弱的国家。

结果正是如此。

1932年,日本发动战争的危险加大了。日本帝国主义者看到欧洲列强和美国为应付经济危机在国内忙得不可开交,决定利用这个机会对防御能力薄弱的中国施加压力,企图把它征服而成为那里的主宰。日本帝国主义者没有向中国宣战,狡诈地利用他们自己制造的“地方事件”,偷偷地把自己的军队开进东三省。日本军队完全占领了东三省,为侵占中国北部和进攻苏联准备了合适的阵地。为了便于自由行动,日本退出了国际联盟,并加紧扩充自己的军备。

这种情况促使美英法三国去加紧扩充它们在远东的海上军备。日本的目的显然是征服中国,并把欧美帝国主义列强从这里赶走。后者就以加紧扩充军备来对付。

但日本还有一个目的,就是侵占苏联的远东地区。苏联当然不能把这种危险置于不顾,于是就来努力加强近东边区的防御能力。

这样,由于日本法西斯化的帝国主义者作祟,在远东形成了第一个战争策源地。

经济危机使资本主义矛盾不仅在远东尖锐化,而且在欧洲也尖锐化了。工农业危机旷日持久,工人大批失业,贫苦阶级生活无着,加强了工农的不满情绪。不满情绪转变成工人阶级的革命义愤。在德国,由于这个国家被战争、被付给英法战胜国的赔款以及经济危机弄得民穷财尽,由于工人阶级被本国资产阶级和英法外国资产阶级压得喘不过气来,不满情绪特别强烈。德国共产党在法西斯分子上台前最后一次国会选举中获得六百万张选票,就雄辩地说明了这一点。德国资产阶级看到,德国保存着的资产阶级民主自由会使它吃苦头,工人阶级可以利用这些自由来开展革命运动。因此它认定,为了在德国保持资产阶级政权,唯一的办法就是消灭资产阶级的自由,把国会化为乌有,建立一个资产阶级民族主义的恐怖专政,一个能够镇压工人阶级而把充满复仇主义情绪的小资产阶级群众作为自己的基础的专政。于是它就叫那个为了欺骗人民而自称国家社会党的法西斯党上台执政,因为它清楚地知道。第一,法西斯党是帝国主义资产阶级中最反动最仇视工人阶级的部分;第二,法西斯党是个复仇主义色彩最浓厚的党,能把千百万怀有民族主义情绪的小资产阶级群众吸引到自己一边。工人阶级的叛徒——德国社会民主党的首领在这方面帮了忙,他们用自己的妥协主义政策替法西斯主义扫清了道路。

就是这些条件使德国法西斯分子在1933年取得了政权。

斯大林同志在党的第十七次代表大会上作总结报告时分析了德国的事变,他说:

\begin{quotation}
“不仅应当把法西斯主义在德国的胜利看作工人阶级软弱的表现,看作替法西斯主义扫清道路的社会民主党叛变工人阶级的结果,而且应当把它看作资产阶级软弱的表现,看作资产阶级已经不能用国会制度和资产阶级民主制的旧方法来实行统治,因而不得不在对内政策上采用恐怖的管理方法的表现……”(斯大林《列宁主义问题》俄文第10版第545页)\footnote{见斯大林《列宁主义问题》第515页。——译者注}
\end{quotation}

德国法西斯分子火烧国会、残酷镇压工人阶级、消灭工人阶级的组织、消灭资产阶级的民主自由。以此表明了他们要实行什么样的对内政策。他们退出国际联盟、公开准备战争,企图用暴力手段按德国的需要来修改欧洲国家的边界,以此表明了他们要实行什么样的对外政策。

这样,由于德国法西斯分子作祟,在欧洲的中心形成了第二个战争策源地。

苏联当然不能把这样严重的事实置于不顾。于是它就警惕地注视着西方事变的进程.并加强它在西部边陲的防御能力。


\subsection[二\q 从限制富农分子的政策进到消灭富农阶级的政策。同歪曲党在集体农庄运动中的政策的行为作斗争。对资本主义成分进行全线进攻。党的第十六次代表大会]{二\\从限制富农分子的政策进到消灭富农阶级的政策。\\同歪曲党在集体农庄运动中的政策的行为作斗争。\\对资本主义成分进行全线进攻。党的第十六次代表大会}

1929—1930年全面铺开的农民加入集体农庄的群众性运动,是党和政府过去全部工作的结果。社会主义工业已发展到开始为农业大批生产拖拉机和农业机器;1928年和1929年粮食收购运动期间对富农进行了坚决的斗争,农业合作社已发展到使农民逐渐习惯了集体经济;第一批集体农庄和国营农场提供了良好的经验,——这一切为过渡到全盘集体化,为农民整村,整区、整个专区地加入集体农庄的运动做好了准备。

向全盘集体化过渡并不是基本农民群众简单地和平地加入集体农庄。而是一场农民反对富农的群众性斗争。实行全盘集体化就是把全村所有的土地转交集体农庄,但其中相当一部分土地是在富农手里,因此农民就把富农从土地上赶走,剥守富农的财产。夺取耕畜和机器,并要求苏维埃政权逮捕和驱逐富农。

所以,全盘集体化就是消灭富农。

这就是在全盘集体化基础上消灭富农阶级的政策。

当时苏联已经有充分的物质基础来铲除富农,打垮他们的反抗,消灭他们这个阶级,并用集体农庄和国营农场的生产来代替他们的生产。

1927年,富农生产了六亿多普特粮食,其中商品粮约一亿三千万普特。而集体农庄和国营农场在1927年能够提供的商品粮仅仅三千五百万普特。1929年,由于布尔什维克党采取发展集体农庄和国营农场的坚定方针,由于社会主义工业在以拖拉机和农业机器供给农村方面取得了成绩,集体农庄和国营农场已成长为一支重大的力量。就在这一年,集体农庄和国营农场生产了不下四亿普特的粮食,其中商品粮已超过一亿三千万普特,即超过富农在1927年提供的数量。而在1930年,集体农庄和国营农场应该提供而且确实提供了的商品粮已达四亿多普特,即远远地超过了富农在1927年提供的效量。

这样,由于我国经济方面的阶级力量发生了变动,由于用集体农庄和国营农场的生产来代替富农的粮食生产所必需的物质基础已经具备,布尔什维克党就有可能从限制富农的政策过渡到在全盘集体化基础上消灭富农阶级的新政策。

1929年以前苏维埃政权实行的是限制富农的政策。苏维埃政权向富农征收高额赋税,要求他们按照同定价格把粮食卖给国家,颁布土地租佃法把富农的土地使用限制在一定的范围内,颁布个体农民经济使用雇佣劳动法来限制富农经济的规模。但是苏维埃政权还没有实行消灭富农的政策,因为土地租佃法和劳动雇佣法容许富农存在,而禁止剥夺富农财产的命令又对此给予了一定的保障。这样的政策限制了富农的增长,使那些经不住这种限制的个别富农阶层受到排挤和陷于破产。但这一政策并没有消灭富农阶级的经济基础,井没有消灭富农。这是限制富农的政策,而不是消灭富农的政策。这政策在一定时期,即在集体农庄和国营农场力量还薄弱、不能用自己的粮食生产代替富农的生产的时期,是必要的。

1929年底,由于集体农庄和国营农场的发展,苏维埃政权放弃这个政策而实行了个急剧的转变。它采取了消灭的政策,采取了消灭富农阶级的政策。废除了土地租佃法和劳动雇佣法,从而使富农失去了土地和雇工。取消了禁止剥夺富农财产的命令。允许农民没收富农的耕畜、机器和其他农具转交集体农庄。富农被剥夺了。如同1918年资本家在工业中被剥夺一样,不过有这样一个区别:富农的生产资料这一次不是转归国家,而是转归联合起来的农民,即转归集体农庄。

这是一个极其深刻的革命变革,是从社会的旧质态到新质态的飞跃,按其结果来说,与1917年10月的革命变革具有同等的意义。

这个革命的特点,就在于它上有国家政权的倡议,下有千百万农民群众反对富农盘剥、争取自由的集体农庄生话这一斗争的直接支持。

这个革命一举解决了社会主义建设中的三个根本同题:

(一)它消灭了我国人数最多的一个剥削阶级,即作为资本主义复辟支柱的富农阶级;

(二)它使我国人数最多的一个劳动阶级,即农民阶级,从产生资本主义的个体经济的道路转上了公共经济、集体农庄经济、社会主义经济的道路;

(三)它在农业这个最广泛的,生活必需的、但又最落后的国民经济部门中,为苏维埃政权建立了社会主义基础。

这样就在国内消灭了资本主义复辟的最后根源,同时创造了建成社会主义国民经济所必需的新的具有决定意义的条件。

斯大林同志1929年在论证消灭富农阶级的政策和指出全盘集体化这一群众性农民运动的结果时写道:

\begin{quotation}
“世界各国资本家梦想在苏联恢复资本主义的最后一点希望——‘神圣的私有制原则’正在破灭,正在化为泡影。被他们看作资本主义滋养料的农民正在大批地离开被颂扬的‘私有制’旗帜而走上集体制的轨道,走上社会主义的轨道。恢复资本主义的最后一点希望正在破灭。”(斯大林《列宁主义问题》俄文第10版第296页)\footnote{见斯大林《列宁主义问题》第336页。——译者注}
\end{quotation}

联共(布)中央在1930年1月5日《关于集体化的速度和国家帮助集体农庄建设的办法》这一历史性决议中,明文规定了消灭富农阶级的政策。这一决议充分估计到了苏联不同地区的不同条件,充分估计到了苏联不同地区对集体化准备的不同程度。

当时规定了不同的集体化速度。联共(布)中央按照集体化的速度把苏联各地区分为三类。

属于第一类的是对集体化最有准备,拖拉机较多、国营农场较多、在过去的粮食收购运动中同富农斗争的经验较多的那些最重要的产粮区,即北高加索(库班、顿河、捷列克)、伏尔加河中游、伏尔加河下游。中央提议这一类产粮区在1931年春基本完成集体化。

第二类产粮区,即乌克兰、中央黑土区、西伯利亚、乌拉尔、哈萨克斯坦等地,可以在1932年春基本完成集体化。

其余各州、边区和共和国(莫斯科州、南高加索,中亚细亚各共和国等地),集体化完成的时间可以到五年计划期末,即到1933年。

党中央认为,由于集体化速度日益增长,必须加速建立生产拖拉机,农具等等的工厂。同时,中央要求“坚决反对在集体农庄运动现阶段轻视马匹牵引作用的倾向,因为选种倾向台导致随意处置和变卖马匹”。\footnote{见《苏联共产党决议汇编》第4分册第114页。——译者注}

1929—1930年度给集体农庄的信贷增加了一倍(达五亿卢布)。

规定由国家出资给集体农庄进行土地规划。

在这个决议中有一个极重要的指示:集体农庄运动在现阶段的主要形式是只把基本生产资料集体化的农业劳动组合。

中央十分严肃地提醒各级党组织注意,必须“反对任何从上面对集体农庄运动‘发号施令’的做法,因为这会造成一种危险,把集体化当作儿戏,而不是开展真正社会主义的集体化竞赛”(《联共(布)决议汇编》俄文版第2册第662页)\footnote{同上,第115页。——译者注}。

中央的这一决议把如何贯彻执行党在农村的新政策的问题讲清楚了。

在消灭富农和实行全盘集体化这一政策的基础上,蓬勃的集体农庄运动全面铺开了。农民整村、整区地加入集体农庄,扫除了前进路上的富农,摆脱了富农的盘剥。

可是,在集体化取得巨大成就的同时,在党的工作人员的实际工作中也很快暴露出一些缺点,即对党的集体农庄建设政策的歪曲。尽管中央提醒不要在集体化取得成就时头脑过于发热,许多党的工作人员还是不顾当时当地的条件,不顾农民加入集体农庄的准备程度,人为地加速集体化。

当时出现了违背集体农庄建设自愿原则的现象。有些地区不贯彻自愿原则,而以所谓“剥夺富农财产”、剥夺选举权等威胁手段来强迫加入集体农庄。

有些地区不做集体化的准备工作,不去耐心地解释党的集体化政策的一些原则,而是官僚主义地在上面发号施令,造成虚报集体农庄数字的现象,人为地扩大集体化的百分数。

中央指示集体农庄运动的基本环节是只把基本生产资料实行公有的农业劳动组合,但有些地方不顾这一指示,鲁莽地跳过劳动组合而径直组织公社,把住宅、自用奶牛、小性畜和家禽等等都实行公有。

某些州的领导人刚一看见集体化取得成绩就头脑发热,违背了中央关于集体化速度和期限的明确指示。莫斯科州为了追求浮夸的数字,竟责成该州工作人员在1930年春完成集体化,虽然他们至少还有三年的时间(到1932年底)。在南高加索和中亚细亚,违背指示的情况更为严重。

富农及其应声虫就利用这种过火行为进行挑拨,提出组织公社而不组织农业劳动组合,提出立刻把住宅、小牲畜和家禽实行公有。同时,富农还鼓动农民在加入集体农庄前把牲畜杀掉,诱骗农民说牲畜到集体农庄内“反正会被没收”。阶级敌人指望地方组织在集体化运动中所犯的过火行为和错误会激怒农民,会激起反对苏维埃政权的叛乱。

由于党组织的错误和阶级敌人的直接挑拨,1930年2月下半月,在集体化普遍取得无可置疑的成就的情况下,有些地区出现了农民严重不满的危脸征兆。有的地方,富农及其走狗甚至煽起了农民公开反对苏维埃。

党中央在接到一连串警报、知道党的路线遭到歪曲因而集体化有失败的危险之后,就立刻开始纠偏,努力把党的干部引上迅速改正错误的道路。1930年3月2日,根据中央决定,发表了斯大林同志的《胜利冲昏头脑》一文。这篇文章警告了所有由于集体化成就而头脑发热,犯了严重错误和离开了党的路线的人,警告了所有企图用行政强迫手段使农民转上集体农庄道路的人。文章特别强调了集体农庄建设的自愿原则,并指出在规定集体化的速度和方法时必须考虑苏联不同地区的不同条件。斯大林同志提醒说,集体农庄运动的基本环节是农业劳动组合,这种劳动组合只把基本生产资料,主要是粮食生产方面的基本生产资料实行公有,而宅旁园地、住宅、一部分奶牛、小牲畜、家禽等等不实行公有。

斯大林同志的文章具有极大的政治意义。这篇文章帮助党组织改正了错误,极其有力地回击了苏维埃政权的敌人,因为这些敌人原以为他们一定能利用这种过火行为来煽起农民反对苏维埃政权。广大农民群众确信,布尔什维克党的路线与某些地方发生的鲁莽的“左的”过火行为毫无共同之处。这篇文章使农民群众安了心。

为了把斯大林同志的文章所发动的纠正过火行为和纠正错误的运动进行到底,联共(布)中央决定对这些错误再一次进行打击,于1930年3月15日公布了《关于反对歪曲党在集体农庄运动中的路线》的决议。

这个决议详细分析了所犯的错误,认为这是离开党的列宁斯大林路线的结果,是直接违背党的指令的结果。

中央指出;“左的”过火行为是对阶级敌人的直接帮助。

中央提议:“撤换那些不善于或不愿意同歪曲党的路线的行为作坚决斗争的工作者。”(《联共(布)决议汇编》俄文版第2册第663页)\footnote{见《苏联共产党决议汇编》第4分册第120页。——译者注}

中央对几个犯了政治错误而又不能改正错误的州和边区的党组织(莫斯科州、南高加索)的领导进行了改组。

1930年4月3日,发表了斯大林同志的《答集体农庄庄员同志们》一文。文章指出,在农民问题上发生错误的根源和集体农庄运动中的主要错误是:不正确地对待了中农,违背了列宁的建立集体农庄的自愿原则,违背了列宁关于必须估计到苏联不同地区的不同条件的原则,跳过了劳动组合而径直组织公社。

由于采取了这一切措施,党克服了一些地区的地方工作人员的过火行为。

只是因为中央非常坚决,善于逆潮流而进,才把党内很大一部分因取得成就而头脑发热,离开党的路线使劲往下滑的干部及时引上了正确道路。

党把歪曲党在集体农庄运动中的路线的行为克服了。

结果就把集体农庄运动的成就巩固了。

结果就为集体农庄运动的进一步蓬勃发展打下了基础。

在党采取消灭富农阶级的政策以前,为消灭资本主义成分而对资本主义成分进行的严重进攻,主要是在城市方面、工业方面。农业、农村暂时还落后于工业,落后于城市。因此,进攻还是片面的、不全面的、非总攻性质的。但是现在,农村的落后已开始从画面上消失,农民为消灭富农而进行的斗争已十分清楚地呈现出来,党已进而采取消灭富农的政策,所以对资本主义成分的进攻有了总攻的性质,片面的进攻转变成了全线的进攻。到召开党的第十六次代表大会时,对资本主义成分的总攻已经全线展开了。

党的第十六次代表大会于1930年6月26日开幕。出席这次大会的有一千二百六十八名有表决杈的代表和八百九十一名有发言权的代表,代表着一百二十六万零八百七十四名党员和七十一万一千六百零九名预备党员。

党的第十六次代表大会是作为“社会主义在全线展开大规模进攻、消灭富农阶级和实现全盘集体化的代表大会”(斯大林)\footnote{见《斯大林全集》第12卷第295页。——译者注}载入党的史册的。

斯大林同志在中央政治报告中指出:布尔什维克党在开展社会主义进攻中取得了极其重大的胜利。

在社会主义工业化方面,工业在国民经济总产值中的比重已超过农业的比重。在1929—1930经济年度,工业的份额在整个国民经济总产值中已至少占万分之五十三,而农业的份额约占百分之四十七。

党的第十五次代表大会时,即在1926—1927年度,全部工业的总产值还只等于战前水平的百分之一百零二点五,而到第十六次代表大会时,即在1929—1930年度,已约为战前水平的百分之一百八十了。

重工业,即生产生产资料的机器制造业,更加强大了。

\begin{quotation}
斯大林同志在全场的热烈掌声中说:“……我们正处在由农业国变为工业国的前夜。”\footnote{见《斯大林全集》第12卷第232页。——译者注}
\end{quotation}

斯大林同志解释说,但是不能把工业发展的高速度和工业发展的水平混为一谈。虽然社会主义工业发展的速度是空前的,但按工业发展的水平我们还远远落后于先进资本主义国家。例如,电力生产的情形就是如此,尽管苏联在电气化方面已取得了重大的成就。金属生产的情形也是如此。苏联的生铁产量,根据计划在1929—1930年度末应为五百五十万吨,而德国1929年的生铁冶炼量为一千三百四十万吨,法国为一千零四十五吨。为了在最短期间消灭我国技术和经济落后的情形,必须进一步加快我国工业的发展速度,必须同企图减低社会主义工业发展速度的机会主义分子作最坚决的斗争。

\begin{quotation}
斯大林同志指出。“……那些胡说必须减低我国工业发展速度的人,是社会主义的敌人,是我们阶级敌人的代理人。”(《列宁主义问题》俄文第10版第369页)\footnote{见《斯大林全集》第12卷第240页。——译者注}
\end{quotation}

在第十五年计划第一年度计划顺利完成和超额完成后,群众中提出了“五年计划四年完成”的口号。在许多先进工业部门(石油工业、泥炭工业、普通机器制造业、农业机器制造业、电机工业),计划执行得非常顺利,甚至可在两年半至三年内完成这些部门的五年计划。这就证实了“五年计划四年完成”的口号是完全现实的,并揭穿了那些缺乏信心、怀疑这个口号能实现的人的机会主义。

第十六次代表大会委托党中央“保证今后仍然以战斗的布尔什维克的速度进行社会主义建设,真正做到五年计划四年完成”\footnote{见《苏联共产党决议汇编》第4分册第136页。——译者注}。

到党的第十六次代表大会对,苏联的农业发展实现了极其重大的转变。广大农民群众完全转到社会主义方面来了。截至1930年5月1日,在各产粮州的主要产粮区,集体化的农户已达百分之四十至五十(而在1928年春为百分之二三)。集体农庄的耕地面积已达三千六百万公顷。

这样,中央1930年1月5日决议中所规定的那个提高了的计划(三千万公顷)超额完成了。而集体农庄建设的五年计划,在两年内超额百分之五十多完成了。

集体农庄的商品产量三年内增加了三十九倍多。1930年,国家从集体农庄(国营农场不算)取得的商品粮,已占国内粮食的全部商品产量的一半多。

这就是说。今后决定农业命运的将不是个体农户而是集体农庄和国营农场了。

如果说在农民加入集体农庄的群众性运动以前,苏维埃政权主要是依靠社会主义工业。那么今后它也要开始依靠农业中迅速增长的社会主义成分即集体农庄和国营农场了。

正如党的第十六次代表大会的一个决议所指出的,集体农民已经成了“苏维埃政权真正的可靠的支柱”\footnote{见《苏联共产党决议汇编》第4分册第171页。——译者注}。


\subsection[三\q 改造国民经济一切部门的方针。技术的作用。集体农庄运动的进一步发展。机器拖拉机站中的政治部。五年计划四年完成的总结。社会主义的全线胜利。党的第十七次代表大会]{三\\改造国民经济一切部门的方针。技术的作用。\\集体农庄运动的进一步发展。机器拖拉机站中的政治部。\\五年计划四年完成的总结。社会主义的全线胜利。\\党的第十七次代表大会}

在重工业特别是机器制造业不仅已经建立和巩固、而且向前发展得相当迅速的时候,党面临的一个首要任务是:在现代化新技术的基础上改造国民经济的一切部门。必须向燃料工业、冶金工业、轻工业、食品工业、森林工业、军事工业、运输业和农业提供现代化的新技术,提供新机床和新机器。由于对农产品和工业品的需求大量增长,必须使国民经济一切部门的产量增加一两倍。但是如果没有充分的现代化的新设备来供给工厂、国营农场和集体农庄,就不可能做到这一点,因为旧的设备不能使产量提高得这么快。

不改造国民经济各个基本部门,就不能满足国家及其国民经济日益增长的、更高的需要。

不改造,社会主义的全线进攻就不能进行到底,因为要打垮和彻底消灭城乡资本主义成分,不仅要依靠新的劳动组织和新的所有制,而且要依靠新的技术。依靠自己技术的优越。

不改造,就不能在技术和经济方面赶上并超过先进的资本主义国家,因为从工业的发展速度看,苏联已超过这些国家,但从工业的发展水平看,从产量看,苏联还大大落后于它们。

为了消灭这种落后,必须用新的技术来供给我国整个国民经济,必须在现代化新技术的基础上改造国民经济一切部门。

这样,技术就有了决定性的意义。

阻碍这件事情的,并不是新机器和新机床不够,因为机器制造工业已能提供新的设备,而是我们的经济工作人员不正确地对待技术,低估技术在改造时期的作用,鄙薄技术。我们的经济工作人员认为:技术是“专家”的事情,是交给“资产阶级专家”去做的一种次要的事情;党员经济工作人员不应干预生产技术;他们应该抓的不是技术,而是更重要的事情,即对生产的“一般”领导。

这样,让资产阶级“专家”管生产上的事情,而党员经济工作人员自己则从事“一般”领导,即签署公文。

用不着证明,采取这样的态度,“一般”领导必然变成“一般”地空谈领导,为签公文而签公文,在一纸公文上瞎忙。

当然,在党员经济工作人员这样鄙薄技术的情况下,我们就不仅永远不能超过、而且也永远不能赶上先进的资本主义国家。这样对待技术,而且是在改造时期,就会使我国注定要落后,使我们的发展速度注定要降低。这样对待技术实质上掩盖和掩饰了一部分党员经济工作人员内心深处的愿望——放慢、降低工业发展速度而为自己造成一种“安静的环境”,办法就是把生产的担子推给“专家”。

必须使党员经济工作人员面向技术,使他们对技术产生兴趣,向他们指明:掌握新技术是布尔什维克经济工作人员切身的事情,不掌握新技术就有使我们祖国永远落后,永远不能翻身的危险。

这个任务不解决就不能前进。

斯大林同志1931年2月在工业工作人员第一次代表会议上的演说在这方面起了极重大的作用。

\begin{quotation}
斯大林同志在演说中说:“人们有时问:不能稍微放慢速度,延缓进展吗?不,不能,同志们!决不能减低速度!……延缓速度就是落后。而落后者是要挨打的。但是我们不愿意挨打。不,我们绝对不愿意!

旧俄的历史,其中有一点,就是它因为落后而不断挨打。蒙古的可汗打过它。土耳其的贵族打过它。瑞典的封建主打过它。波兰和立陶宛的地主打过它。英国和法国的资本家打过它。日本的贵族打过它。大家都打过它,就是因为它落后。……

我们比先进国家落后了五十年至一百年。我们应当在十年内跑完这一段距离。或者我们做到这一点,或者我们被人打倒。……

我们至多在十年内就应当跑完我们落后于先进资本主义国家的距离。我们有一切‘客观的’可能性来做到这一点。所缺乏的只是真正利用这些可能性的本领。而这是取决于我们自己的。并且仅仅是取决于我们自己的!已经是我们学会利用这些可能性的时候了。已经是抛弃那种不干预生产的陈腐方针的时候了。已经是领会另一个方针,即适合于目前时期的要干预一切的新方针的时候了。如果你是厂长,你就要干预一切事务,就要熟悉一切,什么都不要忽略过去,就要学习再学习。布尔什维克应当掌握技术。已经是布尔什维克自己成为专家的时候了。在改造时期,技术决定一切。”(斯大林《列宁主义问题》俄文版第10版第444—446页)\footnote{见斯大林《列宁主义问题》第399—402页。——译者注}
\end{quotation}

斯大林同志的演说的历史意义,在于它结束了党员经济工作人员对技术的鄙薄态度,使他们面向技术,开辟了为布尔什维克自己掌握技术而斗争的新时期,从而促进了国民经济改造工作的开展。

从这时起,技术不再由资产阶级“专家”所垄断而变成了布尔什维克经济工作人员自己切身的事情,而“专家”这一鄙薄的称呼变成了掌握技术的布尔什维克的光荣称号。

从这时起,必然会出现而后来也确实出现了一批一批、成千成万掌握了技术和能够领导生产的红色专家。

这是工人阶级和农民的新的即苏维埃的生产技术知识分子,他们现在是我们经济领导中的基本力量。

所有这些都必然会促进并且也确实促进了国民经济改造工作的开展。

改造工作不仅在工业和运输业方面进行,而且还以更快的速度在农业方面进行。这也可以理解,因为农业拥有的机器本来就比其他部门少,它比哪一个部门都更需要得到新机器。而现在特别需要加紧用新机器供给农业,因为现在集体农庄的建设每月每周都有新的发展,也就是说每月每周都提出了供给它成千上万台拖拉机和农业机器的新要求。

1931年,集体农庄运动有了新的发展。就主要产粮区来说,加入集体农庄的农户已占农户总数百分之八十以上。全盘集体化在这些地区基本上已经完成。在二等产粮区和经济作物区,加入集体农庄的农户已占农户总数的百分之五十以上。二十万个集体农庄和四千个国营农场的耕地面积已占全国耕地面积的三分之二,而个体农民只占三分之一。

这是社会主义在农村的巨大胜利。

但是,集体农庄建设暂时还没有向深度发展,而只是向广度发展;还没有向改进集体农庄工作和集体农庄干部的质量方面发展,而只是向增加集体农庄的数量和扩大集体化的地区的方面发展。其所以如此,是因为集体农庄积极分子的成长、集体农庄干部的成长,赶不上集体农庄本身数量的增加。因此,新集体农庄中的工作并不总是令人满意的,而集体农庄本身暂时还幼弱,不巩固。农村缺少集体农庄所必需的能读会写的人(会计、管理员、秘书),农民缺乏经营集体农庄大经济的经验,这些也妨碍了集体农庄的巩固。在集体农庄里都是些昨天的个体农民。他们只有经营小块土地的羟验,还没有领导集体农庄大经济的经验。要获得这样的经验是需要时间的。

由于这些情况,集体农庄工作在最初一个时期暴露了一些严重的缺点。当时发现集体农庄的劳动还组织得不好,劳动纪律松弛。有许多集体农庄不是按劳动日而是按人口分配收入。常有这样的时候,懒汉竟比埋头苦干不耍滑的庄员分的粮食还多。由于集体农庄领导方面的这些缺点,庄员们对工作的切身利害感降低了,许多人甚至在大忙季节也不出工,一部分集体农庄庄稼直到下雪时还没有收割,并且收也收得不细,糟蹋很厉害。机器和马匹无人照管,工作没有专人负责,也削弱了集体农庄经济,减少了集体农庄的收入。

在那些被过去的富农和富农走狗钻进集体农庄窃取了某些职务的地区,情况特别严重。被剥夺了财产的富农往往跑到另外一个没有人认识他们的地区,钻进那里的集体农庄进行破坏和捣乱。有时候由于党和苏维埃的工作人员缺乏警惕性,富农竞钻进了本地区的集体农庄。过去的富农之所以易于钻进集体农庄,是因为他们对集体农庄急剧改变了斗争策略。从前,富农公开反对集体农庄,用残酷手段来对付集体农庄积极分子、先进庄员,躲在角落里对他们放冷抢,焚烧他们的房屋和仓库等等。当时富农想用这种手段吓倒农民群众,不让他们加入集体农庄。现在,公开反对集体农庄的斗争已经失败,他们就改变自己的策略。他们现在已经不用半截枪杀人了,而是装得很温和、驯服、听话和完全像个苏维埃人的样子。他们钻进集体农庄之后,就暗中进行破坏。他们到处设法从内部瓦解集体农庄,破坏集体农庄的劳动纪律,把计算收成和计算劳动的工作搅乱,富农企图灭绝集体农庄的马匹,并且居然害死了许多马匹。富农故意让马匹感染上鼻疽、疥癣和其他病症,不给马匹任何照料,如此等等。富农破坏拖拉机和其他机器。

富农所以能够欺骗庄员和进行破坏活动而不受到惩罚,是因为集体农庄还幼弱和没有经验,而集体农庄的干部还没有成长起来。

为了消灭富农对集体农庄的破坏活动和加速巩固集体农庄,必须迅速地、认真地给集体农庄以人员、意见和领导方面的帮助。

布尔什维克党给了集体农庄这样的帮助。

1933年1月,党中央通过决议:在为集体农庄服务的机器拖拉机站中成立政治部。为了帮助集体农庄,向农村派出了一万七千名党的工作人员去担任政治部工作。

这是一种重大的帮助。

机器拖拉机站政治部在两年(1933年和1934年)内做了大量的工作,来消除集体农庄工作的缺点,培养集体农庄的积极分子,巩固集体农庄,清除集体农庄中的敌对分子,即从事破坏活动的富农分子。

政治部光荣地完成了所担负的任务:在组织上经济上巩固了集体农庄,培养了新的集体农庄干部,整顿了集体农庄的业务领导,提高了庄员群众的政治水平。

全苏集体农庄突击队员第一次代表大会(1933年2月)和斯大林同志在这次大会上的演说,对提高庄员群众为巩固集体农庄而斗争的积极性起了巨大的作用。

斯大林同志在演说中对比了农村中集体农庄以前的旧制度和集体农庄这个新制度,他说:

\begin{quotation}
“在旧制度下,农民进行单干,用古老陈旧的方法和旧式农具工作,为地主和资本家、为富农和投机分子工作,自己过着半饥半饱的生活而使别人发财致富。在新的集体农庄制度下,农民按劳动组合的方式共同工作,用新式农具——拖拉机和农业机器工作,为自己和自己的集体农庄工作,过着没有资本家和地主、没有富农和投机分子的生活,为了使自己的物质生活状况和文化生活状况一天比一天改善而工作。”(《列宁主义问题》俄文第10版第528页)\footnote{见斯大林《列宁主义问题》第494—495页。——译者注}
\end{quotation}

斯大林同志在演说中指出了农民走上集体农庄道路以后所取得的实际成绩。布尔什维克党帮助千百万贫农群众加入了集体农庄,摆脱了富农盘剥。加入集体农庄并在农庄中使用着最好的土地和最好的生产工具的千百万贫农群众,从前都过着半饥半饱的生活,现在在农庄里上升到了中农的水平,成了生活有保障的人。

这是集体农庄建设道路上的第一步,第一个成绩。

斯大林同志说,第二步就是要把庄员——不论是从前的贫农或从前的中农——都提得更高,要使集体庄员都成为生活富裕的人,使所有农庄都成为布尔什维克式的农庄。

\begin{quotation}
斯大林同志说:“现在,集体农庄庄员要成为生活富裕的人,只需要一件事情,就是在集休农庄里诚实地工作,正确地使用拖拉机和机器,正确地使用耕畜,正确地耕种土地,爱护集体农庄的财产。”(列宁主义问题)俄文第10版第532—533页)\footnote{同上,第500页。——译者注}
\end{quotation}

斯大林同志的演说深深印在千百万庄员的心里,成了集体农庄的具体的战斗纲领。

到1934年底,集体农庄已成为牢固而不可战胜的力量了。当时,加入集体农庄的农户已占全苏联农户总数的四分之三左右,集体农庄的耕地面积已占全国耕地面积的百分之九十左右。

1934年苏联农业中使用的拖拉机已有二十八万一千台,联合收割机三万二千台。1934年的春播工作比1933年早十五至十天完成,比1932年早三十至四十天完成,而粮食收购计划则比1933年早三个月完成。

这样,由于党和工农国家的巨大帮助,集体农庄在两年之内就巩固了。

集体农庄制度的牢固的胜利和随之而来的农业的高涨,使苏维埃政权有可能取消面包和其他食品的配给制而规定食品可以自由购买。

由于作为临时政治机关而设立的机器拖拉机站政治部完成了自己的任务,中央决定把它们改为通常的党的机关,使其同当时党的区委会合并。

所有这些成就,无论在农业方面或在工业方匝,都是由于五年计划胜利完成而获得的。

到1933年初已看得很明显,第一个五年计划已经完成,已经提前完成,已经在四年零三个月内完成。

这是苏联工人阶级和农民所获得的具有世界历史意义的伟大胜利。

1933年,斯大林同志在党的中央委员会和中央监察委员会一月联席全会上作的报告,对第一个五年计划作了总结。从报告中可以看出,党和苏维埃政权在过去这一时期,在实现第一个五年计划的时期,取得了如下的基本成绩:

(一)苏联由农业国变成了工业国,因为工业产值在国民经济全部生产中的比重已增长到百分之七十。

(二)社会主义经济体系消灭了工业领域的资本主义成分而成了工业中唯一的经济体系。

(三)社会主义经济体系消灭了农业领域的富农阶级而成了农业中的统治力量。

(四)集作农庄制度消灭了农村中的贫穷、困苦现象,几千万贫农变成了生活有保障的人。

(五)社会主义体系在工业中消灭了失业现象,在一些生产部门里保持八小时工作制,在绝大多数企业中已改为七小时工作制,在有害于健康的企业中规定了六小时工作制。

(六)社会主义在国民经济一切部门的胜利消灭了人剥削人的现象。

第一个五年计划的这些成就的意义,首先就在于它们使工人和农民彻底摆脱了剥削的桎梏,并为苏联全体劳动者过富裕而有文化的生活开辟了道路。

1934年1月,召开了党的第十七次代表大会。出席这次大会的有一千二百二十五名有表决权的代表和七百三十六名有发言权的代表,代表着一百八十七万四千四百八十八名党员和九十三万五千二百九十八名预备党员。

大会总结了党在过去这一时期的工作,指出社会主义在经济和文化的一切部门取得了具有决定意义的成就,确认党的总路线取得了全面的胜利。

党的第十七次代表大会以“胜利者的代表大会”载入史册。

斯大林同志在总结报告中指出,苏联在报告所涉及的时期内发生了根本的变化。

\begin{quotation}
“苏联在这个时期内根本改变了样子,抛掉了落后的中世纪的外貌。它由农业国变成了工业国。它由个体小农业的国家变成了大规模机械化集体农业的国家。它由愚昧无知、不识字、没有文化的国家变成了(确切些说,正在变成)人人识字的有文化的国家,这个国家广泛地设立了用苏联各民族语言来教学的高等学校,中等学校和初等学校。”(斯大林《列宁主义问题》俄文第10版第553页)\footnote{见斯大林《列宁主义问题》第524页——译者注}
\end{quotation}

当时,社会主义工业已占我国全部工业的百分之九十九。社会主义农业,即集体农庄和国营农场,已占我国全部耕地面积的百分之九十左右。至于商品流转,资本主义成分已从商业中完全排挤出去了。

列宁在实行新经济政策时说过:我国有五种社会经济成分。第一种成分是宗法式经济,在很大程度上是自然经济,即几乎不进行任何贸易的经济。第二种成分是小商品生产,即占农民大多数的,出卖农产品的农户以及手工业者。这种经济成分在新经济政策头几年包括了大多数居民。第三种成分是私人资本主义,它在新经济政权初期活跃起来了。第四种成分是国家资本主义,主要是租让企业,它没有什么大的发展。第五种成分是社会主义,即当时还幼弱的社会主义工业,在新经济政策初期在国民经济中微不足道的国营农场和集体农庄以及在新经济政策初期同样很幼弱的国营商业和合作社。

列宁指出,在这五种成分中,社会主义成分一定会占绝对优势。

新经济政策的目的就是要使社会主义经济形式获得完全胜利。

而这个目的到党的第十七次代表大会时已经实现了。

\begin{quotation}
斯大林同志在谈到这点时说:“我们现在可以说,第一种、第三种和第四种社会经济成分已经不存在了,第二种社会经济成分已经被排挤到次要地位了,而第五种社会经济成分——社会主义成分是整个国民经济中独占统治地位的唯一的领导力量。”(斯大林《列宁主义问题》俄文第10版第555页)\footnote{见斯大林《列宁主义问题》第526页。——译者注}
\end{quotation}

政治思想领导的问题在斯大林同志报告中占有重要的地位。他提醒党说:虽然党的敌人——形形色色的机会主义者、各种民族主义倾向分子已被击败,但是他们的思想体系的残余还留在一些党员的头脑中,并且时常在兴妖作怪。经济中特别是人们意识中的资本主义残余,是已被击败的反列宁主义集团的思想体系得以死灰复燃的良好土壤。人们意识的发展落后于人们的经济地位。因此,虽然资本主义在经济中已经消灭,但资产阶级观点的残余仍然在人们的头脑中存在着,并且将来还会存在下去。同时必须估计到,应当时刻加以防范的资本主义包围势力力图复活和支持这些残余。

斯大林同志还讲到,在民族问题方面,人们意识中的资本主义残余特别有生命力。布尔什维克党进行了两条路线的斗争,既反对大俄罗斯沙文主义倾向,又反对地方民族主义倾向。在有些共和国(乌克兰、白俄罗斯等等)内,党组织放松了对地方民族主义的斗争,竟让它发展到同敌对势力结合起来、同武装干涉者结合起来,成了危及国家的一种祸害。斯大林同志在答复民族问题上什么倾向是主要危险的问题时说道:

\begin{quotation}
“主要危险就是人们停止和它作斗争因而让它发展到危害国家的那种倾向。”(同上,第587页)\footnote{同上,第563页。——译者注}
\end{quotation}

斯大林同志号召党加强政治思想工作,不断地揭露敌对阶级的和敌视列宁主义的派别的思想体系及其残余。

接着斯大林同志在报告中指出:单是通过正确的决议本身还不能保征事业成功。要保证事业成功,必须正确配备能够实现领导机关决议的人员,并组织对这些决议执行情况的检查。不采取这些组织措施,决议就有变成一纸空文而不能落实的危险。说到这里,斯大林同志援引了列宁的著名原理:组织工作主要就是挑选人员和检查执行情况,同时斯大林同志着重指出,我们实际工作中的主要祸害,就是通过的决议同执行决议、检查这些决议执行情况的组织工作脱节。

为了改进对党和政府的决议执行情况的检查,党的第十七次代表大会建立了直属联共(布)中央的党的监察委员会和直属苏联人民委员会的苏维埃监察委员会,以代替从党的第十二次代表大会以来已经完成了任务的中央监察委员会和工农检察院。

斯大林同志把党在新阶段上的组织任务规定如下:

\begin{quotation}
(一)要使我们的组织工作适应于党的政治路线的要求;

(二)把组织领导提高到政治领导的水平;

(三)使组织领导能够完全保证党的政治口号和决议得到实现。
\end{quotation}

斯大林同志在结束报告时提醒说:虽然社会主义的成就是伟大的,使我们产生了应有的自豪感,但是不要醉心于已得的成就,不要“骄傲自满”,不要高枕而卧。

\begin{quotation}
斯大林同志指出:“……不要使党高枕而卧,而要在党内提高警惕性;不要使党酣睡,而要使它保持战斗准备状态;不要解除党的武装,而要把它武装起来;不要使党涣散,而要使它保持动员状态以实现第二个五年计划。”(《列宁主义问题》俄文第10版第596页)\footnote{见斯大林《列宁主义问题》第574页。——译者注}
\end{quotation}

第十七次代表大会听取了莫洛托夫和古比雪夫两同志关于发展国民经济第二个五年计划的报告。第二个五年计划的任务比第一个五年计划的任务更加宏伟。到第二个五年计划结束的1937年,工业产值和战前水平比,大概增长七倍。在第二个五年计划时期,整个国民经济的基建项目投资规定为一千三百三十亿卢布,而第一个五年计划时期为六百四十多亿卢布。

这样多的基建项目,能保证国民经济一切部门都得到彻底的技术改造。

在第二个五年计划时期,基本上实现农业的机械化。拖拉机的总功率要从1932年的二百二十五万马力,增加到1937年的八百多万马力。还规定要广泛采用各种农艺措施(实行正确的轮作制,用精选的种籽播种,秋耕等等)。

对运输业和邮电业规定要进行工程巨大的技术改造。

提出了进一步提高工农物质生活水平和文化水平的广泛规划。

第十七次代表大会对组织问题很重视,并根据卡冈诺维奇同志的报告通过了关于党和苏维埃建设问题的专门决议。在党的总路线已经取得胜利,党的政策已经由实际生活即由千百万工农的经验检验过了的时候,组织问题就有了更大的意义。第二个五年计划规定的新的复杂任务,要求提高一切部门的工作质量。

\begin{quotation}
代表大会关于组织问题的决议说;“第二个五年计划的基本任务是,彻底消灭资本主义成分,克服经济中和人们意识中的资本主义残余,在最新技术基础上完成整个国民经济的改造,掌握新技术和新企业,实行农业机械化和提高农业生产率。这些任务非常尖锐地提出了提高一切部门的工作质量,首先是组织工作和实际工作的领导质量的问题。”(《联共(布)决议汇编》俄文版第2册第591页)\footnote{见《苏联共产党决议汇编》第4分册第384页。——译者注}
\end{quotation}

第十七次代表大会通过了新的党章,它和旧党章的不同之处首先就是增加了导言部分。党章导言部分对共产党下了一个简短的定义,说明了共产党在无产阶级斗争中的作用和在无产阶级专政机关体系中的地位。新党章详细列出了党员的义务。党章中加进了关于接收党员的更严格的规定和关于同情者小组的条文。党章更详细地阐明了党的组织机构问题,重新拟定,关于党的支部(原称支部,从党的第十七次代表大会起改称基层组织)的条文。新党章还重新拟定了关于党内民主和党的纪律的条文。


\subsection[四\q 布哈林派蜕化为政治上的两面派。托洛茨基两面派分子蜕化为一帮白卫杀人凶手和特务。谢·米·基洛夫遭凶杀。党在加强布尔什维克警惕性方面的措施]{四\\布哈林派蜕化为政治上的两面派。\\托洛茨基两面派分子蜕化为一帮白卫杀人凶手和特务。\\谢·米·基洛夫遭凶杀。\\党在加强布尔什维克警惕性方面的措施}

我国社会主义的成就不仅使我们党,不仅使工人和集体农庄庄员欢欣鼓舞,而且使我们的整个苏维埃知识界,使苏联全体忠实的公民欢欣鼓舞。

这些成就没有使被打倒的剥削阶级的余孽感到高兴,而是使他们更加恼恨。

这些成就使被打倒的阶级的应声虫——布哈林派和托洛茨基派的可怜余孽发了狂。

这些老爷评价工人和集体农庄庄员的成绩时,并不是从欢迎每一个这样的成绩的人民的利益出发,而是从自己那个可怜的脱离实际生活和完全腐化了的派别集团的利益出发。我国社会主义的成就意味着党的政策的胜利,意味着这些老爷的政策的彻底破产,因此他们不仅不承认明显的事实并加入共同的事业,反而为自己的失败和破产向党和人民进行报复,对工人和集体农庄庄员的事业进行捣乱和破坏,炸矿井,烧工厂,在集体农庄和国营农场搞破坏,以便破坏工人和集体农庄庄员的成绩,并在人民中间挑起对苏维埃政权的不满。同时,为了使自己那个可怜的集团免遭揭露和粉碎,他们戴上了一副忠诚于党的假面具,越来越起劲地巴结党,吹捧党、谄媚党,而事实上继续在暗中进行反对工农的破坏活动。

在第十七次代表大会上,布哈林、李可夫和托姆斯基作了忏悔的发言,他们颂扬党,把党的成绩吹得天花乱坠。但是代表大会感觉到,他们的发言有些言不由衷和两面派的味道,因为党要求于党员的不是吹捧和颂扬党的成绩,而是在社会主义战线上忠诚地工作,但这一点是布哈林派早已没有的了。党看到,实际上这些老爷是通过自己的虚伪发言同他们在会外的同伙遥相呼应,教他们耍两面派,叫他们不要放下武器。

在第十七次代表大会上,托洛茨基派的季诺维也夫和加米涅夫也发了言,他们过甚其词地斥责自己的错误,也过甚其词地赞扬党的成绩。但是代表大会不能不看到,无论是令人作呕的自我斥责,还是甜言蜜语的歌功颂德,都是这些老爷心地龌龊和做贼心虚的另一种表现。不过党还不知道。也没有料到,这些老爷在台上甜言蜜语的同对,已经在准备凶杀谢·米·基洛夫了。

1934年12月1日,在列宁格勒斯莫尔尼宫,谢·米·基洛夫被人用手枪凶杀了。

当场捕获的凶手,原来就是由季诺维也夫反苏集团参加者在列宁格勒组织的一个反革命地下集团的成员。

全党敬爱的,工人阶级敬爱的谢·米·基洛夫被杀害的消息,引起了我国劳动者无比的愤慨和深切的悲痛。

调查材料表明,1933—1934年,前季诺维也夫反对派的成员在列宁格勒组织了一个以所谓“列宁格勒总部”为首的反革命地下恐怖集团。这个集团的宗旨是杀害共产党的领导人。谢·米·基洛夫是预定的第一名牺牲者。从这个反革命集团的成员的口供中知道,他们同几个资本主义国家的代表有勾结,从他们那里领取经费。

这个组织中被揭露的成员,由苏联最高法院军法处判处了枪毙的极刑。

很快查出存在着一个反革命地下“莫斯科总部”。调查材料和审判表明,季诺维也夫、加米涅夫、叶甫多基莫夫和这个组织的其他领导人在培植自己同伙的恐怖主义情绪方面,在准备杀害中央委员和苏联政府成员方面,起了卑鄙的作用。

这些人的两面手法和卑鄙龌龊到了这种程度,以致季诺维也夫这样一个组织和指使杀害谢·米·基洛夫、催促凶手赶快下毒手的人,居然写了一篇颂扬死者基洛夫的悼文,要求把它登载出来。

季诺维也夫分子在法庭上装出悔过自新的样子,其实他们就在这时也还在耍两面派。他们隐瞒了自己和托洛茨基的勾结,隐瞒了他们同托洛茨基分子一起卖身投靠法西斯间谍机关的事实,隐瞒了他们的特务活动和暗害活动。季诺维也夫分子在法庭上隐瞒了自己和布哈林派的勾结,隐瞒了法西斯分子的雇佣走狗托洛茨基派—布哈林派联合匪帮的存在。

后来查明,基洛夫同志就是被这个托洛茨基派—布哈林派联合匪帮杀害的。

还在当时,即在1935年,就已很清楚,季诺维也夫集团是个暗藏的白卫组织,完全应该把它的成员当作白卫分子来严办。

一年后知道,杀害基洛夫的真正的、直接的、实际的组织者和准备杀害其他中央委员的组织者,就是托洛茨基、季诺维也夫、加米涅夫和他们的同谋者。季诺维也夫、加米涅夫、巴卡也夫、叶甫多基莫夫,皮克里、伊·恩·斯米尔诺夫、穆拉契科夫斯基、帖尔瓦加年、勒因哥里德等被交付法庭审判。这些当场捕获的罪犯,不得不在法庭上当众承认,他们不仅组织了杀害基洛夫的事件,而且还作了准备要杀害党和政府的其他一切领导人。后来调查材料表明,这些恶棍走上了组织破坏活动的道路,走上了充当特务的道路。1936年在莫斯科举行的审判,揭穿了这些人在道义上和政治上极其骇人听闻的堕落,揭穿了这些人用假装对党表示忠诚的两面派声明所掩盖的最下流的卑鄙勾当和叛卖行为。

犹大托洛茨基是这一大帮凶手和特务的主要指使人和组织者。季诺维也夫、加米涅夫及其托洛茨基主义的仆从,是托洛茨基的帮手和反革命指令的执行者。他们进行着使苏联在受到帝国主义者侵犯时遭到失败的准备工作,他们是主张工农国家失败的失败主义者,他们是德日法西斯分子的可恶奴仆和走狗。

各级党组织从谢·米·基洛夫凶杀案的审判中应该得出的基本教训是,要消灭自己的政治盲目病,消灭自己的政治麻木病,提高自己和全体党员的警惕性。

党中央在它为谢·米·基洛夫被凶杀一事发表的给各级党组织的信中指示说:

\begin{quotation}
(一)“必须铲除机会主义的好心肠,它的根源就是错误地以为随着我们力量的增长,敌人会越来越驯服、善良。这种想法是根本不正确的。这是那种硬要大家相信敌人将不声不响地爬进社会主义,最终将成为真正社会主义者的右倾思想的翻版。布尔什维克决不应高枕而卧,决不应马虎从事。我们不要好心肠,而要警惕性,真正布尔什维克的革命警惕性。要记住,敌人愈绝望,就愈要采取‘极端手段’,作为他们同苏维埃政权斗争的唯一的最后挣扎的手段。要记住这一点,并保持警惕。”

(二)“必须把下列工作提到应有的高度:在党员中讲授党史,研究我党历史上的一切反党集团,研究他们反对党的路线的斗争手段,他们的策略,尤其要研究我们党同反党集团作斗争的策略和手段,即保证我们党战胜并彻底击溃这些集团的策略和手段。必须使党员不仅知道党怎样反对和战胜立宪民主党人、社会革命党人、孟什维克和无政府主义者,而且知道党怎样反对和战胜托洛茨基派,‘民主集中派’、‘工人反对派’、季诺维也夫派、右倾分子、右的‘左’的畸形儿等等。不应该忘记,熟悉和了解我党的历史是完全保证党员具有革命警惕性所必需的最重要的手段。”
\end{quotation}

在这一时期具有重大意义的,是1933年开始的把混进来的和异己的分子清除出党的队伍的工作,特别是在谢·米·基洛夫被凶杀后进行的仔细审查党员证件和更换党员证件的工作。

在审查党员证件以前,许多党组织中任意和随便对待党证的情况很严重。在好些地方党组织中发现,党员登记工作混乱到了根本不能容忍的地步,敌人利用了这一点来实现他们的卑鄙目的,他们以党证作掩护来搞特务,暗害等等活动。许多党组织的领导人竟把接收新党员和发党证的事情交给一些很不重要的人员去作,甚至往往交给一些完全没有经过考查的党员去作。

1935年5月13日,党中央在给各级组织的一封专门讲党证的登记、发给和保存问题的信中,建议各级组织对党员证件进行一次仔细的审查,“在我们自己党的屋子里整顿一下布尔什维克的秩序”。

审查党员证件的工作具有很大的政治意义。在党中央全会1935年12月25日关于审查党员证件工作的总结这一决议中指出,这次审查对于巩固联共(布)的队伍是一项非常重要的组织和政治措施。

在党员证件审查和更换以后,恢复了接收新党员的工作。对此,联共(布)中央要求,接收新党员时,不要用集体接收的方式,而要用严格地个别地接收的方式,“从为社会主义而奋斗的不同岗位上经受了考验的工人(首先是工人)、农民和劳动知识分子中”接收“我国真正先进的、真正忠于工人阶级事业的优秀分子”入党\footnote{见《苏联共产党决议汇编》第4分册第460页。——译者注}。

中央在恢复接收新党员的工作时,要求各级党组织记住,敌对分子今后还会企图钻进联共(布)的队伍。因此:

\begin{quotation}
“每个党组织的任务就是要竭力提高布尔什维主义的警惕性,高举列宁党的旗帜,保证不让异己分子、敌对分子和偶然分子钻进党的队伍。”(联共(布)中央1936年9月29日的决议,载于1936年《真理报》第270号)
\end{quotation}

布尔什维克党清洗和巩固了自己的队伍,消灭了党的敌人,坚决地克服了歪曲党的路线的行为,更加紧密地团结在党中央的周围。在党中央的领导下,党和苏维埃国家过渡到了新的阶段。即完成无阶级的社会主义社会建设的阶段。


\subsection{简短的结论}

1930—1934年,布尔什维克党解决了无产阶级革命在夺取政权以后最困难的历史任务:使千百万小私有农户转上集体农庄道路,转上社会主义道路。

富农这一人数最多的剥削阶级被消灭和基本农民群众转上集体农庄道路,导致了资本主义在国内的最后根源的消灭、社会主义在农业中的完全胜利,苏维埃政权在农村的完全巩固。

集体农庄在克服了许多组织方面的困难后已完全巩固,并且走上了富裕生活的道路。

由于第一个五年计划的完成,我国建立了社会主义经济的坚固基础——头等的社会主义重工业和机械化的集体农业,消灭了失业,消灭了人剥削人的现象,为不断改善我们祖国劳动者的物质生活和文化生活状况创造了条件。

我国工人阶级、集体农庄庄员和一切劳动者取得这样伟大的成就,是因为党和政府实行了大胆的、革命的和英明的政策。

资本主义包围势力力图削弱和破坏苏联的威力,就加紧进行他们的“工作”:在苏联内部组织杀人凶手、暗害分子和特务的匪帮。法西斯分子在德国和日本上台后,资本主义包围势力的反苏活动特别加紧起来。法西斯主义找到了托洛茨基派和季诺维也夫派这样一批忠实的仆役来从事特务活动、暗害活动、恐怖活动和破坏活动,来促使苏联遭到失败,以便恢复资本主义。

苏维埃政权果断地惩罚了这些人类蟊贼,无情地惩治了这些人民的敌人和祖国的叛徒。

