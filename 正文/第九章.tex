\section[第九章\q 布尔什维克党在过渡到恢复国民经济的和平工作时期(1921—1925年)]{第九章\\ 布尔什维克党在过渡到恢复国民经济的\\和平工作时期 \\{\zihao{3}(1921—1925年)}}

\subsection[一\q 武装干涉和国内战争结束以后的苏维埃国家。恢复时期的困难]{一\\武装干涉和国内战争结束以后的苏维埃国家。\\恢复时期的困难}

苏维埃国家在结束战争以后,开始过渡到和平经济建设的轨道。必须医治战争创伤。必须恢复被破坏的国民经济,整顿工业、运输业和农业。

但是,过渡到和平建设这件事,必须在非常困难的环堆下进行。国内战争的胜利得来不易。国家被四年帝国主义战争和三年反革命武装干涉战争弄得贫穷不堪。

1920年的农业总产值,只等于战前的一半左右。而且战前水平还是沙俄那个贫困农村的水平。不仅如此,1920年还有许多省份歉收。农民经济情况困难。

工业情况更坏,已处于破坏状态。1920年的大工业产值,比战前几乎减少了七分之六。大多数工厂停工,矿场和矿井被破坏、被淹没。冶金业情况特别严重。生铁在1921年全年总共只炼了十一万六千三百吨,只等于战前生铁产量的百分之三左右。燃料不足。运输业遭破坏。国内原有的金属和布匹的储备差不多都已用完。国内的生活必需品,如面包、脂油、肉类、鞋类、衣服、火柴、食盐、煤油、肥皂,都痛感不足。

当战争还在进行的时候,人们对这种物品缺乏的情况还能忍受,有时甚至没有觉察。但是现在战争没有了,人们突然感到这种情况不堪忍受,要求立刻加以消除。

农民中间出现了不满情绪。在国内战争的炮火中建立了和巩固了工人和农民的军事政治联盟。这个联盟的建立是有一定的基础的:农民从苏维埃政权方面取得土地和免除地主富农压迫的保障,工人按照余粮收集制从农民方面取得粮食。

现在这种基础已经不够了。

苏维埃国家为了国防的需要,当时不得不按余粮收集制收集农民的所有余粮。不实行余粮收集制,不实行战时共产主义政策,国内战争就不可能获得胜利。战时共产主义政策是由于战争和武装干涉而被迫采取的。当战争还在进行的时候,农民接受余粮收集制,没有觉察商品不足,但是当战争已经结束、地主卷土重来的威胁已经过去的时候,农民就开始对征收全部杂粮的办法即余粮收集制感到不满,而要求供给他们充足的商品了。

整个战时共产主义制度,正如列宁指出的,同农民的利益发生了抵触。

不满情绪也侵入了工人阶级。无产阶级承受了国内战争的主要重担,英勇忘我地进行了反对自卫分子和武装干涉者群寇的斗争,进行了消除经济破坏和饥荒的斗争。最有觉悟的、富有自我牺牲精神和遵守纪律的优秀工人,表现了如火如荼的社会主义热忱。但是,极其严重的经济破坏也影响了工人阶级。还在开工的少数工厂常常长时间停产。工人们不得不从事手工业,制作打火机和从事小宗的粮食买卖。无产阶级专政的阶级基础开始削弱了,工人阶级的队伍日益涣散。一部分工人跑到农村去,不再成为工人,脱离了本阶级。一部分工人由于饥饿和疲惫产生了不满情绪。

一个问题摆在党的面前:要制定党在国内经济生活一切问题上的新方针,以适应新的情况。

于是党就着手制定关于经济建设问题的新方针。

但是阶级敌人并没有睡觉。他们企图利用经济上的困难,企图利用农民的不满情绪。白卫分子和社会革命党人策动的富农叛乱在西伯利亚、乌克兰、唐波夫省(安东诺夫叛乱)都发生了。各种反革命分子——孟什维克、社会革命党人、无政府主义者、白卫分子、资产阶级民族主义者,都积极活动起来。敌人采取了新的策略手段来反对苏维埃政权。他们现在涂上了一层苏维埃色彩,已经不提“打倒苏维埃”这种破产了的旧口号,而是提“拥护苏维埃,但是不要共产党员参加”这样的新口号。

喀琅施塔得的反革命叛乱,就是阶级敌人实行新策略的明显表现。这次叛乱发生在党的第十次代表大会开幕前一星期,即1921年3月。领导叛乱的是同社会革命党人、孟什维克和外国的代表有勾结的白卫分子。起初,叛乱者力图打出“苏维埃”的招牌,掩盖他们恢复资本家地主的政权和所有制的意图。他们提出“没有共产党员参加的苏维埃”的口号。反革命势力企图利用小资产阶级群众的不满情绪,想在假装拥护苏维埃的口号下推翻苏维埃政权。

促成喀琅施塔得叛乱的有两个情况:军舰上水兵成分的变坏和喀琅施塔得布尔什维克组织的薄弱。参加过十月革命的老水兵,几乎个个上了前线,参加红军队伍英勇作战去了。补充到海军中的新兵没有受过革命的锻炼。这些新兵都是完全没有受过训练的农民群众,反映了农民对余粮收集制的不满情绪。至于当时喀琅施塔得的布尔什维克组织,由于多次动员上前线,它已经大大削弱。这两个情况使社会革命党人、孟什维克和白卫分子有可能混进并控制了喀琅施塔得。

叛乱分子掌握了这个头等要塞、舰队和大批武器弹药。国际反革命势力已在欢庆胜利。但是敌人高兴得太早了。叛乱很快被苏维埃军队镇压下去。当时为了镇压喀琅施塔得的叛乱分子,党派去了自己的优秀子弟——以伏罗希洛夫同志为首的第十次代表大会代表。红军战士踏着薄冰向喀琅施塔得前进。冰面踩破了,许多人被淹死。必须向几乎是坚不可摧的喀琅施塔得炮台冲击。对革命的忠诚和勇敢、为苏维埃政权捐躯的决心占了上风。喀琅施塔得要塞由红军猛力攻克。喀琅施塔得叛乱被肃清了。


\subsection[二\q 党内关于工会问题的争论。党的第十次代表大会。反对派的失败。过渡到新经济政策]{二\\党内关于工会问题的争论。党的第十次代表大会。\\反对派的失败。过渡到新经济政策}

党中央委员会,党中央委员会的列宁多数清楚地知道,在战争结束并过渡到和平经济建设以后,没有理由再保持由战争和封锁的情况所造成的规定太死的战时共产主义制度。

中央懂得,余粮收集制已经没有必要了,必须代之以粮食税,好使农民能随意处理自己生产的大部分余粮。中央懂得,这种办法能活跃农业,扩大工业发展所必需的粮食和经济作物的生产,活跃国家的商品运转,改善城市供应,为工农联盟建立新的基础即经济的基础。

中央也知道,活跃工业是首要的任务。但是它认为如果不吸引工人阶级及其工会参加,就不可能活跃工业;只要说服工人相信经济破坏同武装干涉和封锁一样,也是人民的危险的敌人,他们是能够吸引来参加这项工作的;只要党和工会对工人阶级不是像在前线那样采用军事命令(在前线确实需要采用命令),而是通过说服的途径,采用说服的方法,就一定能做到这一点。

然而并非所有的党员都像中央这样想。托洛茨基派、“工人反对派”、“左派共产主义者”、“民主集中派”等反对派小集团都处于思想混乱的状态,在向和平经济建设轨道过渡的困难面前动摇不定。党内有不少从前的孟什维克、从前的社会革命党人、从前的崩得分子、从前的斗争派和俄国边沿地区的各种半民族主义者。他们大部分都参加了某个反对派小集团。这些人不是真正的马克思主义者。不懂得经济发展规律,没有受过列宁主义的党性锻炼,面只是加强了反对派小集团的思想混乱和动摇。其中有的认为不需要削弱规定太死的战时共产主义制度,相反需要“再拧紧螺丝”。有的认为党和国家应该把恢复国民经济的事情丢开不管,应该把这件事情完全交给工会。

很明显,既然思想这么混乱,党内某些成分中间就一定有一些人,爱好争论的人,各种各样的反对派“首领”,会竭力强迫党进行争论。

结果正是如此。

争论是从工会的作用问题开始的,虽然工会问题当时并不是党的政策的主要问题。

挑起争论、反对列宁、反对中央委员会中的列宁多数的急先锋是托洛茨基。他唯恐天下不乱,在1920年11月初全俄工会第五次代表会议的党员会上提出了“拧紧螺丝”和“整刷工会”的可疑口号。托洛茨基提出立刻把“工会国家化”的要求。他反对对工人群众采取说服方法。他主张把军事方法搬到工会里来。托洛茨基反对在工会里扩大民主,反对工会机关按选举产生。

托洛茨基派反对说服方法(不采用说服方法,工人组织是无法进行活动的),而主张采用赤裸裸的强制方法,赤裸裸的命令手段。凡是托洛茨基派把持工会领导的地方,他们都通过自己的政策在工会里引起了冲突、分裂和瓦解。托洛茨基派通过自己的政策来挑动非党工人群众反对党,分裂工人阶级。

实际上,工会问题争论的意义,要比工会问题本身广泛得多。如后来俄共(布)中央全会(1925年1月17日)决议所指出的,实际上当时争论的是“关于如何对待反对战时共产主义的农民,关于如何对待非党工人群众,总的是关于党在国内战争已告结束的时期如何对待群众的问题”(《联共(布)决议汇编》俄文版第1册第651页)\footnote{见《苏联共产党决议汇编》第2分册第527页。——译者注}。

跟在托洛茨基后面的还有其他的反党集团:“工人反对派”(施略普尼柯夫、梅德维捷夫、柯伦泰等人)、“民主集中派”(萨普龙诺夫、德罗布尼斯、鲍古斯拉夫斯基、奥新斯基、弗·斯米尔诺夫等人)、“左派共产主义者”(布哈林、普列奥布拉任斯基)。

“工人反对派”提出把整个国民经济交给“全俄生产者代表大会”去管理的口号。他们想把党的作用化为乌有,否定无产阶级专政在经济建设中的作用。“工人反对派”把工会同苏维埃国家和共产党对立起来。他们认为工人阶级的最高组织形式不是党而是工会。“工人反对派”实际上是个无政府工团主义的反党集团。

“民主集中派”(民集派)集团要求各派别和集团能完全自由。民集派也如托洛茨基派一样,力图破坏党在苏维埃和工会中的领导作用。列宁称民集派为“叫喊得最响亮者”\footnote{见《列宁全集》第32卷第36页。——译者注}的派别,而把它的纲领称为社会革命党—孟什维克的纲领。

布哈林帮助托洛茨基反对列宁和反对党。布哈林同普列奥布拉任斯基、谢烈布利雅柯夫、索柯里尼柯夫一起成立了“缓冲”集团。这个集团维护和掩护最凶恶的派别分子托洛茨基派。列宁称布哈林的行动是“思想瓦解达到顶点”\footnote{同上,第33页。——译者注}。不久,布哈林派就同托洛茨基联合起来反对列宁了。

列宁和列宁主义者集中主要火力打击托洛茨基派,因为托洛茨基派是反党集团的主力。列宁和列宁主义者揭露了托洛茨基派把工会同军事组织混淆起来,向他们指出不能把军事组织的方法搬到工会里来。针对各反对派集团的纲领,列宁和列宁主义者拟定了自己的纲领。这个纲领指出:工会是学习管理的学校,是学习主持经济的学校,是共产主义的学校。工会应该在自己的全部工作中贯彻说服方法。只有这样,工会才能发动全体工人消除经济破坏,才能吸引他们参加社会主义建设。

在同反对派集团的斗争中,党的各个组织团结在列宁的周围。莫斯科的斗争特别紧张。反对派在这里集中了自己的主要力量,打算夺取首都的组织。但是,莫斯科的布尔什维克给了派别分子的这种阴谋以坚决的回击。乌克兰党组织内的斗争也很尖锐。乌克兰布尔什维克在当时的乌共(布)中央书记莫洛托夫同志领导下,打败了托洛茨基派和施略普尼柯夫派。乌克兰共产党仍然是列宁党的可靠支柱。在巴库,在奥尔忠尼启则同志领导下,打垮了反对派。在中亚细亚,拉·卡冈诺维奇同志领导了反对反党集团的斗争。

主要的地方党组织全都拥护列宁的纲领。

1921年3月8日,召开了党的第十次代表大会。出席这次大会的有六百九十四名有表决权的代表,代表着七十三万二千五百二十一名党员。有发言权的代表有二百九十六人。

大会对工会问题的争论做了总结,并以压倒多数通过了列宁的纲领。

列宁在会上致开幕词时说,争论是一种不能容许的奢侈品。他指出,敌人正指望共产党发生内讧和分裂。

鉴于派别集团的存在对于布尔什维克党和无产阶级专政有莫大的危险,第十次代表大会对党的统一问题特别注意。列宁作了关于这个问题的报告。大会谴责了一切反对派集团,指出它们“实际上在帮助无产阶级革命的阶级敌人”\footnote{见《苏联共产党决议汇编》第2分册第69页。——译者注}。

大会责令立刻解散一切派别集团,并责成各级组织密切注意禁止任何派别活动;而且凡是不执行代表大会此项决议的,都将被立即无条件地开除出党。大会授权中央委员会,在中央委员违反纪律、恢复或进行派别活动时,可以采取党内一切处分办法,直到把他们开除出中央委员会和开除出党。

所有这些决定,都写进了由列宁提出、经大会通过的《关于党的统一》的专门决议中。

在这个决议中,大会提醒全体党员注意,在目前第十次代表大会这个时期,由于许多情况加剧了国内小资产阶级居民的动摇,特别需要保持党的队伍的统一和团结,保持无产阶级先锋队的意志的统一。

\begin{quotation}
决议指出:“但是,还在全党争论工会问题以前,党内就已经显露出派别活动的某些征兆,即产生了几个具有特殊纲领、力求在一定程度上闹独立并建立其集团纪律的集团。必须使一切觉悟的工人都清楚地了解,任何派别活动都是有害的,都是不能容许的,因为派别活动事实上必然要削弱齐心协力的工作,使混进执政党内的敌人又能加紧活动来加深(党的)分裂,并利用这种分裂来达到反革命的目的。”
\end{quotation}

大会在这个决议中接着说:

\begin{quotation}
“无产阶级的敌人竭力利用一切离开共产主义的坚定路线的倾向,最明显的例子就是喀琅施塔得叛乱。当时,世界各国的资产阶级反革命势力和白卫分子都急忙表示,只要能推翻俄国的无产阶级专政,他们甚至情愿接受苏维埃制度的口号;当时,社会革命党人和所有资产所级反革命势力在喀琅施塔得事件中,运用了仿佛是为维护苏维埃政权而起义反对俄国苏维埃政府的口号。这些事实充分证明,只要能削弱和推翻俄国无产阶级革命的支柱,白卫分子都会竭力装扮而且善于装扮成共产主义者,甚至装扮得比共产主义者‘更左’。喀琅施塔得叛乱前夜在彼得格勒发现的孟什维克传单,也同样表明了孟什维克在利用俄国共产党内部的意见分歧时所采用的方式:口头上装作反对叛乱,拥护苏维埃政权,只是要给苏维埃政权加上一些不大的所谓的修正,实际上在鼓舞和支持在喀琅施塔得举行叛乱的社会革命党人和白卫分子。”
\end{quotation}

决议指出,党的宣传应当从保持党的统一和实现无产阶级先锋队的意志的统一是保证无产阶级专政胜利的基本条件这一观点,来详细说明派别活动的危害和危险。

大会决议说,另一方面党的宣传应当揭露苏维埃政权的敌人采用的新的策略手法的特点。

\begin{quotation}
决议指出:“现在,这些敌人已经知道在公开的白卫旗帜下进行反革命活动是没有希望的了,所以他们竭力利用俄国共产党内部的意见分歧,设法使政权转归表面上最象是承认苏维埃政权的那些政治集团,用这种办法来推进反革命事业。”(《联共(布)决议汇编》俄文版第1册第373—374页)\footnote{见《苏联共产党决议汇编》第2分册第63—64页。——译者注}
\end{quotation}

决议接着指出,党的宣传“还应当阐明过去革命的经验,这些经验证明反革命势力总是支持那些与极端革命政党最相似的小资产阶级集团,以便动摇并推翻革命专政,促使资本家地主反革命势力获得完全的胜利”。\footnote{见《苏联共产党决议汇编》第2分册第64—65页。——译者注}

同《关于党的统一》的决议密切相联的另一个决议是《关于我们党内的工团主义和无政府主义倾向》,这个决议也是由列宁提出、经代表大会通过的。第十次代表大会在这个决议中谴责了所谓“工人反对派”。大会确认宣传无政府工团主义倾向的思想和共产党员的称号不能相容,并号召全党同这种倾向作坚决的斗争。

第十次代表大会通过了关于从余粮收集制过渡到粮食税即关于过渡到新经济政策的极其重要的决议。

从战时共产主义到新经济政策这一转变,显示了列宁政策的无比的英明和远见。

大会决议说明了用粮食税代替余粮收集制的问题。实物粮食税征收额要比余粮收集制征收额低些。粮食税额必须在春播以前公布。明确规定了纳税期限。纳税后剩下的全部粮食完全归农民支配,即可以自由出卖。列宁在报告中指出,贸易自由在开始时,会使国内资本主义有某种活跃。必须容许私人贸易和准许私营工厂主开设小企业。但是用不着怕它。列宁认为:少许的商品流转自由能造成农民经营的兴趣,提高他们的劳动生产率,使农业迅速高涨;在这个基础上,国营工业将得到恢复,私人资本将被排挤,积蓄了人力物力以后,就可以建立强大的工业——社会主义的经济基础,然后转入坚决进攻,以消灭国内资本主义的残余。

战时共产主义是用冲击、用正面进攻来夺取城乡资本主义分子的堡垒的尝试。在这个进攻中,党向前跑得太远,有脱离自己根据地的危险。列宁这时主张稍许后退一点,暂时退到较接近于自己后方的地方去,由冲击堡垒转到较为长期地包围堡垒,待积蓄起力量后,重新开始进攻。

托洛茨基派和其他反对派认为新经济政策纯粹是退却。这样的解释是对他们有利的,因为他们的路线是要恢复资本主义。这是对新经济政策极其有害的反列宁主义的解释。实际上,在新经济政策实行一年以后,在党的第十一次代表大会上,列宁就宣布退却已经结束,并提出了一个口号:“准备向私人经营的资本进攻。”(《列宁全集》俄文第3版第27卷第218页)\footnote{见《列宁全集》第36卷第593页。——译者注}

反对派分子是些蹩脚的马克思主义者,对布尔什维克政策方面的问题一窍不通,他们既不了解新经济政策的实质.也不了解新经济政策开始时实行的退却的性质。关于新经济政策的实质,上面已经说过了。关于退却的性质,那么有各种各样的退却。有时候,党或军队因为遭到了失败,不得不实行退却。在这种情况下,党或军队实行退却,是为了保存自己、保存人力,以利再战。列宁在实行新经济政策时根本不是要实行这种退却,因为党在国内战争时期不仅没有遭到失败,没有被击败,反而击败了武装干涉者和白卫分子。但也有这样的时候,获得了胜利的党或军队在进攻中向前跑得太远,不能保证得到后方根据地的支持。这就造成严重的危险。在这种情况下,一个有经验的党或军队为了不脱离自己的根据地,通常都要稍许后退一点,跟自己的后方靠得近些,以便同自己的后方根据地更牢固地联结起来,保证自己要什么有什么,然后更有信心地、确有把握地重新实行进攻。列宁在实行新经济政策时所采取的,正是这种暂时的退却。列宁向共产国际第四次代表大会报告实行新经济政策的原因时,直截了当地说过,“我们在经济进攻中前进得太远了,我们没有给自己保证足够的根据地”\footnote{见《列宁全集》第2版第4卷第661页。——译者注},因此必须暂时向有保证的后方退却。

反对派可悲,就在于他们由于自己的愚昧无知而不了解而且至死也不了解在新经济政策下实行退却的这种特点。

第十次代表大会关于新经济政策的决定,保证了工人阶级和农民能结成巩固的经济联盟来建设社会主义。

大会的另一个决议,即关于民族问题的决议,也是服从于这个基本任务的。关于民族问题的报告是斯大林同志作的。斯大林同志说,我们已经消灭了民族压迫,但是这还不够。现在的任务是要消灭旧时代的沉重遗产,即过去的被压迫民族在经济,政治和文化上的落后状态。必须帮助他们在这方面赶上俄国中部。

其次,斯大林指出了民族问题上的两种反党倾向,即大国(大俄罗斯)沙文主义和地方民族主义。大会谴责了这两种倾向,认为这两种倾向对于共产主义和无产阶级国际主义都是有害的和危险的。同时,大会集中主要火力打击了当时的主要危险——大国主义,即打击了大俄罗斯沙文主义者在沙皇制度下对非俄罗斯民族所采取的那种态度的残余和遗毒。


\subsection[三\q 新经济政策的初步总结。党的第三次大会。苏维埃社会主义共和国联盟的成立。列宁患病。列宁的合作社计划。党的第十二次代表大会]{三\\新经济政策的初步总结。党的第三次大会。\\苏维埃社会主义共和国联盟的成立。\\列宁患病。列宁的合作社计划。\\党的第十二次代表大会}

新经济政策的实行遇到了党内不坚定分子的抵抗。这种抵抗来自两个方面。一方面,是“左的”空谈家,如洛明纳泽和沙茨金等这种类型的政治畸形儿,他们“证明”说,实行新经济政策就是放弃十月革命的成果,回到资本主义,毁灭苏维埃政权。这些人由于政治上无知和不了解经济发展规律,所以不懂得党的政策,惊惶失措,在自己周围散布灰心失望的情绪。另一方面,是公开的投降主义者,如托洛茨基、拉狄克、季诺维也夫、索柯里尼柯夫、加米涅夫、施略普尼柯夫、布哈林、李可夫之流,他们不相信我国社会主义发展的可能,拜倒在资本主义“威力”面前,并且力图巩固资本主义在苏维埃国家中的阵地——要求对国内外的私人资本作巨大让步,要求按租让原则或按吸收私人资本参加混合股份公司的原则把苏维埃政权在国民经济中的许多命脉交给私人资本。

这两种人都是敌视马克思主义、列宁主义的。

党揭露并孤立了这两种人。党给了惊惶失措者和投降主义者坚决的回击。

这种对党的政策的抵抗再次提醒我们,必须把不坚定分子清洗出党。为此中央大力进行了巩固党的工作,在1921年进行了清党。清党工作吸收了非党员参加,在公开的会议上进行。列宁建议彻底从党内清除“……欺骗分子、官僚化分子、不忠诚和不坚定的共产党员,以及虽然‘改头换面’但心里依然故我的孟什维克”(《列宁全集》俄文第3版第27卷第13页)\footnote{见《列宁选集》第2版第4卷第564页——译者注}。

清党结果共开除党员十七万人,占全体党员的百分之二十五左右。

清洗工作大大巩固了党,改善了党的社会成分,加强了群众对党的信任,提高了党的威信。党的团结和纪律性增强了。

新经济政策实行的第一年,就证明这个政策是正确的。过渡到新经济政策,大大加强了工农在新的基础上的联盟。无产阶级专政更加坚强有力了。富农匪患差不多已全部肃清。余粮收集制取消后,中农帮助了苏维埃政权同富农匪帮作斗争。苏维埃政权掌握着国民经济的全部命脉;大工业,运输业、银行、土地、国内商业和对外贸易。党使经济战线的情况有了转变。农业进展很快。工业和运输业取得了初步的成就。暂时还很缓慢,但是扎扎实实的经济高涨开始了。工人和农民已经感到和看到,党走在正确的道路上。

1922年3月,召开了党的第十一次代表大会。出席这次大会的有五百二十二名有表决权的代表,代表着五十三万二干名党员,即比上次代表大会时少了一些。有发言权的代表有一百六十五人。党员数量减少,是由于开始了清党。

党在代表大会上对新经济政策第一年的实行情况做了总结。根据这个总结,列宁在代表大会上宣布:

\begin{quotation}
“我们退却已经一年了。现在我们应当代表党来说:已经够了!退却所要达到的目的已经达到了。这个时期就要结束,或者说已经结束。现在提出的另一个目标,就是重新配置力量。”(《列宁全集》俄文第3版第27卷第238页)\footnote{见《列宁选集》第2版第4卷第629页——译者注}
\end{quotation}

列宁指出,新经济政策意味着资本主义和社会主义之间你死我活的斗争。“谁战胜谁”,这就是摆在我们面前的问题。为了胜利,必须保证工人阶级和农民、社会主义工业和农民经济的结合,办法是大力发展城乡商品流转。为此必须学会管理经济,必须学会文明经商。

这个时期,党的任务的链条中的主要一环是商业。不解决这个任务,就不能扩展城乡商品流转,不能巩固工农经济联盟,不能提高农业,不能使工业走出破坏状态。

当时苏维埃商业还很薄弱。商业机构很薄弱,共产党员还没有经商的技能,对敌人耐普曼\footnote{耐普曼是俄语“нэпман”一词的音译,指新经济政策时期的资本主义分子——译者注}还没有进行过研究,还没有学会同他们作斗争。私商耐普曼趁苏维埃商业薄弱,把布匹和其他畅销商品的贸易抓到手上。关于组织国营商业和合作社商业的问题有了重大的意义。

第十一次代表大会以后,经济工作热火朝天地开展起来了。国家遭到的歉收的后果,已被顺利地消除。农民经济恢复得很迅速。铁路运输已有所改善。重新开工的工厂日益增多。

1922年10月,苏维埃共和国庆祝了巨大的胜利:红军和远东游击队从日本武装干涉者手中解放了武装干涉者所占领的最后一块苏维埃国土海参崴。

这时,苏维埃国家的全部领土已经肃清了武装干涉者,而社会主义建设和国防的任务又要求进一步加强苏维埃国家各民族的联盟,于是各苏维埃共和国更紧密地联合起来组成一个统一的国家联盟的问题提上了日程。必须联和各族人民的力量来建设社会主义。必须建立巩固的国防。必须保证我们祖国各民族全面发展。为了这个目的,必须使苏维埃国家各族人民更加接近起来。

1922年12月,召开了全苏苏维埃第一次代表大会。在这次会上。根据列宁和斯大林的提议,建立了苏维埃各族人民自愿的国家联合。即苏维埃社会主义共和国联盟(苏联)。最初加入苏联的有俄罗斯苏维埃联邦社会主义共和国、南高加索苏维埃联邦社会主义共和国、乌克兰苏维埃社会主义共和国和白俄罗斯苏维埃社会主义共和国。不久,在中亚细亚成立了三个独立的加盟苏维埃共和国,即乌兹别克苏维埃共和国,土库曼苏维埃共和国和塔吉克苏维埃共和国。现在,所有这写共和国都按自愿和平等原则联合成一个统一的苏维埃国家联盟——苏联。同时每个共和国都保有自由退出苏联的权利。

苏维埃社会主义共和国联盟的成立,意味着苏维埃政权的巩固和布尔什维克党的列宁斯大林民族政策的伟大胜利。

1922年11月,列宁在莫斯科苏维埃全会上发表了演说。列宁在总结苏维埃政权成立五年来的情况时,表示坚信“新经济政策的俄国将变成社会主义的俄国”\footnote{见《列宁全集》第33卷第401页——译者注}。这是他向全国所作的最后次演说。1922年秋,党遭到了重大的不幸:列宁患重病了。全党和全体劳动者都感到,列宁患病如同他们自己遭到了巨大的痛苦。大家都为亲爱的列宁的生命担忧。但是,列宁即使在病中也没有停止自己的工作。列宁在病势已很沉重的时候,还写了好几篇很重要的文章。在这最后一批文章中,他总结了过去的工作,并拟定了在我国通过吸引农民参加社会主义建设来建成社会主义的计划。在这个计划中,列宁提出了一个吸引农民参加建成社会主义的事业的合作社计划。

列宁认为一般合作社,特别是农业合作社,是千百万农民易于接受和了解的由小的个体经济过渡到大的生产协作组织即集体农庄的遭路。列宁指出,我国农业发展的道路,应该是通过合作社吸收农民参加社会主义建设.逐渐把集体制原则应用于农业,起初是农产品的销售方面,然后是农产品的生产方面。列宁指出,在无产阶级专政和工农联盟的条件下,在保证无产阶级对农民实行领导的条件下,在社会主义工业存在的条件下,正确地组织起来的、拥有千百万农民的生产合作社,是能用来在我国建成完全的社会主义社会的手段。

1923年4月,召开了党的第十二次代表大会。这是布尔什维克夺取政权后列宁不能参加的第一次代表大会。出席这次大会的有四百零八名有表决权的代表,代表着三十八万六千名党员,即比上次党代表大会时少了一些。这是因为当时在继续清党,有相当一个百分数的党员被开除出党。有发言权的代表有四百一十七人。

党的第十二次代表大会在自己的决议中,考虑了列宁在最后一批文章和书信中所作的各项指示。

大会坚决回击了所有认为新经济政策是退出社会生义阵地,把自己的阵地交给资本主义的分子和提议接受资本主义盘剥的分子。在会上作这种提议的,是托洛茨基的拥护者拉狄克和克拉辛。他们提议向外国资本家投降,把苏维埃国家生命攸关的工业部门租让给外国资本家。他们据说偿还被十月革命废除了的沙皇政府债务。党把这些投降主义的提议痛斥为叛卖性的提议。党不是不利用租让政策,但这只能以有利于苏维埃国家的部门和规格为限。

布哈林和索柯里尼柯夫在代表大会以前就提议取消对外贸易垄断制。这个提议也是他们认为新经济政策是把自己的阵地交给资本主义的结果。列宁当时痛斥了布哈林,说他是投机的耐普曼和富农的保护人。第十二次代表大会坚决驳斥了对对外贸易垄断制的不可动摇性的侵犯。

大会还回击了托洛茨基强迫党对农民采取毁灭政策的企图。大会指出,不要忘记小农经济在国内占优势的事实。大会强调说,发展工业,包括发展重工业,不应同农民群众的利益相抵触,而应同他们的利益相结台,应有利于全体劳动居民。这些决定是反对托洛茨基的。因为他提议用剥削农民经济的办法来建设工业,因为他事实上不承认无产阶级和农民联盟的政策。

同时。托洛茨基还提议关闭普梯洛夫和布良斯克等等具有国防意义的大工厂,据他说,这些工厂不赢利。大会气愤地否决了托洛茨基的提议。

第十二次代表大会根据列宁向大会提出的书面建议,成立了中央监察委员套和工农检查院的联合机构。这个机关负有维护我们党的统一、巩固党和国家的纪律,全力改进苏维埃国家机关等项重要任务。

大会对民族问题十分注意。这个问题的报告人是斯大林同志。斯大林同志强调了我们在民族问题上的政策的国际意义。西方和东方的被压迫民族都把苏联看作是解决民族问题和消灭民族压迫的榜样。斯大林同志指出,必须大力消灭苏联各民族经济上和文化上的不平等。他号召全党坚决反对民族问题上的两种倾向——大俄罗斯沙文主义和地方资产阶级民族主义。

大会揭露了民族主义倾向分子以及他们对少教民族的大国主义政策。当时发言反对党的有格鲁吉亚的民族主义倾向分子穆吉万等人。这些民族主义倾向分子反对成立南高加索联邦,反对巩固南高加索各民族的友谊。他们对格鲁吉亚其他民族的态度是十足的大国沙文主义。他们把一切非格鲁吉亚人,特别是阿尔明尼亚人,都迁出梯弗里斯,用法律规定格鲁吉亚女子嫁培非格鲁吉亚人要丧失格鲁吉亚籍。托洛茨基、拉狄克、布哈林、斯克雷普尼克和拉柯夫斯基支持格鲁吉亚民族主义倾向分子。

大会以后不久,召开了各民族共和国民族问题工作者的专门会议。会上揭露了苏丹—加里也夫等人的鞑靼资产阶级民族主义者集团和斐祖拉·霍札也夫等人的乌兹别克民族主义倾向分子集团。

党的第十二次代表大会总结了新经济政策实行两年的结果。这个总结使人鼓舞和对最后胜利充满信心。

\begin{quotation}
斯大林同志在代表大会上说道:“我们党依然是团结一致的,它经受住了最伟大的转变,正举着展开的大旗前进。”\footnote{见《斯大林全集》第5卷第180页——译者注}
\end{quotation}


\subsection[四\q 克服恢复国民经济困难的斗争。托洛茨基派趁列宁患病加紧积极活动。党内的又一次争论。托洛茨基的失败。列宁的逝世。为纪念列宁而吸收党员。党的第十三次代表大会]{四\\克服恢复国民经济困难的斗争。\\托洛茨基派趁列宁患病加紧积极活动。\\党内的又一次争论。托洛茨基的失败。\\列宁的逝世。为纪念列宁而吸收党员。\\党的第十三次代表大会}

恢复国民经济的头几年,就取得了显著的成就。到1924年,各方面都有了提高。播种面积从1921年起就已大大增加,说明农民经济日趋巩固。社会主义工业增长和发展了。工人阶级的人数大大增加。工资提高了。工农生活比1920—1921年好过了,改善了。

但是,还没有消除的经济破坏的后果,仍然令人感觉得到。工业还落后于战前水平,工业的增长大大落后于国家需要的增长。到1923年底,还有一百万左右失业者,因为国民经济增长缓慢,不可能消灭失业现象。商业的发展时好时环,因为城市产品价格过高,这种过高的价格是耐昔曼和我们商业组织中的耐普曼分子强加给国家的。因此,苏维埃卢布极不稳定,币值降低。这一切都阻碍了工农生活状况的改善。

到1923年秋,由于我们的工业和商业机关违反了苏维埃的价格政镣,经济困难有些加重。工业品和农产品的价格相差太大。粮食价格很低,而工业品价格过高。工业中杂费开支很大,这就把商品价格提高了。农民出卖粮食所得的货币迅速贬值。加上当时盘踞在最高国民经济委员会里的托洛茨基分子皮达可夫又向经济工作人员发出一项罪恶的指令,从出卖工业品方面多赚利润,放手提高价格,美其名是为了发展工业。实际上,这种耐普曼的口号只能缩小工业生产的基础和破坏工业。在这种条件下,农民买城市商品不合算,就不买了。销售出现了危机,影响了工业。工资发不出,引起工人不满。在某些工厂中,最落后的工人停工不干了。

党中央委员会定出了克服这一切困难和缺点的办法。采取了消除销售危机的种种措施。降低了日用品的价格。决定实行币制改革——采用稳定的货币切尔克。整顿了给工人发放工资的工作。采取了通过苏维埃机关和合作社机关发展商业、把各种私商和投机商从商业中排挤出去的措施。

当时本应鼓起劲来齐心协力地进行工作。忠实于党的人是这样想和这样做的。但是托洛茨基派却不是这样。他们趁列宁病重不能视事,向党和党的领导发动了新的进攻。他们以为击败党和推翻党的领导的有利时机已经到来。他们在反党的斗争中利用了一切;1923年秋德国和保加利亚革命的失败,国内的经济困难,列宁的患病。正是在党的领袖卧病不起这个苏维埃国家的困难时刻,托洛茨基开始对布尔什维克党进行攻击。他把党内一切反列宁主义分子纠集在自己周围,炮制了一个旨在反对党,党的领导和党的政策的反对派纲领。这个纲领叫做四十六个反对派分子的声明。在反对列宁党的斗争中,所有的反对派集团——托洛茨基派、民粹派,以及“左派共产主义者”和“工人反对派”的残余,都联台起采了。他们在自己的声明中预言苏维埃政权必遭严重的经济危机、必遭灭亡,要求各派别和集团能自由活动,说这是摆脱现状的唯一出路。

这就是要竭力恢复被党的第十次代表大会根据列宁的提议禁止了的派别。

托洛茨基派没有提出过任何关于改进工业或农业,关于改进国内商品流转、改善劳动者的生活状况的具体问题。因为他们对这些问题根本不感兴趣。他们感兴趣的只有一点:趁列宁不能视事,恢复党内的派别,动摇党的基础,动摇党的中央。继四十六人纲领之后,托洛茨基又发表了一封信,在这封信中他污蔑党的干部,对党进行了一系列新的诽谤性的责准。托洛茨基在这封信中搬弄着党已听他讲过不止一次的孟什维主义陈词滥调。

托洛茨基派首先攻击党的机关。他们知道,没有巩固的党机关,党就不能生存和活动。反对派企图动摇、破坏这个机关,企图把党员同党的机关对立起来,把党内的青年同党的老干部对立起来。托洛茨基在自己的信中想在青年学生身上打主意,想在不知道党同托洛茨基主义斗争的历史的青年党员身上打主意。托格茨基为了争夺青年学生而对他们阿谀奉承,称他们为“党的最可靠的晴雨表”,同时又说列宁主义老近卫军在蜕化。他指桑骂槐,用第二国际首领的蜕化来卑鄙地影射布尔什维克老近卫军也在走着这条路。托洛茨基企图通过叫喊党在蜕化来掩盖他自己的蜕化和他自己的反党阴谋。

反对派分子的两个文件,即四十六人纲领和托洛茨基的信,由托洛茨基派散发给各区和各支部,并交付党员讨论。

党接到了要党争论的挑战。

这样,现在也如党的第十次代表大会前夜进行工会问题争论时一样,托洛茨基派又强迫党来进行全党争论了。

党虽然忙于更为重要的经济问题,但仍接受了挑战而宣布进行争论。

全党都参加了争论。斗争很激烈,莫斯科的斗争特别尖锐。托洛茨基派力图首先夺得首都组织。但是争论并没有帮托洛茨基派的忙。争论只是使他们丢了脸。托洛茨基派无论在莫斯科或在全苏联各地,都遭到了惨败。只有少数大学支部和机关支部拥护托洛茨基派。

1924年1月,召开了党的第十三次代表会议。会议听取了斯大林同志对争论所作的总结报告。会议谴责,托洛茨基反对派,指出它足党内一种离开马克思主义的小资产阶级倾向。会议的决议后来由党的第十三次代表大会和共产国际第五次代表大会批准了。国际共产主义无产阶级支持了布尔什维克党反对托洛茨基主义的斗争。

但是,托洛茨基派并投有停止其破坏话动。1924年秋,托洛茨基发表了《十月的教训》一文,他在文章中企图用托洛茨基主义来偷换列宁主义。这篇文章完全是诬蔑我们党和党的领袖列宁的。共产主义和苏维埃政权的一切敌人,都把这本诽谤性小册子奉为至宝。党愤怒地回击了托洛茨基对布尔什维主义的英勇历史的诬蔑。斯大林同志揭穿了托洛茨基用托洛茨基主义偷换列宁主义的企图,斯大林同志在自己的发言中指出:“党的任务就是要埋葬托洛茨基主义这一思潮\footnote{见《斯大林全集》第6卷第309页——译者注}。”

斯大林同志的1924年出版的理论著作《论列宁主义基础》,对于从思想上粉碎托洛茨基主义和捍卫列宁主义具有重大的意义。这本小册子是对列宁主义的精辟阐述和理论上的深刻论证。它在当时和现在都起了用马克思列宁主义理论这个锐利的武器武装世界各国布尔什维克的作用。

在反对托洛茨基主义的战斗中,斯大林同志把党团结在它的中央周围,并动员起全党为社会主义在我国胜利而继续斗争。斯大林同志证明了,从思想上粉碎托洛茨基主义,是保证继续向社会主义胜利前进的必要条件。

斯大林同志在总结同托洛茨基主义斗争的这一时期时说:

\begin{quotation}
“不粉碎托洛茨基主义。就不能在新经济政策条件下取得胜利,就不能把目前的俄国变成社会主义的俄国。”\footnote{见《斯大林全集》第7卷第31页——译者注}
\end{quotation}

但是,党的列宁政策所取得的成就,由于党和工人阶级遭到的最大的不幸而显得黯然无光。1924年1月21日,我们的领袖和导师,布尔什维克党的创始人列宁,在莫斯科附近哥尔克村逝世了。全世界工人阶级把列宁逝世看作最沉痛的损失。在列宁安葬的那天,国际无产阶级宣布一切业都停止工作五分钟。铁路停运了,工厂停工了。全世界劳动者怀着极大的悲痛送别自己的父亲和导师,最好的朋友和保护者列宁。

苏联工人阶级以更加紧密地团结在列宁党的周围来纪念列宁的逝世。在这些哀悼的日子里,每个觉悟工人都仔细考虑了自己应如何对待执行着列宁遗嘱的共产党。党中央委员会收到了成千上万非党工人请求接收他们入党的申请书。中央委员会欢迎先进工人发起的这一运动,宣布大批接收先进工人入党,宣布为纪念列宁而吸收党员。成千上万的工人在这次加入了党。加入党的都是决心为党的事业、为列宁的事业而献身的人。当时在一个短时期内就有二十四万多工人加入了布尔什维克党的队伍。工人阶级的先进分子,即最觉悟最革命最勇敢最守纪律的分子加入到党里来了。这就是为纪念列宁而吸收党员的运动。

列宁的逝世表明,我们党同工人群众多么亲密,工人们对列宁的党多么珍爱。

在哀悼列宁的日子里,斯大林同志代表全党在苏联苏维埃第二次代表大会上作了伟大的宣誓。他说:

\begin{quotation}
“我们共产党人、是具有特种性格的人,我们是由特殊材料制成的。伟大的无产阶级战略家的军队,列宁同志的军队,就是由我们这些人组成的。在这个军队里做一个战士,是再光荣不过的了。以列宁同志为创始人和领导者的这个党的党员称号,是再高尚不过的了。……

列宁同志和我们永别时嘱咐我们要珍重党员这个伟大称号,并保持这个伟大称号的纯洁性。列宁同志,我们谨向你宣誓:我们一定要光荣地执行你的这个遗嘱!……

列宁同志和我们永别时嘱咐我们要保护我们党的统一,如同保护眼珠一样。列宁同志,我们谨向你宣誓:我们也一定要光荣地执行你的这个遗嘱!……

列宁同志和我们永别时嘱咐我们要保护并巩固无产阶级专政。列宁同志,我们谨向你宣誓:我们也一定不遗余力来光荣地执行你的这个遗嘱!……

列宁同志和我们永别时嘱咐我们要竭力巩固工农联盟。列宁同志,我们谨向你宣誓:我们也一定要光荣地执行你的这个遗嘱!……

列宁同志始终不倦地对我们说明我国各族人民自愿联盟的必要性,说明我国各族人民在共和同联盟内实行兄弟合作的必要性。列宁同志和我们永别时嘱咐我们要巩固并扩大共和国联盟。列宁同志,我们谨向你宣誓:我们也一定要光荣地执行你的这个遗嘱!……

列宁曾屡次向我们指出,巩固红军和改善红军状况是我们党的最重要的任务之一。……同志们,我们来宣誓:我们一定不遗余力地来巩固我们的红军,巩固我们的红海军……

列宁同志和我们永别时嘱咐我们要忠实于共产国际的原则。列宁同志,我们谨向你宣誓:我们一定奋不顾身地来巩固并扩大全世界劳动者的联盟——共产国际!”\footnote{见《斯大林全集》第6卷第42—46页——译者注}
\end{quotation}

这就是布尔什维克党对自己的永垂不朽的领袖列宁的誓言。

1924年5月,举行了党的第十三次代表大会。出席这次大会的有七百四十八名有表决权的代表,代表着七十三万五千八百八十一名党员。党员数量比上次代表大会时大大增加,是因为为纪念列宁而吸收党员时党的队伍增加了约一十五万新党员。有发言权的代表有四百一十六人。

大会一致谴责了托洛茨基反对派的纲领,肯定它是一种脱离马克思主义的小资产阶级倾向,是对列宁主义的修正,同时批准了党的第十二次代表会议《关于党的建设》和《关于争论的总结》这两个决议。

从巩固城乡结合的任务出发,大会指示进一步扩大工业,首先是轻工业,同时强调必须迅速发展冶金业。

大会批准建立国内商业人民委员部,并向一切商业机关提出了控制市场、把私人资本从商业领域排挤出击的任务。

大会提出了扩大国家对农民的低利贷款而把高利贷者从农村中排挤出去的任务。

大会提出了用各种方法使农民群众合作化的口号作为农村工作的主要任务。

最后,大会指出了为纪念列宁而吸收党员的巨大意义,并号召全党注意对青年党员——首先是为纪念列宁而吸收的青年党员加强列宁主义基础的教育。


\subsection[五\q 苏联在恢复时期结束时的情形。我国社会主义建设和社会主义胜利问题。季诺维也夫—加米涅夫的“新反对派”。党的第十四次代表大会。国家社会主义工业化方针]{五\\苏联在恢复时期结束时的情形。\\我国社会主义建设和社会主义胜利问题。\\季诺维也夫—加米涅夫的“新反对派”。\\党的第十四次代表大会。国家社会主义工业化方针}

布尔什维克党和工人阶级在新经济政策道路上已经奋战了四年多了。恢复国民经济的英勇工作行将结束。苏联的经济实力和政治实力日益增长。

这时国际形势已有变化。资本主义抵挡住了群众在帝国主义战争后的第一次革命进攻。德国、意大利、保加利亚、波兰和其他许多国家的革命运动都被镇压下去了。各妥协主义的社会民主党的领袖在这方面帮了资产阶级的忙。革命的暂时退潮到来了。西欧资本主义的暂时局部稳定,即资本主义阵地的局部巩固到来了。但是资本主义的稳定并没有消除使资本主义社会分裂的各种基本矛盾。恰恰相反,资本主义的局部稳定使工人同资本家的矛盾、帝国主义同殖民地民族的矛盾、各个国家的帝国主义集团的矛盾尖锐化了。资本主义的稳定酝酿着各资本主义国家矛盾的新爆发和新危机。

除资本主义的稳定外,还有苏联的稳定。但是这两种稳定是根本不同的。资本主义的稳定预示着资本主义的新危机。苏联的稳定则标志着社会主义国家经济实力和政治实力的进一步增长。

虽然西方的革命失败了,但是苏联的国际地位仍在继续巩固,虽然速度较为缓慢。

1922年,苏联被邀请参加在意大利的热那亚举行的国际经济会议。在热那亚会议上,各帝国主义政府因资本主义各国革命遭到失败而气焰嚣张,企图对苏维埃共和国施加新的压力,不过这次是采取外交形式。帝国主义者向苏维埃共和国提出了蛮横无理的要求。他们要求把十月革命宣布国有的工厂归还给外国资本家,要求偿还沙皇政府的一切债务。只有这样,帝国主义国家才答应给苏维埃国家少量的贷款。

苏联拒绝了这些要求。

热那亚会议没有什么结果。

英国外交大臣寇松1923年通过最后通牒再一次进行干涉的尝试,也遭到了应有的回击。

资本主义国家在试探了苏维埃政权的稳固性、确信苏维埃政权已不可动摇之后,就相继来同我国恢复外交关系。1924年,同英法日意四国恢复了外交关系。

很清楚,苏维埃国家已经争得整整一个和平喘息的时期了。

国内的形势也发生了变化。布尔什维克党领导下的工农的忘我工作,已经开花结果。国民经济迅速地增长了。1924—1925经济年度,农业已接近战前规模,达到战前水平的百分之八十七。苏联大工业的产值在1925年已约占战前工业产值的四分之三。1924—1925年度,苏维埃国家已经能够向基本建设投资三亿八千五百万卢布。国家电气化计划在顺利执行中。国民经济中的社会主义命脉巩固了。反对私人工商业资本的斗争取得了重大胜利。

由于经济的高涨,工农的物质生活状况得到了进一步改善。工人阶级人数迅速增加。工资增长了。劳动生产率提高了。农民的物质生活状况大大改善。1924—1925年度,工农国家已能拿出近二亿九千万卢布来帮助力量单薄的农民。在工农生活状况改善的基础上,群众的政治积极性大大提高。无产阶级专政巩固了。布尔什维克党的威信和影响增长了。

国民经济的恢复接近结束。但是,对苏维埃国家来说,对一个建设社会主义的国家来说,单单恢复经济,单单达到战前水平是不够的。战前水平是个落后国家的水平。必须继续前进。苏维埃国家争得的长时间的喘息,保证了继续建设的可能性。

但是这里尖锐地提出了关于前途、关于我国发展即我国建设的性质的问题,关于社会主义在苏联的命运的问题。苏联的经济建设应该循着什么方向进行,循着社会主义方向,还是循着其他什么方向?我们是应当建成并且能够建成社会主义经济,还是我们注定要为另一种经济即资本主义经济去准备肥沃土壤呢?一般说来,苏联有没有建成社会主义经济的可能呢,如果有,那么能不能在资本主义国家革命推迟和资本主义处于稳定状态的条件下做到呢?能不能在一方面竭力巩固和扩大我国社会主义力量,同时又暂时让资本主义得到某种发展的新经济政策道路上,建成社会主义经济呢?要怎样来建设社会主义的国民经济呢,从哪里开始这种建设呢?

所有这些问题都在恢复时期快结束时提到了党的面前,已经不是作为理论问题,而是作为实践问题,作为日常的经济建设问题提到了党的面前。

对所有这些问题,都必须给以直接而明确的回答,好让我们从事工农业建设的党的经济工作者和全体人民,知道要朝什么方向走——朝社会主义走,还是朝资本主义走?

如果对这些问题不给以明确的回答,那我们在建设方面的全部实际工作就会成为没有前途的工作,盲目的工作,徒劳无益的工作。

党对所有这些问题都给了明确而肯定的回答。

党回答说,是的,在我国能够而且必须建成社会主义经济,因为我国有建成社会主义经济、建成完全的社会主义社会所必需的一切。1917年10月,工人阶级建立了自己的政治专政,在政治上战胜了资本主义。从那时起,苏维埃政权采取了一切措施,来粉碎资本主义的经济实力和创造建成社会主义国民经济所必需的条件。这些措施就是:剥夺资本家和地主;变土地、工厂、铁路和银行为全民财产;实行新经济政策;建设社会主义国营工业;实行列宁的合作社计划。现在,主要任务是要在全国展开社会主义新经济的建设,从而在经济上也彻底击败资本主义。我们的全部实际工作,我们的一切行动,都应服从于实现这个主要任务的要求。工人阶级能做到这点,并且一定会做到这点。实现这个宏伟的任务,应当从国家工业化开始。国家的社会主义工业化,是展开社会主义国民经济的建设必须由以开始的基本环节。无论西方革命的推迟,无论非苏维埃国家资本主义的局部稳定,都不可能阻止我们向社会主义前进。新经济政策只会促进这一事业,因为党实行新经济政策正是为了促进我国国民经济的社会主义基础的建设。

这就是党对社会主义建设能否在我国胜利这个问题的回答。

但是党知道,这还不是社会主义在一国胜利的问题的全部。在苏联建成社会主义是人类历史上最伟大的转折。是苏联工人阶级和农民的具有全世界历史意义的胜利。但这终究是苏联内部的事情,仅仅是社会主义胜利问题的一部分。这个问题的另一部分就是它的国际方面。斯大林同志在论证社会主义在一国胜利的原理时不止一次地指出,应当把这个问题的两个方面,即国内方面和国际方面分开。至于这个问题的国内方面,即国内各阶级的相互关系方面,苏联的工人阶级和农民完全能够在经济上战胜本国的资产阶级,建成完全的社会主义社会。但是这个问题还有一个国际方面,即外部关系方面,苏维埃国家同资本主义国家、苏联人民同国际资产阶级的关系方面。国际资产阶级仇恨苏维埃制度,总在寻找机会对苏维埃国家进行新的武装干涉,在苏联作恢复资本主义的新尝试。由于苏联暂时还是唯一的社会主义国家,而其他国家仍然是资本主义国家,所以存在着资本主义对苏联的包围,从而产生资本主义武装干涉的危险。很清楚,只要存在资本主义包围,也就存在资本主义武装干涉的危险。苏联人民单靠本身力量,能不能消灭这种外来危险即资本主义武装干涉苏联的危险呢?不,不可能。其所以不可能,是因为要消灭资本主义武装干涉的危险,必须消灭资本主义的包围,而要消灭资本主义的包围,就至少要有几个国家的无产阶级革命取得胜利才能做到。但由此就应得出结论:社会主义在苏联的胜利,即资本主义经济制度的消灭和社会主义经济制度的建成,还不能算是最后胜利,因为外国进行武装干涉和试图复辟资本主义的危脸仍然没有消灭,因为社会主义国家仍然缺少免除这种危险的保障。要消灭外国资本主义武装干涉的危险,就必须消灭资本主义的包围。

当然,苏联人民及其红军在苏维埃政权实行正确政策的条件下,能给外国资本主义的新的武装干涉以应有的回击,正如他们在1918—1920年给了资本主义的第一次武装干涉以回击一样。但这还不是说,资本主义的新的武装干涉的危险就此消灭了。第一次武装干涉的失败,并没有消灭新的武装干涉的危险,因为武装干涉的危险的根源,即资本主义的包围,还继续存在。只要资本主义的包围还存在,即使新的武装干涉失败,也不会消灭武装干涉的危险。

由此得出结论:资本主义国家里无产阶级革命的胜利,是同苏联劳动者休戚相关的事情。

这就是党关于社会主义在我国胜利问题的方针。

中央要求把这个方针提交即将召开的党的第十四次代表会议去讨论,以便得到代表会议的批准和通过,成为全体党员必须遵守的党规党法。

党的这个方针使反对派分子大为震惊。其所以使他们大为震惊,首先是因为党使这个方针带有具体实践的性质,把它和国家社会主义工业化的实际计划联系起来,并要求把这个方针变成党法,变成党的第十四次代表会议的决议,而为全体党员所必须遵守。

托洛茨基派反对党的方针,提出了一个对立的孟什维主义的“不断革命论”,一个只有作为对马克思主义的嘲弄才可以称作马克思主义理论的“理论”,一个否认社会主义建设在苏联有胜利可能的“理论”。

布哈林派不敢公开反对党的方针,但他们还是偷偷地用自己的资产阶级和平长入社会主义的“理论”来同党的方针对抗,并用“发财吧”的“新”口号来补充自己的“理论”。照布哈林派的说法,社会主义的胜利不是消灭资产阶级,而是培植资产阶级并使之发财致富。

季诺维也夫和加米涅夫也曾一度跳出来,说社会主义在苏联不可能取得胜利,因为苏联在技术上经济上落后。但是后来他们被迫缩回去了。

党的第十四次代表会议(1925年4月)谴责了公开的和暗藏的反对派分子的所有这些投降主义“理论”,批准了党争取社会主义在苏联胜利的方针,并通过了相应的决议。

季诺维也夫和加米涅夫无可奈何,只好赞成这个决议。但是党知道,他们只是暂缓同党作斗争,因为他们拿定主意到党的第十四次代表大会上再来“向党开火”。他们在列宁格勒纠集了自己的同伙,组织了所谓的“新反对派”。

1925年12月,召开了党的第十四次代表大会。

这次大会是在党内气氛紧张的情况下进行的。整整一个列宁格勒代表团,这个党员最集中地区的代表团,居然准备出来反对自己的中央,这种情况有党以来还没有过。

出席大会的有六百六十五名有表决权的代表和六百四十一名有发言权的代表,代表着六十四万三千名党员和四十四万五千名预备党员,即比上一次代表大会时略微少一些。这是对混杂有反党分子的大学支部和机关支部进行了局部清洗的结果。

斯大林同志作了中央委员会的政治报告。他对苏联政治和经济实力增长的情景作了清晰的描绘。由于苏维埃经济制度的优越,无论工业或农业都在比较短的时期内得到了恢复,并接近了战前水平。虽然有这些成就,斯大林同志还是提议不要以此为满足,因为这些成就并不能消除我国仍然是个落后的农业国这一事实。当时农业生产占全部产值的三分之二。而工业仅仅占三分之一。斯大林同志说,党面临着一个迫切的问题,就是要把我国变为经济上不依赖资本主义国家的工业国。这一点可以做到,而且必须做到。为国家的社会主义工业化而斗争,为社会主义的胜利而斗争,现在成了党的中心任务。

\begin{quotation}
斯大林同志指出:“把我国从农业国变成能自力生产必需的装备的工业国,——这就是我们总路线的实质和基础。”\footnote{见《斯大林全集》第7卷第294页。——译者注}
\end{quotation}

国家的工业化能保证我国的经济独立,加强我国的国防力量,创造社会主义在苏联胜利所必需的条件。

季诺维也夫派反对党的总路线。同斯大林的社会主义工业化计划相对抗,季诺维也夫分子索柯里尼柯夫提出了一个流行于帝国主义豺狼中的资产阶级计划。按照这个计划,苏联应当仍然是个农业国,主要生产原料和粮食用以向国外出口,而从国外进口自己所不生产并且也不应生产的机器。在1925年的条件下,这个计划分明是让工业发达的外国在经济上奴役苏联,是为了满足资本主义国家的帝国主义豺狼的贪欲而使苏联工业永远处于落后状态。

采纳这个计划,就等于把我国变成资本主义世界的软弱无力的农业附庸,使我国在资本主义包围面前成为无以自卫的弱国,说到底就是葬送苏联的社会主义事业。

大会痛斥了季诺维也夫派的经济“计划”,指出它是奴役苏联的计划。

“新反对派”又使出另外一招,他们硬说什么(无视列宁的意见!)我们的国营工业不是社会主义工业,又说什么(也是无视列宁的意见!)中农不能成为工人阶级在社会主义建设中的同盟者。但这也救不了他们的命。

大会痛斥了“新反对派”的这些胡说,指出它们是反列宁主义的。

斯大林同志揭露了“新反对派”的托洛茨基主义—孟什维主义实质。他指出,季诺维也夫和加米涅夫不过是重弹列宁当初无情批驳过的、党的敌人的滥调。

很清楚,季诺维也夫派就是伪装得很不高明的托洛茨基派。

斯大林同志着重指出,建立工人阶级同中农在社会主义建设中的坚固联盟,是党的最重要的任务。他指出当时党内在农民问题上存在两种危害这个联盟的倾向。第一种倾向是低估和小看富农的危险,第二种倾向是在富农面前张皇失措而低估中农的作用。对于哪一种倾向更坏这个问题,斯大林同志回答说:“这两种倾向,无论第一种倾向或第二种倾向都坏。如果这两种倾向发展下去,它们就会瓦解和断送党。幸而我们党内有能够消灭第一种倾向和第二种倾向的力量。”\footnote{见《斯大林全集》第7卷第278页。——译者注}

党确实把“左”右两种倾向都粉碎和消灭了。

党的第十四次代表大会在总结经济建设问题的讨论时,一致否决了反对派分子的投降主义计划,并在自己的著名决议中写道:

\begin{quotation}
“在经济建设方面,代表大会认为我国——无产阶级专政的国家拥有‘建成完全的社会主义社会所必需的一切’(列宁)\footnote{见《列宁选集》第2版第4卷第682页。——译者注}。代表大会认为,为社会主义建设在苏联的胜利而斗争是我们党的基本任务。”\footnote{见《苏联共产党决议汇编》第3分册第77页。——译者注}
\end{quotation}

第十四次代表大会批准了新的党章。

从第十四次代表大会起,我们党开始称为苏联共产党(布尔什维克),简称联共(布)。

季诺维也夫派在代表大会上被击败后,并没有向党屈服。他们开始了反对第十四次代表大会决议的斗争。第十四次代表大会一结束,季诺维也夫立刻召集了共青团列宁格勒省委会议(团省委的领导人是由季诺维也夫、查鲁茨基、巴卡也夫、叶甫多基莫夫、库克林,萨发罗夫等两面派分子用仇恨我党列宁中央的精神培养出来的)。在这次会议上,共青团列宁格勒省委通过了苏联列宁共产主义青年团历史上从来没有过的决议,即拒绝服从党的第十四次代表大会决议的决议。

但是列宁格勒共青团的季诺维也夫派领导人根本没有反映列宁格勒广大共青团员的情绪。因此,这些领导人很容易地被打垮了,列宁格勒共青团组织很快又站上了它在共青团中应有的位置。

在第十四次代表大会结束时,一批大会代表,即莫洛托夫、基洛夫,伏罗希洛夫、加里宁、安得列也夫等同志,被派到列宁格勒去。必须向列宁格勒组织的党员揭露,用欺骗手段取得代表资格的列宁格勒代表团在代表大会上的立场,是犯罪的、反布尔什维主义性质的。各单位的党员大会听了关于代表大会情况的介绍,会开得很激烈。重新开了一次列宁格勒党

组织紧急代表会议。列宁格勒党组织的绝大多数党员(百分之九十七以上)完全赞同党的第十四次代表大会的决议,并谴责了反党的季诺维也夫“新反对派”。这个“新反对派”当时已经成了光杆司令了。

列宁格勒的布尔什维克,仍然站在列宁—斯大林党的前列。

斯大林同志在总结党的第十四次代表大会的工作时写道:

\begin{quotation}
“联共(布)第十四次代表大会的历史意义就在于它彻底揭露了新反对派的错误,斥责了新反对派的不相信的态度和叫苦的行为,明确地指出了进一步为社会主义而斗争的道路,给党指出了胜利的前途,因而用对社会主义建设必获胜利的坚强信念武装了无产阶级。”(斯大林《列宁主义问题》俄文第10版第150页)\footnote{见斯大林《列宁主义问题》第178—179页。——译者注}
\end{quotation}


\subsection{简短的结论}

过渡到恢复国民经济的和平工作的年代,是布尔什维克党历史上最重要的时期之一。党在紧张的形势下实现了从战时共产主义政策到新经济政策的困难转变。党加强了工人和农民在新的基础即经济的基础上的联盟。成立了苏维埃社会主义共和国联盟。

在新经济政策道路上,恢复国民经济的工作取得了具有决定意义的成就。苏维埃国家卓有成效地渡过了国民经济发展中的恢复时期,开始过渡到另一个时期,国家工业化的时期。

从国内战争向和平的社会主义建设过渡——特别是在最初——伴随着巨大的困难。布尔什维主义的敌人,联共(布)队伍中的反党分子,在这整个时期内一直都在拼命反对列宁的党。领导这些反党分子的是托洛茨基。他在这个斗争中的帮手是加米涅夫、季诺维也夫和布哈林。反对派分子指望在列宁逝世后瓦解布尔什维克党的队伍,分裂党,使党产生不相信社会主义能在苏联胜利的心理。实际上,托洛茨基派是企图在苏联建立一个新资产阶级的政治组织,建立另外一个党,即搞资本主义复辟的党。

党在列宁旗帜下团结在自己的列宁中央周围,团结在斯大林同志周围,挫败了托洛茨基派以及他们在列宁格勒的新朋友们,即季诺维也夫—加米涅夫的新反对派。

布尔什维克党积蓄了人力物力,把国家引上了新的历史阶段——社会主义工业化的阶段。

