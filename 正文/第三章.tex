\section[第三章\q 孟什维克和布尔什维克在日俄战争和俄国第一次革命时期(1904—1907年)]{第三章\\ 孟什维克和布尔什维克在日俄战争和\\俄国第一次革命时期\\{\zihao{3}(1904—1907年)}}

\subsection[一\q 日俄战争。俄国革命运动的继续高涨。彼得堡的罢工。1905年1月9日工人在冬宫前举行的游行示威。示威群众遭到枪杀。革命的开始]{一\\ 日俄战争。俄国革命运动的继续高涨。\\ 彼得堡的罢工。\\ 1905年1月9日工人在冬宫前举行的游行示威。\\ 示威群众遭到枪杀。革命的开始}

从十九世纪末起,各帝国主义国家为称霸太平洋和瓜分中国开始加紧进行争夺。沙俄也参加了争夺。1900年,沙皇军队伙同日德英法等国军队用空前残暴的手段镇压了中国人民反对外国帝国主义者的起义。在此以前,沙皇政府已强迫中国把辽东半岛连同旅顺口要塞割让给俄国。俄国取得了在中国领土上修筑铁路的权利。俄国在北满修筑了中东铁路,并调兵守卫该路。北满被沙俄用武力占领了。沙皇政府的势力已伸展到朝鲜。俄国资产阶级拟定了在满洲成立“黄俄罗斯”的计划。

沙皇政府在远东方面进行侵略的时候,碰到了另一个强盗日本;当时日本已迅速地变成一个帝国主义国家,也企图侵略亚洲大陆,首先是从中国下手。日本也如沙俄一样力图把朝鲜和满洲据为已有。日本当时已梦想占领库叶岛和远东地区。英国害怕沙俄势力在远东加强,所以暗中支持日本。日俄战争逼近了。寻找新市场的大资产阶级和最反动的地主阶层,推动沙俄政府去进行这场战争。

日本不待沙皇政府宣战先开始了战争。日本在俄国境内设置了周密的间谍网,它知道对方在这场斗争中没有准备。1904年1月,日本不宣而战,向俄军要塞旅顺口发动突然袭击,重创了旅顺口的俄国舰队。

日俄战争就这样开始了。

沙皇政府本来指望这次战争能帮助它稳定政局、阻止革命。但是它失算了。战争更加动摇了沙皇制度。

装备恶劣、训练不良、由一些庸碌无能和贪污腐败的将军所指挥的俄国军队屡战屡败。

资本家、官吏和将军们在战争中大发横财。盗窃之风盛行一时。军队的供给很坏。正当缺乏炮弹的时候,军队却收到一车厢一车厢的神像,好象是在嘲笑他们一样。士兵们痛心地说:“日本人用炮弹打我们,我们却用神像打他们。”专车不去运输伤员,却去运输沙皇将军抢来的财物。

日军包围了旅顺口要塞,接着就把它占领了。沙皇军队遭到多次失败之后,在沈阳城下被击溃。沙皇三十万大军在这次战役中死伤和被俘的人数达十二万。接着,沙皇从波罗的海派往旅顺口解围的舰队,也在对马海峡被彻底击溃而覆没。对马之败是一次灭顶之灾:沙皇派去的二十艘军舰中,十三艘被击沉击毁,四艘被俘。战争结果是沙俄遭到了完全的失败。

沙皇政府不得不与日本缔结可耻的和约。日本占领了朝鲜,从俄国手中夺得了旅顺口和半个库页岛。

人民群众不要这场战争,并认识到这场战争对俄国的害处。由于沙俄落后,人民付出了高昂的代价。

布尔什维克和孟什维克对这场战争采取了不同的态度。

盂什维克包括托洛茨基在内堕落到护国主义的立场,即主张保卫沙皇、地主和资本家的“祖国”。

列宁和布尔什维克与此相反,认为沙皇政府在这场掠夺战争中失败有好处,因为它会削弱沙皇制度,加强革命力量。

沙皇军队的失败向广大人民群众揭示了沙皇制度的腐朽。人民群众对沙皇制度的憎恨与日俱增了。列宁写道“旅顺口的陷落是专制制度陷落的开始”。

沙皇想用战争扼杀革命,但得到了相反的结果。日俄战争加速了革命。

沙俄的资本主义压迫,因有沙皇制度的压迫而强化。工人不仅感到资本主义剥削和苦役劳动的痛苦,而且感到全体人民没有权利的痛苦。因此,觉悟的工人力求领导城乡一切民主分子反沙皇制度的革命运动。农民因没有土地,因受许多农奴制残余的束缚而喘不过气来,他们遭受着地主和富农的盘剥。沙俄境内各族人民受到本民族地主资本家和俄罗斯地主资本家的双重压迫。1900—1903年的经济危机已经加重了劳动群众的苦难,而战争又使苦难更加深重。战争的失败加深了群众对沙皇制度的憎恨。人民已到了忍无可忍的地步。

由此可见,引起革命的原因实在是太多了。

1904年12月,在巴库布尔什维克委员会领导下举行了组织得很好的巴库工人大罢工,结果是工人获得胜利,工人和石油业主订立了俄国工人运动史上第一个集体合同。

巴库罢工成了南高加索和俄国许多地区革命高涨的开端。

\begin{quotation}
“巴库罢工是全俄一二两月光荣发动的信号。”(斯大林)
\end{quotation}

这次罢工好象是预示大革命风暴即将来临的雷前闪电。

1905年1且9日(22日)的彼得堡事件,就是革命风暴的开始。

1905年1月3日,彼得堡最大的普梯洛夫工厂(现为基洛夫工厂)开始举行罢工。罢工的起因是厂里解雇了四名工人。普梯洛夫工厂的罢工迅速扩大,彼得堡其他工厂也相继加入。这次罢工变成总罢工了。运动迅猛地发展起来。沙皇政府决定一开始就把运动镇压下去。

早在1904年,即普梯洛夫工厂罢工以前,警察局以通过奸细加邦神父在工人中间建立了自己的组织“俄罗新工厂工人大会”。这个组织在彼得堡各区都设有自己的分会。当罢工开始时,加邦神父在他这个组织的会上提出了一个挑衅的计划:让全体工人在1月9日集合起来,举着教堂旗幡和沙皇画像,和平列队前往冬宫向沙皇呈递陈述本身疾苦的请愿书。他说,沙皇一定会出来接见人民,倾听和满足人民的要求。加邦是为沙皇的保安局效劳:挑起枪杀工人的惨剧,把工人运动淹没于血泊中。但这个警察计划却反过来对着沙皇政府了。

请愿书在工人的集会上讨论过,并作了一些修改。布尔什维克也在这些集会上讲了话,不过他们没有明说自己是布尔什维克。由于他们的影响,在请愿书上加进了关于言论出版自由、工人结社自由、召集立宪会议来改变俄国国家制度、在法律上人人平等、政教分离、停止战争、实行八小时工作制、土地归农民等要求。

布尔什维克在这些会上发言时向工人指明,自由不是用向沙皇请愿的方法获得的,而是靠拿起武器去争取。布尔什维克警告工人会遭到枪击。但他们阻止不住往冬宫请愿的游行。很大一部分工人还相信沙皇会帮助他们。强有力的运动席卷了群众。

彼得堡工人在请愿书上写道:

\begin{quotation}
“我们,彼得堡市的工人,偕同我们的妻室儿女和老弱父母,特来向皇上请求公道和保护。我们生活困苦,备受压迫,当牛做马,遭受着欺凌侮辱和非人的待遇……我们已再三忍耐,但是我们日甚一日地被推入困苦、无权和愚昧的深渊,暴政专横压制着我们……忍耐已经到了极限。我们已经到了与其让这种难以忍受的痛苦继续下去还不如死去为好的可怕时刻……”
\end{quotation}

1905年1月9日清晨,工人们前往当时沙皇所在的冬宫。工人们带着全家——妻子、孩子和老人——去见沙皇,他们手无寸铁,只是抬着沙皇的画像,举着教堂的旗帜,唱着祷告歌。上街的队伍总共有十四万多人。

尼古拉二世并没有和他们讲友爱。他下令枪杀手无寸铁的工人。这一天有一千多工人被沙皇军队打死,有两千多工人被打伤。彼得堡的街头染遍了工人的鲜血。

布尔什维克是和工人们同去了的。他们中有许多人被打死或被逮捕。布尔什维克当时就在染遍工人鲜血的街头向工人解释,谁是这一残酷暴行的祸首,应该怎样同他作斗争。

1月9日从此称为“流血星期日”。工人在1月9日得到了血的教训。工人对沙皇的信念在这天被枪毙了。他们懂得了,只有用斗争才能争得自己的权利。1月9日傍晚,各工人区开始构筑街垒。工人们说:“沙皇揍了我们,那我们也要揍他!”

沙皇制造血腥暴行的可怕消息传遍了全国。全体工人阶级、全国人民义愤填膺。每一个城市里的工人都用罢工来抗议沙皇的暴行,并提出了政治要求。工人现在已是喊着“打倒专制制度”的口号上街了。在1月间,罢工人数达到很大的数字……四十四万。一个月内参加罢工的工人人数超过了过去整整十年的罢工人数。工人运动上升到了极大的高度。

革命在俄国开始了。


\subsection[二\q 工人的政治罢工和游行示威。农民革命运动的增长。“波将金”号装甲舰上的起义]{二\\ 工人的政治罢工和游行示威。\\ 农民革命运动的增长。\\ “波将金”号装甲舰上的起义}

1月9日以后,工人的革命斗争具有更加尖锐的政治性质。工人群众开始由经济罢工和支持性罢工转到政治罢工,转到游行示威,在某些地方甚至开始转到武装抵抗沙皇军队。在彼得堡、莫斯科、华沙、里加和巴库这些集中了大量工人的大城市里,罢工进行得特别顽强和有组织。五金工人走在斗争着的无产阶级的前列。先进工人队伍用自己的罢工振奋了觉悟较低的阶层,发动了整个工人阶级去作斗争。社会民主党的影响迅速增长了。

五一游行示威在许多地方引起了群众与军警的冲突。在华沙,示威群众受到枪击而死伤者达数百人。华沙工人响应波兰社会民主党的号召,举行了总罢工以示抗议。罢工和游行示威在5月间一天也没有停止过。全俄各地参加五月罢工的工人在二十万以上。巴库、洛兹、伊万诺沃-沃兹涅先斯克的工人,都卷入了总罢工。罢工工人和示威群众同沙皇军队冲突的事件日益增多。敖德萨、华沙、里加、洛兹和其他许多城市都曾发生过这样的冲突。

在波兰的大工业中心洛兹市,斗争进行得特别激烈。洛兹工人在市区的街道上构筑了几十座街垒。同沙皇军队进行了三天巷战(1905年6月22—24日)。在这里,武装发动与总罢工汇合起来了。列宁认为这些战斗是俄国工人的第一次武装发动。

伊万诺沃—沃兹涅先斯克工人的罢工是夏季罢工中特别出色的一次。这次罢工从1905年5月底开始一直坚持到8月初,几乎持续了两个半月。参加罢工的工人约有七万,其中许多是妇女。

这次罢工是布尔什维克北方委员会领导的。在城外的塔尔卡河畔,差不多每天都有几千工人举行集会。工人在这些大会上讨论了自己的需求。在工人大会上常有布尔什维克出来发言。为了镇压罢工,沙皇当局命令军队驱散工人。向工人开枪射击。几十个工人被打死,数百个工人受伤。城内宣布了戒严。但工人还是继续坚持,拒绝复工。工人和他们的家属忍饥挨饿,但不屈服。只是到了极端疲惫的时候,工人才不得已去上工。罢工锻炼了工人。工人阶级在这次罢工中作出了勇敢、坚定、沉着和团结的榜样。在这次罢工中,伊万诺沃—沃兹涅先斯克的工人受到了真正的政治教育。

伊万诺沃—沃兹涅先斯克的工人在这次罢工期间建立了工人代表苏维埃,它实际上是俄国最初的工人代表苏维埃之一。

工人的政治罢工震撼了全国。农村也跟着城市发动起来了。从春天起就开始了农民的骚动。一群一群的农民起来造地主的反,捣毁地主的田庄、糖厂和酒厂,焚烧地主的楼房和庄院。许多地方的农民夺取地主的土地,大批砍伐地主的林木,要求把地主土地转交给人民。农民把地主的粮食和其他食品夺来分给饥民。地主们惊惶万分,不得不逃往城市。沙皇政府调遣士兵和哥萨克去镇压农民起义。军队开枪射击农民,逮捕“祸首”,拷打和折磨他们。但农民并不停止斗争。

在俄国中部,伏尔加河流域和南高加索(特别是在格鲁吉亚),农民运动不断扩大。

社会民主党人深入农村。觉中央发出了告农民书:《农民们,请听我们说》。特维尔、萨拉托夫,波尔塔瓦、切尔尼果夫、叶加特林诺斯拉夫、梯弗里斯和其他许多省份的社会民主党委员会都发表了告农民书。社会民主党人在农村中开大会,成立农民小组,建立农民委员会。1905年夏天,许多地方发生了社会民主党人组织的农业工人罢工。

但这还只是农民斗争的开始。农民运动只扩展到八十五个县,即约近沙俄欧洲部分总县数的七分之一。

工人和农民的运动以及俄国军队在日俄战争中的多次失败,对军队发生了影响。沙皇制度的这个支柱动摇了。

1905年6月,黑海舰队的“波将金”号装甲舰上爆发了起义。该舰当时停泊在离正在举行工人总罢工的敖德萨不远的地方。起义的水兵惩治了他们切齿痛恨的军官,把装甲舰开到了敖德萨。“波将金”号装甲舰转到革命方面来了。

列宁对这次起义极为重视。他认为布尔什维克必须领导这一运动,使其能与工农群众和地方驻军的运动汇合起来。

沙皇派了一批军舰来镇压“波将金”号,但这些军舰上的水兵拒绝对自己的起义同伴射击。革命的红旗在“波将金”号装甲舰上飘扬了好几天。但布尔什维克党在1905年还不像后来在1917年那样是领导运动的唯一政党。当时在“波将金”号上有许多孟什维克、社会革命党人和无政府主义者。因此,虽然有一批社会民主党人参加了起义,但起义还是没有一个正确的和有充分经验的领导。一部分水兵在决定关头动摇了。黑海舰队中的其他军舰没有来响应这艘起义的装甲舰。革命的装甲舰因为缺乏煤炭和粮食,不得不开到罗马尼亚岸边,向罗马尼亚当局投降。

“波将金”号装甲舰的水兵起义最后失败了。后来落到沙皇政府手里的水兵被交付法庭审判,一部分被处死,一部分被流放服苦役。但起义这一事实本身却有特别重大的意义。“波将金”号装甲舰的起义是陆海军中第一个群众性的革命发动,是沙皇军队的很大一支部队第一次转到革命方面来。这次起义使工人和农民,特别是士兵群众和水兵群众自己更加认识、更加了解了陆海军必须与工人阶级联合、与人民联合的思想。

工人向群众性政治罢工和游行示威的转变,农民运动的加强,人民与军警的武装冲突,以及黑海舰队中的起义,——这一切说明人民武装起义的条件正在成熟。这种情形使自由资产阶级不得不认真行动起来。它害怕革命,同时又用革命恐吓沙皇。它是想勾结沙皇反对革命,要沙皇“为人民”实行小小的改良以便“稳定”人心,分裂革命力量,借以防止“革命惨象”。自由派地主们说:“必须割点土地给农民,不然他们就会割死我们的。”自由资产阶级准备同沙皇分掌政权。列宁在当时谈到工人阶级的策略和自由资产阶级的策略时写道:“无产阶级在进行斗争,资产阶级在窃取政权。”

沙皇政府继续用残暴手段镇压工农,但它不能不明白,单用高压手段是对付不了革命的。因此,除高压手段外,它还采取了随机应变的政策。一方面,它通过自己的奸细唆使俄国各族人民互相摧残,制造蹂躏犹太人的暴行,挑拨阿尔明尼亚人和鞑靼人互相残杀。另方面,它又答应召集缙绅会议\footnote{缙绅会议是十六世纪至十七世纪俄国的中央等级代表机关。——译者注}或国家杜马\footnote{国家杜马是沙皇俄国于1906—1917年按反民主选举法产生的代议机关。——译者注}之类的“代议机构”,并委托大臣布里根拟定一种不让杜马拥有立法权的杜马法案。所有这些办法的采用,都是为了分裂革命力量,使人民中间的温和阶层脱离革命。

布尔什维克宣布抵制布里根杜马,决心拆穿这套嘲弄人民代表机关的把戏。

反之,孟什维克决定不拆杜马的台,而且认为必须去参加。


\subsection[三\q 布尔什维克和孟什维克在策略问题上的分歧。党的第三次代表大会。列宁的《社会民主党在民主革命中的两种策略》一书。马克思主义政党的策略基础]{三\\ 布尔什维克和孟什维克在策略问题上的分歧。\\ 党的第三次代表大会。\\ 列宁的《社会民主党在民主革命中的两种策略》一书。\\ 马克思主义政党的策略基础}

革命把社会各个阶层都发动起来了。革命引起的国内政治生活中的转变推功它们离开了旧日的习惯了的地位,迫使。它们变更自己的部署来适应新的环境。每个阶级、每个政党,都在努力制定自己的策略,自己的行动路线,自己对其他阶级的态度,自己对政府的态度。甚至沙皇政府也不得不定出一种在它看来很不寻常的新策略,即答应召集布里根杜马这种所谓的“代议机构”。

社会民主党也必须定出自己的策略。其所以必须这样作,是因为革命在不断高涨,是因为无产阶级面前摆着急待解决的实际问题:组织武装起义问题,推翻沙皇政府问题,建立临时革命政府问题,社会民主党参加这个政府的问题,对农民的态度问题,对自由资产阶级的态度问题,等等。社会民主党必须定出一个统一而周密的马克思主义策略。

但是由于孟什维克的机会主义和分裂行动,俄国社会民主党当时已经分裂成为两个派别。虽然当时的分裂还不能认为是完全的分裂,虽然这两个派别形式上还不是两个不同的党,但事实上它们却很像两个不同的党,各有自己的中央,各有自己的报纸。

孟什维克除了他们和党内的多数在组织上的旧分歧之外,还加上了策略问题上的新分歧,这就使分裂更加深了。

由于没有统一的党,也就没有统一的党的策略。

如果立刻召开党的第三次例行的代表大会,由它来制定统一的策略,并责成少数忠实地执行大会决议,服从大会多数的决议,那也许是摆脱当时状况的出路。布尔什维克当时向孟什维克建议的正是这样的出路。但孟什维克根本就不愿听人提到第三次代表大会。布尔什维克认为使党继续缺乏党所批准而为全体党员所必须执行的策略是一种罪恶,决定自己担负起发起召开第三次代表大会的责任。

所有的党组织,不论是布尔什维克的或孟什维克的,都被邀请参加代表大会。但孟什维克拒绝参加第三次代表大会,而决定召开自己的代表大会。他们把自己的代表大会叫做代表会议,因为他们的代表人数很少;但实际上这是个代表大会,是孟什维克的党代表大会,因为它的决议是全体孟什维克必须执行的。

1905年4月,在伦敦召开了俄国社会民主党第三次代表大会。出席大会的有二十四名代表,代表着二十个布尔什维克委员会。所有大的党组织,都派有代表参加。

大会谴责了孟什维克是“党内分裂出去的部分”,然后就转到下一个议题,即制定党的策略的问题。

与代表大会同时,在日内瓦召开了孟什维克的代表会议。

“两个代表大会——两个党”\footnote{参看《列宁全集》第36卷第580页。——译者注}——这就是列宁对当时情况的评论。

代表大会和代表会议所讨论的实际上都同样是策略问题,但双方就这些问题却通过了完全相反的决议。代表大会和代表会议各自通过的两种不同性质的许多决议表明,党的第三次代表大会和孟什维克代表会议之间、布尔什维克和孟什维克之间在策略问题上存在着极深的分歧。

以下就是这些分歧的要点。

党的第三次代表大会的策略路线。大会认为,虽然目前发生的革命是资产阶级民主性的革命,虽然它在目前不能越出资本主义所容许的范围,但愿意这个革命完全胜利的首先是无产阶级,因为这个革命的胜利将使无产阶级有可能组织起来,在政治上得到提高,获得政治上领导劳动群众的经验和本领,并从资产阶级革命过渡到社会主义革命。

只有农民才会支持无产阶级这种争取资产阶级民主革命完全胜利的策略,因为没有革命的完全胜利,农民就不能推翻地主而获得地主的土地。因此,农民是无产阶级的天然同盟者。

自由资产阶级不愿意这个革命完全胜利,因为它需要沙皇政权这个皮鞭来对付它最害怕的工人和农民,所以它会努力保存沙皇政权,只是把沙皇政权的权力稍微限制一下。因此,自由资产阶级将力图在君主立宪制度基础上用同沙皇妥协的办法来结束革命。

只有由无产阶级来领导革命,只有身为革命领袖的无产阶级保证同农民的联盟,只有使自由资产阶级陷于孤立,只有由社会民主党来积极参加组织反沙皇制度的人民起义,只有因起义胜利而成立能够根除反革命势力并召开全民立宪会议的临时革命政府,只有社会民主党不拒绝在顺利条件下参加这个临时革命政府以便把革命进行到底,——只有在这一切条件下,革命才能获得胜利。

孟什维克代表会议的策略路线。因为革命是资产阶级性质的革命,所以只有自由资产阶级才能做革命的领袖。无产阶级不应与农民接近,而应与自由资产阶级接近。这里主要的是不要用自己的革命性吓跑自由资产阶级,不要给由资产阶级以退出革命的借口,因为自由资产阶级退出革命,革命就会削弱下去。

也许起义会获得胜利,但社会民主党在起义胜利后却应当靠边,以免吓跑自由资产阶级。也许起义的结果会成立临时革命政府,但社会民主党在任何条件下都不应去参加,因为这个政府不会是社会主义性质的政府,而主要的是社会民主党参加这个政府并坚持自己的革命立场会吓跑自由资产阶级,从而破坏革命。

从革命的前途着想,最好是召集缙绅会议或国家杜马之类的代议机构,工人阶级可以从外面对它施加压力,以便把它变成立宪会议,或推动它去召开立宪会议。

无产阶级有它特殊的、纯粹工人的利益,它应该管的正是这种利益,而不应妄想充当资产阶级革命的领袖,因为这个革命是一般政治的革命,所以它关系到一切阶级而不仅关系到无产阶级。

简单说来,俄国社会民主工党内两派的两种策略,就是如此。

列宁在《社会民主党在民主革命中的两种策略》这本具有历史意义的著作中,对孟什维克的策略提出了经典性的批评,对布尔什维克的策略作了英明的论证。

这本书出版于1905年7月,即党的第三次代表大会闭会两个月之后。照书名来看,也许会觉得列宁在这本书中只讲到资产阶级民主革命时期的策略问题,并且只讲到俄国孟什维克。实际上,他批评孟什维克的策略,也就是揭露国际机会主义的策略;他在论证马克思主义者在资产阶级革命时期的策略并把资产阶级革命和社会主义革命区分开时,同时也就规定了从资产阶级革命向社会主义革命过渡的时期马克思主义策略的基础。

以下就是列宁在《社会民主党在民主革命中的两种策略》这本小册子中所发挥的基本策略原理。

(一)始终贯彻于列宁这本书中的基本策略原理,就是认为无产阶级能够并且应当做俄国资产阶级民主革命的领袖,做俄国资产阶级民主革命的领导者。

列宁承认这个革命的资产阶级性质,因为正如他说的那样,这个革命“决不能直接越出民主革命的范围”\footnote{见《列宁选集》第2版第1卷第578页。——译者注}。但他认为这个革命不是上层的革命,而是能把全体人民、全体工人阶级、全体农民发动起来的人民革命。因此列宁认为,孟什维克企图缩小资产阶级革命对于无产阶级的意义,降低无产阶级在这个革命中的作用,使无产阶级避开这个革命,就是背叛无产阶级的利益。

\begin{quotation}
列宁写道:“马克思主义教导无产者不要避开资产阶级革命,不要不关心资产阶级革命,不要把革命中的领导权让给资产阶级,相反地,要尽最大的努力参加革命,最坚决地为把革命进行到底而奋斗。”(《列宁全集》俄文第3版第8卷第58页)\footnote{同上,第543页。——译者注}
\end{quotation}

\begin{quotation}
列宁又说:“我们不应当忘记,现在除了充分的政治自由,除了民主共和制……便没有而且也不会有其他可以加速社会主义到来的手段。”(同上,第104页)\footnote{同上,第601页。——译者注}
\end{quotation}

列宁预料到革命可能有两种结局:

(1)或者结局是彻底战胜沙皇制度,是推翻沙皇制度并建立民主共和国;

(2)或者是力量不够,结局就会是沙皇同资产阶级靠牺牲人民利益做成交易,就会是一纸残缺不全的宪法,甚至多半是一种嘲弄宪法的把戏。

无产阶级愿意达到最好的结局,即彻底战胜沙皇制度。但这种结局只有在无产阶级成为革命的领袖,领导者时才能实现。

\begin{quotation}
列宁写道;“革命的结局将取决于工人阶级是成为在攻击专制制度方面强大有力、但在政治上软弱无力的资产阶级助手,抑或是成为人民革命的领导者。”(《列宁全集》俄文第3版第8卷第32页)\footnote{见《列宁选集》第2版第1卷第513页。——译者注}
\end{quotation}

列宁认为无产阶级完全可能避免替资产阶级当助手的命运,而成为资产阶级民主革命的领导者。根据列宁的看法,过种可能有如下述:

第一,“无产阶级按其地位来说是最先进的和唯一彻底革命的阶级,所以它负有在俄国的一般民主革命运动中起领导作用的使命”(同上,第75页)\footnote{同上,第563页。——译者注}。

第二,无产阶级有其不依赖资产阶级而独立的政党,这个政党使它能够团结成为“统一的和独立的政治力量”(同上)\footnote{同上。——译者注}。

第三,无产阶级比资产阶级更愿意革命彻底胜利,因此,“在某种意义上说来,资产阶级革命对无产阶级要比对资产阶级更加有利”(同上,57页)\footnote{同上,第541页。——译者注}。

\begin{quotation}
列宁写道;“对资产阶级有利的是依靠旧制度的某些残余,例如君主制度、常备军等等来反对无产阶级。对资产阶级有利的是资产阶级革命不过分坚决地扫除旧制度的一切残余,而留下其中的某一些,就是说,要这个革命不十分彻底,不进行到底,不坚决无情。……对资产阶级更有利的是要资产阶级民主性的种种必要的改革比较缓慢地、渐进地、谨慎地和不坚决地进行,即用改良的办法而不用革命的办法进行……要这些改革尽可能少地去发扬小百姓即农民特别是工人的革命的自动性、主动性和毅力,因为不这样的话,工人就会更容易如法国人所说的‘把枪枝从右肩移到左肩’,就是说,更容易用资产阶级革命供给他们的武器,用这个革命给予他们的自由,用清除了农奴制的基地上所产生的民主设施,来反对资产阶级本身。反之,对工人阶级更有利的是要资产阶级民主性的种种必要的改革恰恰不是经过改良的道路,而是经过革命的道路来实现,因为改良的道路是一种迁延时日的、迟迟不前的、使人民机体中腐烂部分的消亡过程缓慢得引起万般痛苦的道路。由于这一部分的腐烂而首先感到痛苦和最感到痛苦的是无产阶级和农民。革命的道路是迅速开刀、使无产阶级受到的痛苦最少的道路,是直接割去腐烂部分的道路,是对君主制度以及和君主制度相适应的令人作呕的、卑鄙龌龊的、腐败不堪的、臭气熏天的种种设施让步最少和顾忌最少的道路。”(《列宁全集》俄文第3版第8卷第57—58页)\footnote{见《列宁选集》第2版第1卷第541—542页。——译者注}
\end{quotation}

\begin{quotation}
列宁接着说:“因此,无产阶级也站在为共和制而斗争的最前列,它轻蔑地拒绝它所鄙视的那些劝它注意资产阶级会退出的愚蠢意见。”(同上,第94页)\footnote{同上,第389页。——译者注}
\end{quotation}

为要把无产阶级领导革命的可能变为现实,为要使无产阶级在事实上成为资产阶级革命的领袖、领导者,列宁认为至少要有两个条件。

第一,就是要无产阶级有一个愿意彻底战胜沙皇制度而且自愿接受无产阶级领导的同盟者。这是进行领导这个思想本身所要求的,因为领导者没有被领导者,就不成其为领导者;领袖没有被率领者,就不成其为领袖。列宁认为农民就是这样的同盟者。

第二,就是要把同无产所级争夺革命领导权、竭力想由自己充当革命的唯一领导者的那个阶级逐出领导舞台,并使其陷于孤立。这也是进行领导的这个思想本身所要求的,因为这种思想根本不容有两个革命领导者存在。列宁认为自由资产阶级就是这样的阶级。

\begin{quotation}
列宁写道:“只有无产阶级才能成为彻底的民主战士。只有农民群众加入无产阶级的革命斗争,无产阶级才能成为战无不胜的民主战士。”(《列宁全集》俄文第3版第8卷第65页)\footnote{见《列宁选集》第2版第1卷第551—552页。——译者注}
\end{quotation}

又说:

\begin{quotation}
“农民中有大批的半无产者,同时还有小资产阶级分子。这使得它也不稳定,因而迫使无产阶级团结成为一个阶级性十分严格的党。但是农民的不稳定和资产阶级的不稳定根本不同,因为农民现在所关心的与其说是无条件地保存私有制,不如说是夺取私有制主要形式之一的地主土地。农民虽然不会因此而成为社会主义者,不会因此而终止其为小资产阶级,但是它能够成为全心全意地和最彻底地拥护民主革命的力量。只要给农民以教育的革命事变进程不会因资产阶级叛变和无产阶级失败而过早地中断,农民就必然会成为这样的力量。在上述的条件下,农民必然会成为革命和共和制的支柱,因为只有获得了完全胜利的革命才能使农民获得土地改革方面的一切,才能使农民获得他们所希望,所幻想而且是他们真正必需的一切。”(同上,第94页)\footnote{同上,第589页。——译者注}
\end{quotation}

列宁分析了孟什维克的反对意见,即认为布尔什维克这样一种策略“会迫使资产阶级退出革命,从而缩小革命的规模”这种意见,并给这种意见下了一个评语,说它是“叛卖革命的策略”,是“变无产阶级为资产阶级可怜走卒的策略”。\footnote{同上,第589页。——译者注}列宁当时写道:

\begin{quotation}
“谁真正了解农民在胜利的俄国革命中的作用,他就不能够说革命的规模会因资产阶级退出而缩小。因为事实上只有当资产阶级退出,而农民群众以积极革命者的资格同无产阶级一起奋斗的时候,俄国革命的规模才会真正开始发展起来;只有那时,才会有资产阶级民主革命时代可能有的那种真正最广大的革命规模。我们的民主革命要坚决进行到底,就应当依靠那些能把资产阶级的必不可免的不彻底性麻痹起来的力最,即恰巧能做到‘迫使它退出’……的力量。”(《列宁全集》俄文第3版第8卷第95—96页)\footnote{见《列宁选集》第3版第1卷第590—591页。——译者注}
\end{quotation}

列宁在《社会民生党在民主革命中的两种策略》一书中所发挥的关于无产阶级是资产阶级革命的领袖这一基本策略原理,关于无产阶级在资产阶级革命中的领导权(领导作用)的基本策略原理,就是如此。

这是马克思主义政党关于资产阶级民主革命中策略问题的新方针,它与马克思主义武库中过去存在过的策略方针根本不同。从前,例如西方各国资产阶级革命中的领导作用始终是落在资产阶级手中,无产阶级不管有意无意,总是充当资产阶级的助手,而农民始终是资产阶级的后备力量。马克思主义者当时认为这样的情形多少是不可避免的,但他们同时声明说,无产阶级在这种情形下应当尽可能坚持自己最近的阶级要求,并应当有它自己的政党。现在,在新的历史环境中,按照列宁的方针,情形已经改变成选样,就是无产阶级成了资产阶级革命的领导力量,资产阶级被排除于革命领导之外,而农民变成了无产阶级的后备力量。

有人说普列汉诺夫“也曾主张”无产阶级领导权,这是一种误会。普列汉诺夫向无产阶级领导权思想献过媚眼,而且不嫌在口头上承认这个思想,这是事实,但在实际上他是反对这个思想的实质的。无产阶级的领导权意味着在无产阶级和农民联盟这一政策的条件下,在孤立自由资产阶级这一政策的条件,实现无产阶级在资产阶级革命中的领导作用,但大家知道,普列汉诺夫反对孤立自由资产阶级的政策,主张同自由资产阶级妥协的政策,反对无产阶级和农民联盟的政策。实际上,普列汉诺夫的策略方针是孟什维克否认无产阶级领导权的方针。

(二)列宁认为胜利的人民武装起义是推翻沙皇制度和争得民主共和国的最重要的手段。列宁同孟什维克相反,他认为“一般民主革命运动已使武装起义成为必要”,认为“组织无产阶级举行起义一事”已经“提到日程上来,成为党的极重要的、主要的和必要的任务之一”,并认为必须“采取最有效的措施来武装无产阶级和保证有可能直接领导起义”(《列宁全集》俄文第3版第8卷第75页)\footnote{见《列宁选集》第2版第1卷第564页。——译者注}。

为要引导群众去实行起义,并使起义本身成为全民的起义,列宁认为必须提出这样一种口号,必须向群众发出这样一种号召,实现这种口号和号召能充分调动群众的革命主动性,组织他们去举行起义而瓦解沙皇制度的政权机构。列宁认为党的第三次代表大会关于策略问题的决议就是这样的口号,而他的《社会民主党在民主革命中的两种策略》一书就是为维护这些决议而写的。

他认为这样的口号就是:

(1)采用“群众政治罢工……这种罢工在起义开始时和起义进程中都能有重要的意义”(同上,第75页)\footnote{同上,第563—564页。——译者注};

(2)组织“立刻用革命的方法实现八小时工作制以及工人阶级的其他迫切要求”(同上,第47页)\footnote{同上,第529页。——译者注};

(3)“立刻组织革命农民委员会”,以便用革命的方法“实行一切民主改革”,直到没收地主土地(《列宁全集》俄文第3版第8卷第88页)\footnote{见《列宁选集》第2版第1卷第581页。——译者注};

(4)武装工人。

这里特别值得注意的有两点:

第一,就是用革命的方法在城市中实现八小时工作制和在农村中实现民主改革的策略,就是说做到既不顾政府,也不顾法律,而是藐视政府和法制,打破现行法律,用无所顾忌的手段自动建立新秩序。这是一种新的策略手段,采用这种手段就能使沙皇制度的政权机构瘫痪,并调动群众的积极性和创造主动性。在这一策略的基础上也就成长起来了城市中的革命罢工委员会和农村中的革命农民委员会。后来前者发展成了工人代表苏维埃,后者发展成了农民代表苏维埃。

第二,就是采用群众性政治罢工,采用政治总罢工,这种罢工后来在革命进程中对群众起了极大的革命动员作用。这是无产阶级手中一种新的很重要的武器,是从前在马克思主义政党实践中未曾有过而后来得到了大家公认的武器。

列宁认为人民起义胜利的结果,应是沙皇政府被临时革命政府所取代。临时革命政府的任务是巩固革命的成果,镇压反革命势力的反抗,实现俄国社会民主工党的最低纲领。列宁认为不实现这些任务就无法彻底战胜沙皇制度。但要实现这些任务和彻底战胜沙皇制度,临时革命政府就不应该是一个普通的政府,而应该是获得了胜利的两个阶级即工人和农民专政的政府,应该是无产阶级和农民的革命专政。列宁根据马克思提出的“在革命之后,任何临时性的国家机构都需要专政,并且需要强有力的专政”\footnote{见《马克思恩格斯选集》第1卷第308页。——译者注}这一著名原理得出结论说,临时革命政府要想保证彻底战胜沙皇制度,就不能不是无产阶级和农民的专政。

\begin{quotation}
列宁写道“革命对沙皇制度的彻底胜利,就是实现无产阶级和农民的革命民主专政。……这样的胜利正好就是专政,就是说,它必不可免地要依靠军事力量,依靠群众武装,依靠起义,而不是依靠某种用‘合法的’、‘和平的方法’建立起来的机关。这只能是专政,因为实现无产阶级和农民所迫切需要而且绝对需要的改革,一定会引起地主、大资产者和沙皇制度方面的拼命反抗。没有专政,就不可能摧毁这种反抗,就不可能打破反革命的企图。但是,这当然不是社会主义的专政,而是民主主义的专政。它不能触动(如果不经过革命发展中的一系列中间阶段的话)资本主义的基础。它至多只能实行有利于农民的彻底重分土地的办法,实行彻底的和完全的民主主义,直到共和制为止,把一切亚洲式的、奴役性的特征不仅从农村生活中而且从工厂生活中连根铲掉,奠定认真改善工人生活状况并提高其生活水平的基础,最后——就先后次序而言的最后,不是就重要性而占的最后——把革命火焰延烧到欧洲去。这样的胜利还丝毫不会把我国的资产阶级革命变为社会主义革命;民主革命不会直接越出资产阶级社会经济关系的范围;但是这样一种胜利,对俄国和全世界的将来的发展,都有极其重大的意义。除了已经在俄国开始的革命的这种彻底胜利以外,再没有什么东西能把全世界无产阶级的革命毅力提高到这种程度,再没有什么东西能把达到全世界无产阶级完全胜利的道路缩得这样短。”(《列宁全集》俄文第3版第8卷第62—63页)\footnote{见《列宁选集》第2版第1卷第547—548页。——译者注}
\end{quotation}

至于社会民主党应怎样对待临时革命政府以及社会民主党可否参加这个政府,那末列宁完全坚持党的第三次代表大会关于这个问题的决议,该决议说:

\begin{quotation}
“如果力量对比及其他不能预先准确判定的因素对我们有利,我们党可以派全权代表参加临时革命政府,以便同一切反革命企图作无情的斗争,捍卫工人阶级的独立利益;这样参加临时革命政府的必要条件是:党对自己的全权代表进行严格的监督,并坚定不移地保持社会民主党的独立性,因为社会民主党力求实现完全的社会主义革命,就这一点说,它同一切资产阶级政党是不可调和地敌对的;无论社会民主党是否有可能参加临时革命政府,都必须在最广泛的无产阶级群众中进行宣传,使他们懂得,社会民主党领导下的武装起来的无产阶级为了保卫、巩固和扩大革命的成果,必须经常对临时政府施加压力。”(《列宁全集》俄文第3版第8卷第37页)\footnote{见《列宁选集》第2版第1卷第517—518页。——译者注}
\end{quotation}

孟什维克反驳说,临时政府毕竟是资产阶级政府,社会民主党人如果不愿意重犯法国社会党人米勒兰参加法国资产阶级政府的错误,就不能去参加这样的政府。列宁批驳这种意见时指出,孟什维克在这里混淆了两件不同的事情,暴露出他们没有能力用马克思主义观点看待问题;当时在法国,说的是社会党人在国内缺乏革命形势的时候参加反动的资产阶级政府,所以社会党人不应该参加这样的政府;而现在在俄国,说的是社会党人在革命达到高潮的时候参加争取革命胜利的革命资产阶级政府,所以社会民主党人可以去参加并且在顺利条件下应该去参加这样的政府,以便不仅“从下面”、从外面,而且“从上面”、从政府内部去打击反革命。

(三)列宁在力争资产阶级革命胜利和成立民主共和国时,丝毫没有想停留在民主阶段上,把革命运动的规模限制在最多完成资产阶级民主性的任务这个范围之内。恰恰相应,列宁认为民主任务一完成,无产阶级和其他被剥削群众争取实现社会主义革命的斗争马上就应开始。列宁知道这一点,所以他认为社会民主党必须采取一切措施使资产阶级民主革命开始转变为社会主义革命。列宁要求实现无产阶级和农民的专政,并不是为了在革命战胜沙皇制度以后就把革命结束,而是为了尽量延长革命状态,彻底消灭反革命残余,把革命火焰延烧到欧洲去,并在这个时候使无产阶级在政治上得到启发,组织成为一支伟大的军队,然后就开始直接过渡到社会主义革命。

列宁在讲到资产阶级革命的规模、讲到马克思主义政党应使这个革命具有什么样的规模时写道:

\begin{quotation}
“无产阶级应当把民主革命进行到底,这就要把农民群众联合到自己方面来,以便用强力打破专制制度的反抗,并麻痹资产阶级的不稳定性。无产阶级应当实现社会主义革命,这就要把居民中的半无产者群众联合到自己方面来,以便用强力打破资产阶级的反抗,并麻痹农民和小资产阶级的不稳定性。这就是无产阶级的任务,而新火星派(指孟什雏克。——编者注)在他们关于革命规模的一切议论和决议中,却把这些任务看得非常狭隘。”(《列宁全集》俄文第3版第8卷第96页)\footnote{见《列宁选集》第2版第1卷第591页。——译者注}
\end{quotation}

还说:

\begin{quotation}
“领导全体人民,特别是领导农民来为充分的自由,为彻底的民主革命,为共和制奋斗!领导一切被剥削的劳动者来为社会主义奋斗!革命无产阶级的政策实际上就应当是这样;工人政党在革命时期中应当用来贯彻和决定每一种策略手段和每一个实际步骤的阶级口号就是过样。”(《列宁全集》俄文第3版第8卷第105页)\footnote{见《列宁选集》第2版第1卷第602页。——译者注}
\end{quotation}

为了不致留下任何暧昧不明的地方,列宁在《两种策略》一书出版两个月后,又在《社会民主党对农民运动的态度》一文中解释道:

\begin{quotation}
“我们将立刻由民主革命开始向社会主义革命过渡,并且恰恰是按照我们的力量,按照有觉悟有组织的无产阶级的力量,开始向社会主义革命过渡。我们主张不断革命。我们决不半途而废。”(同上,第186页)\footnote{同上,第643页——译者注}
\end{quotation}

这是资产阶级革命与社会主义革命相互关系问题上一个新的指导思想,这是认为到资产阶级革命终结时就要在无产阶级周围重新配置力量以便直接过渡到社会主义革命的新理论,即资产阶级民主革命转变为社会主义革命的理论。

在确定这个新思想的时候,列宁所依据的是:第一,马克思在十九世纪四十年代末《告共产主义者同盟书》中所提出的不断革命的著名原理;第二,马克思在1856年致恩格斯的信中所说的必须把农民革命运动和无产阶级革命结合起来的著名思想。马克思在这封信里说:“德国的全部同题将取决于是否有可能由某种再版的农民战争来支持无产阶级革命。”\footnote{见《马克思恩格斯选集》第4卷第334页——译者注}但马克思的这些英明思想后来在马克思和恩格斯的著作中没有得到发挥,而第二国际的理论家又用各种办法把它们埋藏起来,不再提起。于是一个任务落到了列宁肩上,就是要让马克思的这些被遗忘的原理重见天日,把它们完全恢复过来。但列宁恢复马克思的这些原理时,并没有局限于——并且也不能局限于——把它们简单地重复一遍,而是加以发展,加工成一个严整的社会主义革命论,其中加进了社会主义革命必不可少的新的成分,即无产阶级与城乡半无产者分子联盟是无产阶级革命胜利的条件这一原理。

这个思想粉碎了西欧各国社会民主觉人所持的策略立场,他们认为在资产阶级革命以后,农民群众包括贫农群众在内一定会离开革命,因此在资产阶级革命以后会出现一个漫长的间歇时期,一个长达五十年到一百年甚至更久的“平静”时期,那时无产阶级将“和平地”受着剥削,而资产阶级将“合法地”牟取暴刺,直到新的社会主义的革命到来。

这是新的社会主义革命论,认为进行社会主义革命不是由孤立无援的无产阶级反对整个资产阶级,而是无产阶级担任领导者,有居民中的半无产者即千百万“被剥削的劳动群众”作为同盟者。

按照这个理论,无产阶级在资产阶级革命中在同农民结成联盟的条件下所实现的领导权,应当转变为无产阶级在社会主义革命中在同其他被剥削的劳动群众结成联盟的条件下所实现的领导权;而无产阶级和农民的民主专政则应为无产阶级的社会主义专政准备基础。

这个理论推翻了西欧各国社会民主党人所持的流行理论,这种流行理论否认城乡半无产者群众的革命潜力,认为“除资产阶级和无产阶级外,没有其他可为我国反政府运动或革命运动所依靠的社会力量”(这是普列汉诺夫的一句话,是西欧各国社会民主党人观点的典型)\footnote{见《普列汉诺夫文集》俄文版第3卷第119页。——译者注}。

西欧各国社会民主党人认为,无产阶级在社会主义革命中将单独去反对整个资产阶级,将在没有同盟者的情况下去反对一切非无产者阶级和阶层。他们不愿意估计这样一个事实:资本不仅剥削着无产阶级,而且剥削着千百万城乡半无产者阶层;这些阶层备受资本主义压迫,所以能够成为无产阶级在争取把社会从资本主义压迫下解放出来的斗争中的同盟者。因此,西欧各国社会民主党人认为,欧洲实现社会主义革命的条件还没有成熟,且有等到无产阶级由于社会的经济继续发展而在民族中占大多数、在社会中占大多数的时候,这种条件才算成熟。

列宁的社会主义革命论彻底推翻了西欧各国社会民主党人的过种反无产阶级的腐朽思想。

当时在列宁这个理论中还没有作出社会主义可能在单独一个国家内获得胜利的直接结论。但这个理论已包含有一切或几乎一切早晚作出这种结论所必需的主要成分。

大家知道,列宁在1915年,即过了十年,就作出了这样的结论。

以上就是列宁在《社会民主党在民主革命中的两种策略》这部具有历史意义的著作中所发挥的基本策略原理。

列宁这部著作的历史意义首先在于:它从思想上粉碎了孟什维克的小资产阶级策略方针;武装了俄国工人阶级去进一步开展资产阶级民主革命,去对沙皇制度进行新的冲击;向俄国社会民主党人指出了资产阶级革命必须转变为社会主义革命的光辉前景。

但列宁这部著作的意义还不止于此。它的不可估量的意义就在于它用新的革命论丰富了马克思主义,给布尔什维克党的革命策略奠定了基础,而我国无产阶级在1917年正是依靠这一策略战胜了资本主义制度。


\subsection[四\q 革命的进一步高涨。1905年10月的全俄政治罢工。沙皇政府的退却。沙皇的宣言。工人代表苏维埃的出现]{四\\革命的进一步高涨。1905年10月的全俄政治罢工。\\沙皇政府的退却。沙皇的宣言。工人代表苏维埃的出现}

1905年秋,革命运动已遍及全国各地。运动汹涌澎湃地发展起来了。

9月19日,莫斯科开始了印刷工人的罢工。罢工浪潮扩展到彼得堡和其他许多城市。在莫斯科本市,印刷工人罢工得到了其他产业部门工人的支援而变成了政治总罢工。

10月初,莫斯科—喀山铁路线开始罢工。过了一天,整个莫斯科铁路枢纽站都罢工了。罢工浪潮很快就席卷全国所有的铁路线。邮政局和电报局停止了工作。全俄各城市工人纷纷举行有成千成万人参加的群众大会,并决定停止工作。一个个工厂,一个个城市,一个个地区,都相继卷入罢工。小职员、学生、知识分子(律师、工程师和医生)也加入了罢工工人的洪流。

十月政治罢工发展成了全俄罢工,几乎包括了全国所有的地区,直至最边远的地区,几乎包括了所有的工人,直至最落后阶层的工人。参加这次政治总罢工的,仅仅产业工人就约有一百万,人数相当多的铁路工人和邮电职员等等还未计算在内。国内全部生活陷于停顿。政府已经瘫痪了。

工人阶级领导了人民群众反对专制制度的斗争。

布尔什维克提出的群众性政治罢工的口号产生了应有的结果。

十月总罢工显示了无产阶级运动的力量和声势,迫使吓得要死的沙皇颁布了10月17日宣言。1905年10月17日的宣言答应为人民施行“公民自由的不可动摇的原则:人身的真正不可侵犯,信仰、言论、集会和结社的自由”。答应召集立法杜马,吸收各阶级的居民来参加选举。

这样,布里根的谘议性杜马被革命力量扫除了。布尔什维克抵制布理根杜马的策略被证明是正确的。

虽然如此,但10月17日宣言终究是对人民群众的欺骗,是沙皇的诡计,是沙皇为了麻痹轻信者、赢得时问、聚集力量、然后打击革命所需要的一种喘息时机。沙皇政府口头上答应给予自由,实际上一点切实的东西也没有给。除了许诺之外,工人农民还没有从政府那里得到任何东西。10月21日实行了很小一部分政治犯的赦免,而不是群众期待的政治大赦。同时,政府为了拆散人民力量,组织过许多次蹂躏犹太人的血腥暴行,使成千累万的人牺牲了生命;而为了摧残革命势力,还成立了匪帮式的警察团体:“俄罗斯人民同盟”和“米哈伊尔·阿尔汉格尔同盟”。在这两个团体中间起重大作用的是一些反动的地主、商人、神父和半刑事犯的流氓,所以人民称之为“黑帮”。黑帮分子在警察协助下,公开殴打和杀害先进工人、革命知识分子和大学生,焚烧和射击群众大会和公民集会的场所。沙皇宣言产生的结果看来就是如此。

当时民间流行着这样评论沙皇宣言的歌谣:

\begin{quotation}
“沙皇心发颤,颁布一宣言;

死者得自由,活人进牢监。”
\end{quotation}

布尔什维克向群众解释说,10月17日宣言是一个骗局。他们斥责政府在颁布宣言后的所作所为是挑衅。布尔什维克号召工人拿起武器,准备武装起义。

工人更加努力地成立战斗队了。他们已经明白,政治总罢工争得的10月17日的第一个胜利,要求他们继续努力、继续斗争,去推翻沙皇制度。

列宁在评价10月17日宣言时,说它是力量对比暂时处于某种均势的瞬间,就是说,当时无产阶级和农民迫使沙皇发表了宣言,但还无力推翻沙皇制度,而沙皇制度已不能单用旧的手段维持统治,所以不得不在口头上许诺“公民自由”和“立法”杜马。

在十月政治罢工那些疾风暴雨的日子里,在同沙皇斗争的烈火中,工人群众的革命创造力创造了新的强大的武器——工人代表苏维埃。

工人代表苏维埃是各工厂代表组成的会议,它是世界上从未有过的工人阶级群众性政治组织。1905年初次产生的苏维埃,是1917年无产阶级在布尔什维克党领导下建立起来的苏维埃政权的雏形。苏维埃是表现人民创造精神的新的革命的形式。它纯粹是各革命阶层居民打破沙皇政府的一切法律和规章制度而创立起来的。它是人民奋起反对沙皇制度的自动性的表现。

布尔什维克把苏维埃看作革命政权的萌芽。他们认为苏维埃的力量和作用完全取决于起义的力量和成功。

孟什维克不认为苏维埃是革命政权的萌芽机关,也不认为它是起义机关。他们把苏维埃看作地方自治机关,如民主产生的城市自治机关之类。

1905年10月13日(26日),彼得堡所有的工厂进行了工人代表苏维埃的选举。当天夜间就举行了苏维埃的第一次会议。继彼得堡之后,莫斯科也成立了工人代表苏维埃。

彼得堡工人代表苏维埃是俄国最大的工业中心和革命中心的苏维埃,是沙皇帝国首都的苏维埃,本应该在1905年革命中起决定性的作用。但由于孟什维克的拙劣的领导,它没有完成自己的任务。大家知道,当时列宁还不在彼得堡,还在国外。孟什维克趁列宁不在,钻进了彼得堡苏维埃,并夺得了领导权。在这样的条件下,赫卢斯塔列夫、托洛茨基、帕尔乌斯等孟什维克分子能扭转彼得堡苏维埃的方向去反对起义的政策,是毫不奇怪的。他们不是设法使士兵同苏维埃接近并使两者在共同的斗争中联合起来,反而要求把士兵撤出彼得堡。苏维埃不是把工人武装起来,使他们作好起义准备,而是裹足不前,反对准备起义。

莫斯科的工人代表苏维埃在革命中完全起着不同的作用。莫斯科苏维埃从成立时起就执行了彻底革命的政策。莫斯科苏维埃是由布尔什维克领导的。由于布尔什维克的努力,在莫斯科除了工人代表苏维埃外,还成立了士兵代表苏维埃。莫斯科苏维埃成了武装起义的机关。

在1905年10—12月期间,许多大城市和几乎所有的工人中心都成立了工人代表苏维埃。当时还作了组织陆海军士兵代表苏维埃并把它们同工人代表苏维埃统一起来的尝试。有些地方还成立过工农代表苏维埃。

苏维埃的影响是巨大的。虽然它们往往是自发产生,没有定型,成分上不固定,但它们的行动却象一个政权机关。苏维埃用夺取手段实现了出版自由,确立了八小时工作制,号召人民拒绝向沙皇政府纳税。在个别场合,它们还没收沙皇政府的资金来满足革命的需要。


\subsection[五\q 十二月武装起义。起义的失败。革命的退却。第一届国家杜马。党的第四次(统一)代表大会]{五\\十二月武装起义。起义的失败。革命的退却。\\第一届国家杜马。党的第四次(统一)代表大会}

1905年10月和11月,群众革命斗争继续轰轰烈烈地发展着。工人的罢工仍在继续。

1905年秋,农民反对地主的斗争规模很大。农民运动席卷了全国三分之一以上的县份。萨拉托夫、唐波夫、切尔尼果夫、梯弗里斯、库泰依斯和其他一些省份发生过真正的农民起义。但农民群众冲击的力量还是不足。农民运动还缺乏组织和领导。

在许多城市,如梯弗里斯、海参崴、塔什干、撒马尔汗、库尔斯克、苏胡姆、华沙、基辅、里加等,士兵中的骚动也更厉害了。在喀琅施塔得,在塞瓦斯托波尔的黑海舰队水兵中,都爆发了起义(1905年11月)。但这些起义由于彼此分散,都被沙皇政府镇压下去了。

一些部队和军舰的起义,往往是由军官的虐待和伙食的恶劣(如所谓“豌豆暴动”)等造成的。许多起义的水兵和士兵还没有明确意识到必须推翻沙皇政府,必须坚决把武装斗争继续下去。起义的水兵和士兵太和气、太慈善,他们常常错误地把起义开始时逮捕的军官放掉,听了长官的诺言和劝告就平息下来。

革命已经发展到马上要举行武装起义了。布尔什维克号召群众举行武装起义反对沙皇和地主,向群众说明武装起义已不可避免。布尔什维克不停地准备武装起义。在士兵和水兵中进行了革命工作,在军队中建立了党的军事组织。在许多城市中组织了工人战斗队,在战斗队员中进行了武器使用的训练。组织了在国外购置枪械并把它们秘密运回俄国的工作。参加组织枪械运输工作的有党内著名的工作人员。

1905年11月,列宁回到了俄国。在这些日子里,列宁避开沙皇的宪兵和特务,直接参加了武装起义的准备。他在布尔什维克的《新生活报》上发表的文章,成了对党的日常工作的指示。

在这期间,斯大林同志在南高加索进行了大量革命工作。斯大林同志揭露并狠批了孟什维克,指出他们是反对革命、反对武装起义。他坚决地准备工人去迎接同专制制度的决战。在沙皇宣言发表那天,斯大林同志在梯弗里斯群众大会上向工人说道:

\begin{quotation}
“为了真正获得胜利,我们需要什么呢?为了这点,需要三件东西:第一是武装,第二是武装,第三也还是武装。”
\end{quotation}

1905年12月,在芬兰的塔墨尔福斯召开了布尔什维克代表会议。虽然布尔什维克和孟什维克形式上还同在一个社会民主党内,实际上他们是两个不同的党,各有自己单独的中央。在这次会议上,列宁和斯大林第一次见面了。在此以前,他们互相用书信或通过其他同志来保持联系。

从塔墨尔福斯代表会议的决议中必须指出的有两个决议:一个是关于恢复实际上已经分裂成两个党的党的统一的问题,另一个是关于抵制第一届杜马即所谓维特杜马的问题。

由于此时莫斯科已经开始武装起义,代表会议按照列宁的意见赶忙结束了自己的工作,而代表们也就回到各地亲自参加起义去了。

但沙皇政府也没有睡觉。它也在作决战的准备。沙皇政府同日本媾和减轻了自己的困难处境,就转而对工人和农民实行进攻。沙皇政府在农民起义的许多省份宣布戒严,颁布了“就地正法”、“格杀勿论”的残暴命令,并下令逮捕革命运动的领导人和驱散工人代表苏维埃。

在这种情况下,莫斯科的布尔什维克和受他们领导并与广大工人群众紧相联系的莫斯科工人代表苏维埃,决定立刻准备武装起义。12月5日(18日),莫斯科委员会通过决议:向苏维埃提议宣布政治总罢工,并在斗争进程中把它转变为起义。这个决议在许多工人群众大会上得到了拥护。莫斯科苏维埃考虑到工人阶级的意志,一致决定开始政治总罢工。

莫斯科无产阶级开始起义时已有自己的战斗组织,约有一千个战斗队员,其中半数以上是布尔什维克。莫斯科许多工厂也有战斗队。起义者方面总共约有两千战斗队员。工人指望卫戍部队保持中立,指望把一部分卫戍部队分化和争取过来。

12月7日(20日),莫斯科开始了政治罢工。但这次罢工没能扩展到全国,它在彼得堡就没有得到足够的支持,这种情况从一开始就减少了起义胜利的机会。尼古拉铁路(现为十月铁路)仍然留在沙皇政府手中。这条线路上的运行没有中断,所以政府能够把近卫团从彼得堡调到莫斯科来镇压起义。

在莫斯科本市,卫戍部队已经动摇了。工人开始起义时多少还指望得到它们的支援。但革命者错过了时机,结果沙皇政府把卫戍部队的骚动镇压下去了。

12月9日(22日),莫斯科出现了第一批街垒。接着莫斯科许多街道都筑满了街垒。沙皇政府出动了火炮。它调来了超过起义者数倍的兵力。几千武装工人进行了九天英勇的斗争。沙皇政府只是从彼得堡、特维尔和西部边区调来了几个团,才把起义镇压了下去。各地的起义领导机关在战斗开始前夜不是已被破获,便是已被隔绝。莫斯科布尔什维克委员会也被破获了。武装发动变成了彼此没有联系的各个区的起义。各区失去了领导它们的中心,又没有全市共同的斗争计划,所以主要是局限于防御。正如列宁后来所指出的。这是莫斯科起义力量薄弱的主要根源,同时也是这次起义遭到失败的原因之一。

起义在莫斯科的红色勃列斯尼亚区进行得特别顽强和激烈。红色勃列斯尼亚区是起义的主要堡垒和中心。这里集中了布尔什维克所领导的精锐的战斗队。但是红色勃列斯尼亚区遭到了火与剑的镇压,淹没在血泊和炮击后的烈火之中了。莫斯科的起义被镇压下去了。

起义不仅在莫斯科发生过。革命的起义还席卷了其他许多城市和地区。克拉斯诺雅尔斯克、莫托维里哈(现为皮尔姆)、诺沃罗西斯克、索尔莫沃、塞瓦斯托波尔和喀琅施塔得等城市,都爆发过武装起义。

俄国境内各被压迫民族也奋起进行了武装斗争。格鲁吉亚起义几乎席卷全境。在乌克兰的顿巴斯一带,即在戈尔洛夫卡、亚力山大罗夫斯克、鲁干斯克(现为伏罗希洛夫格勒),起义的规模很大。在拉脱维亚,斗争进行得很顽强。在芬兰,工人成立了赤卫队,并举行了起义。

但所有这些起义,也如莫斯科起义一样,都被沙皇政府用惨无人道的残暴手段镇压下去了。

孟什维克和布尔什维克对十二月武装起义作了不同的评价。

孟什维克普列汉诺夫在武装起义以后对党提出责备,说“本来就用不着拿起武器”。孟什维克硬说起义是不必要的和有害的事情,说在革命中不用起义也可以,说不用武装起义而用和平斗争手段就可以取得胜利。

布尔什维克斥责这种评价是叛卖性的评价。他们认为莫斯科武装起义的经验恰巧证明工人阶级的武装斗争有获得胜利的可能。列宁回答普列汉诺夫“本来就用不着拿起武器”这种责备时说道:

\begin{quotation}
“正好相反,本来应该更坚决、更果敢和更主动地拿起武器,本来应该向群众说明单靠和平罢工是不行的,必须进行英勇无畏和毫不留情的武装斗争。”(《列宁全集》俄文第3版第10卷第50页)\footnote{见《列宁全集》第2版第1卷第666页。——译者注}
\end{quotation}

1905年十二月起义是革命的最高点。在12月,沙皇专制政府打败了起义。从十二月起义失败时起,便开始了革命逐渐退却的转变。革命由高涨转为逐渐低落。

沙皇政府急忙利用这次失败来彻底粉碎革命。沙皇的刽子手和狱吏展开了他们血腥的工作。讨伐队在波兰、拉脱维亚、爱沙尼亚、南高加索和西伯利亚横行无忌。

但是革命还没有被镇压下去。工人和革命农民是缓缓退却的,是且战且退的。更多的工人阶层加入了斗争。1906年有一百多万工人参加罢工。1907年参加罢工的有七十四万。农民运动在1906年上半年席卷了沙俄一半左右的县份,下半年运动所及的县份仍占总县份数的五分之一。陆海军中的骚动仍在继续。

沙皇政府在同革命的斗争中并不是只采用高压手段。它用高压手段获得初步的成功之后,就决定用另一种手段来给革命以新的打击,即召开新的所谓“立法”的杜马。它打算用召开这种杜马的办法来引诱农民离开革命,从而断送革命。1905年12月,沙皇政府颁布了法令,要召开新的所谓“立法”的杜马,以表示与布尔什维克用抵制手段扫除了的那个旧的“谘议性”的布里根杜马有所不同。沙皇的选举法当然是反民主的。选举不是普遍的。半数以上的居民,例如妇女和二百多万工人,被根本剥夺了选举权。选举不是平等的。选民被分成四个选民团,即当时所谓的土地所有者(地主)选民团、城市(资产阶级)选民团、农民选民团和工人选民团。选举不是直接的,而是多级的。选举实际上不是无记名的。选举法保证一小撮地主资本家在杜马中比之千百万工农占有极大的优势。

沙皇想利用杜马来引诱群众脱离革命。当时很大一部分农民还相信经过杜马可以获得土地。立宪民主党人、孟什维克和社会革命党人欺骗工农,说什么不经过起义、不经过革命就可以得到人民所需要的制度。为了同这种对人民的欺骗作斗争,布尔什维克根据塔墨尔福斯代表会议的决议,宣布并实行了抵制第一届国家杜马的策略。

工人进行反对沙皇制度的斗争时,再次把党的力量统一起来,把无产阶级政党统一起来。布尔什维克根据塔墨尔福斯代表会议关于统一问题的著名决议,支持工人的这一要求,并向孟什维克提议召开党的统一代表大会。孟什维克在工人群众的压力下,不得不同意实行统一。

列宁是主张统一的,但他所主张的是不会是革命问题上的意见分歧掩盖起来的统一。调和派(波格丹诺夫、克拉辛等人)力图证明布尔什维克和孟什维克间并没有什么严重分歧,他们给党带来了很大的危害。列宁竭力反对调和派,要求布尔什维克带着自己的纲领去参加代表大会,好让工人们明白布尔什维克所采取的是什么立场,以及统一是在什么基础上实行的。布尔什维克制定了这样的纲领.井把它交给党员们讨论。

1906年4月,在斯德哥尔摩(瑞典)召开了称为俄国社会民主工党统一代表大会的第四次代表大会。出席这次大会的有一百一十一名有表决权的代表,代表着五十七个地方党组织。此外,出席这次大会的还有各民族的社会民主党代表:崩得代表三人,波兰社会民主党代表三人,拉脱维亚社会民主党组织代表三人。

布尔什维克组织因为在十二月起义时和起义后遭到破坏,没有能都派代表出席。除此之外,孟什维克在1905年的“自由日子”里接纳了大批同革命马克思主义毫无共同之处的小资产阶级知识分子加入自己的队伍。只要指出如下一点就足以说明,就是梯弗里斯的孟什维克(当时梯弗里斯的产业工人很少)所选派的大会代表竟与无产阶级最大的组织彼得堡组织所选派的代表人数相等。因此在斯德哥尔摩代表大会上孟什维克占了多散,虽然是不大的多数。

大会成分既是如此,也就决定了大会在许多问题上通过的决议具有孟什维主义的性质。

这次大会只实现了形式上的统一。实际上布尔什维克和孟什维克仍然是各自保持原有的观点,各有自己独立的组织。

第四次代表大会所讨论的最主要的问题是:土地问题,对目前形势和无产阶级阶级任务的估计,对国家杜马的态度,组织问题。

虽然孟什维克在这次大会上占了多数,但他们为了不使工人疏远自己,仍不得不接受列宁所提出的关于党章第一条即党员资格这一条的条文。

在土地问题上,列宁维护土地国有的主张。列宁认为土地国有只有在革命获得胜利的时候,只有在推翻沙皇制度以后才能实现。在这种场合下实行土地国有,能使无产阶级容易联合农村贫民过渡到社会主义革命。土地国有要求无偿地夺取(没收)全部地主土地而转交给农民。布尔什维克的土地纲领是在号召农民进行反对沙皇和地主的革命。

孟什维克采取了另一种立场。他们坚持土地地方公有的纲领。按照这个纲领,地主土地不是交给农民公社支配,甚至也不是交给它们使用,而是交给地方自治机关(或者说地方自治局)支配。农民必须按各人的力量租佃这种土地。

孟什维克的土地地方公有纲领是一个妥协主义的因而是对革命有害的纲领。它不能动员农民进行革命斗争,它不是要彻底消灭地主土地占有制。孟什维克的纲领是要革命半途而废。孟什维克不愿意发动农民起来革命。

大会以多数票通过了孟什维克的纲领。

在讨论关于对目前形势的估计和关于国家杜马这两项决议案时,孟什维克特别明显地暴露了自己反无产阶级的、机会主义的本性。孟什维克马尔丁诺夫公开反对无产阶级在革命中的领导权。斯大林同志回答孟什维克时直截了当地提出问题:

\begin{quotation}
“或者是无产阶级掌握领导权,或者是民主资产阶级掌握领导权,——这就是党内存在着的一个问题,这就是我们意见分歧的所在。”\footnote{见《斯大林全集》第1卷第220页。——译者注}
\end{quotation}

至于讲到国家杜马,孟什维克在自己的决议案中竭力替它吹嘘,说它是解决革命问题,使人民摆脱沙皇制度的最好的工具。布尔什维克则相反,认为杜马是沙皇制度的一种软弱无力的附属品,是掩盖沙皇制度腐朽机体的一块屏风,一旦沙皇制度感到它碍事时,立刻就会把它抛弃的。

在第四次代表大会上被选进中央委员会的是三名布尔什维克和六名孟什维克。参加中央机关报编辑部的全是孟什维克。

很清楚,党内斗争将继续下击。

第四次代表大会以后,布尔什维克和孟什维克之间的斗争更加激烈了。在那些形式上统一的地方组织里,往往由两个报告人来介绍代表大会的情况,一个代表布尔什维克,一个代表孟什维克。对两条路线讨论的结果,各地方组织中的大多数党员,在大多数场合都是站在布尔什维克方面。

实际生活愈来愈证明布尔什维克正确。第四次代表大会所选出的孟什维克的中央,愈来愈暴露出自己的机会主义,暴露出自己完全没有领导群众革命斗争的能力。1906年夏秋两季,群众革命斗争重新加强起来。在喀琅施塔得和斯维阿波尔格,水兵举行了起义;农民反地主的斗争加剧了。而孟什维克的中央提出的却是机会主义的口号,群众并没有跟这些口号走。


\subsection[六\q 第一届国家杜马的解散。第二届国家杜马的召开。党的第五次代表大会。第二届国家杜马的解散。俄国第一次革命失败的原因。]{六\\第一届国家杜马的解散。第二届国家杜马的召开。\\党的第五次代表大会。第二届国家杜马的解散。\\俄国第一次革命失败的原因。}

沙皇政府觉得第一届国家杜马不够驯服,就在1906年夏天把它解散了。沙皇政府变本加厉地镇压人民,在全国各地展开了讨伐队的迫害活动,并宣布要在最短时期内召开第二届国家杜马。沙皇政府公然骄横起来。它看见革命在走向低落,已经不害怕革命了。

布尔什维克应当决定参加还是抵制第二届杜马的问题。当讲到抵制的时候,布尔什维克通常是指积极的抵制,而不是指简单地和消极地拒绝参加选举。布尔什维克认为,积极的抵制是提醒人民防止沙皇把他们由革命道路引上沙皇“宪制”道路的一种革命手段,是打破沙皇这种阴谋并组织人民去重新进攻沙皇制度的一种手段。

对布里根杜马实行抵制的经验,证明抵制“是由事变完全证实了的唯一正确的策略”(《列宁全集》俄文第3版第10卷第27页)\footnote{见《列宁全集》第11卷第124页。——译者注}。这次抵制是成功的,因为它不仅提醒人民防范了沙皇宪制道路的危险,而且在杜马还没有产生以前就把它搞垮了。这次抵制之所以成功,是因为它是在革命向上高涨的时候实行的,并且是依靠着这种高涨实行的,而不是在革命低落的时候实行的,因为搞垮杜马只有在革命高涨条件下才能做到。

抵制维特杜马,即抵制第一届杜马,是在十二月起义失败以后沙皇已经获得胜利的时候,即在可以认为革命是在走向低落的时候实行的。

\begin{quotation}
列宁写道:“但是不言而喻,当时还没有理由认为这个胜利(指沙皇的胜利。——编者注)是决定性的胜利。1906年夏天一连串分散的、局部的军队起义和罢工是1905年十二月起义的继续。抵制维特杜马的口号是争取集中和联合这些起义的口号。”(《列宁全集》俄文第3版第12卷第20页)\footnote{见《列宁选集》第2版第1卷第715页。——译者注}
\end{quotation}

对维特杜马的抵制虽然也大大破坏了这届杜马的威信,并削弱了一部分人民对杜马的信任,却未能把杜马搞垮。其之所以未能把它搞垮,是因为这次抵制,如现在清楚地看到的,是在革命低落、革命处于低潮的形势下实行的。因此,1906年对第一届杜马进行的抵制没有成功。对于这一点,列宁在《共产主义运动中的“左派”幼稚病》这本著名的小册子中写道:

\begin{quotation}
“1905年布尔什维克对‘议会’的抵制,使革命无产阶级增加了非常宝贵的政治经验,表明了,在把合法的同不合法的斗争形式,议会的同议会外的斗争形式互相配合的时候,善于拒绝议会的斗争形式,有时是有益的,甚至是必要的。……1906年布尔什维克抵制‘杜马’,虽然是一个不算大的、易于纠正的错误,但毕竟已经是一个错误。……关于个别人所说的话,作相当的修改,也适用于政治和政党。聪明的人并不是不犯错误的人。不犯错误的人是没有而且也不可能有的。聪明人是不犯重大错误并且又能容易而迅速地纠正错误的人。”(《列宁全集》俄文第3版第25卷第182—183页)\footnote{见《列宁选集》第2版第4卷第192页。——译者注}
\end{quotation}

讲到第二届国家杜马的时候,列宁认为由于形势改变和革命低落的关系,布尔什维克“应当把抵制国家杜马的问题重新研究一下”(《列宁全集》俄文第3版第10卷第26页)\footnote{见《列宁全集》第11卷第123页。——译者注}。

\begin{quotation}
列宁写道:“历史已经表明:当杜马召集起来的时候,就有可能在它内部和在它周围进行有益的鼓动;同革命农民接近而反对立宪民主党的策略,在杜马内部是有可能实行的。”(同上,第29页)\footnote{同上,第127页。——译者注}
\end{quotation}

由此可见,不仅要善于在革命高涨时坚决进攻,在最前列进攻,而且要善于在已经没有高涨形势时正确地退却,在最后面退却,要善于根据已经改变的形势来改变策略,不要乱糟糟地退却,而要有组织地、镇静地、毫不慌张地退却,以求利用最小一点可能来使干部免遭敌人的打击,重新组织队伍,积蓄力量,为重新向敌人进攻作好准备。

布尔什维克决定参加第二届杜马的选举了。

但布尔什维克参加杜马,并不是要像孟什维克那样在杜马中同立宪民主党人联合起来进行同杜马分不开的“立法”工作,而是要利用杜马讲坛来宣传革命。

反之,孟什维克的中央却号召同立宪民主党人达成选举协议,在杜马中支持立宪民主党人,把杜马看作能够约束沙皇政府的立法机关。

大多数党组织都反对孟什维克中央所采取的政策。

布尔什维克要求召开党的下一届代表大会。

1907年5月,在伦敦召开了党的第五次代表太舍。当时俄国社会民主工党(包括各民族的社会民主党组织在内)共有十五万党员。这次大会共有三百三十六名代表参加。布尔什维克有一百零五名;孟什维克有九十七名。其余的人代表着各个民族的社会民主党组织,即波兰、拉脱维亚的社会民主党组织和崩得,这些组织是由上次代表大会接收加入俄国社会民主工党的。

托洛茨基企图在会上结成自己的单独的中派小集团,即半孟什维克的小集团,可是谁也没有跟他走。

布尔什维克因为得到波兰代表和拉脱维亚代表的拥护,在会上获得了稳定的多数。

会上引起斗争的一个基本问题,就是对各资产阶级政党的态度问题。关于这个问题,早在第二次代表大会上布尔什维克和孟什维克之间就已进行过斗争。大会对一切非无产阶级政党——黑帮、十月党人、立宪民主党人和社会革命党人——都作出了布尔什维主义的评价,并规定了对这些政党所应采取的布尔什维主义的策略。

大会赞同布尔什维克的政策,并通过决议要同“俄罗斯人民同盟”、保皇派、贵族联合会等黑帮政党,以及“十月十七日同盟”(十月党人)、工商党和“和平革新”党作无情的斗争。所有这些政党都是明显的反革命政党。

至于自由资产阶级的立宪民主党,大会主张同它进行不调和的揭露性的斗争。大会要求揭露立宪民主党虚伪骗人的“民主主义”立场,要求对自由资产阶级妄想领导农民运动的企图进行斗争。

对于所谓民粹派的或劳动派的政党(人民社会党、劳动团、社会革命党人),大会主张揭露他们冒充社会主义者的企图。同时,大会认为可以同这些政党达成个别的协议,以便共同进行和同时进行反对沙皇制度和立宪民主党资产阶级的斗争,因为这些政党当时还是民主派政党,还代表着城乡小资产阶级的利益。

还在代表大会以前,孟什维克就已提出召开所谓“工人代表大会”。孟什维克的计划是要召开一个由社会民主党人、社会革命党人和无政府主义者参加的代表大会,由这个“工人”代表大会建立一个既像是“非党的党”,又像是“广泛的”小资产阶级的无纲领的工人党。列宁揭穿了孟什维克这种极有害的企图,指出他们是想取消社会民主工党而把工人阶级的先进部队融化到小资产阶级群众中去。大会严厉地斥责了孟什维克的“工人代表大会”口号。

在大会工作中占有特殊地位的是工会问题。孟什维克主张工会“中立”,即反对党对工会的领导作用。大会否决了孟什维克的提案,通过了布尔什维克提出的工会问题决议案。这个决议案指出,党应力求在思想上政治上领导工会。

第五次代表大会表明布尔什维克在工人运动中获得了巨大的胜利。但布尔什维克并没有沾沾自喜,安于既得的胜利。这不是列宁对他们的教诲。布尔什维克知道,同孟什维克的斗争还在前头。

斯大林同志在1907年发表的《一个代表的札记》一文中,对这次代表大会的结果作了如下的评价:

\begin{quotation}
“在革命的社会民主主义的旗帜下把全俄国的先进工人事实上联合成一个全俄国的统一的党,——这就是伦敦代表大会的意义,这就是它的一般性质。”
\end{quotation}

斯大林同志在这篇文章中举出了说明代表大会成分的具体材料。原来,布尔什维克参加这次大会的代表主要是从大工业地区(彼得堡、莫斯科、乌拉尔、伊万诺沃—沃兹涅先斯克等地)选出的,而孟什维克参加这次大会的代表则是从手工业工人和半无产者占优势的小生产地区,以及几个纯粹的农民地区选出的。

\begin{quotation}
斯大林同志对代表大会的情况作出总结说:“很明显,布尔什维克的策略是大工业无产者的策略,是阶级矛盾特别明显和阶级斗争特别激烈的地区的策略。布尔什维主义是真正无产者的策略。另一方面,同样很明显,孟什维克的策略主要是手工业工人和农村半无产者的策略,是阶级矛盾不很明显和阶级斗争还隐蔽着的地区的策略。孟什维主义是无产阶级中半资产阶级分子的策略。数字就是这样说明的。”(《俄国社会民主工党第五次代表大会记录》1935年俄文版第\Rmnum{11}页和第\Rmnum{12}页)\footnote{见《斯大林全集》第2卷第50页和第52页。——译者注}
\end{quotation}

第五次代表大会后不久,沙皇政府举行了所谓六三政变。1907年6月3日,沙皇解散了笫二届杜马。社会民主党的杜马党团共有六十五名代表,都被捕流放到西伯利亚。新的选举法颁布了。工农权利被进一步削减。沙皇政府继续实行进攻。

沙皇大臣斯托雷平对工农大肆进行血腥镇压。成千累万的革命工人和农民惨遭讨伐队枪毙或绞杀。革命者在沙皇刑讯室里受尽各种刑罚和折磨。工人组织,首先是布尔什维克,受到特别残酷的迫害。沙皇的密探拼命搜寻当时匿居芬兰的列宁。他们想摧残革命领袖。1907年12月,列宁冒了很大的危险才逃出毒手,再度流亡国外。

艰苦的斯托雷平反动年代到来了。

这样,第一次俄国革命最后是失败了。

造成失败的有以下几个原因:

(一)当时在革命中还没有工农反沙皇制度的巩固的联盟。农民虽然奋起进行了反地主的斗争,并同工人结成了联盟去反对地主,但他们还不了解不推翻沙皇就不可能推翻地主,还不了解沙皇同地主是一鼻孔出气的,甚至相当一部分农民还信任沙皇,还对沙皇的国家杜马寄托着希望。由此,有许多农民不愿同工人结成联盟去推翻沙皇制度。当时农民相信妥协主义的社会革命党胜过相信真正的革命者布尔什维克。其结果是农民反对地主的斗争缺乏组织。列宁指出:

\begin{quotation}
“……农民的行动过于散漫、无组织和没有充分采取攻势,而这也就是革命遭到失败的根本原因之一。”(《列宁全集》俄文第3版第19卷第354页)\footnote{见《列宁全集》第23卷第255页。——译者注}
\end{quotation}

(二)相当一部分农民不愿同工人一起去推翻沙皇制度的这种心理,也影响到军队的行动,因为军队大多数是由身穿军服的农民子弟组成的。虽然沙皇军队个别部队中也发生过骚动和起义,但大多数士兵还是帮助了沙皇镇压工人的罢工和起义。

(三)工人的行动也不够协调一致。先进的工人阶级队伍在1905年展开了英勇的革命斗争。较为落后的阶层,即工业最不发达省份里那些住在农村的工人,却发动得比较迟缓。他们参加革命斗争特别踊跃是在1906年,但这时工人阶级的先锋队已经大大地削弱了。

(四)工人阶级是革命的先进的基本的力量,但工人阶级党的队伍还没有必要的统一和团结。工人阶级的党俄国社会民主工党分成了两个集团,即布尔什维克集团和孟什维克集团。布尔什维克执行彻底的革命路线,号召工人推翻沙皇制度。孟什维克却以其妥协主义的策略阻碍了革命,使相当一部分工人迷失了方向,分裂了工人阶级。因此,工人在革命中的行动并非始终都是协调一致,而工人阶级既然还没有自己队伍的统一,也就不能成为真正的革命领袖。

(五)西欧帝国主义者帮助了沙皇专制政府镇压了1905年革命。外国资本家害怕丧失他们在俄国的投资和巨额收入。此外,他们还担心一旦俄国革命胜利,其他国家的工人也会起来革命。用此,西欧帝国主义者就来帮助刽子手沙皇。法国银行家贷给沙皇一笔巨款来镇压革命。德国皇帝使数万大军作好准备,想用武装干涉来援助俄国沙皇。

(六)1905年9月缔结的日俄和约给沙皇帮了大忙。战争的失败和革命的迅猛发展,迫使沙皇匆忙签订了和约。战争的失败削弱了沙皇制度,而和约的缔结巩固了沙皇的地位。


\subsection{简短的结论}

俄国第一次革命是我国发展中整整一个历史阶段。这个历史阶段分为两个时期。在第一个时期,革命利用了沙皇在满洲战场上遭受失败而实力削弱这一点,扫除了布里根杜马,并接二连三取得了沙皇的让步,而走向高涨,从10月的政治总罢工发展为12月的武装起义。在第二个时期,沙皇与日本媾和后恢复了元气,利用自由资产阶级对革命的恐惧,利用农民的动摇,把维特杜马投给他们作为施舍,并转而向工人阶级、向革命发起进攻。

在不过是三年的革命时期(1905—1907年),工人阶级和农民受到了他们在三十年平常的和平发展时期所不能受到的丰富的政治教育。革命时期的几年,使得和平发展条件下几十年也无法使人看清楚的事情看清楚了。

革命揭示出:沙皇制度是人民的死敌,沙皇制度是只有坟墓才能使它伸直的驼背。

革命表明:自由资产阶级不想同人民联盟而想同沙皇联盟;自由资产阶级是反革命力量,同它妥协就等于背叛人民。

革命表明:只有工人阶级才能成为资产阶级民主革命的领袖;只有工人阶级才能排除立宪民主党自由资产阶级,使农民摆脱它的影响,消灭地主,把革命进行到底,扫清通往社会主义的道路。

最后,革命表明:劳动农民虽然动摇,但毕竟是能同工人阶级结成联盟的唯一的重大力量。

在革命时期,俄国社会民主工党内有两条路线即布尔什维克路线和孟什维克路线斗争着。布尔什维克采取扩展革命的方针,主张用武装起义推翻沙皇制度,实现工人阶级领导权,孤立立宪民主党资产阶级,同农民联盟,成立由工农代表组成的临时革命政府,使革命达到胜利的结局。反之,孟什维克采取收缩革命的方针。他们所主张的是改良和“改善”沙皇制度而不是用起义推翻沙皇制度,是自由资产阶级领导权而不是无产阶级领导权,是同立宪民主党资产阶级联盟而不是同农民联盟,是召开国家杜马,把它作为全国“革命势力”的中心,而不是成立临时革命政府。

于是孟什维克滚进了妥协主义泥潭,成了向工人阶级传播资产阶级影响的人,事实上成了工人阶级中的资产阶级代理人。

事实表明,只有布尔什维克是党内和国内革命、马克思主义的力量。

可以理解,既然有这样严重的意见分歧,俄国社会民主工党实际上已分裂成了两个党,即布尔什维克党和孟什维克党。党的第四次代表大会丝毫没有改变党内的实际状况,只是把党的形式上的统一保持下来并把它稍微巩固了一下。党的第五次代表大会朝着党的实际上的统一前进了一步,并且这种统一是在布尔什维主义旗帜下进行的。

党的第五次代表大会在总结革命运动时斥责了孟什维克的妥协主义路线,赞同了布尔什维克的革命马克思主义路线。于是大会再次证实了已由俄国第一次革命全部进程证实了的事情。

革命表明:布尔什维克善于在形势要求进攻时就去进攻;他们学会了在最前列进攻和引导人民进行冲击。但除此而外,革命还表明:布尔什维克也善于在形势不利、革命走向低落时有秩序地退却;布尔什维克学会了正确地退却,毫不慌张、毫不忙乱地退却,以求保存干部,积蓄力量,在根据新的形势把队伍重新组织起来后再去向敌人进攻。

不善于正确地进攻,就不能战胜敌人。

不善于正确地退却,毫不慌张、毫不慌乱地退却,就不能在遭受失败时避免覆灭。

