\section[第一章\q 为在俄国建立社会民主工党而斗争(1883—1901年)]{第一章\\ 为在俄国建立社会民主工党而斗争\\ {\zihao{3}(1883—1901年)}}

\subsection[一\q 俄国农奴制度的废除和工业资本主义的发展。现代工业无产阶级的出现。工人运动的最初阶段]{一\\ 俄国农奴制度的废除和工业资本主义的发展。\\ 现代工业无产阶级的出现。\\ 工人运动的最初阶段}

沙皇俄国走上资本主义发展的道路,要比其他各国晚。在十九世纪六十年代以前,俄国的工厂还很少。地主贵族的农奴制经济占着主要地位。工业在农奴制度下无法得到真正的发展。强制性的农奴劳动使农业的劳动生产率非常低。经济发展的整个进程要求消灭农奴制度。沙皇政府由于在克里木战争期间遭到军事失败而削弱,又慑于农民反对地主的“骚动”,不得不于1861年废除农奴制度。

但在农奴制度废除以后,地主还是继续压迫农民。地主对农民进行掠夺,在“解放”农民时剥夺了、割去了农民先前享有的一大部分土地。农民就把这部分土地称为“割地”。农民为了自身的“解放”,被迫向地主交付将近二十亿卢布的赎金。

农奴制度废除以后,农民不得不在最苛刻的条件下租佃地主的土地。农艮除向地主交纳货币租金外,还往往被迫用自己的农具和马匹去替地主白白耕种一定数量的土地。这就叫作“工役”或“劳役”。农民往往不得不把自己收成的一半作为实物地租交给地主。这就叫作“对分制”。

可见,当时情形几乎完全同农奴制度存在时一样,唯一不同的是,这时的农民已有了人身自由,不能再被当作物品来买卖了。

地主用各种掠夺方法(地租、罚款),把落后的农民经济的脂膏榨取净尽。大多数农民因受地主的压迫,不能改善自己的经济。所以,革命前的俄国农业极端落后,时常发生歉收和饥荒。

农奴制经济的残余、苛重的赋税和付给地主的大量赎金(这些往往超过农民经济的收入),引起农民群众破产和贫困,迫使农民离乡背井出外谋生。农民进入工厂。厂主获得了廉价的劳动力。

一大群警察局长、巡官、宪兵、警察和乡丁骑在工农头上,他们保护沙皇、资本家和地主,反对劳动群众,反对被剥削者。肉刑一直存在到1903年。虽然农奴制度已经废除,但农民由于极小的过失,由于没有交纳赋税,仍然遭受鞭笞。工人常受警察和哥萨克毒打,特别是在工人忍受不了厂主的虐待而停工即举行罢工的时候。工人和农民在沙俄时代没有任何政治权利。沙皇专制制度是人民的死敌。

沙俄是各族人民的监狱。沙俄境内许多非俄罗斯民族完全没有权利,经常受到各种侮辱和欺凌。沙皇政府唆使俄罗斯居民把各民族地区的土著民族看作下等种族,正式把他们叫作“异族”,培植鄙视和仇视他们的心理。沙皇政府故意挑起民族纠纷,怂恿一个民族反对另一个民族,组织蹂躏犹太人的暴行,在南高加索挑拨鞑靼人和阿尔明尼亚人互相残杀。

在各民族地区,一切或几乎一切国家职务都由俄罗斯官吏充任。各个机关和法庭的一切事务都采用俄语。禁止用民族语文出版书报,学校里禁止用本民族语文教课。沙皇政府力图扼杀民族文化的任何表现,对一切非俄罗斯民族实行强迫“俄罗斯化”的政策,沙皇制度是残害非俄罗斯民族的刽子手和掌刑人。

农奴制度废除以后,俄国工业资本主义虽然还受农奴制残余的阻碍,但已发展得相当迅速了。在1865—1890年这二十五年内,单是大工厂和铁路的工人,就由七十万零六千人增加到一百四十三万三千人,即增加了一倍以上。

在九十年代,俄国资本主义大工业发展得更为迅速了。到九十年代末,大工厂、矿业和铁路的工人,单以欧俄五十省来说,就已增加到二百二十万零七干人,而以全俄来说,则已增加到二百七十九万二千人。

这已是现代工业无产阶级,无论就其在大资本主义企业中的团结性来说,还是就其战斗的革命的品质来说,它同农奴制时代的工厂工人.同小手工业和其他一切工业的工人根本不同。

九十年代的工业高涨,首先跟加紧修筑铁路分不开。十年内(1890—1900年)一共修筑了二万一千多俄里新铁路。铁路需要大量的金属(制造路轨、机车和车辆),需要愈来愈多的煤炭、石油等燃料,这就引起了冶金工业和燃料工业的发展。

在革命前的俄国,也如在一切资本主义国家一样,工业高涨年代总是跟工业危机和工业停滞年代交替着,因而使工人阶级受到严重打击,使数十万工人陷于失业和贫困。

虽然俄国资本主义在农奴制度废除后有很迅速的发展,但俄国经济发展程度仍比其他资本主义国家落后得多。绝大多数居民还是从事农业。列宁在《俄国资本主义的发展》这本有名的著作中,引用了1897年全国人口调查册中的重要数字,指明当时经营农业的人口约占全人口六分之五,而从事大小工业、商业、铁路和水路运输业、建筑业、森林采伐等等的人口总共只占全人口六分之一左右。

由此可见,俄国虽有资本主义的发展,但它还是一个经济落后的农业国家,小资产阶级国家,即小私有的、生产率很低的个体农民经济还占主要地位的国家。

资本主义不仅在城市,而且也在农村发展起来。农民这一在革命前的俄国人数最多的阶级,日益瓦解,日益分化。在农村里,一方面从最富裕的农民中间产生出富农上层,即农村资产阶级;另一方面,又有许许多多农民陷于破产,贫苦农民即农村无产者和半无产者的人数,逐渐增加起来。中农人数一年比一年减少。

1903年,俄国约有一千万农户。据列宁在《给农村贫民》这本小册子中所作的计算,这个户数中至少有三百五十万是无耕马的农民。这些贫苦户通常只种很小一块土地,其余的土地租给富农,而自己出外谋生。贫苦农民按其地位说来,跟无产阶级非常接近。列宁称他们为农村的无产者或半无产者。

另一方面,一百五十万户富农(总农户数是一千万),占有全部农民耕地的半数。这个农民资产阶级靠盘剥贫农和中农,靠剥削长工和短工的劳动而发财致富,变成农业资本家。

早在十九世纪七十年代,特别是八十年代,俄国工人阶级就已开始觉醒起来,同资本家进行斗争了。沙俄时代的工人生活非常困苦。在八十年代,工厂里的工作时间至少是十二个半小时,纺织工业中甚至长达十四以至十五小时。对女工童工劳动的剥削采用得很广。童工劳动时间虽与成年工人相等,但所领的工资也如女工一样,要比成年男工少得多。工资非常低。大部分工人每月只能领得七八个卢布。就是金属加工厂和铸造厂里工资最高的工人,每月至多也只能领得三十五个卢布。根本没有什么劳动保护,结果造成工人大量的残废和死亡。根本没有什么工人保险,看病完全自费。居住条件非常恶劣。工人集体宿舍一间矮小的黑屋子,要住十至十二个工人。厂主时常克扣工人的工资,强迫工人在厂主开设的店铺高价购买食品,并用罚款的办法掠夺工人。

工人们开始商量,共同向厂主提出要求,以改善他们难以忍受的生活状况。他们撂下工作——宣布罢工。七十年代和八十年代最初发生的那些罢工,通常是由于罚款过高、发工资时实行蒙骗,降低计件工资标准等引起的。

在最初举行的罢工中,工人常因忍无可忍而毁坏机器,打破厂房玻璃,捣毁厂主的店铺和办事处。

先进工人开始明白,要同资本家顺利地进行斗争,必须组织起来。工人协会相继出现了。

1875年。在敖德萨成立了“南俄工人协会”。这个最初的工人组织存在了八九个月,后来就被沙皇政府破坏了。

1878年,在彼得堡成立了由木工哈尔土林和钳工奥勃诺尔斯基领导的“俄国北方工人协会”。协会纲领说,协会的任务与西方社会民主工党的任务相同。协会的最终目的是实现社会主义革命,即“推翻国内现存的政治经济制度,因为它是极不公平的制度”。协会的组织者之一奥勃诺尔斯基曾在国外住过一些时候,他在那里对马克思主义的社会民主党和马克思所领导的第一国际的活动有所了解。这一点在“俄国北方工人协会”纲领上有了反映。协会认为自己的当前任务是为人民争取政治自由和政治权利(言论自由、出版自由、集会权利等等)。当前要求中还包括限定工作日。

协会有会员二百人,还有同样数目的同情者。协会开始参加工人的罢工,领导工人的罢工。这个工人协会也被沙皇政府破坏了。

然而,工人运动继续发展,席卷了愈来愈多的地区。八十年代发生过很多次罢工。五年(1881—1886年)间发生的罢工至少有四十八次,共有八万工人参加。

1885年奥列哈沃—祖也沃的莫罗佐夫工厂的大罢工,在革命运动史上具有特别重大的意义。

当时这个厂大约有八千工人。劳动条件一天比一天坏。从1882年至1884年,工资减过五次,1884年那次一下就把计件工资标准减了四分之一,即减了百分之二十五。此外,厂主莫罗佐夫还用各种罚款剥削工人。罢工后的审判表明,工人每挣一卢布工资。厂主就用罚款从中扣去三十至五十戈比。工人忍受不了这种掠夺,于1885年1月宣布了罢工。这次罢工是事先准备好的。领导罢工的是先进工人彼得·莫伊先科,他先前是“俄国北方工人协会”的会员,有革命工作的经验。在罢工前夜,莫伊先科同其他一些最觉悟的织布工人一起拟定了向厂主提出的许多要求,这些要求并经工人秘密会议通过。首先工人们要求停止抢劫式的罚款。

这次罢工被武力镇压下去了。六百多工人被捕,其中几十人被交付法庭审判。

1885年,伊万诺沃—沃兹涅先斯克的一些工厂也举行过同样的罢工。

第二年,沙皇政府慑于工人运动的增长,被迫颁布了罚款法,规定所罚的款项不得落入厂主的私囊,而必须用于工人自己的需要。

根据莫罗佐夫厂和其他厂的罢工经验,工人们懂得了:通过有组织的斗争,他们可以争得很多的东西。工人运动中开始涌现出了许多坚决捍卫工人阶级利益的能干的领导者和组织者。

同时,由于工人运动的增长和西欧工人运动的影响,第一批马克思主义组织开始在俄国建立起来。

\subsection[二\q 俄国民粹主义和马克思主义。普列汉诺夫及其“劳动解放社”。普列汉诺夫对民粹主义的斗争。马克思主义在俄国的传播]{二\\ 俄国民粹主义和马克思主义。\\ 普列汉诺夫及其“劳动解放社”。\\ 普列汉诺夫对民粹主义的斗争。\\ 马克思主义在俄国的传播}

在马克思主义团体出现以前,民粹派在俄国进行过革命工作,他们是马克思主义的敌人。

俄国第一个马克思主义团体出现于1883年,这就是“劳动解放社”。它是格·瓦·普列汉诺夫因进行革命活动受沙皇迫害、被迫逃亡国外、侨居日内瓦时所组织的。

普列汉诺夫先前本是一个民粹主义者。他在国外了解了马克思主义之后,就同民粹主义决裂而成为一个杰出的马克思主义宣传家。

“劳动解放社”在俄国传播马克思主义这点上进行过很多工作。它把马克思和恩格斯的《共产党宣言》、《雇佣劳动与资本》、《社会主义从空想到科学的发展》以及其他著作译成俄文,在国外刊印后秘密散布到俄国国内。格·瓦·普列汉诺夫、查苏利奇、阿克雪里罗得以及该社的其他参加者,还写过许多解释马克思恩格斯学说、解释科学社会主义思想的著作。

无产阶级的伟大导师马克思和恩格斯与空想社会主义者相反,最先说明了,社会主义不是幻想家(空想主义者)的臆造,而是现代资本主义社会发展的必然结果。他们指出,资本主义制度定将崩溃,正如农奴制度已经崩溃一样;资本主义造成了自身的掘墓人,即无产阶级。他们指出,只有无产阶级的阶级斗争,只有无产阶级战胜资产阶级,才能使人类摆脱资本主义,摆脱剥削制度。

马克思和恩格斯教导无产阶级要认识本身的力量,认识本身的阶级利益,联合起来同资产阶级坚决作斗争。马克思和恩格斯发现了资本主义社会发展的规律,科学地证明了,资本主义社会的发展以及这个社会里的阶级斗争必然导致资本主义的崩溃,导致无产阶级的胜利,导致无产阶级专政。

马克思和恩格斯教导说,摆脱资本政权并把资本主义所有制变为公有制,不是用和平手段可以达到的;要达到这一步,工人阶级必须用革命暴力反对资产阶级,实现无产阶级革命,建立自己的政治统治即无产阶级专政,以便镇压剥削者的反抗,并建立起新社会即无阶级的共产主义社会。

马克思和恩格斯教导说,工业无产阶级是资本主义社会里最革命的阶级,因而也是最先进的阶级;只有无产阶级这样一个阶级,才能把一切不满意资本主义的势力集合到自己周围,引导他们去冲击资本主义。但要战胜旧世界和建立无阶级的新社会,无产阶级必须建立自己的即马克思和恩格斯称为共产党的工人政党。

正是俄国第一个马克思主义团体,普列汉诺夫的“劳动解放社”,进行了传播马克思恩格斯观点的工作。

当“劳动解放社”在国外用俄文出版刊物,举起马克思主义旗帜的时候,俄国还没有社会民主主义运动。首先必须在理论上、思想上为这个运动开拓道路。当时在传播马克思主义和开展社会民主主义运动道路上所遇到的主要思想障碍,是在先进工人和怀有革命情绪的知识分子中最为流行的民粹主义现点。

随着俄国资本主义的发展,工人阶级成了能够进行有组织的革命斗争的强大先进力量。但民粹派不了解工人阶级的先进作用。俄国民粹派错误地认为.主要的革命力量不是工人阶级而是农民,单靠农民“骚动”就能把沙皇和地主政权推翻。民粹派不熟悉工人阶级。他们不了解,不和工人阶级联盟,没有工人阶级领导,单是农民不可能战胜沙皇制度和地主。民粹派不了解,工人阶级是社会上最革命最先进的阶级。

起初,民粹派企图发动农民去进行反对沙皇政府的斗争。为着这一目的,革命的知识青年就穿起农民衣服,跑到农村去,用当时的说法就是“到民间去”。由此产生了“民粹派”这一名称。但农民并没有跟他们走,因为他们对农民也并不真正熟悉,并不真正了解。大多数民粹派分子被警察逮捕了。于是民粹派决定不要人民,单靠自己的力量继续进行反对沙皇专制制度的斗争,结果犯了更严重的错误。

民粹派的秘密团体“民意党”准备行刺沙皇。1881年3月1日,民意党人果然用炸弹把沙皇亚历山大二世炸死了。但这并没有使人民获得丝毫益处。刺杀个别人物,并不能推翻沙皇专制制度,并不能消灭地主阶级。一个沙皇刚被刺死,另一个沙皇亚历山大三世又代之而起。在亚历山大三世统治下,工人和农民的生活更坏了。

民粹派采取这种刺杀个别人物、实行个人恐怖的斗争手段来反对沙皇制度,是错误的、对革命有害的。个人恐怖政策,是从民粹派所谓“英雄”是积极的而“群氓”是消极的,“群氓”应等待“英雄”建立丰功伟绩这一谬论出发的。这一谬论认为,只有个别杰出人物才能创造历史,而群众、人民、阶级,或如民粹派作家们轻蔑地称呼的“群氓”,不会自觉地有组织地行动,只能盲目地跟着“英雄”走。因此,民粹派拒绝在农民和工人阶级中进行群众性的革命工作,转而采取个人恐怖手段。民粹派强迫当时影响最大的革命家之一斯切潘·哈尔土林停止组织革命工人协会的工作,而完全去干恐怖活动。

民粹派用刺杀压迫者阶级的个别代表人物这种对革命无益的行动,转移了劳动群众同压迫者阶级作斗争的注意力。他们阻碍了工人阶级和农民的革命主动性与积极性的发挥。

民粹派妨碍工人阶级了解自己在革命中的领导作用,阻碍创立工人阶级的独立政党。

虽然民粹派的秘密组织已被沙皇政府破坏,但民粹主义观点在有革命情绪的知识分子中仍然保持了很久。民粹派的残余拼命抵抗马克思主义在俄国的传播,阻挠工人阶级组织起来。

因此,只有同民粹主义作斗争,马克思主义才能在俄国成长壮大起来。

“劳动解放社”开展了反对民粹派错误观点的斗争,指出民粹派的学说和民粹派的斗争方式对工人运动极为有害。

普列汉诺夫在反民粹派的著作中指出,虽然民粹派也自称为社会主义者,但民粹派的观点与科学社会主义没有丝毫共同的地方。

普列汉诺夫第一个给了民粹派的错误观点以马克思主义的批评。普列汉诺夫一方面对民粹派观点给以一针见血的打击,同时光辉地捍卫了马克思主义的观点。

受到普列汉诺夫致命打击的是民粹派的哪些基本错误观点呢?

第一,民粹派认为:资本主义在俄国是种“偶然”现象,资本主义不会在俄国发展起来,因此无产阶级也不会成长和发展起来。

第二,民粹派并不认为工人阶级是革命中的先进阶级。他们妄想不要无产阶级而达到社会主义。他们认为知识分子所领导的农民以及他们视为社会主义萌芽和基础的农民公社,是主要的革命力量。

第三,民粹派对于整个人类历史进程持着错误而有害的观点。他们不知道也不懂得社会的经济和政治发展的规律。他们在这方面是些完全落后的人。按照他们的意见,创造历史的不是阶级,不是阶级斗争,而是个别杰出人物即“英雄”;群众、“群氓”、人民、阶级是盲目地跟着“英雄”走的。

普列汉诺夫在同民粹派作斗争和揭露他们的时候,写了许多马克思主义的著作,当时俄国马克思主义者就是靠这些著作进行学习和得到培养的。普列汉诺夫的《社会主义与政治斗争》、《我们的意见分歧》、《论一元论历史观之发展》等著作,为马克思主义在俄国的胜利扫清了基地。

普列汉诺夫在他的著作中叙述了马克思主义的基本问题。他在1895年出版的著作《论一元论历史观之发展》,有特别重大的意义。列宁指出,这本书“培养了一整代俄国马克思主义者”(《列宁全集》俄文第3版第14卷第347页)\footnote{见《列宁全集》第16卷第267页(本书译者注中提到的书均指中文版本)。——译者注}。

普列汉诺夫在反民粹派的著作中证明说,像民粹派那样提出问题,即问资本主义应否在俄国发展,是很荒谬的。普列汉诺夫用事实证明说,问题在于俄国已经走上了资本主义发展的道路。并且没有什么力量能使它离开这条道路。

革命者的任务不是阻止资本主义在俄国发展,——这是他们无论如何也做不到的。革命者的任务是要依靠资本主义的发展所造成的强大革命力量即工人阶级,发展它的阶级意识,把它组织起来,帮助它建立自己的工人政党。

普列汉诺夫把民粹派第二个基本错误观点,即否认无产阶级在革命斗争中能起先进作用的观点,也批倒了。民粹派把无产阶级在俄国的出现看作“历史上的不幸”,撰文指摘“无产阶级化是一种病害”。普列汉诺夫捍卫了马克思主义学说,认为它完全适用于俄国。他证明说,虽然农民在人数上占优势,无产阶级在人数上比较少,革命者却正是应当把自己的主要希望寄托于无产阶级,寄托于它的增长。

为什么正是应当寄托于无产阶级呢?

因为无产阶级虽然现在人数很少,但它是同最先进的经济形式即大生产相联系的,因而是具有远大前途的劳动阶级。

因为无产阶级这个阶级一年年增长着,在政治上发展着,由于大生产中的劳动条件而容易组织起来,由于自己的无产者地位而最有革命性,因为它在革命中失去的只是自己身上的锁链。

农民却不是这样。

农民(这里是指个体农民。——编者注)虽然人数众多,但它是同最落后的经济形式即小生产相联系的,因而是没有并且也不可能有远大前途的劳动阶级。

农民这个阶级不仅不增长,反而一年年分化为资产阶级(富农)和贫农(无产者、半无产者)。除此而外,他们由于本身分散而不如无产阶级那样容易组织起来,他们由于自己所处的小私有者的地位而不如无产阶级那样乐于投入革命运动。

民粹派认为俄国到达社会主义不会通过无产阶级专政,而会通过他们视为社会主义萌芽和基础的农民公社。然而公社不是并且也不可能是社会主义的基础或萌芽,因为在公社里占统治地位的是富农,即剥削贫农、雇农和力量单薄的中农的“土豪”。当时有名无实的公社土地占有制和间或实行过的按人口重分土地的办法,丝毫没有改变这种情况。享用土地的是公社里拥有耕畜、农具和种籽的社员,即富裕的中农和富农。无耕马的农民、贫农和一般力量单薄的农民,却不得不把土地让给富农,自己去受人雇用,去当雇农。农民公社事实上是掩饰富农的豪强的一种方便形式,是沙皇政府按连环保原则向农民征税的一种便利的工具。因此,沙皇政府没有触动过农民公社。把这样的公社当作社会主义的萌芽或基础是很可笑的。

普列汉诺夫把民粹派第三个基本错误观点,即认为“英雄”、杰出人物及其思想在社会发展中起头等重要作用而群众、“群氓’、人民、阶级起不了什么作用这一观点,也批倒了。普列汉诺夫斥责民粹派为唯心主义,他证明说,真理不在唯心主义方面,而在马克思恩格斯的唯物主义方面。

普列汉诺夫发挥并论证了马克思主义的唯物主义观点。他根据马克思主义的唯物主义证明,决定社会发展的,归根到底不是杰出人物的愿望和思想,而是社会存在的物质条件的发展,是社会存在所必需的物质资料的生产方式的变更,是各个阶级在物质资料生产方面相互关系的变更,是各个阶级为着自己在物质资料生产和分配方面的作用和地位而进行的斗争。不是思想决定人们的社会经济地位,而是人们的社会经济地位决定人们的思想。如果杰出人物的思想和愿望与社会的经济发展背道而驰,与先进阶级的要求背道而驰,那么这种杰出人物就会变成无用之物,反之,如果杰出人物的思想和愿望正确反映社会经济发展的要求,正确反映先进阶级的要求,那他们就能成为真正杰出的人物。

对于民粹派所谓群众是群氓、只有英雄才能创造历史并把群氓变为人民的论断,马克思主义者的回答是:并不是英雄创造历史,而是历史创造英雄,也就是说,不是英雄创造人民,而是人民创造英雄并推动历史前进。英雄,杰出人物,只有当他们能正确理解社会发展条件,理解应当如何改善这些条件的时候,才能在社会生活中起重大的作用。英雄、杰出人物如果不能正确理解社会发展条件,以至俨然以历史“创造者”自居,不顾社会的历史要求而一意孤行,那他们就会变成滑稽可笑、一钱不值的倒霉人物。

民粹派就是这种倒霉的英雄。

普列汉诺夫的著作,他对民粹派的斗争,大大地消除了民粹派在革命知识分子中的影响。但从思想上粉碎民粹主义还远未完成。这个任务,即彻底打垮民粹主义,打垮这个马克思主义敌人的任务,落到列宁的肩上了。

自从“民意党”被破坏以后,大多数民粹派分子很快就放弃了反沙皇政府的革命斗争,而主张同沙皇政府调和妥协。在八十年代和九十年代,民粹派已变成富农利益的代表者了。

“劳动解放社”先后拟定了两个俄国社会民主党纲领草案(第一个草案在1884年,第二个草案在1887年)。这对俄国马克思主义社会民主党建党准备工作是很重要的一步。

但“劳动解放社”也有严重的错误。在它的第一个纲领草案中,还有民粹派观点的残余,还包含有个人恐怖的策略。其次,普列汉诺夫没有注意到,无产阶级在革命进程中能够而且应当引导农民前进,并且只有同农民联盟,才能战胜沙皇制度。再次,普列汉诺夫把自由资产阶级看作是能够给革命以援助——虽然是不可靠的援助——的力量;至于农民,那他在某些著作中却完全忽略了。例如他说:

\begin{quotation}
“除资产阶级和无产阶级外,没有其他可为我国反政府运动或革命运动所依靠的社会力量。”(《普列汉诺夫文集》俄文版第3卷第119页)
\end{quotation}

普列汉诺夫的这些错误观点,就是他后来的孟什维克主义观点的萌芽。

无论“劳动解放社”或当时的马克思主义小组,都还没有在实践上同工人运动联系起来。这还是马克思主义理论、马克思主义思想、社会民主党纲领原理在俄国产生和巩固起来的时期。在1884—1894年这十年中,社会民主党还只是以个别人数不多的团体和小组的形式存在,它们同群众性的工人运动还没有联系,或是很少联系。当时社会民主党好像一个还没有诞生但已在母亲胎胞里发育着的婴儿,如列宁所说,还处在“胚胎发育的过程中”\footnote{见《列宁选集》第2版第1卷第386页。——译者注}。

列宁指出:“劳动解放社”“只是在理论上为社会民主党奠定了基础,跨出了走向工人运动的第一步”\footnote{见《列宁全集》第20卷第275页。——译者注}。

在俄国把马克思主义同工人运动结合起来,并把“劳动解放社”的错误纠正过来的任务,只得由列宁来解决了。


\subsection[三\q 列宁革命活动的开始。彼得堡“工人阶级解放斗争协会”]{三\\ 列宁革命活动的开始。\\ 彼得堡“工人阶级解放斗争协会”}

布尔什维克主义创始人弗拉基米尔·伊里奇·列宁,1870年生于辛比尔斯克市(现为乌里杨诺夫斯克市)。1887年,列宁进了喀山大学,但不久就因参加学生革命运动被捕并被开除学籍。列宁在喀山加入了费多谢也夫组织的马克思主义小组。自从列宁迁居萨马拉后,很快就以列宁为中心成立了萨马拉第一个马克思主义者小组。还在那个时候,列宁就以深知马克思主义而使大家惊服了。

1893年底,列宁迁居彼得堡。列宁最初发表的几次言论,已经给彼得堡马克思主义小组参加者留下了强烈的印象。由于非常深知马克思的学说,善于把马克思主义应用于当时俄国的经济政治环境,对工人事业的胜利有坚定强烈的信心,具有卓越的组织才能,列宁成了彼得堡马克思主义者公认的领导者。

列宁受到了他所指导的那些小组的先进工人的热烈爱戴。

工人巴布什金回忆列宁给工人小组讲课的情形时说:“我们听的课生动有趣.大家听了都非常满意。经常赞叹我们讲师的智慧。”

1895年,列宁把彼得堡所有的马克思主义工人小组(当时已有二十个左右)统一成了“工人阶级解放斗争协会”,从而为建立革命的马克思主义的工人政党作了准备。

列宁向“斗争协会”提出了密切联系群众性的工人运动、并从政治上加以领导的任务。列于提出,要从专门在宣传小组的少数先进工人中宣传马克思主义,转到在广大的工人阶级群众中进行迫切的政治鼓动。这个向群众性鼓动的转变,对于俄国工人运动以后的发展有重大的意义。

在九十年代,工业正值高涨时期。工人数量增加了。工人运动加强了。据不完全的统计,1895—1899年的罢工工人数目不下于二十二万一千。工人运动成了全国政治生活中的重大力量。现实生活本身证实了马克思主义者在同民粹派作斗争时所捍卫的观点,即工人阶级在革命运动中能起先进的作用。

在列宁领导下,“工人阶级解放斗争协会”把工人为改善劳动条件,缩短工作日和增加工资等经济要求进行的斗争同反对沙皇制度的政治斗争联系起来。“斗争协会”从政治上教育了工人。

在列宁领导下,彼得堡“工人阶级解放斗争协会”第一次在俄国实现了社会主义同工人运动的结合。哪个工厂一罢工,“斗争协会”因为通过自己小组成员而很熟悉各企业的情形,立刻就能印发传单、印发社会主义的宣言来响应。这些传单揭露厂主虐待工人的情形,说明工人应当怎样为本身的利益斗争,载明工人群众的要求。这些传单把资本主义的病害如工人生活困苦、工人每天要干十二至十四小时极其繁重的劳动,工人毫无权利等等,都揭露无遗。同时这些传单又提出了相应的政治要求。1894年底,列宁在工人巴布什金参加下,写了第一个这样的鼓动传单和给彼得堡的谢勉尼柯夫工厂罢工工人的号召书。1895年秋,列宁写了声援托伦顿厂男女罢工工人的传单。这个厂是英国资本家开办的,他们赚了亿万的利润。这里的工作日长达十四小时以上,而织布工人每月不过挣七卢布左右。罢工结果是工人获得了胜利。在很短一个时期,“斗争协会”印发了几十种这样的告各工厂工人的传单和号召书。每一份传单都有力地鼓舞了工人的斗志,工人看到,社会主义者是帮助他们,保护他们的。

1896年夏,在“斗争协会”领导下举行了彼得堡三万纺织工人的大罢工。基本的要求是缩短工作日。在这次罢工的压力下,沙皇政府不得不于1897年6月2日颁布法令,把工作日限定为十一小时半。在法令颁布以前,工作日根本是不限定的。

1895年12月,列宁被沙皇政府逮捕。但列宁在监狱里也没有停止革命斗争。他提出种种意见和指示来帮助“斗争协会”,从监狱里寄出他写的小册子和传单。列宁在监狱里写了小册子《谈谈罢工》和揭露沙皇政府的专横暴戾的传单《告沙皇政府》。列宁在监狱里还写了党纲草案(用牛奶写在一本医书的字行中间)。

由于彼得堡“斗争协会”的强有力的推动,俄国其它城市与地区的工人小组也相继统一成为这样的协会。九十年代中期,南高加索出现了一些马克思主义的组织。1894年,莫斯科成立了“工人协会”。九十年代末,西伯利亚产生了“社会民主主义联盟”。在九十年代,伊万诺沃—沃兹涅先斯克、雅罗斯拉夫里和科斯特罗马出现了马克思主义团体,后来这些团体统一成了“社会民主党北方协会”。九十年代后半期,顿河岸罗斯托夫、叶加特林诺斯拉夫、基辅,尼古拉也夫、土拉、萨马拉、喀山,奥列哈沃-祖也沃等城市都相继成立了社会民主主义的团体和协会。

彼得堡“工人阶级解放斗争协会”的意义,如列宁所说,就在于它是依靠着工人运动的革命政党的第一个不容忽视的萌芽。

列宁后来就是根据彼得堡“斗争协会”的革命经验来进行俄国马克思主义社会民主党的建党工作的。

自从列宁和他的亲密战友们被捕后,彼得堡“斗争协会”的领导成分大大改变了。新的人物出头露面,他们自称为“青年人”,而把列宁和他的战友称为“老头子”。他们开始实行错误的政治路线。他们说,必须叫工人只进行反对厂主的经济斗争,而政治斗争是自由资产阶级的事情,政治斗争的领导权应当属于自由资产阶级。

这些人就被称为“经济派”。

这是俄国马克思主义组织中第一个妥协主义的、机会主义的集团。

\subsection[四\q 列宁反对民粹主义和“合法马克思主义”的斗争。列宁提出的工农联盟思想。俄国社会民主工党第一次代表大会]{四\\ 列宁反对民粹主义和“合法马克思主义”的斗争。\\ 列宁提出的工农联盟思想。\\ 俄国社会民主工党第一次代表大会}

虽然普列汉诺夫在八十年代对民粹主义那一套观点已经给了一大打击,但民粹派观点在九十年代初期还博得一部分革命青年的同情。有一部分青年继续认为俄国可以避免资本主义发展的道路,认为将来在革命中起主要作用的是农民而不是工人阶级。民粹派的残余竭力阻碍马克思主义在俄国的传播,攻击马克思主义者,竭力诽谤他们。当时必须从思想上彻底粉碎民粹主义,以保证马克思主义的进一步传播,为建立社会民主党创造条件。

这个任务由列宁完成了。

列宁在《什么是“人民之友”以及他们如何攻击社会民主主义者?》(1894年)一书中,彻底揭穿了民粹派冒充“人民之友”其实是人民之敌的真面目。

九十年代的民粹派,实际上早已放弃了任何反沙皇政府的革命斗争。自由主义的民粹派主张同沙皇政府和解。列宁谈到当时的民粹派时写道:“他们简直以为只要向这个政府客客气气地请求一下,它就会把一切都安顿得妥妥贴贴。”(《列宁全集》俄文第3版第1卷第161页)\footnote{见《列宁全集》第1卷第238页。——译者注}

九十年代的民粹派闭眼不看农村贫民的生活状况、农村中的阶级斗争和富农对贫农的剥削,而一味赞美富农经济的发展。他们事实上代表富农的利益。

同时,民粹派又在他们的杂志上拼命攻击马克思主义者。他们故意颠倒是非,歪曲俄国马克思主义者的观点,硬说马克思主义者是希望农村破产,是想“让每个农夫到工厂的锅炉里去受熬煎”。列宁在揭露民粹派这种荒谬批评时指出,问题不在于马克思主义者的“愿望”,而在于俄国资本主义的真实发展进程,而在这个进程中,无产阶级的人数必然要增加起来。但无产阶级将成为资本主义制度的掘墓人。

列宁指出,愿意消灭资本家地主的压迫、消灭沙皇制度的真正的人民之友,并不是民粹派,而是马克思主义者。

列宁在《什么是“人民之友”》书中,第一次提出了工农革命联盟是推翻沙皇制度、推翻地主资产阶级的主要手段这一思想。

列宁在这个时期所写的许多著作中,批判了民粹主义者的主要派别民意党人所运用,后来又由民粹派继承者社会革命党人所运用的那些政治斗争手段,特别是个人恐怖的策略。列宁认为这种策略是对革命运动有害的,因为这种策略用单个英雄人物的斗争来代替群众的斗争。这种策略意味着不相信人民革命运动。

列宁在《什么是“人民之友”》一书中,规定了俄国马克思主义者的基本任务。列宁认为俄国马克思主义者首先应当把零散的马克思主义小组组织成一个统一的社会主义工人政党。接着列宁又指出,正是俄国工人阶级同农民结成联盟,将推翻沙皇专制制度,然后俄国无产阶级同被剥削的劳动群众结成联盟,和世界各国无产阶级一道,循着公开的政治斗争大道,走向胜利的共产主义革命。

由此可见,列宁早在四十多年前就已正确地指出了工人阶级的斗争道路,确定了工人阶级是社会的先进革命力量,确定了农民是工人阶级的同盟者。

列宁和他的拥护者所进行的斗争,还在九十年代就从思想上把民粹主义彻底粉碎了。

列宁反对“合法马克思主义”的斗争也有巨大的意义。如历史上常见的那样,一个大的社会运动通常总有一些暂时的“同路人”混入。所谓“合法马克思主义者”就是这样的“同路人”。当时马克思主义在俄国已经广泛传播。于是,资产阶级的知识分子开始披上马克思主义的外农。他们常在合法的即沙皇政府准许的报刊上发表文章,因此被称为“合法马克思主义者”。

他们按自己的方式同民粹主义进行斗争,但他们是想利用这个斗争和马克思主义旗帜来使工人运动服从和适应资产阶级社会的利益,资产阶级的利益。他们抛弃了马克思学说中最主要的东西,即关于无产阶级革命和无产阶级专政的学说。最著名的合法马克思主义者彼得·司徒卢威竭力赞美资产阶级,并号召大家“承认我们不文明,去向资本主义学习”,而不是去进行反资本主义的革命斗争。

列宁认为在反民粹派的斗争中可以同“合法马克思主义者”达成暂时协议,利用他们去反对民粹派。例如共同出版过一本反民粹派的文集。但列宁同时又非常尖锐地批评了“合法马克思主义者”,揭露了他们那种自由资产阶级的本性。

后来,这些“同路人”中的许多人都成了立宪民主党(俄国资产阶级的主要政党)党员,而在国内战争时期则成了彻头彻尾的白卫分子。

除彼得堡、莫斯科,基辅等处的“斗争协会”外,在俄国西部各民族边区也有社会民主主义组织出现。九十年代,一些马克思主义分子从波兰民族主义的党里分化出来,成立了“波兰和立陶宛社会民主党”。九十年代末,有几个拉脱维亚社会民主主义组织成立起来。1897年10月,在俄国西部省份成立了犹太社会民主主义总联盟(简称崩得)。

1898年,彼得堡、莫斯科、基辅、叶加特林诺斯拉夫等几个城市的“斗争协会”和崩得,作了统一成为一个社会民主党的第一次尝试。为了这一目的,它们于1898年3月在明斯克举行了俄国社会民主工党第一次代表大会。

出席俄国社会民主工党第一次代表大会的只有九名代表。列宁没有出席这次大会,因为他当时在西伯利亚流放。大会选出的党中央委员会不久就被破获了。用大会名义发表的《宣言》,有许多地方还不能令人满意。《宣言》回避了无产阶级夺取政权的任务,根本没有提到无产阶级领导权问题,回避了无产阶级在反沙皇制度和反资产阶级斗争中的同盟者问题。

大会在决议和《宣言》中宣告了俄国社会民主工党的成立。

俄国社会民主工党第一次代表大会的意义也就在于它完成了这个正式手续,起了很大的革命宣传作用。

虽然第一次代表大会已经举行过了,但实际上马克思主义的社会民主党在俄国并没有建立起来。这次大会没有把各个马克思主义小组和组织统一起来,没有在组织上把它们联成一体。各地方组织的工作还没有统一的路线,没有党纲和党章,没有中央统一的领导。

由于这些以及其他许多原因,各地方组织中思想上的涣散开始增长起来,结果就给“经济主义”这个机会主义派别在工人运动中的加强造成了良好的条件。

只是靠了列宁和他创办的《火星报》多年紧张的工作,才战胜了这种涣散现象,克服了机会主义的动摇,为俄国社会民主工党的成立做好了准备。

\subsection[五\q 列宁反对“经济主义”的斗争。列宁《火星报》的出现]{五\\ 列宁反对“经济主义”的斗争。\\列宁《火星报》的出现}

列宁没有出席俄国社会民主工党第一次代表大会。他当时在西伯利亚流放,住在舒申斯克村,是因“斗争协会”案在彼得堡监禁很久以后被沙皇政府放逐到那里去的。

但列宁就是在流放中也还在进行革命工作。列宁在流放中写完了极其重要的科学著作《俄国资本主义的发展》,完成了从思想上粉碎民粹主义的事业,在那里他还写了《俄国社会民主主义者的任务》这本有名的小册子。

列宁虽然脱离了直接的实际的革命工作。但他仍然设法同实际工作者保持着联系,从流放地跟他们通信,向他们了解情况,给他们出主意。当时列宁特别关心“经济派”问题。他比谁都明白,“经济主义”是妥协主义、机会主义的基本组成部分;“经济主义”在工人运动中获得胜利就会使无产阶级的革命运动受到破坏,使马克思主义遭到失败。

所以“经济派”刚一露头,列宁就给以迎头痛击。

“经济派”硬说,工人只应进行经济斗争,而政治斗争应当让自由资产阶级去搞,工人应当支持自由资产阶级。列宁认为“经济派”这样宣传,就是背叛马克思主义,就是否认工人阶级需要有独立的政党,就是企图把工人阶级变为资产阶级的政治附庸。

1899年,一部分“经济派”分子(普罗科波维奇、库斯柯娃等人,他们后来都成了立宪民主党人)发表了自己的宣言。他们公开反对革命的马克思主义,要求放弃建立无产阶级的独立政党,放弃工人阶级独立的政治要求,“经济派”认为政治斗争是自由资产阶级的事情,而工人只要同厂主进行经济斗争就行了。

列宁一读到这个机会主义文件,就把附近的政治犯即被流放的马克思主义者召集起来,经过商议,以列宁为首的十七个同志提出了一个揭露“经济派”观点的激烈抗议书。

这个由列宁起草的抗议书在全俄各地马克思主义组织中间传播开来,大大促进了马克思主义思想和马克思主义政党在俄国的发展。

俄国“经济派”所宣传的观点,也就是外国社会民主党内那些反对马克思主义的所谓伯恩施坦派即机会主义者伯恩施坦的信徒们所宣传的观点。

因此,列宁对“经济派”作斗争,同时也就是对国际机会主义作斗争。

为反对“经济主义”,为建立无产阶级的独立政党而进行的斗争,基本上是列宁创办的秘密报纸《火星报》进行的。

1900年初,列宁和“斗争协会”的其他一些会员从西伯利亚流放地回到了俄罗斯。列宁立意创办一个大型的全俄的马克思主义的秘密报纸。当时在俄国已有许多规模很小的马克思主义小组和组织,但它们还没有联成一体。当时正如斯大林同志所说,“手工业方式和小组习气从上到下腐蚀着党,思想上的混乱是党内生活的特征”\footnote{见《斯大林全集》第6卷第143页。——译者注},所以创办全俄秘密报纸,是俄国革命马克思主义者的基本任务。只有这样的报纸,才能把零散的马克思主义组织联成一体,为建立真正的政党作好准备。

但由于警察的迫害,这样的报纸不能在沙俄境内出版。否则,过不了一两月就会被沙皇密探发觉而被捣毁。因此列宁决定把它拿到国外出版。报纸在国外用最薄最结实的纸张刊印,再秘密转寄到俄国。有几号《火星报》在俄国境内由巴库、基什涅夫,西伯利亚的秘密印刷所进行了翻印。

1900年秋,弗拉基米尔·伊里奇到国外去同“劳动解放社”的同志们磋商全俄政治报纸的出版问题。这个想法,列宁在流放中就已十分周详地考虑过。列宁从流放地回来的途中,曾为这个问题在乌发、普斯科夫、莫斯科、彼得堡等好几个地方开过会。每到一处,他都和同志们约定秘密通信的密码、寄送出版物的地址等等,并同他们讨论了将来斗争的计划。

沙皇政府感觉到,列宁是它最危险的敌人。沙皇的暗探,一个叫祖巴托夫的宪兵,在秘密的呈报中说,“现时在革命中再没有比乌里杨诺夫更重要的人物了”,因此他认为最好是把列宁刺死。

列宁到国外后,同“劳动解放社”,也就是同普列汉诺夫、阿克雪里罗得和维·查苏利奇,商定了共同出版《火星报》。整个出版计划,从头至尾都是列宁拟定的。

1900年12月,《火星报》创刊号在国外出版了。报头下面载有一句名言(题词):“星火可以燎原。”这是十二月党人\footnote{十二月党人是俄国贵族革命者,因1825年12月起义反对沙皇政府和农奴制而得名。——译者注}在西伯利亚流放地赋诗回答诗人普希金的致意时所用的一个句子。果然,列宁点燃的“火星”,后来燃成了燎原的革命烈火,把地主贵族的沙皇君主制和资产阶级的政权烧成了灰烬。

\subsection{简短的结论}

俄国马克思主义社会民主工党,首先是在反对民粹主义,反对民粹主义那些有害于革命事业的错误观点的斗争中建立起来的。

只有从思想上彻底批倒民粹派观点,才能为建立俄国马克思主义工人政党扫清基地。在十九世纪八十年代,普列汉诺夫及其“劳动解放社”给了民粹主义一个有决定意义的打击。

列宁在九十年代完成了从思想上粉碎民粹主义的事业。彻底打垮了民粹主义。

1883年成立的“劳动解放社”,为在俄国传播马克思主义作了大量的工作,在理论上为社会民主党奠定了基础,跨出了走向工人运动的第一步。

随着俄国资本主义的发展,工业无产阶级的人数迅速增长起来。在八十年代中期,工人阶级已走上有组织的斗争的道路,走上有组织的罢工这种群众性发动的道路。但是马克思主义小组和团体还只从事于宣传,不了解必须转到工人阶级中去进行群众鼓动,因此它们也就没有在实践中同工人运动联系起来,没有领导工人运动。

列宁组织的彼得堡“工人阶级解放斗争协会”(1895年)在工人中进行了群众鼓动,领导了群众罢工,因而开辟了一个新的阶段,即转到在工人中进行群众鼓动,把马克思主义同工人运动结合起来的阶段。彼得堡“工人阶级解放斗争协会”是俄国无产阶级革命政党的第一个萌芽。继彼得堡“斗争协会”之后,各大工业中心和各个边沿地区都相继成立了马克思主义者的组织。

1898年进行了第一次(虽然没有成功)把各个马克思主义的社会民主主义组织统一成为一个党的尝试,即举行了俄国社会民主工党第一次代表大会。但这次大会还没有把党建立起来,因为既没有制定党纲和党章,也没有造成一个中央统一的领导,各个马克思主义小组和团体之间几乎没有任何联系。

为了把零散的马克思主义组织统一起来、联成一体而成为一个党,列宁提出并且实现了创办革命马克思主义者第一个全俄报纸《火星报》的计划。

“经济派”是这个时期建立统一的工人政党的主要敌人。他们否认有必要成立这样一个党,赞成各个团体保持零散状态和手工业方式。列宁和他创办的《火星报》就是对准他们开火的。

《火星报》最初几号的出版(1900—1901年),意味着转向一个新的时期,即把一些零散的团体和小组真正组成为一个统一的俄国社会民主工党的时期。

