\section[第八章\q 布尔什维克党在外国武装干涉和国内战争时期(1918—1920年)]{第八章\\ 布尔什维克党在外国武装干涉和国内战争时期 \\{\zihao{3}(1918—1920年)}}

\subsection[一\q 外国武装干涉的开始。国内战争的第一个时期]{一\\ 外国武装干涉的开始。\\国内战争的第一个时期}

苏维埃政权缔结布列斯特和约并通过一系列经济方面的革命措施而得到巩固(当时西方正在酣战),引起了西方帝国主义者特别是协约国帝国主义者极大的惊慌。

协约国帝国主义者担心,德俄两国缔结和约会有利于德国的作战地位,而相应地增加协约国前线军队的困难。其次,他们担心,俄德之间建立和平会加强各国和各战场要求和平的趋向,从而破坏战争事业,破坏帝国主义者的事业。最后,他们担心,苏维埃政权在一个幅员辽阔的国家里的存在以及它在推翻那里的资产阶级政权后在国内所取得的成就,会使西方工人和士兵受到传染,因为西方工人和士兵对旷日持久的战争深感不满,可能效法俄国人,掉转枪口反对本国的统治者和压迫者。因此,各协约国政府决定开始对俄国进行武装干涉(干预),以推翻苏维埃政权而成立资产阶级政权,认为这样一来就能恢复俄国的资产阶级秩序,取消对德和约,恢复对德奥的作战。

协约国帝国主义者所以乐意干这种黑暗勾当,还因为他们深信苏维埃政权不稳固,以为只要苏维埃政权的敌人稍微努一把力,苏维埃政权必然很快灭亡。

苏维埃政权的成就和巩固,引起了被推翻的阶级(地主阶级和资本家阶级)。被打倒的党派(立宪民主党、孟什维克、社会革命党、无政府主义者及各种资产阶级民族主义者)、白卫将军和哥萨克军官等等更大的惊慌。

所有这些敌对分子,从十月革命胜利的最初几天起,就到处叫嚷:苏维埃政权在俄国没有根基,它注定要失败,它经过一两个星期、一个月、至多两三个月就会灭亡。但是由于苏维埃政权在敌人诅咒下继续存在而且不断巩固,俄国内部苏维埃政权的敌人不得不承认:苏维埃政权要比他们先前想象的强大得多;要推翻苏维埃政权,需要一切反革命势力作很大的努力,来一场激战。因此,他们决定广泛地进行反革命叛乱工作来纠集反革命力量,搜罗军事干部,组织叛乱,首先是在哥萨克和富农聚居的地区组织叛乱。

由此可见,还在1918年上半年就已经形成了两股准备推翻苏维埃政权的明显势力:协约国外国帝国主义者和俄国内部的反革命势力。

这两股势力中的任何一股势力,都没有足够的力量单独推翻苏维埃政权。俄国的反革命势力虽然有反苏维埃政权的暴动所必需的相当的军事干部以及相当的人力——主要是哥萨克上层分子和富农,但没有金钱和武器。反之,外国帝国主义者有金钱和武器,但不可能“抽出”充分的兵力来进行武装干涉,这不仅因为这些力量必须用于对德奥作战,而且因为这些力量用于反苏维埃政权会不十分可靠。

反苏维埃政权的斗争条件迫使国内外这两股反苏势力联合起来。而这种联合就在1918年上半年形成了。

这样,就形成了以俄国内部苏维埃政权敌人的反革命叛乱为内应的外国对苏维埃政权的武装干涉。

这样,喘息时机结束了,俄国的国内战争,即俄国各民族的工人和农民反对苏维埃政权内外敌人的战争开始了。

英法日美帝国主义者没有宣战就发动了武装干涉,虽然这种武装干涉是对俄国的战争,并是一种最坏的战争。这些“文明”强盗不声不响地偷偷摸到俄国边境,驱使自己的军队在俄国登陆。

英国人和法国人驱使军队在俄国北部登陆,占领了阿尔汉格尔斯克和牟尔曼新克,支持当地的白卫叛乱,推翻了苏维埃政权,成立了白卫的“俄国北方政府”。

日本人驱使军队在海参崴登陆,夺取了沿海边区,解散了苏维埃,支持白卫叛乱分子,使这些叛乱分子得以在后来恢复资产阶级秩序。

在北高加索,科尔尼洛夫、阿列克谢也夫和邓尼金这几个将军在英法支持下组织了白卫“志愿军”,发动了哥萨克上层分子的叛乱,开始向苏维埃大举进攻。

在顿河一带,克拉斯诺夫和马蒙托夫两个将军在德帝国主义者秘密支持下(德国人不敢公开支持他们,因为同俄国订有和约)发动了顿河哥萨克的叛乱,占领了顿河区,开始向苏维埃大举进攻。

在伏尔加河中游和西伯利亚,由于英国人和法国人的策动,造成了捷克斯洛伐克军的叛乱。这个军是战俘组成的,苏维埃政府允许他们经过西伯利亚和远东开回本国,但是中途他们受社会革命党人和英国人法国人的利用而举行了反苏维埃政权的叛乱。这个军的叛乱成了一个信号,伏尔加河流域和西伯利亚的富农、沃特金斯克工厂和伊热夫斯克工厂同情社会革命党的工人跟着也举行叛乱。伏尔加河流域成立了萨马拉白卫-社会革命党人政府。鄂木斯克成立了西伯利亚白卫政府。

德国没有参加而且也不可能参加英法日美联盟所进行的这次武装干涉,因为——至少是因为——它同这个联盟还处于交战状态。但是,虽然如此,虽然订有俄德和约,每个布尔什维克都深知威廉皇帝的德国政府也如英法日美干涉者一样,是苏维埃国家的凶恶敌人。而德帝国主义者也确实是在竭尽一切努力来孤立、削弱和消灭苏维埃国家。他们从苏维埃俄国(诚然是按他们和乌克兰拉达\footnote{乌克兰拉达即乌克兰反革命的资产阶级民族主义政府,1917年成立,1918年被推翻。——译者注}订立的“条约”)夺去了乌克兰,应乌克兰白卫拉达之请派兵进驻乌克兰,残暴地掠夺和压迫乌克兰人民,禁止他们同苏维埃俄国保持任何联系。他们从苏维埃俄国夺去了南高加索,应格鲁吉亚和阿捷尔拜疆的民族主义者之请派去了德国和土耳其的军队,在梯弗里斯和巴库横行霸道。他们千方百计(虽然是暗地里)用武器和粮食援助在顿河区进行叛乱的克拉斯诺夫将军反对苏维埃政权。

这样,苏维埃俄国同自己主要的粮食、原料和燃料产区的联系被切断了。

苏维埃俄国在这一时期很困难。面包不够。肉类不够。饥饿折磨着工人。莫斯科和彼得格勒的工人每人每两天只能得到八分之一磅的面包。甚至还有根本得不到面包的时候。工厂因为缺乏原料和燃料而停工或几乎停工。但是工人阶级并没有灰心丧气。布尔什维克党并没有灰心丧气。这一时期所遭到的极大困难和为克服困难而进行的殊死斗争,证明工人阶级中蕴藏着无穷无尽的力量,证明布尔什维克党的威信具有极其巨大和无法估计的力量。

党宣布全国为军营,并把全国的经济生活和文化政治生活转入战时轨道。苏维埃政府宣布“社会主义祖国在危急中”,号召人民进行抗战。列宁提出“一切为了前线”的口号,几十万工人和农民志愿加入红军、奔赴前线。党员和共青团员约有一半上了前线。党发动人民起来进行卫国战争,抗击外国武装干涉军的侵犯,消灭被革命推翻了的剥削阶级的叛乱。列宁组织的工农国防委员会负责为前线供应人员、粮食、服装和武器。志愿兵改为义务兵,使红军得到几十万新兵的补充;红军在短时期内就成了一支百万大军。

尽管国内状况十分困难,尽管红军很年轻、还来不及巩固,但由于采取种种防御措施,已经取得了初步的成就。克拉斯诺夫将军已经被撵出他认为保证能固守的察里津,而且被赶到顿河区以外。邓尼金将军的行动被限制在北高加索的狭小地区,科尔尼洛夫将军则在与红军交战时被打死。捷克斯洛伐克军和社会革命党白卫匪帮被赶出喀山、辛比尔斯克和萨马拉,并被压迫到乌拉尔。英国驻莫斯科使团团长洛卡尔特策动白卫分子萨文可夫在雅罗斯拉夫里进行的叛乱已被粉碎,洛卡尔特被捕。刺杀了乌里茨基、沃洛达尔斯基同志并恶毒地谋害过列宁的社会革命党人,因搞反布尔什维克的白色恐怖而受到红色恐怖的惩治,在俄国中部一切较为重要的地点都已被打垮。

年轻的红军在同敌人的战斗中锻炼成长起来。

当时在红军中任政治委员的共产党员,在巩固红军,进行政治教育、加强战斗力和纪律方面起了决定的作用。

布尔什维克党知道,红军的这些成就还不能解决问题,还仅仅是它的初步成就。党知道,前面还有新的更加严重的战斗,只有同敌人进行长期的严重的斗争之后,国家才有可能收复失去的粮食、原料和燃料产区。因此,布尔什维克加紧进行长期战争的准备,决定使整个后方都来为前线服务。苏维埃政府实行了战时共产主义。苏维埃政权除对大工业实行监督外,对中小工业也实行监督,以便积蓄日用品供应军队和农村。它实行了粮食贸易垄断制,禁止私人买卖粮食,规定了余粮收集制,以便掌握农民的余粮数字、搞好粮食储备、向军队和工人供应粮食。最后,它实行了遍及于一切阶级的劳动义务制。党迫使资产阶级参加强制的体力劳动,从而腾出工人去从事其他的对前线更为重要的工作,这样就实现了“不劳动者不得食”的原则。

为了适应极其困难的防御条件而暂时采取的这一整套措施,就叫做战时共产主义。

国家准备同苏维埃政权的内外敌人进行长期的严重的国内战争。它应当使军队人数到1918年底增加两倍。它应当积蓄供应这个军队的物资。

列宁当时指出:

\begin{quotation}
“我们原来决定到春天建立一支一百万人的军队,现在我们需要三百万人的军队了。我们能够有这样多的军队。我们一定会有这样多的军队。”\footnote{见《列宁全集》第28卷第87页。——译者注}
\end{quotation}


\subsection[二\q 德国的军事失败。德国的革命。第三国际的成立。党的第八次代表大会]{二\\德国的军事失败。德国的革命。\\第三国际的成立。党的第八次代表大会}

正当苏维埃国家准备对外国武装干涉进行新的战斗的时候,在西方,在各交战国的后方和前线发生了具有决定意义的事变。当时德国和奥地利被战争和粮荒压得喘不过气来。英、法和北美还能不断挖掘新的潜力,而德国和奥地利连最后一点潜力也快耗尽了。精疲力竭到了极点的德国和奥地利,眼看很快就要失败。

同时在德奥两国内部,民怨沸腾,怨恨战争无休无止、招致灭亡,怨恨两国帝国主义政府把人民弄到精疲力竭和饥饿的境地。十月革命的伟大革命影响,布列斯特和约前就发生过的苏维埃士兵同奥德士兵在前线的联欢,以及后来同苏维埃俄国停战媾和本身的影响,在这里也起了作用。俄国人民通过推翻本国帝国主义政府而结束了可恨的战争,这个实例不能不给奥德工人以教育。而德军方面那些原先在东线,到布列斯特和约签订后又调到西线的士兵,通过讲述他们先前怎样同苏维埃士兵联欢以及苏维埃士兵怎样摆脱了战争,也不能不在西线造成军心的瓦解。至于奥地利军队,还要更早一些,就由于同样的原因而开始瓦解了。

由于这一切情况,德军中要求和平的趋向加强了,他们已经没有先前那样的战斗力了,他们开始在协约国军进攻之下节节败退;而在德国本国,于1918年11月爆发了革命,推翻了威廉及其政府。

德国不得不承认战败,并向协约国求和。

这样,德国这个头等强国一下子就降到二等强国的地位。

从苏维埃政权的地位来看,这种情况有某些消极的作用,因为它使对苏维埃政权进行武装干涉的协约国变成了欧亚两洲的统治力量,使它们有可能对苏维埃国家加强武装干涉和实行封锁,加紧围困苏维埃政权。结果正是如此,这一点我们往下就可看到。但是从另一方面来看,这种情况还有更为重大的积极的作用,即从根本上缓和苏维埃国家处境的作用。第一,苏维埃政权有了可能废除掠夺性的布列斯特和约,停付赔款和进行公开的斗争(军事的和政治的)来把爱沙尼亚、拉脱维、白俄罗斯、立陶宛、乌克兰和南高加索从德帝国主义的压迫下解放出来。第二(这也是主要的),在欧洲的心脏德国存在共和制度和工兵代表苏维埃,必然会使欧洲各国革命化,而且也确实使他们革命化了,这就不能不巩固俄国苏维埃政权的地位。当然德国的革命是资产阶级革命而不是社会主义革命,而苏维埃是资产阶级议会的驯服工具,因为在苏维埃中占统治地位的社会民主党人是俄国孟什维克那样的妥协派。因为德国革命是软弱无力的。德国白卫分子可以任意杀害罗·卢森堡和卡·李卜克内西这样著名的革命家而不受到制裁,仅从这一件事也可以看出德国的革命软弱无力到何等地步。但它终究是一场革命。威廉被推翻了,工人挣脱了锁链,单是这一点就不能不发动西方的革命,不能不引起欧洲国家革命的高潮。

欧洲革命高潮开始了。奥地利的革命运动展开了。匈牙利成立了苏维埃共和国。欧洲各国共产党在革命浪潮基础上出现了。

现在有了把各国共产党统一为第三国际即共产国际的现实基础。

1919年3月,在莫斯科各国共产党第一次代表大会上,由列宁和布尔什维克发起成立了共产国际。虽然帝国主义者的封锁和迫害阻挠了许多代表来莫斯科,但是欧美各重要国家的代表还是参加了第一次代表大会。列宁领导了大会的工作。

列宁在关于资产阶级民主和无产阶级专政的报告中,说明了苏维埃政权的意义,指出它是真正的劳动者的民主。大会通过了告国际无产阶级的宣言,号召他们为在各国实现无产阶级专政、为苏维埃在各国胜利而坚决奋斗。

大会成立了共产国际执行委员会,作为第三国际即共产国际的执行机关。

这样,就成立了新型的国际无产阶级革命组织——共产国际,即马克思列宁主义的国际。

1919年3月,一方面协约国反动联盟加紧反对苏维埃政权,另一方面欧洲(主要是战败国)的革命高潮大大缓和了苏维埃国家的处境,在这样一种矛盾情况下,我们党召开了第八次代表大会。

出席这次大会的有三百零一名有表决权的代表,代表着三十一万三千七百六十六名党员。有发言权的代表有一百零二人。

列宁在会上致开幕词,第一句话就是悼念大会开幕前夜逝世的布尔什维克党优秀组织者之一雅·米·斯维尔德洛夫同志。

大会通过了新党纲。党纲说明了资本主义及其最高阶段即帝国主义的特征,党纲对比了两种国家制度,即资产阶级民主制度和苏维埃制度。党纲详细地指出了党在争取社会主义斗争中的具体任务;把对资产阶级的剥夺进行到底,按照统一的社会主义计划管理全国经济,使工会参加组织国民经济的工作,实行社会主义劳动纪律,在国民经济中由苏维埃机关监督利用专家,逐渐地有计划地吸收中农参加社会主义建设。

大会采纳了列宁的建议:在党纲上除载明帝国主义是资本主义的最高阶段这一定义外,还把党的第二次代表大会通过的旧党纲说明工业资本主义和简单商品经济的那一部分写进去。列宁认为,在党纲上必须估计到我国经济的复杂情况,指出国内存在着几种不同的经济成分,其中包括中农所代表的小商品经济。因此,在讨论党纲时,列宁坚决反对布哈林的反布尔什维主义观点,因为布哈林建议把关于资本主义、关于小商品生产、关于中农经济的条文从党纲上删去。布哈林的这种观点,就是孟什维克和托洛茨基的否认中农在苏维埃建设中的作用的观点。同时,布哈林又抹杀富农分子正从小农商品经济中产生和滋长的事实。

列宁还驳斥了布哈林和皮达可夫在民族问题上的反布尔什维主义观点。他们反对把关于民族自决权的条文写进党纲,反对民族平等,借口是这个口号会妨碍无产阶级革命的胜利,妨碍各民族无产者的联合。列宁驳倒了布哈林和皮达可夫这种极其有害的大国沙文主义观点。

关于对中农的态度问题,在党的第八次代表大会工作中占着重要的地位。由于实现了著名的土地法令,农村愈来愈中农化了,现在中农已在农村居民中占多数。中农动摇于资产阶级和无产阶级之间,他们的情绪和态度对于国内战争和社会主义建设的命运至关重要。国内战争的结局如何,在许多方面取决于中农倒向哪一边,取决于哪一个阶级(是无产阶级还是资产阶级)能吸引中农跟自己走。捷克斯洛伐克军、白卫分子、富农、社会革命党人、孟什维克1918年夏在伏尔加河流域所以能推翻苏维埃政权,就是因为相当一部分中农支持了他们。富农在俄国中部举行叛乱时情形也是如此。但是从1918年秋天起,中农群众在情绪上开始转向苏维埃政权方面。农民已经看到,白军胜利的结果就是:地土重新掌权,农民土地被夺,农民遭受掠夺,鞭笞和折磨。贫农委员会击败富农,也促进了农民情绪的转变。因此列宁在1918年11月提出了如下的口号:

\begin{quotation}
“善于同中农达成协议。一分钟也不放弃对富农的斗争,只是牢牢地依靠贫农。”(《列宁全集》俄文第3版第23卷第294页)\footnote{见《列宁选集》第2版第3卷第612页。——译者注}
\end{quotation}

当然,中农还没有完全停止动摇,但是他们比过去更接近苏维埃政权,更牢靠地支持苏维埃政权了。党的第八次代表大会所规定的对中农的敢策,大大促进了这点。

第八次代表大会是党对中农政策的一个转折点。列宁的报告和大会的决议,决定了党在这个问题上的新路线。大会要求党的组织和全体共产党员严格地把中农与富农区别开、划分开,通过关心中农的需要把中农吸引到工人阶级方面来。必须用说服的方法,而绝不要用强制、暴力去克服中农的落后性。因此,大会指示,在农村中实行社会主义措施(成立公社和农业劳动组合)时,不允许采取强制手段。凡涉及中农切身利益的场合,都要同他们达成实际的协议,例如在确定社会主义改造的方式上要对中农让步。大会提出实行这样的政策:在保持无产阶级领导作用的情况下,同中农结成巩固的联盟。

列宁在第八次代表大会上所宣布的对中农的新政策,要求无产阶级依靠贫农、同中农保持巩固的联盟、对富农作斗争。在第八次代表大会以前,党大体上是实行中立中农的政策。这就是说,党争取中农不站到富农方面去,不站到一般资产阶级方面去。但是现在这已经不够了。第八次代表大会从中立中农的政策转到同中农结成巩固的联盟,以反对白卫匪帮和外国武装干涉,并顺利进行社会主义建设。

大会对基本农民群众即中农所采取的路线,对于胜利地结束反对外国武装干涉及其白卫走狗的国内战争,起了决定的作用。1919年秋,正当需要在苏维埃政权和邓尼金之间做出选择的时候,农民支持了苏维埃,无产阶级专政就战胜了自己最危险的敌人。

关于红军的建设问题在会上占有特殊地位。会上出现了所谓“军事反对派”。它联合了不少先前的“左派共产主义者”。但是,除了已被打垮的“左派共产主义”集团的代表人物,参加“军事反对派”的,还有从未参加任何反对派但对托洛茨基在军队中的领导表示不满的工作者。大多数军人代表都激烈反对托洛茨基,反对他崇拜来自旧沙皇军队的军事专家(其中有一部分在国内战争时期直接背叛了我们),反对他对军队中的老布尔什维克干部采取傲慢和敌视的态度。会上举出的许多“来自实践”的例子证明,托洛茨基曾企图枪毙许多他所不喜欢的在前线担任军事领导职务的共产党员,借以帮助敌人;只是由于中央的干涉和军事工作人员的抗议,这些同志才幸免于死。

“军事反对派”虽然反对托洛茨基对党的军事政策的歪曲,但是在军事建设的许多问题上维护不正确的观点。列宁和斯大林坚决地反对了“军事反对派”,因为这个派别维护军队中的游击主义残余,反对建立正规红军,反对利用军事专家,反对铁的纪律——没有铁的纪律军队就不可能成为真正的军队。斯大林同志在反驳“军事反对派”时要求建立一支纪律非常严格的正规军。

\begin{quotation}
斯大林同志说“或者我们建立起一支有严格纪律的真正工农的、主要是农民的军队而保卫住共和国,或者我们遭到灭亡。”\footnote{参看《斯大林全集》第4卷第222页。——译者注}
\end{quotation}

大会否决了“军事反对派”的一系列提案,同时给了托洛茨基以打击,要求改进中央军事机关的工作,要求加强共产党员在军队中的作用。

由于大会成立的军事小组进行了工作.大会关于军事问题的决议获得一致通过。

大会关于军事问题的决议,使红军得到了加强,使它跟党进一步接近了。

其次,大会讨论了党和苏维埃的建设问题,即党在苏维埃工作中的领导作用问题。大会在讨论这个问题时,回击了萨普龙诺夫-奥新斯基的机会主义集团,因为他们否认党在苏维埃工作中的领导作用。

最后,由于新党员大批涌入党内,大会通过了关于改善党的社会成分和重新进行登记的决议。

这是第一次清党的开始。


\subsection[三\q 武装干涉的加剧。苏维埃国家被封锁。高尔察克的进攻及其被粉碎。邓尼金的进攻及其被粉碎。三个月的喘息时机。党的第九次代表大会]{三\\武装干涉的加剧。苏维埃国家被封锁。\\高尔察克的进攻及其被粉碎。\\邓尼金的进攻及其被粉碎。\\三个月的喘息时机。党的第九次代表大会}

协约国把德国和奥地利打败之后,决定投入大批兵力来反对苏维埃国家。在德国战败,德军退出乌克兰和南高加索后,英法两国就取代了德国,把自己的军舰开进了黑海,派兵在敖德萨和南高加索登陆。协约国武装干涉者在占领区横行霸道,残暴到竟然整批整批地屠杀工农。到后来占领土尔克斯坦时,武装干涉者甚至猖狂到将邵武勉、菲奥列托夫、查帕里泽,马里根、阿集兹别科夫、柯尔加港夫等二十六个巴库布尔什维克领导同志解到里海东岸,在社会革命党人协助下把他们残暴地枪杀了。

不久,武装干涉者宣布对俄国实行封锁。所有与外界来往的海上的和其他的通道,都被切断了。

于是,苏维埃国家陷入了几乎四面受围的境地。

当时协约国把主要希望寄托在他们在西伯利亚鄂木斯克的傀儡海军上将高尔察克身上。高尔察克被宣布为“俄国最高执政”。俄国一切反革命势力都听命于他。

于是,东线成了主要战线。

1919年春,高尔察克纠集了一支庞大的军队,差不多推进到伏尔加河畔。为反对高尔察克派去了布尔什维克的优秀力量,动员了共青团员和工人。1919年4月,红军大败高尔察克。不久,高尔察克的军队开始全线退却。

正当红军在东线的攻势达到高潮的时候,托洛茨基提出一个可疑的计划:在乌拉尔停下来,对高尔察克军队停止追击,把军队从东线调往南线。党中央明白,不能让乌拉尔和西伯利亚留在高尔察克手中,不能让他在法国日本和英国人的帮助下恢复元气和重新站稳脚跟,并由此否定了这个计划,指示继续进攻。托洛茨基不同意这一指示,提出辞职。中央拒绝了托洛茨基的辞职,同时责令他立刻停止参加对东线战事的领导。红军开始更猛烈地向高尔察克展开进攻,使高尔察克遭到了一连串新的失败,并在白军后方的强大游击运动支持下从叛军手里解放了乌拉尔和西伯利亚。

1919年夏,帝国主义者责成领导西北方面(波罗的海沿岸,彼得格勒附近)反革命势力的尤登尼奇将军向彼得格勒进攻,借以转移红军对东线的注意。彼得格勒附近两个炮台的守备部队受旧军官的反革命煽动,发动了反苏维埃政权的叛乱,而在前线司令部中又发现了反革命叛乱。彼得格勒岌岌可危。但是由于苏维埃政权采取了措施,在工人和水兵的支持下,从白军手中解放了两个叛乱的炮台,打败了尤登尼奇的军队,把尤登尼奇驱逐到爱沙尼亚去了。

尤登尼奇在彼得格勒附近的失败,有利于反高尔察克的斗争。1919年底,高尔察克军队被彻底击溃。高尔察克本人被俘,根据革命委员会的判决在伊尔库茨克被枪决。

这样高尔察克就完结了。

当时,西伯利亚民间流传着一首嘲讽高尔察克的歌谣:

\begin{quotation}
“英国的军装,

法国的肩章,

日本的烟叶,

鄂木斯克的执政王。

军装穿破了,

肩章脱落了,

烟叶吸完了,

执政王不见了。”
\end{quotation}

武装干涉者看到高尔察克没有实现他们所寄托予他的希望,就改变了他们进攻苏维埃共和国的计划。敖德萨的陆战队不得不撤回,因为武装干涉者的军队同苏维埃共和国军队接触后受到了革命精神的感染,开始起来反对自己的帝国主义统治者。例如,敖德萨法国士兵举行了起义。因此现在,在高尔察克被击溃以后,协约国把主要注意力转向科尔尼洛夫的同僚和“志愿军”的组织者邓尼金将军身上。当时邓尼金正在南俄库班地区进行反苏维埃政权的勾当。协约国供应邓尼金军队大批武器装备,驱使他北上去反对苏维埃政权。

于是,南线这次成了主要战线。

1919年夏,邓尼金开始了他对苏维埃政权的大进军,托洛茨基把南线工作搞得一塌糊涂,使我军接二连三遭到失败。到10月中旬,白军控制了整个乌克兰,攻占了奥勒尔,逼近到供应我军子弹、步枪和机关枪的土拉。白军向莫斯科进逼。苏维埃共和国的形势非常危急。党敲响警钟,号召人民奋起抵抗。列宁提出了“大家都去同邓尼金作斗争”的口号。在布尔什维克鼓舞下,工人和农民集中了全力来歼灭敌人。

为了组织好歼灭邓尼金的战事,中央把斯大林、伏罗希洛夫、奥尔忠尼启泽、布琼尼四同志派往南线。托洛茨基被撤销领导南线红军作战的职权。在斯大林同志来到之前,南线指挥部同托洛茨基一起曾制定了一个计划,准备从察里津经顿河草原前往诺沃罗两斯克,对邓尼金施行主要突击,但红军在顿河草原会遇到完全没有道路的地带,并且还要经过当时很大一部分居民还受白卫影响的哥萨克地区。斯大林同志尖锐地批评了这个计划,并向中央提出了自己的歼灭邓尼金的计划:取道哈尔科夫—顿巴斯—罗斯托夫对邓尼金实行主要突击。这个计划保证我军能迅速前进去攻打邓尼金,因为在我们工作过的工农地区,人心显然是归向我们的。

此外,这个地区有稠密的铁路网,使我军有可能按时获得一切必需品的供给。最后,实行这个计划就能解放顿巴斯,保证我国燃料的供给。

党中央采纳了斯大林同志的计划。1919年10月下半月,邓尼金经过激烈的抵抗以后,在奥勒尔附近和沃龙涅什附近的两次决战中被红军击败。邓尼金开始迅速退却,随后在我军追击下向南逃窜。1920年初,整个乌克兰和北高加索都从白军手中解放了。

在南线两次决战的时候,帝国主义者又把尤登尼奇军调来进攻彼得格勒,以牵制我南线兵力,缓和邓尼金军队的处境。白军进抵彼得格勒城下。彼得格勒英勇的无产阶级挺身捍卫第一个革命之城。和往常一样,共产党员战斗在最前面。经过多次激烈的战斗,白军被击败,重新被逐出我国国境,被赶到爱沙尼亚去了。

这样,邓尼金完结了。

高尔察克和邓尼金被击溃以后,出现了一个短暂的喘息时机。

帝国主义者看到白卫军队被击败,武装干涉不成功,苏维埃政权在全国巩固起来,而在西欧,武装干涉者对苏维埃共和国的战争又使工人的反战情绪不断高涨,于是他们开始改变自己对苏维埃国家的态度。1920年月,英法意三国决定停止对苏维埃俄国的封锁。

这是武装干涉墙壁上打开的一大缺口。

当然这并不是说苏维埃国家已消灭了武装干涉和结束了国内战争。帝国主义对波兰进犯的危险依然存在。远东,南高加索和克里术的武装干涉者还没有被彻底赶走。但是苏维埃国家获得了暂时的喘息时机,可以集中更多的力量进行经济建设。党有了可能来处理经济问题。

国内战争时期,许多熟练工人因为工厂关闭而离开了生产。

现在党动员熟练工人回到生产岗位,从事本行工作。几千名共产党员被派去恢复情况严重的运输业。如果不恢复运输,就无法切实地恢复基本工业部门。粮食耕作加强了和改进了。开始制定俄罗斯电气化计划。近五百万现役红军战士由于存在战争危险,暂时还不能复员。因此一些红军部队被改编为劳动军,用来搞经济建设。工农国防委员会改成了劳动国防委员会。为了协助它工作,成立了国家计划委员会。

在这种形势下,1920年3月底召开了党的第九次代表大会。

出席这次大会的有五百五十多名有表决权的代表,代表着六十一万一千九百七十八名党员。有发言权的代表有一百六十二人。

大会规定了国家在运输业和工业方面当前的经济任务,并特别指出工会必须参加经济建设。

大会对统一的经济计划问题特别注意,这个计划要求首先发展运输业、燃料业和冶金业。关于整个国民经济电气化的问题在这个计划中占有主要的地位,这是列宁提出的“10—20年的伟大纲领\footnote{见《列宁全集》第35卷第434页。——译者注}。后来在这个基础上产生了俄罗斯国家电气化委员会的那个著名的计划,现在它已远远超额完成了。

大会回击了反党的“民主集中派”集团,因为这个集团反对在工业中实行工长制和厂长个人负责制,而坚持在工业领导方面事事都实行“集体领导”,完全不顾专人职责。这个反党集团中的主要角色是萨普龙诺夫、奥新斯基和弗·斯米尔诺夫。在代表大会上支持他们的有李可夫和托姆斯基。


\subsection[四\q 波兰地主对苏维埃国家的进攻。弗兰格尔将军的袭击。波兰计划的破产。弗兰格尔的溃败。武装干涉的结束]{四\\波兰地主对苏维埃国家的进攻。\\弗兰格尔将军的袭击。波兰计划的破产。\\弗兰格尔的溃败。武装干涉的结束}

虽然高尔察克和邓尼金被击溃,虽然苏维埃国家不断扩大自己的领土,从白军手中解放了北部边区、土尔克斯坦、西伯利亚、顿河区、乌克兰等等,虽然协约国被迫取消了对俄国的封锁,但是协约国仍然不甘心承认苏维埃政权坚不可摧,苏维埃政权不可战胜。因此,他们决定再作一次武装干涉苏维埃国家的尝试。这次他们决定,一方面利用反革命资产阶级民族主义者,波兰国家的实际首脑皮尔苏茨基,另一方面还利用在克里木收集了邓尼金的残兵败将并从那里威胁顿巴斯和乌克兰的弗兰格尔将军。

按照列宁的说法,地主波兰和弗兰格尔是国际帝国主义企图用来掐死苏维埃国家的两只手。

波兰人的计划是,占领苏维埃乌克兰的河西地区\footnote{指乌克兰境内德涅泊河以西地区。——译者注},占领苏维埃白俄罗斯,在这两个地区恢复波兰地主政权,把波兰国界扩大到“由一海到另一海”,即由但泽到敖德萨,并且帮助弗兰格尔击败红军,在苏维埃俄国恢复地主资本家政权,以报答弗兰格尔对他们的帮助。

这个计划得到了各协约国的赞同。

苏维埃政府为了保持和平、防止战争,试图同波兰进行谈判,但没有获得任何结果。皮尔苏茨基无意谈和,只想打仗。刚刚同高尔察克和邓尼金作战的红军打得疲惫不堪,已经疲于应付波兰军队的进攻。

短暂的喘息时机到来了。

1920年4月,波兰军队入侵苏维埃乌克兰,占领了基辅。同时,弗兰格尔也转入进攻,开始威胁顿巴斯。

作为对波兰进攻的回答,红军部队展开了全线反攻。南线红军部队解放了基辅,把波兰地主赶出了乌克兰和白俄罗斯之后,一鼓作气,一直打到加里西亚的里沃夫城下,而西线红军部队则逼近了华沙。波兰地主军队的彻底失败,已经指日可待了。

还是托洛茨基及其在红军总司令部中的拥护者的可疑行动,破坏了红军的成功。由于托洛茨基和图哈切夫斯基的过错,西线红军部队向华沙方面的进攻毫无组织,没有让部队巩固已占领的阵地,先头部队前进得越迅速,预备队和弹药越落在后方,先头部队由于没有弹药,没有预备队,战线过长而易于被突破。由于这一切,当波军一个不大的集团突破我西线一点的时候,缺乏弹药的我军不得不实行退却。至于进抵里沃夫城下,压迫着该处波军的我南线部队,“革命军事委员会主席”托洛茨基却禁止他们攻占里沃夫,并命令他们把骑兵集团军即南线主力远近地调往东北,似乎是去援助西线,虽然不难明白,攻占里沃夫是对西线唯一可能的和最好的援助。但是骑兵集团军退出南线和离开里沃夫,实际上意味着我军在南线也实行退却。这样,就是托洛茨基自毁长城的命令迫使我南线部队实行莫名其妙的、毫无根据的退却,而使波兰地主兴高采烈。

这是直接的援助,但不是对我西线的援助,而是对波兰地主和协约国的援助。

过了几天,波兰军队的进攻被阻止住了,我军开始准备向波军发起新的反击。但是波军无力再战,又怕红军反击,不得不放弃占领乌克兰的河西地区和白俄罗斯的基辅而愿意同苏维埃共和国缔结和约。1920年10月20日,在里加同波兰缔结了和约,根据和约波兰保留了加里西亚以及靠近俄罗斯的部分。

苏维埃共和国同波兰缔结和约后,决定消灭弗兰格尔。弗兰格尔从英法方面得到了最新式的武器:装甲车、坦克、飞机和装具。他拥有白卫突击部队,主力是军官部队。但弗兰格尔没能给也在库班和顿河区登陆的部队纠集起多少可观的农民和哥萨克力量。然而弗兰格尔推进到了顿巴斯附近,威胁着我们的产煤区。苏维埃政权之所以处于很困难的境地,还因为红军这时已经相当疲惫。当时红军战士必须在空前困难的条件下推进,既要攻打弗兰格尔的军队,又要消灭援助弗兰格尔的马赫诺尤政府主义匪帮。但是,虽然弗兰格尔拥有技术上的优势,虽然红军根本没有坦克,红军还是把弗兰格尔驱逐到克里木半岛上。1920年11月,红军攻占了皮列柯普筑垒阵地,冲进了克里木,击溃了弗兰格尔军队,从白军和武装干涉者手中解放了克里木。克里木成为苏维埃的了。

武装干涉时期以波兰大国主义计划的破产和弗兰格尔的溃败而宣告结束。

1920年底,南高加索开始从阿塞拜疆资产阶级民族主义者木沙瓦特派、格鲁吉亚民族主义者孟什维克、阿尔明尼亚达什纳克党人的压迫下解放出来。苏维埃政权在阿塞拜疆、阿尔明尼亚和格鲁吉亚胜利了。

但这还不是说武装干涉可以完全停止。日军在远东的武装干涉一直继续到1922年。此外,还有过组织武装干涉的新尝试(在东方有哥萨克军队首领明诺夫和男爵翁格恩,在卡累里亚有芬兰白军1921年的干涉)。但是苏维埃国家的主要敌人,武装干涉的主要力量,到1920年底已经被击溃了。

外国武装干涉者和俄国自卫分子的反苏维埃战争,以苏维埃胜利而告终。

苏维埃共和国保卫住了自己的国家主权和自由生存。

这是外国武装干涉和国内战争的结束。

这是苏维埃政权的历史性胜利。


\subsection[五\q 苏维埃国家怎样和为什么战胜了英法日波武装干涉和俄国资产阶级地主白卫反革命的联合势力?]{五\\苏维埃国家怎样和为什么战胜了英法日波武装干涉\\和俄国资产阶级地主白卫反革命的联合势力?}

只要看看武装干涉时期的欧美重要报刊,马上就可以肯定:没有一个有名的著作家(军事的或非军事的)或军事专家相信,苏维埃政权会获得胜利。相反,所有有名的著作家、军事专家、研究各国和各民族革命的历史学家即所谓学者,都异口同声地叫喊,苏维埃政权的末日即将来临,苏维埃政权的失败必不可免。

他们确信武装干涉必获胜利的根据是,苏维埃国家还没有已经组织好的红军,还得临时建立,即所谓边打边建,而武装干涉者和白卫分子拥有比较现成的军队。

其次,他们的根据是,红军缺乏有经验的军事干部,因为这样的干部大多数都跑到反革命方面去了,而武装干涉者和白卫分子拥有这样的干部。

再次,他们的根据是,由于俄国军火工业落后,红军的武器弹药数量少、质量差,而从其他国家获得军用品又不可能,因为俄国被包围得水泄不通,而武装干涉者和白卫分子的军队却大批获得而且今后会继续获得头等的武器、弹药和军服的供给。

最后,他们的根据是,武装干涉者和白卫分子的军队当时占领了俄国最富饶的产粮区,而红军失去了这样的地区,痛感粮食不足。

的确,所有这些缺点和不足在红军部队中确实都有过。

在这方面,但也只是在这方面,武装干涉者老爷们是说得完全对的。

既然如此,为什么有这么多严重缺陷的红军战胜了没有这些缺陷的武装干涉者和白卫分子的军队呢?

(一)红军之所以胜利,是因为红军所捍卫的苏维埃政权的政策是符合人民利益的正确政策,人民认识到和了解到这一政策是正确的政策,是他们自己的政策,对它坚持到底。

布尔什维克知道,为不正确的、人民所不支持的政策而斗争的军队,是不可能获得胜利的。武装干涉者和白卫分子的军队就是这样的军队。武装干涉者和白卫分子的军队拥有一切:有经验的老指挥官、头等的武器、弹药、军服,粮食。但是就少一件——俄国各族人民的支持和同情,因为俄国各族人民不愿意支持也不可能支持武装干涉者和白卫“执政”的反人民政策。因此,武装干涉者和白卫分子的军队失败了。

(二)红军之所以胜利,是因为它彻底忠实于和献身于自己的人民,因此人民也爱戴它、支持它,把它看作自己的,情同骨肉的军队。红军是人民的儿子,只要它像儿子对待母亲那样忠实于自己的人民,它就会得到人民的支持,就一定会胜利。而反对本国人民的军队一定会失败。

(三)红军之所以胜利,是因为苏维埃政权发动了整个后方,整个国家来为前线的需要服务。军队如果没有坚固的,全力支持前线的后方,必然要遭到失败。布尔什维克知道这点,正因为如此,他们把全国变成了一个为前线供应武器、弹药、军服、粮食、兵员的军营。

(四)红军之所以胜利,是因为:(甲)红军战士了解战争的目的和任务,认识到了这些目的和任务的正确;(乙)由于认识到了战争的目的和任务的正确,红军战士加强了纪律性和战斗力;(丙)因此,广大红军战士在同敌人的斗争中,往往表现出无比的自我牺牲精神和空前的普遍的英雄主义。

(五)红军之所以胜利,是因为红军的后方和前线的领导核心是布尔什维克党,这是一个由于自己的团结和纪律而统一的党,一个由于具有革命精神、具有为共同事业的胜利牺牲一切的决心而强有力的党,一个善于组织千百万群众并在复杂环境中正确领导他们的出色的党。

\begin{quotation}
列宁说:“只因为党当时时刻警戒,因为党纪律严明,因为党的威信统一了各机关、各部门,使几十、几百、几千以至几百万的人都遵照中央提出的口号一致行动,只是因为忍受了前所未有的牺牲——只因为有这一切,才使目前的奇迹能够发生。只因为有这一切,我们才能在协约国帝国主义者和全世界帝国主义者两次、三次以至四次的进攻中获得了胜利。”(《列宁全集》俄文第3版第25卷第96页)\footnote{见《列宁选集》第2版第4卷第158—159页。——译者注}
\end{quotation}

(六)红军之所以胜利,是因为:(甲)它在自己的队伍中锻炼出了伏龙芝、伏罗希洛夫、布琼尼等这样一些新型的军事领导者;(乙)在它的队伍中战斗的有柯托夫斯基、哈巴也夫、拉佐、肃尔斯,帕尔霍缅柯等许多有才能的英雄;(丙)对红军进行政治教育的是列宁、斯大林、莫洛托夫、加里宁、斯维尔德洛夫、卡冈诺维奇、奥尔忠尼启泽、基洛夫、古比雪夫、米高扬、日丹诺夫、安德列也夫、彼得罗夫斯基、雅罗斯拉夫斯基、捷尔任斯基、沙金柯、美赫利斯、赫鲁晓夫、什维尔尼克、什基里亚托夫等这样一些活动家;(丁)红军有政治委员这样出色的组织者和鼓动者,他们通过自己的工作团结了红军战士的队伍,培养了战士遵守纪律和英勇作战的精神,有力地(迅速和无情地)制止了个别指挥员的叛变行为;另一方面,又大胆和坚决地维护了献身于苏维埃政权并能果断地领导红军部队的那些指挥员(不论是党员还是非党员)的威信和荣誉。

\begin{quotation}
列宁说:“没有政治委员,我们就没有红军。”\footnote{见《列宁选集》第31卷第155页。——译者注}
\end{quotation}

(七)红军之所以胜利,是因为在白卫军队的后方,即在高尔察克、邓尼金、克拉斯诺夫和弗兰格尔的后方,有出色的党员和非党员布尔什维克在进行地下工作。他们发动工农起义反对武装干涉者和白卫分子,破坏苏维埃政权的敌人的后方,从而给红军的进展创造了有利条件。大家知道,乌克兰、西伯利亚、远东、乌拉尔、白俄罗斯、伏尔加河流域的游击队,通过破坏白卫分子和武装干涉者的后方,给了红军不可估量的帮助。

(八)红军之所以胜利,是因为苏维埃国家在反对白卫反革命势力和外国武装干涉的斗争中并不是孤立的,苏维埃政权的斗争及其成功,赢得了全世界无产者的同情和援助。当帝国主义企图用武装干涉和封锁来扼杀苏维埃共和国的时候,他们国家的工人站在苏维埃方面,援助苏维埃。这些敌视苏维埃共和国的国家的工人同资本家的斗争,促使帝国主义者不得不放弃武装干涉。英法和其他参加武装干涉的国家的工人举行罢工,拒绝装载军用品去援助武装干涉者和白卫将军,建立了以“不许干涉俄国”作为自己口号的“行动委员会”。

\begin{quotation}
列宁说:“只要国际资产阶级向我们举起拳头来,他们的手就会被本国工人抓住。”(《列宁全集》俄文第3版第25卷第405页)\footnote{见《列宁全集》第31卷第276页。——译者注}
\end{quotation}


\subsection{简短的结论}

被十月革命打倒的地主资本家同白卫将军们一起,牺牲祖国的利益,同各协约国政府勾结,想共同用武力来进攻苏维埃国家,来推翻苏维埃政权。在这个基础上组织了协约国的武装干涉和俄国边境地区的白卫叛乱,结果俄国同粮食和原料产区的联系被切断了。

德国的军事失败和两个帝国主义联盟的欧洲战争的结束,导致协约国的加强,武装干涉的加强,给苏维埃国家增加了新的困难。

另一方面,德国革命和欧洲国家中开始的革命运动,却给苏维埃政权造成了有利的国际环境,缓和了苏维埃国家的处境。

布尔什维克党发动工人和农民进行卫国战争,反对外国侵略者和资产阶级地主的白卫匪帮。苏维埃共和国及其红军把协约国的傀儡高尔察克、尤登尼奇、邓尼金、克拉斯诺夫、弗兰格尔一个接一个地相继打败,把协约国的另一个傀儡皮尔苏茨基驱逐出了乌克兰和白俄罗斯,从而击退了外国武装干涉,把干涉军赶出了苏维埃国境。

这样,国际资本对社会主义国家的第一次武装进攻就以彻底失败而告终。

被革命打倒的党派社会革命党、孟什维克、无政府主义者、民族主义者等,在武装干涉时期支持白卫将军和武装干涉者,对苏维埃共和国搞反革命阴谋,对苏维埃活动家搞恐怖活动。这些十月革命前在工人阶级中还有某些影响的党派,在国内战争时期已在人民群众面前完全暴露出是反革命党派。

国内战争和武装干涉时期,是这些党派在政治上灭亡和共产党在苏维埃国家彻底胜利的时期。

