\section[第六章\q 布尔什维克党在帝国主义战争时期。俄国第二次革命(1914年—1917年3月)]{第六章\\ 布尔什维克党在帝国主义战争时期。\\俄国第二次革命 \\{\zihao{3}(1914年—1917年3月)}}

\subsection[一\q 帝国主义战争的发生和起因]{一\\帝国主义战争的发生和起因}

1914年7月14日(27日),沙皇政府宣布总动员。7月19日(8月1日),德国向俄国宣战。

俄国加入战争了。

在战争开始以前很久,列宁、布尔什维克早已预见到战争不可避免。列宁在几次社会党人国际代表大会上都发表了讲话,提议确定社会党人在战争爆发时所应采取的革命行动路线。

列宁指出,战争是资本主义不可避免的伴侣。抢劫他国领土,侵占和掠夺殖民地,夺取新的市场,这些已经不止一次地成为资本主义国家进行侵略战争的原因。战争也如剥削工人阶级一样,对资本主义国家来说是一种自然和当然的事情。

尤其是当资本主义于十九世纪末和二十世纪初完全发展到它的最高和最后阶段,即发展成帝国主义的时候,战争更是不可避免的了。在帝国主义时代,强大的资本家联合组织(垄断组织)和银行已在资本主义各国生活中起着决定的作用。金融资本已成为资本主义国家的主宰,金融资本需要新的市场,需要侵占新的殖民地,需要新的资本输出场所,需要新的原料产地。

但到十九世纪末,地球上的全部领土都已被资本主义各国瓜分完毕。加以在帝国主义时代,资本主义发展是极不平衡和跳跃式的:从前居第一位的国家现在工业发展得比较慢,而从前落后的国家却迅速地跳跃前进,赶上并超过它们。各帝国主义国家经济上军事上的实力对比发生了变化。出现了重新分割世界的趋向。重新分割世界的战争使帝国主义战争不可避免。

1914年的战争就是重新分割世界和势力范围的战争。这次战争是各帝国主义国家准备已久的。发动这次战争的祸首是世界各国的帝国主义者。

特别努力准备了这次战争的,一方面是德国和奥地利,另一方面是英法和依赖于它们的俄国。1907年成立了三国协定(或称协约),即英法俄三国联盟。组成另一帝国主义联盟的是德国、奥匈帝国和意大利。但意大利在1914年战争开始时退出了这个联盟,随后便加入了协约国。支援德国和奥匈帝国的有保加利亚和土耳其。

德国准备帝国主义战争,是要从英法两国手中夺取殖民地,从俄国手中夺取乌克兰、波兰和波罗的海沿岸。德国建筑了巴格达铁路,威胁到英国在近东的统治。英国害怕德国海上军备的增长。

沙俄力图分割土耳其,想要侵占黑海通地中海的海峡(例如达达尼尔海峡),想要夺取君士坦丁堡。沙皇政府还打算夺取奥匈帝国的加里西亚。

英国力图通过战争打败它的危险的竞争者德国,因为战前德国商品在世界市场上日益排挤英国商品。此外,英国还打算从土耳其手中夺取美索不达米亚和巴勒斯担,并在埃及站稳脚跟。

法国资本家力图从德国手中夺取盛产煤铁的萨尔矿区,以及在1870—1871年战争时德国从法国手中占去的亚尔萨斯—洛林区。

由此可见,帝国主义战争是由两个资本主义国家集团间的最大矛盾引起的。

这次重新分割世界的掠夺战争牵连到所有帝国主义国家的利益,所以后来日本、美国和其他许多国家也被卷进去。

战争成为世界性的了。

这次帝国主义战争是资产阶级背着本国人民极端秘密地准备的。当战争爆发时,每个帝国主义政府都竭力证明,不是它侵犯了邻国,而是邻国侵犯了它。资产阶级欺骗人民,隐瞒战争的真正目的,隐瞒战争的帝国主义的掠夺性质。每个帝国主义政府都说,进行战争是为了保卫自己的祖国。

第二国际的机会主义者帮助资产阶级欺骗人民。第二国际的社会民主党人卑鄙地背叛了社会主义事业,背叛了无产阶级国际团结事业。他们不仅没有起来反对战争,反而在保卫祖国的幌子下帮助资产阶级去挑动各交战国工农互相残杀。

俄国不是偶然站到英法协约国方而参加帝国主义战争的。必须注意到,在1914年以前,俄国各个最重要的工业部门都掌握在外国资本手里,主要是英法比三国即协约国资本手里。俄国最重要的冶金工厂由法国资本家把持着。整个说来,冶金业差不多有四分之三(百分之七十二)依赖于外国资本。煤炭工业的情形、顿巴斯的情形,也是如此。石油开采约有一半是在英法资本手里。很大一部分俄国工业利润流进了外国银行,主要是英法两国银行。所有这些情况,再加上沙皇向英法两国借的几十亿债款使沙皇政府紧紧依附于英法帝国主义,把俄国变成了这些国家的纳贡国,变成了它们的半殖民地。

俄国资产阶级指望开始战争可以改善一下自己的状况:获得新的市场,从军事定货和军需供给中赚得暴利,利用战争局势顺便把革命运动镇压下去。

沙俄参加战争是没有充分准备的。俄国工业大大落后于其他资本主义国家。俄国工厂多半是些设备破旧的老厂。农业由于存在着半农奴制地产和大批贫困破产的农民,不能成为进行长期战争的巩固的经济基础。

沙皇主要是依靠农奴制地主、黑帮大地主同大资本家结成联盟,操纵着俄国和国家杜马,他们完全支持沙皇政府的对内对外政策。俄国帝国主义资产阶级满心希望沙皇专制政府成为一个铁拳头。一方面保证它夺得新的市场和新的领土,另一方面把工农革命运动镇压下去。

自由资产阶级的政党立宪民主党以反政府派自居。却无条件地支持沙皇政府的对外政策。

小资产阶级的政党社会革命党和孟什维克,从战争一开始就在社会主义幌子下帮助资产阶级欺骗人民,隐瞒这次战争的帝国主义的掠夺性质。他们鼓吹必须保卫,必须保护资产阶级的“祖国”而抵抗“普鲁士野蛮人”,他们支持“国内和平”政策,这样,他们就是帮助俄国沙皇政府进行战争,象德国社会民主党人帮助德国皇帝政府进行战争反对“俄罗斯野蛮人”一样。

只有布尔什维克党仍然忠于革命国际主义的伟大旗帜,坚定不移地采取马克思主义的立场:坚决反对沙皇专制制度,反对地主资本家,反对帝国主义战争。布尔什维克党从战争一开始就认定:发动这次战争不是为了保卫祖国,而是为了地主资本家去侵占别国领土,掠夺别国人民,因此工人必须坚决向这次战争宣战。

工人阶级支持布尔什维克党。

固然,战争开始时笼罩着知识分子和富农阶层的资产阶级爱国主义狂热,也熏染了一小部分工人。但这主要是流氓式的“俄罗斯人民同盟”中的分子,以及一部分同情社会革命党和孟什维克的工人。他们当然没有反映而且也不可能反映工人阶级的情绪。正是这些分子参加了沙皇政府在战争开始时组织的资产阶级沙文主义的游行。


\subsection[二\q 第二国际各党转到本国帝国主义政府方面。第二国际瓦解为各个社会沙文主义党]{二\\第二国际各党转到本国帝国主义政府方面。\\第二国际瓦解为各个社会沙文主义党}

列宁不止一次提醒过,要注意第二国际的机会主义和第二国际领袖们的不坚定性。他始终强调说,第二国际领袖们只是口头上反对战争,一旦战争爆发,他们就会改变自己的立场而投到帝国主义资产阶级方面去,就会成为战争的拥护者。战争一开始就证实了列宁的预言。

1910年第二国际哥本哈根代表大会曾通过决议,说社会党人应当在国会中投票反对军事拨款。1912年巴尔干战争时,第二国际巴塞尔代表大会曾发表声明,说各国工人认为,为资本家增加利润而互相残杀是种罪恶。在口头上,在决议上,就是这样讲的。

而当帝国主义战争轰隆一声爆发,必须实现这些决议的时候,第二国际领袖们却成了无产阶级的叛徒和变节者。成了资产阶级的奴仆,成了战争的拥护者。

1914年8月4日,德国社会民主党在国会中投票赞成军事拨款,投票支持帝国主义战争。法英比和其他国家的绝大多数社会党人也是这样做的。

第二国际已不存在了。它事实上已瓦解为各个互相进行战争的社会沙文主义党。

各国社会党领袖背叛无产阶级而采取了社会沙文主义和保卫帝国主义资产阶级的立场。他们帮助帝国主义政府愚弄工人阶级,用民族主义毒药毒害工人阶级。这些社会主义叛徒在保卫祖国的幌子下挑动德国工人去反对法国工人,挑动英国工人和法国工人去反对德国工人。第二国际内只有很少一部分人仍然坚持国际主义立场,逆潮流而进,虽然不是十分坚定,不是完全明确,但毕竟是逆潮流而进的,

只有布尔什维克党才毫不犹豫地立刻举起了坚决反对帝国主义战争的旗帜。列宁在1914年秋拟定的关于战争的提纲中指出,第二国际的瓦解不是偶然的。第二国际是被机会主义者断送的,而革命无产阶级的优秀代表还在很早以前就已预告过必须反对这些机会主义者。

第二国际各党在战前就传染上了机会主义。机会主义者公开鼓吹放弃革命斗争,鼓吹“资本主义和平长入社会主义”的理论。第二国际不愿同机会主义作斗争,而主张同机会主义和睦相处,让它巩固起来。第二国际既对机会主义采取调和政策,于是自己也成了机会主义的了。

帝国主义资产阶级靠着它从殖民地、从剥削落后国家获得的利润,用较高的工资和其他的小恩小惠来不断地收买熟练工人上层,即所谓工人贵族。从这个工人阶层中产生了不少工会和合作社的领导者、地方议会和国会的议员、报刊和社会民主党组织的工作人员。在战争时,这些人害怕失掉自己的地位,于是就成了革命的敌人,成了本国资产阶级和本国帝国主义政府的最狂热的维护者。

机会主义者成了社会沙文主义者。

社会沙文主义者包括俄国孟什维克和社会革命党人在内,鼓吹工人在国内要同资产阶级实行阶级和平,在国外要同其他国家人民进行战争。他们向群众隐瞒战争的真正祸首,说他们本国的资产阶级不是造成战争的祸首。许多社会沙文主义者都成了本国帝国主义政府中的部长。

暗藏的社会沙文主义者,即所谓中派,对于无产阶级事业也是同样危险的。中派分子考茨基、托洛茨基和马尔托夫等人,极力为公开的社会沙文主义者洗刷、辩护。即同社会沙文主义者一起背叛无产阶级,只不过用了一些专门欺骗工人阶级的“左的”反战词句来掩盖自己的叛变行为罢了。事实上中派是支持战争的,因为中派提议在表决用于战争的拨款时不投票反对而只限于弃权,就等于是支持战争。他们也同社会沙文主义者一样,要求在战争期间放弃阶级斗争,以免妨碍本国帝国主义政府进行战争。中派分子托洛茨基在战争和社会主义的所有重大问题上,都是反对列宁,反对布尔什维克党的。

列宁从战争爆发时起,就开始聚集力量,准备建立新的国际即第三国际。布尔什维克党中央委员会在1914年11月的反战宣言\footnote{指《战争和俄国社会民主党》一文,见《列宁选集》第二版第二卷第568—574页。——译者注}中,就已提出了成立第三国际来代替已经遭到可耻破产的第二国际这一任务。

1915年2月,在协约国社会党人伦敦代表会议上,李维诺夫同志受列宁委托作了发言。李维诺夫要求比法两国社会党人(王德威尔得、桑巴、盖得)退出资产阶级政府并完全同帝国主义者决裂,放弃同他们合作。他要求所有社会党人都同本国帝国主义政府作坚决的斗争,并谴责投票赞成军事拨款的行为。但李维诺夫的呼声在这个会上没有得到响应。

1915年9月初,在齐美尔瓦尔得召开了国际主义者第一次代表会议。列宁称这次会议是国际反战运动发展中的“第一步”\footnote{见《列宁全集》第21卷第362—367页。——译者注}。列宁在这次会上组织了一个齐美尔瓦尔得左派。但在这个齐美尔瓦尔得左派中,只有以列宁为首的布尔什维克党采取了唯一正确的、贯彻到底的反战立场。齐美尔瓦尔得左派用德文出版了《先驱》杂志,上面刊载过列宁的文章。

1916年,在瑞士的昆塔尔村召开了国际主义者第二次代表会议。这次会议称为第二次齐美尔瓦尔得代表会议。这时差不多在所有国家中都已有一批国际主义者分离出来,国际主义分子同社会沙文主义者的分裂已经更加明显了。而主要的是,这时群众自己已因受战争和战争灾难的影响而左倾了。昆塔尔宣言的拟定是会上互相斗争的各个集团妥协的结果。它和齐美尔瓦尔得宣言相比前进了一步。

但是昆塔尔代表会议也没有采纳布尔什维克政策的基本原则:变帝国主义战争为国内战争,使本国帝国主义政府在战争中失败,成立第三国际。但是昆塔尔代表会议终究促进了国际主义分子分离出来的过程,后来正是这些分子组成了共产国际即第三国际。

列宁批评了罗莎·卢森堡和卡尔·李卜克内西这样一些左派社会民主党人中不彻底的国际主义者的错误,同时又帮助他们采取了正确的立场。


\subsection[三\q 布尔什维克党在战争、和平与革命问题上的理论和策略]{三\\布尔什维克党在战争、和平与革命问题上的理论和策略}

布尔什维克不是那种只是感叹和平和局限于宣传和平的和平主义者(和平派),像大多数左派社会民主党人那样。布尔什维克主张用积极的革命斗争来争取和平,直到推翻好战的帝国主义资产阶级的政权。布尔什维克把和平事业同无产阶级革命胜利的事业联系在一起,认为推翻帝国义资产阶级的政权是消灭战争,取得公正的和平,即取得不割地不赔款的和平的最可靠的手段。

布尔什维克反对孟什维克和社会革命党人背弃革命的行为和在战争时期保持“国内和平”的叛卖性口号,而提出“变帝国主义战争为国内战争”的口号。这个口号就是说,劳动群众,包括武装的工人和穿着军服的农民,如果想摆脱战争而赢得公正的和平,就应掉转枪口去反对本国资产阶级并推翻他们的政权。

布尔什维克反对孟什维克和社会革命党人保护资产阶级祖国的政策,而提出“使本国政府在帝国主义战争中失败”的政策。这就是说,必须投票反对军事拨款,在军队中成立秘密的革命组织,支持前线士兵联欢,组织工农的反战革命发动,并把这种发动转变为反对本国帝国主义政府的起义。

布尔什维克认为在帝国主义战争中对人民害处最少的是沙皇政府在军事上失败,因为这种失败有助于人民战胜沙皇制度,有助于工人阶级顺利地摆脱资本主义奴隶制和帝国主义战争。同时,列宁认为不仅俄国革命者,而且一切交战国工人阶级的革命政党,都应实行使本国帝国主义政府失败的政策。

布尔什维克不是反对一切战争。他们只是反对掠夺性的战争,反对帝国主义战争。布尔什维克认为战争有两种:

(一)正义的,非掠夺性的、解放性的战争,其目的或者是保卫人民抵御外来的侵犯和奴役人民的企图,或者是把人民从资本主义奴隶制度下解放出来,或者是把殖民地和附属国从帝国主义者压迫下解放出来;

(二)非正义的、掠夺性的战争,其目的是掠夺和奴役别的国家和别国人民。

布尔什维克拥护前一种战争。至于后一种战争,布尔什维克以为必须对它进行坚决的斗争,直到举行革命和推翻本国帝国主义政府。

列宁在战争时期所写的理论著作,对于全世界工人阶级都有巨大的意义。1916年春,列宁写了《帝国主义是资本主义的最高阶段》一书。列宁在这本书中指出:帝国主义是资本主义的最高阶段,这时它已由“进步的”资本主义变成了寄生的资本主义,变成了腐朽的资本主义;帝国主义是垂死的资本主义。这当然不是说资本主义会不经过无产阶级革命而自行死亡,不是说资本主义会自己连根烂掉。列宁始终教导说,不经过工人阶级的革命,就不可能推翻资本主义制度。因此,列宁在这本书中肯定了帝国主义是垂死的资本主义之后,同时说明:“帝国主义是无产阶级社会革命的前夜。”\footnote{见《列宁选集》第2版第2卷第737页。——译者注}

列宁指出:资本主义的压迫在帝国主义时代日益加剧;在帝国主义条件下,无产阶级对资本主义制度的愤恨日益增长,资本主义国家内部革命爆发的因素日益成熟。

列宁指出:在帝国主义时代,殖民地和附属国内革命危机日益尖锐,对帝国主义愤恨因素日益增长,反帝解放战争的因素日益增加。

列宁指出:在帝国主义条件下,资本主义发展的不平衡和资本主义的矛盾特别尖锐;争夺商品销售市场和资本输出场所的斗争,争夺殖民地、争夺原料产地的斗争,使得重新分割世界的周期性帝国主义战争成为不可避免。

列宁指出:正是由于资本主义发展的这种不平衡而发生的帝国主义战争,削弱着帝国主义的力量,并使帝国主义战线有可能在它最薄弱的地方被突破。

列宁根据这一切得出结论说:无产阶级在某一个地方或某几个地方突破帝国主义战线是完全可能的;社会主义首先在几个国家或者甚至在单独一个国家内胜利是可能的,由于各国资本主义发展的不平衡,社会主义同时在所有国家内胜利是不可能的;社会主义将首先在一个或者几个国家中获得胜利,而其余的国家在一段时期内将仍然是资产阶级的国家。

列宁在帝国主义战争时期所写的两篇不同的文章中,把这个英明的结论表述如下:

\begin{quotation}
(一)“经济政治发展的的不平衡是资本主义的绝对规律。由此就应得出结论:社会主义可能首先在少数或者甚至在单独一个资本主义国家内获得胜利。这个国家内获得胜利的无产阶级既然剥夺了资本家并在本国组织了社会主义生产,就会起来反对其余的资本主义的世界,把其他国家的被压迫阶级吸引到自己方面来……”(摘自1915年8月写的《论欧洲联邦口号》一文)(《列宁全集》俄文第3版第18卷第232—233页)\footnote{见《列宁选集》第2版第2卷第709页。——译者注}

(二)“资本主义的发展在各个国家是极不平衡的。而且在商品生产的条件下也只能是这样。由此可以得出一个坚定不移的结论:社会主义不能在所有国家内同时获得胜利。它将首先在一个或者几个国家中获得胜利,而其余的国家在一段时期内将仍然是资产阶级的或者资产阶级以前时期的国家。这就不仅要引起摩擦,而且要引起其他各国资产阶级公然企图扑灭社会主义国家中胜利的无产阶级。在这种情形下发生的战争,就我们方面说来是合理的和正义的战争。这是争取社会主义、争取把其他各国人民从资产阶级压迫下解放出来的战争。”(摘自1916年秋写的《无产阶级革命的军事纲领》一文)(《列宁全集》俄文第3版第19卷第325页)\footnote{同上,第873页。——译者注}
\end{quotation}

这是新的完备的社会主义革命论,是关于社会主义可能在单个国家内获得胜利、关于社会主义胜利条件、关于社会主义胜利前途的理论,而这一理论的基础是列宁早在1905年写的《社会民主党在民主革命中的两种策略》这本小册子中就已规定了的。

这个理论同帝国主义以前的资本主义时期流行于马克思主义者当中的指导思想根本不同;当时马克思主义者认为社会主义在一个国家内胜利是不可能的,认为社会主义将在所有的文明国家内同时获得胜利。列宁根据他在《帝国主义是资本主义的最高阶段》这一卓越著作中所阐述的关于帝国主义阶段的资本主义的论据,改变了这种已经过时的思想,提出了新的理论,即认为社会主义在所有国家内同时胜利是不可能的,而社会主义在单独一个资本主义国家内胜利是可能的。

列宁的社会主义革命论的不可估量的意义,不仅在于它用新的理论丰富和推进了马克思主义。这个理论的意义还在于它向各个国家的无产者指出了革命的前途,使他们能充分发挥主动性去冲击本国的资产阶级,教导他们利用战争环境去组织这样的冲击,加强了他们对无产阶级革命胜利的信心。

以上就是布尔什维克在战争、和平与革命问题上的理论和策略。

布尔什维克根据这一理论和策略在俄国进行了他们的实际工作。

在战争开始时,杜马中的布尔什维克代表巴达也夫、彼得罗夫斯基、穆拉诺夫、萨莫依洛夫和沙果夫不顾警察的残酷迫害。巡视了许多地方组织,在那里作了关于布尔什维克对战争和革命态度问题的报告。1914年11月,布尔什维克国家杜马党团开会讨论对战争的态度问题。开到第三天,全体代表都被捕了。法庭判决剥夺所有这些代表的权利,判处他们在西伯利亚东部终身流放。沙皇政府给国家杜马中的布尔什维克代表定的是“叛国”罪。

在法庭上揭示出来的杜马代表的活动情况,使我们党感到光荣。布尔什维克代表在沙皇法庭上表现得很英勇,他们把沙皇法庭变成了揭露沙皇政府侵略政策的讲坛。

当时为此案受到审讯的加米涅夫却是另一种表现。他由于胆怯成性,一遇到危险就背弃了布尔什维克党的政策。加米涅夫在法庭上声明他在战争问题上同布尔什维克意见不合,并为证明这点请求法庭把孟什维克约尔丹斯基传来作证。

布尔什维克进行了大量工作来反对为战争服务的军事工业委员会,反对孟什维克想使工人服从帝国主义资产阶级影响的企图。资产阶级切身需要在大众面前把帝国主义战争说成是全民的战争。资产阶级在战争时期建立了自己的全俄组织——地方自治局联合会和市政公所联合会,因而能在很大程度上左右国家事务。同时,它还想使工人也服从它的领导和影响。资产阶级为此想出了一个办法——在军事工业委员会下设立“工人小组”。孟什维克支持资产阶级的这种想法。把工人代表拉进这些军事工业委员会对资产阶级是有利的,因为有些工人当了代表,就会到工人群众中间进行鼓动,说必须在炮弹厂、大炮厂、枪械厂、子弹厂以及其他进行国防生产的工厂提高劳动生产率。“一切为了战争,一切用于战争”,——这就是资产阶级的口号。实际上,这个口号是说:“不顾一切地靠军事订货和侵占别国领土发财吧。”孟什维克积极参加了资产阶级玩弄的这种冒牌爱国主义勾当。他们帮助资本家拼命鼓动工人参加军事工业委员会下面的“工人小组”的选举。布尔什维克反对这种花招。他们主张抵制军事工业委员会,并卓有成效地进行了选种抵制。但一部分工人仍然在著名的孟什维克格沃兹迭夫和奸细阿布罗西莫夫领导下参加了军事工业委员会的活动。当工人代表在1915年9月集合起来进行军事工业委员会“工人小组”决选时,才发现大多数代表都反对参加“32人小组”。大多数工人代表通过了强烈反对参加军事工业委员会的决议,说明工人的任务是为和平、为推翻沙皇制度而斗争。

布尔什维克在陆军和海军中也开展了巨大的工作。他们向士兵和水兵群众说明谁是造成这场骇人听闻的战争惨祸和人民苦难的罪人,说明革命是人民摆脱帝国主义大屠杀的唯一出路。布尔什维克在陆军和海军中,在前线和后方部队中建立了支部,散发了反战传单。

布尔什维克在喀琅施塔得成立了“喀琅施塔得军队党组织总干事团”,总干事团同党的彼得格勒委员会保持着密切的联系。在党的彼得格勒委员会下设立了负责在卫戍部队中进行工作的军事局。1916年8月,彼得格勒保安局局长呈报说。“喀琅施塔得干事团工作做得很慎重、很严密,参加者都是一些沉默寡言和作事谨慎的人。该干事团在岸上也派有自己的代表。”

党在前线进行了鼓动,号召交战国军队的士兵联欢,并着重指出:敌人就是世界各国资产阶级;只有变帝国主义战争为国内战争,掉转枪口去反对本国资产阶级及其政府,才能结束战争。个别部队拒绝进攻的事件日益增多了。早在1915年,特别是1916年,就已有过这样的事情。

布尔什维克在波罗的海沿岸地区的北方战线各集团军中进行了特别巨大的工作。1917年初,北方战线一个集团军的总司令鲁兹斯基将军向上级报告,说布尔什维克在北方战线开展了规模巨大的革命工作。

战争是各国人民和国际工人阶级生活中的最大转折。它使各个国家的命运、各国人民的命运和社会主义运动的命运都处于决定的关头。因此,它同时又是对所有以社会主义自命的党派的试金石和考验。这些党派是坚持忠于社会主义事业,国际主义事业呢,还是宁愿背叛工人阶级,卷起自己的旗帜,把它抛到本国资产阶级的脚下呢,——当时的问题就是这样摆着的。

战争表明,第二国际各党没有经住考验,背叛了工人阶级,在本国帝国主义资产阶级面前放下了自己的旗帜。

这些党也不能不如此,因为它们在自己的队伍中间培植机会主义,它们是用对机会主义者,民族主义者让步的精神教育出来的。

战争表明,只有布尔什维克党才光荣地经住了考验,才彻底忠于社会主义事业,忠于无产阶级国际主义事业。

而这也是理所当然的,因为只有新型的党,只有用同机会主义作不调和斗争的精神教育出来的党,只有清除了机会主义和民族主义的党,才能经得起伟大的考验,才能坚持忠于工人阶级的事业,忠于社会主义和国际主义的事业。

布尔什维克党正是这样的党。


\subsection[四\q 沙皇军队在前线的失败。经济破坏。沙皇制度的危机]{四\\沙皇军队在前线的失败。\\经济破坏。沙皇制度的危机}

战争已经进行三年了。战争夺去了数百万人的生命:有的被打死,有的因伤致死,有的死于战争所引起的瘟疫,资产阶级和地主大发战争财。而工人和农民却愈来愈贫穷困苦。战争破坏了俄国的国民经济。约有一千四百万壮劳力被拉去当兵,脱离了生产。工厂纷纷停产。谷物播种面积因缺乏劳力而缩减。居民和前线士兵忍饥挨饿,赤脚露体。战争耗尽了国内的一切资源。

沙皇军队屡战皆败。德军炮兵轰击沙皇军队时弹如雨下,而沙皇军队则缺乏大饱,缺乏炮弹,甚至缺乏步枪,有时三个士兵用一枝枪。还在战争期间就已发觉了沙皇陆军大臣苏霍姆林诺夫同德国特务勾结的卖国行为。苏霍姆林诺夫执行德国间谍机关的指令,破坏前线的弹药供应,不供给前线大炮和枪枝。沙皇的一些大臣和将领自己暗中协助德军获胜:他们同跟德方有勾结的皇后一起把军事秘密泄露给敌军。难怪沙皇军队屡遭失败,不得不退却。到1916年,德军已侵占了波兰边境和波罗的海沿岸的部分地区。

这一切激起了工人、农民、士兵和知识分子对沙皇政府的深恶痛绝,使后方和前线、中心地区和边沿地区人民群众反对战争和反对沙皇制度的革命运动加强和加剧起来。

俄国帝国主义资产阶级也产生了不满情绪。它看见权奸拉斯普庭之流悍然企图同德方单独媾和,而沙皇朝廷上又由他们说了算,感到十分恼火。它愈来愈确信沙皇政府不能进行胜利的战争。它害怕沙皇政府为摆脱自己的困境而去同德方单独媾和。因此,俄国资产阶级决定举行宫廷政变,以废黜沙皇尼古拉二世,另立同资产阶级有勾结的米哈伊尔·罗曼诺夫。资产阶级是想借此一箭双雕:第一,僭取政权而保证继续进行帝国主义战争;第二,用小小的宫廷政变来阻止当时已经汹涌澎湃的人民大革命。

英法政府在这方面完全支持俄国资产阶级。它们知道沙皇不能把战争继续下去。它们害怕沙皇会以同德方单独媾和而就此了事。要是沙皇政府缔结单独和约,英法政府就要失掉俄国这样一个战争中的同盟者,俄国就不仅不能在自己战线上牵制敌军力量,而且不能向法国提供数以万计的俄国精锐部队。因此,它们支持俄国资产阶级举行宫廷政变的尝试。

于是沙皇陷于孤立了。

在前线接连失利的同时,经济破坏也有增无减。1917年1、2月间。粮食、原料和燃料生产方而的破坏,已达到最厉害最尖锐的地步。供给彼得格勒和莫斯科的食品,差不多完全停止了运输。企业相继倒闭。企业的倒闭又使失业人数增加。工人的生活特别困苦不堪。愈来愈多的人民群众确信:要摆脱这种不堪忍受的状况,只有一条出路——推翻沙皇专制制度。

沙皇制度显然经历着毁灭性的危机。

资产阶级想用宫廷政变来解决危机。

但人民按自己的方式把危机解决了。


\subsection[五\q 二月革命。沙皇制度的覆灭。工兵代表苏维埃的成立。临时政府的成立。两个政权并存的局面]{五\\二月革命。沙皇制度的覆灭。工兵代表苏维埃的成立。\\临时政府的成立。两个政权并存的局面}

1917年一开始就发生了1月9日的罢工。在罢工时,彼得格勒、莫斯科、巴库和下新城都举行了游行示威,而且莫斯科参加1月9日罢工的大约有三分之一的工人。当时在特维尔林荫道上有两千示威群众被骑警驱散。在彼得格勒的维波尔格公路上,有士兵加入游行示威的队伍。

彼得格勒警察局报告说:“有总罢工这种想法的人一天比一天多,这种想法已经像1905年那样成为普遍的想法了。”孟什维克和社会革命党人,力图把已经开始的革命运动纳入自由资产阶级需要的轨道。2月14日国家杜马开幕这天,孟什维克提议组织工人游行去向国家杜马请愿,但工人群众跟布尔什维克走了,不是去向杜马请愿,而是去游行示威。

1917年2月18日,彼得格勒的普梯洛夫工厂工人开始罢工。2月22日,大多数大企业的工人也宣布了罢工。2月23日(3月8日)国际妇女节这天,女工们响应布尔什维克彼得格勒委员会的号召,上街游行示威,反对饥饿,反对战争,反对沙皇制度。工人们举行了全彼得格勒的总罢工来支援女工的游行示威。政治罢工开始转变为反对沙皇制度的政治总示威了。

2月24日(3月9日),又举行了声势浩大的游行示威。罢工工人已达二十万左右。

2月25日(3月10日),革命运动席卷了整个工人的彼得格勒。各区的政治罢工转变成全彼得格勒的政治总罢工。到处都在举行游行示威,并同警察发生冲突。工人群众举着的红旗上写着这样的口号:“打倒沙皇!”“打倒战争!”“面包!”

2月26日(3月11日)清晨,政治罢工和游行示威开始转变为起义的尝试。工人解除警察和宪兵的武装,把自己武装起来。但当时同警察发生的武装冲突,却以示威群众在兹那缅斯克广场上遭受枪击而告终。

彼得格勒军区司令哈巴洛夫将军发出布告,说工人必须在2月28日(3月13日)复工,不然就要把他们派往前线。2月25日(3月10日),沙皇给哈巴洛夫将军下令:“着令于明日将京都骚乱悉行制止。”

但要“制止”革命已经办不到了。

2月26日(3月11日)白天,巴甫洛夫团后备营第四连开火了,不过不是向工人开火,而是向那些同工人交火的骑警队开火。当时大力地坚持地展开了争取军队的工作,尤其是女工,她们径直走到士兵面前,同他们欢谈,号召他们帮助人民推翻那个令人痛恨的沙皇专制制度。

当时负责领导布尔什维克党实际工作的,是设在彼得格勒的以莫洛托夫同志为首的我党中央局。2月26日(3月11日),中央局发表宣言,号召继续进行反对沙皇制度的武装斗争,号召成立临时革命政府。

2月27日(3月12日),彼得格勒驻军拒绝向工人开枪,开始转到起义的人民方面来。2月27日早晨起义的士兵还只有一万人,而到晚上就已超过了六万人。

起义的工人和士兵开始拘捕沙皇的大臣和将军,释放狱中的革命者。被释放的政治犯加入了革命斗争事业。

在街上,群众还在同那些架着机关枪盘踞房顶的巡警和宪兵互相射击。但军队迅速转到工人方面来,已决定了沙皇专制制度的命运。

当革命在彼得格勒胜利的消息传到其他城市和前线时,各处的工人和士兵都起来推翻沙皇官吏。

二月资产阶级民主革命胜利了。

革命所以获得了胜利,是因为工人阶级做了革命的先锋,领导了数百万身穿军服的农民群众“争取和平,争取面包,争取自由”的运动。无产阶级的领导权决定了革命的成功。

\begin{quotation}
列宁在革命的初期写道:“无产阶级实现了革命,它表现了英勇精神,它流了鲜血,它率领了最广泛的劳苦大众……”(《列宁全集》俄文第3版第20卷第23—24页)\footnote{见《列宁全集》第23卷第318页。——译者注}
\end{quotation}

1905年的第一次革命准备了1917年的第二次革命的迅速胜利。

\begin{quotation}
列宁写道:“如果不是俄国无产阶级在1905—1907年三年间进行了极其伟大的阶级战斗和表现了革命的毅力,那么第二次革命的进展就不会这样迅速。也就是说这次革命的开始阶段就不会在几天以内完成。”(同上,第13页)\footnote{见《列宁选集》第2版第3卷第2页。——译者注}
\end{quotation}

在革命的最初几无就出现了苏维埃。获得胜利的革命依靠着工兵代表苏维埃。起义的工人和士兵建立了工兵代表苏维埃。1905年的革命表明,苏维埃是武装起义的机关,同时又是革命新政权的萌芽。苏维埃思想已经深入到工人群众意识中,所以他们在推翻沙皇制度后的第二天就实现了这一思想,不过所不同的是,1905年成立的还只是工人代表苏维埃,而1917年2月则由布尔什维克发起成立了工兵代表苏维埃。

当布尔什维克在街头领导群众的直接斗争的时候,妥协主义政党孟什维克和社会革命党却在苏维埃中夺取代表席位以组成自己的多数。其所以形成了这样一种局面,部分是因为当时布尔什维克党的大多数领袖还在监狱和流放地(列宁侨居国外,斯大林和斯维尔德洛夫在西伯利亚流放地),而孟什维克和社会革命党人却在彼得格勒街上自由自在地游逛。因此,妥协主义政党的代表孟什维克和社会革命党人掌握了彼得格勒苏维埃及其执行委员会的领导权。莫斯科和其他许多城市也是这种情况。只在伊万诺沃-沃兹涅先斯克、克拉斯诺雅尔斯克及其他几个城市,苏维埃中的多数才是从一开始就属于布尔什维克。

武装的人民——工人和士兵选派自己的代表到苏维埃去,是把苏维埃当作人民政权机关看待的。他们认为并且相信,工兵代表苏维埃定会实现革命人民的一切要求,并且首先会缔结和约。

但过度的轻信使工人和士兵上了大当。社会革命党人和孟什维克根本没有想结束战争,争取和平。他们想的是利用革命来继续战争。至于革命和人民的革命要求,社会革命党人和孟什维克认为革命已经完结了,现在的任务是巩固革命,转上按“正常的”宪制的原则同资产阶级共处的轨道。因此,彼得格勒苏维埃的社会革命党——孟什维克领导,竭力设法把结束战争的问题、和平的问题压下去,并把政权交给资产阶级。

1917年2月27日(3月12日),国家杜马中的自由派代表根据同社会革命党——孟什维克领导的秘密协定,成立了以第四届杜马主席、地主首领兼保皇派罗将柯为首的国家杜马临时委员会。几天后,国家杜马临时委员会和工兵代表苏维埃执行委员会的社会革命党——孟什维克领导,又背着布尔什维克商定了组织俄国的新政府即资产阶级临时政府,让早在二月革命前就由沙皇尼古拉二世指定充任自己政府首相的李沃夫公爵来担任首脑。参加临时政府的有立宪民主党人的首领米留可夫,十月党人的首领古契柯夫,以及资本家阶级的其他的有名的代表,而作为“民主派”代表参加的是社会革命党人克伦斯基。

结果就是苏维埃执行委员会的社会革命党——孟什维克领导把政权拱手交给了资产阶级,而工兵代表苏维埃知道此事之后,又不顾布尔什维克的抗议而以多数表决认可了社会革命党孟什维克领导的行动。

于是在俄国就形成了如列宁所说的由“资产阶级和资产阶级化的地主”\footnote{见《列宁选集》第2版第3卷第36页。——译者注}的代表所组成的新的国家政权。

但是当时同资产阶级政府并存的还有另一个政权,即工兵代表苏维埃。苏维埃中的士兵代表主要是被征召参战的农民。工兵代表苏维埃是反对沙皇政权的工农联盟机关,同时又是工农政权机关,是工人阶级和农民专政的机关。

于是就形成了两个政权、两个专政即以临时政府为代表的资产阶级专政和以工兵代表苏维埃代表的工农专政二者交错在一起的特殊局面。

结果就形成了两个政权并存的局面。

为什么起初孟什维克和社会革命党人在苏维埃中占了多数呢?

为什么获得胜利的工人和农民自愿地把政权交给了资产阶级的代表呢?

列宁认为,这是因为觉醒起来参加政治生活的千百万人缺乏政治经验。他们大部分是小业主、农民、不久前还是农民的工人,即介于资产阶级和无产阶级之间的人们。当时俄国是欧洲所有的大国中小资产阶级人数最多的国家。在这样一个国家中,“汹涌的小资产阶级浪潮吞没了一切,它不仅在数量上而且在思想上压倒了觉悟的无产阶级,就是说,用小资产阶级的政治观点感染了和俘虏了非常广大的工人群众”(《列宁全集》俄文第3版第20卷第115页)\footnote{见《列宁选集》第2版第3卷第40页。——译者注}。

这个小资产阶级自发势力的浪潮,把小资产阶级的党派孟什维克和社会革命党涌到表面上来了。

列宁指出,另一个原因就是无产阶级的成分在战争时期起了变化,以及无产阶级在革命开始时缺乏足够的觉悟性和组织性。在战争时期,无产阶级本身的成分发生了很大的变化。约有百分之四十的骨干工人被征召入伍了。在战争年代,有很多同无产阶级心理格格不入的小私有者、手工业者和小店主,为逃避征兵而钻进了企业。

工人中间的这些小资产阶级阶层,也就成了小资产阶级政治家孟什维克和社会革命党人吸取养料的土壤。

正因为如此,缺乏政治经验、为小资产阶级自发势力的浪潮所吞没,沉醉于革命的最初胜利的广大人民群众,在革命的头几个月就成了妥协主义政党的俘虏,同意把国家政权让给资产阶级,天真地认为资产阶级政权不会妨碍苏维埃进行自己的工作。

摆在布尔什维克党面前的任务就是在群众中耐心地进行解释工作:揭穿临时政府的帝国主义性质,揭穿社会革命党人和孟什维克的叛卖行为,说明不用苏维埃政府代替临时政府就得不到和平。

于是布尔什维克党就用全力把这项工作担当起来。

它恢复了自己的合法机关报。二月革命后第五天,《真理报》就在彼得格勒开始出版;再过几天,《社会民主党人报》也在莫斯科出版了。党开始出来领导正在放弃对自由资产阶级的信任,放弃对孟什维克和社会革命党人的信任的群众。党耐心地向士兵、向农民解释必须同工人阶级共同行动。党向他们解释,不继续发展革命,不以苏维埃政府来代替资产阶级临时政府,农民就得不到和平,也得不到土地。


\subsection{简短的结论}

帝国主义战争的发生是由于各个资本主义国家发展不平衡,由于几大强国之间的均势遭到破坏,由于帝国主义者需要用战争重新分割世界和造成新的均势。

如果第二国际各党不背叛工人阶级的事业,如果它们不违背第二国际几次代表大会的反战决议,如果它们下决心采取积极行动并发动工人阶级去反对本国帝国主义政府、反对战争挑拨者,那么战争就不会有这样大的破坏性,或许根本就不会发展到这样严重的地步。

布尔什维克党表明,它是坚持忠于社会主义和国际主义事业并组织了国内战争反对本国帝国主义政府的唯一的无产阶级政党。第二国际所有其余的党因为经过它们的上层领导同资产阶级拴在一起,都成了帝国主义的俘虏,投到帝国主义者方面去了。

战争本是资本主义总危机的反映,而战争本身又加剧了这个危机,削弱了世界资本主义。俄国工人和布尔什维克党在世界上第一个成功地利用了资本主义的弱点,突破了帝国主义战线,推翻了沙皇并建立了工兵代表苏维埃。

广大的小资产阶级、士兵以至工人沉醉于革命最初的胜利,满足于孟什维克和社会革命党人所谓今后一切都会好起来的担保,都痴心信任临时政府并给它以支持。

摆在布尔什维克党面前的任务就是要向这些沉醉于最初的胜利的工人和士兵群众说明:现在离革命完全胜利还很遥远;只要政权还在资产阶级临时政府手中而苏维埃还由孟什维克和社会革命党这些妥协派所操纵,人民就既不能获得和平,也不能获得土地,也不能获得面包;要获得完全的胜利,必须前进一步,使政权转归苏维埃。


