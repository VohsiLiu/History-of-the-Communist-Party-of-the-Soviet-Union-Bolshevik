\section[第五章\q 布尔什维克党在第一次帝国主义战争前工人运动高涨年代(1912—1914年)]{第五章\\ 布尔什维克党在第一次帝国主义战争前\\工人运动高涨年代 \\{\zihao{3}(1912—1914年)}}

\subsection[一\q 1912—1914年间革命运动的高涨]{一\\1912—1914年间革命运动的高涨}

斯托雷平实行反动,得势并不长久。一个除了皮鞭和绞架而外不愿给人民任何东西的政府,本来是不可能稳固的。高压手段已成为屡见不鲜的事情,再也不能恐吓人民了。工人在革命失败后的头几年所产生的疲倦心理开始消失。工人又重新奋起斗争了。布尔什维克断定革命必然会重新高涨的预见已被证实。1911年,罢工人数已超过十万,而在过去几年内每年罢工人数却不过五六万。1912年1月举行的党的布拉格代表会议已指出工人运动开始活跃的事实。但革命运动的真正高涨,是在1912年4—5月间由于连纳工人惨遭枪杀而爆发群众性政治罢工的时候开始的。

1912年4月4日,在西伯利亚连纳金矿举行罢工时,沙皇的宪兵队长下令开枪,打死打伤工人五百多。听说一群手无寸铁、和平前去同资方进行谈判的连纳矿工遭到枪杀,全国都沸腾起来了。沙皇专制政府干下这次新的血腥暴行,是为了摧毁矿工的经济罢工,以讨好连纳金矿老板英国资车家。英国资车家和他们的俄国股东靠极无耻地剥削工人,从连纳金矿每年取得七百多万卢布的骇人听闻的利润。他们付给工人极低的工资,供给工人不能食用的、腐烂变质的食品。连纳金矿的六千矿工不堪忍受这种欺压与凌辱而举行了罢工。

连纳枪杀事件发生后,无产阶级在彼得堡、莫斯科以及所有工业中心和工业地区举行了群众性罢工、游行示威和集会以示抗议。

有几个企业的工人在共同通过的决议上写道:“我们万分震惊,一时都不知道说什么好。无论我们提出怎样的抗议,都不能表达我们每个人的沸腾心情于万一。对于我们,无论眼泪或抗议都没有用,唯一有用的就是有组织的群众斗争。”

当沙皇大臣马卡罗夫回答社会民主党党团在国家杜马中对连纳枪杀事件提出的质询而蛮横声言,说“过去如此,将来还会如此!”的时候,工人更加怒不可遏了。参加抗议血腥屠杀连纳工人的政治罢工的人数,增加到三十万。

连纳事件象飓风一样冲破了斯托雷平制度所造成的“沉静”气氛。

斯大林同志1912年在彼得堡布尔什维克《明星报》上谈到这点时写道:

\begin{quotation}
“连纳的枪声击破了沉默的冰层,人民运动的江河奔流起来了。奔流起来了!…现存制度中的一切弊端和祸害,多灾多难的俄国所受的一切苦痛都集中在一件事实上,集中在连纳事件上。这就是连纳的枪声正好成为罢工和游行示威的信号的原因。”\footnote{见《斯大林全集》第2卷第232页。——译者注}
\end{quotation}

取消派和托洛茨基派企图埋葬革命,这是枉费心机。连纳事件表明革命力量仍然活着,工人阶级中积聚了巨大的革命能量。1912年的五一罢工大约有四十万工人参加,罢工带有鲜明的政治性质,它提出的是布尔什维克的革命口号,即成立民主共和国,实行八小时工作制,没收一切地主土地。这三大口号是要把广大的工人以及广大的农民和士兵都团结起来,向专制制度进行革命的冲击。

\begin{quotation}
列宁在《革命的高涨》一文中写道:“全俄无产阶级的轰轰烈烈的五月罢工,以及与罢工相连的游行示威、革命宣言和向工人群众发表的革命演说,都清楚地表明俄国已进入了革命高涨时期。”(《列宁全集》俄文第3版第15卷第533页)\footnote{见《列宁全集》第18卷第88页。——译者注}
\end{quotation}

取消派被工人的革命行动弄得惊惶失措,竟出来反对罢工斗争,称它为“罢工狂”。取消派及其同盟者托洛斯基想用“请愿运动”来代替无产阶级的革命斗争。他们劝工人在要求“权利”(要求取消对结社、罢工等的限制)的“请愿书”上签名,以便把它递交国家杜马。但取消派只征集到一千三百人签名,而团结在布尔什维克所提出的革命口号周围的工人却有几十万。

工人阶级是循着布尔什维克所指引的道路前进的。

当时国内的经济情况有如下述。

还在1910年,工业停滞已由主要工业部门生产的活跃和扩大所代替。生铁冶炼量在1910年为一亿八千六百万普特,1912年为二亿五千六百万普特,而到1913年已增至二亿八千三百万普特。煤炭开采量在1910年为十五亿二干二百万普特,而1913年已达到二十二亿一千四百万普特。

在资本主义工业增长的同时,无产阶级也迅速增长起来。当时工业发展的特点,就是生产进一步集中于大企业和最大企业。1901年,在有五百名以上工人的大企业里做工的工人占工人总数百分之四十六点七,而1910年在这种企业里做工的工人已达工人总数的百分之五十四左右,即占全体工人一半以上。工业集中的这种速度足空前的。甚至在北美这样一个工业发达的国家里,当时在大企业中做工的工人也只占全体体工人三分之一左右。

无产阶级的增长和集中于大企业,是在存在布尔什维克党这样一个革命政党的条件下实现的,因此就使俄国工人阶级变成了全国政治生活中最重大的力量。企业中对工人实行的野蛮剥削方式,再加上沙皇鹰犬横行这种不堪忍受的警察制度,就使每次重大的罢工都具有政治性质。同时,经济斗争同政治斗争结合起来,使群众罢工具有特别巨大的革命力量。

走在工人革命运动前头的是英勇的彼得堡无产阶级,继彼得堡之后是波罗的海沿岸边区、莫斯科市和莫斯科省,然后是伏尔加河流域和俄国南部地区。1913年,运动已扩展到西部边区、波兰和高加索。1912年罢工人数,据官方统计是七十二万五千,而据其他比较完备的统计在百万以上;1913年罢工人数,据官方统计是八十六万一千,而据比较完备的统计是一百二十七万二千。1914年上半年参加罢工的工人,已达一百五十万左右。

这样,1912—1914年间革命的高涨,即罢工运动的规模,已使全国接近于1905年革命开始时的局势了。

无产阶级的革命的群众性罢工具有全民的意义。它的目标是反对专制制度。罢工斗争得到了绝大多数劳动居民的同情。工厂主用同盟歇业来报复工人的罢工。1910年,莫斯科省资本家解雇了五万纺织工人。1914年3月,彼得堡在一天内就有七万工人被解雇。其他企业和其他工业部门的工人用群众性募捐,有时则用支持性罢工来支援举行罢工的和受到同盟歇业打击的同志。

工人运动的高涨和群众性的罢工唤起了农民群众,把他们也卷进了斗争。农民再次挺身起来反对地主,捣毁地主的庄园和富农的独立农庄。在1910—1914年间,总共发生了一万三千多次农民运动。

军队中的革命运动也开始了。1912年在土尔克斯坦驻军中发生了武装暴动,在波罗的海舰队和塞瓦斯托波尔酝酿着起义。

布尔什维克党所领导的革命罢工运动和游行示威表明,工人阶级进行斗争不是为了局部的要求,不是为了“改良”,而是为了把人民从沙皇制度下解放出来。俄国在走向新的革命。

列宁为了更接近俄国,于1912年夏从巴黎迁到加里西亚(原属奥地利)。在这里由列宁主持开过两次中央委员和负责工作人员的会议:一次是1912年底在克拉科夫举行,另一次是1913年秋在克拉科夫附近的波罗宁诺镇举行。这两次会议通过了许多有关工人运动的重大问题的决议:关于革命高潮,关于罢工和党的任务,关于巩固秘密组织,关于社会民主党杜马党团,关于党的报刊,关于保险运动。


\subsection[二\q 布尔什维克的《真理报》。第四届国家杜马中的布尔什维克党团]{二\\布尔什维克的《真理报》。\\第四届国家杜马中的布尔什维克党团}

布尔什维克在彼得堡出版的日报《真理报》,是布尔什维克党用来巩固白己的组织和扩大对群众的影响的强大武器。它是遵照列宁的指示,由斯大林、奥里明斯基和波列塔也夫发起创办的。群众性的工人报纸《真理报》随着革命运动的新高潮而诞生。1912年4月22日(5月5日),《真理报》创刊号出版了。这是工人的真正的节日。为了纪念《真理报》的诞生。决定5月5日为工人出版节。

还在《真理报》创办以前,已出版了专供先进工人阅读的布尔什维克周报《明星报》。《明星报》在连纳事件时期起了很大的作用。它登载过列宁和斯大林动员工人阶级进行斗争的许多战斗性政论文章。但在革命高涨的条件下,周报已不能满足布尔什维克党的需要了。必须出版一种供最广大的工人阶层阅读的群众性的政治日报。《真理报》就是这样的报纸。

在这个时期,《真理报》的作用是特别巨大的。《真理报》争取了广大的工人阶级群众站到布尔什维主义方面来。《真理报》经常受到警察迫害,遭到罚款,为登载书报检查机关所不喜欢的文章和通讯而被没收,它在这种条件下能够存在,完全是靠了数以万计的先进工人的积极支持。《真理报》能够交付巨额罚款,完全是靠了工人的踊跃捐献。被没收的每一号《真理报》,往往仍有相当多的份数能够到达读者手中,因为先进工人半夜就来到印刷厂,把一捆一捆的报纸取走。

沙皇政府在两年半内把《真理报》查封过八次,但《真理报》每次都在工人援助下又用一种新的类似的名称,如《拥护真理报》,《真理之路报)、《劳动的真理》等等,重新出版。

当时《真理报》每天平均销售四万份,而孟什维克的日报《光线报》每天印数不超过一万五六千份。

工人认为《真理报》是工人自己的报纸,对它非常信任,敏感地注视着它发表的意见。每份《真理报》都是辗转传阅,给几十个读者看,培养他们的阶级觉悟,教育他们,组织他们,号召他们进行斗争。

《真理报》上都讲些什么呢?

每一号《真理报》都刊载有几十篇工人通讯,叙述工人的生活情况、他们所过受的残酷剥削、资本家及其管事和工头们对工人的种种压迫和侮辱。这是对资本主义制度的尖锐的一针见血的揭露。《真理报》的简讯常常报导饥饿的失业工人因找工作无望而自杀的消息。

《真理报》经常反映各个工厂和各个工业部门工人的疾苦和要求,叙述工人怎样为自己的要求进行斗争。几乎每一号都载有各个企业罢工的消息。每当发生了大规模的持久的罢工,《真理报》就组织其他企业和其他工业部门的工人募捐援助罢工者。募得的罢工基金有时达几万卢布,这在当时是一笔很大的数目,因为要知道,当时大多数工人每天的收入只有七八十个戈比。这种做法培养了工人的无产阶级团结精神,使他们意识到全体工人利益的一致。

每当听到发生了政治事件,每当听到胜利或失败的消息,工人都要把信件,贺词或抗议书等寄给《真理报》。《真理报》在自己的文章中,用彻底的布尔什维克观点阐明工人运动的任务。一个合法的报纸是不可能直接号召推翻沙皇制度的。它只能暗示。但觉悟的工人能够很好地领会,并把暗示的意思解释给群众。例如,当《真理报》说到“1905年的全部的不折不扣的要求”时,工人们就懂得这是指布尔什维克的几个革命口号,即推翻沙皇制度,成立民主共和国,没收地主土地,实行八小时工作制。

《真理报》在第四届杜马选举前夜组织了先进工人。它揭露了孟什维克主张同自由资产阶级妥协、主张成立“斯托雷平工党”的叛徒立场。《真理报》号召工人投票选举坚持“1905年的不折不扣的要求”的人,即选举布尔什维克。当时的选举是多级的。先在工人大会上选出初级代表,再由初选代表选出复选代表,然后复选代表参加杜马工人代表的选举。在选举那天,《真理报》公布了布尔什维克复选代表的候选名单,号召工人投票选举他们。为了使预定的候选人不致遭到被捕的危险,这种名单是不能预先公布的。

《真理报》帮助了组织无产阶级的发动。当1914年春彼得堡大规模举行同盟歇业,以致不宜宣布群众性罢工的时候,《真理报》就号召工人采取其他的斗争方式,例如在工厂开群众大会,上街游行示威。当时在报纸上不能公开这样讲。但觉悟的工人一读列宁用《论工人运动的形式》这一不惹人注意的标题所写的一篇文章,就懂得了号召的意思,因为文章上说目前必须用工人运动的更高级的形式来代替罢工,意思就是号召举行群众大会和游行示威。

布尔什维克的秘密革命活动就是这样同通过《真理报》对工人群众进行的合法的鼓动工作和组织工作结合起来的。

《真理报》不仅报道了工人的生活、工人的罢工和游行示威。同时《真理报》还系统说明了农民的生活、农民的饥饿痛苦、农奴制地主对农民的剥削、斯托雷平“改革”后农民的好地被富农庄主攫为己有的情况。《真理报》使觉悟工人看到,农村中积蓄了大量的易燃物。《真理报》教导无产阶级说,1905年革命的任务并没有解决,新的革命即将到来。《真理报》教导说,无产阶级在这第二次节命中应当成为人民的真正的领袖和领导者,无产阶级在这次革命中将有革命的农民这样强有力的同盟者。

孟什维克竭力想使无产阶级抛弃革命念头。他们劝告工人说:你们别去考虑什么人民,什么农民的饥饿痛苦,什么黑帮农奴制地主的统治,你们应当一心争取“结社自由”。向沙皇政府呈递这样的“请愿书”。布尔什维克向工人解释说:孟什维克这种放弃革命和放弃同农民联合的宣传,只是有利于资产阶级;工人只要把农民吸引到自己方面来做自己的同盟者,就一定能战胜沙皇制度;象孟什维克这种恶劣的牧师,应当作为革命的敌人一脚踢开。

《真理报》在“农民生活”栏内讲了些什么呢?

我们从1913年的通讯中举出几篇作例子。

来自萨马拉的一篇标题为《一个土地案件》的通讯说:布古尔玛县诺沃哈兹布拉特利村有四十五个农民被控,罪名是说,在把公社土地划给独立田庄主时,他们反抗过土地丈量官。很大一部分被控农民都被判处长期徒刑。

普斯科夫省的一篇简讯说:“普西茨村(札瓦利耶车站附近)农民对乡丁实行了武装反抗。有人受伤。冲突的原因是土地纠纷。乡丁已往普西茨村集结,副省长和检察长已前往视察。”

乌发省的通汛说农民在出卖份地,说饥荒和退社的法令加速了农民丧失土地的过程。例如波利索夫卡村有二十七户农民,共有有五百四十三俄亩耕地。饥荒发生时,有五户永远变卖了三十一俄亩土地,每俄亩卖价是二十五至三十三卢布,但土地实际价值比这贵两倍。村里还有七户抵押了一百七十七俄亩土地,每俄亩押了十八至二十卢布,期限六年,年息百分之十二。如果注意到居民贫困和利率极高的情况,可以确定无疑地说,一百七十七俄亩土地中有一半要落到高利贷者手中,因为债户当中未必有一半在六年内能够偿清这样大的一笔数目。

列宁在《真理报》上发表的《俄国地主的大地产和农民的小地产》一文中,清楚地向工人和农民指明有多大一笔地产握在地主寄生虫手中。仅仅三万个大地主就占有土地约七千万俄亩,面一千万农户总共也只有这样多土地。每个大地主平均占有土地二千三百俄亩,而农民,连富农在内,每户平均不过七俄亩,而且其中五百万力量单薄的农户,即全体农户的半数,每户不过一二俄亩。这些事实清楚地表明,农民遭受贫困和饥饿,根源在于存在着地主的大地产即农奴制的残余,农民只有在工人阶级领导下进行革命,才能摆脱这种残余。

《真理报》经过那些同农村有联系的工人深入到农村中去,唤起先进农民进行革命斗争。

在创办《真理报》时期,各个秘密的社会民主党组织已完全掌握在布尔什维克手中。但杜马党团、报刊、保险基金会和工会等合法组织形式还没有完全从孟什维克手中夺取过来。为了把取消派从工人阶级的合法组织中驱逐出去,布尔什维克必须进行坚决的斗争。这斗争由于有《真理报》的努力而胜利完成了。

《真理报》是为保护党性、为重建群众性工人革命政党而斗争的核心。《真理报》把合法组织团结到布尔什维克党的地下基地周围,把工人运动指向一个确定的目标,即准备革命。

《真理报》拥有大量的工人通讯员。它在一年内就刊载了一万一千多篇工人通讯。但《真理报》不仅通过来信来稿同工人群众保持联系。每天都有很多工人从企业来到编辑部。《真理报》编辑部担负了很大一部分党的组织工作。地方党支部代表常到这里来接头。各工厂党的工作的情报都往这里送。党的彼得堡委员会和中央委员会的指示都从这里转发。

由于布尔什维克为重建群众性革命工人政党同取消派进行了两年半顽强的斗争,到1914年夏天,俄国积极的工人已有五分之四拥护布尔什维克党,拥护“真理派”的策略。举一件事就可以说明。1914年捐款支持工人报纸的七千个工人团体中,有五千六百个团体捐给布尔什维克党报刊,面捐给孟什维克报刊的只有一千四百个团体。但孟什维克在自由资产阶级和资产阶级知识分子中有很多“有钱的朋友”,他们供给了孟什维克报纸所需资金的一半以上。

当时布尔什维克被称为“真理派”。随着《真理报》成长起来了一整代革命的无产阶级,正是这一代人后来进行了十月社会主义革命。《真理报》受到几万以至几十万工人的拥护。革命高涨年代(1912—1814年)给群众性的布尔什维克党打下了坚实的基础,沙皇政府在帝国主义战争时期所采取的任何迫害都没能把这一基础摧毁。

\begin{quotation}
“1912年的《真理报》为布尔什维主义1917年的胜利奠定了基础。”(斯大林)\footnote{见《斯大林全集》第5卷104页。——译者注}
\end{quotation}

党的另一个全俄合法机关,是第四届国家杜马中的布尔什维克党团。

1912年,政府宣布举行第四届杜马选举。我们党对于参加这次选举极为重视。社会民主党的杜马党团和《真理报》是全俄范围的两大合法据点,布尔什维克党就是通过这两个据点在群众中进行自己的革命工作。

布尔什维克党带着自己的口号独立参加杜马选举,并对各个政府党和自由资产阶级(立宪民主党)同时给以打击。布尔什维克进行这次选举运动的口号是成立民主共和国,实行八小时工作制,没收地主土地。

第四届杜马选举在1912年秋天举行。10月初,政府因对彼得堡选举进程不满意,企图侵犯许多大厂工人的选举权。我们党的彼得堡委员会根据斯大林同志的提议,号召各大企业的工人罢工一天作为回答。政府陷入困境只好让步,于是工人在选举大会上有了可能选举他们愿意选举的人。绝大多数工人在表决时都赞成斯大林同志所拟定的给初选代表和杜马代表的《委托书》。《彼得堡工人给自己的工人代表的委托书》提到了1905年没有解决的任务。

\begin{quotation}
《委托书》上说:“…我们认为俄国正处在必将到来的群众运动的前夜,这一运动也许比1905年更加深入……这一运动的先锋,也象1905年一样,将是俄国社会中最先进的阶级即俄国无产阶级。它的同盟者只能是和俄国解放事业休戚相关的多灾多难的农民。”\footnote{见《斯大林全集》第2卷245—246页。——译者注}
\end{quotation}

《委托书》上说,必将到来的人民发动必定是采取在两条战线上斗争的形式,既要反对沙皇政府,又要反对同沙皇制度谋求妥协的自由资产阶级。

列宁对号召工人进行革命斗争的这个《委托书》极为重视。工人们纷纷通过决议来响应这一号召。

布尔什维克在选举中获得了胜利,巴达也夫同志由彼得堡工人选进了杜马。

工人选举杜马代表是同其他居民阶层分开进行的(即所谓工人选民团)。工人选民团选出的九名代表中,六名是布尔什维克党党员,即巴达也夫,彼得罗夫斯基、穆拉诺夫、萨莫依洛夫、沙果夫和马林诺夫斯基(后来发现他是一个奸细)。布尔什维克的代表。是至少拥有工人阶级人数五分之四的各大工业中心选出的。但几十个取消派不是工人选出的,即不是工人选民团选出的。因此在杜马里是七名取消派对六名布尔什维克。起初,布尔什维克和取消派在杜马里组成一个共同的社会民主党党团。但因取消派阻碍布尔什维克进行革命工作,所以布尔什维克代表同他们进行了顽强的斗争后,于1913年10月遵照布尔什维克党中央指示退出了联合的社会民主党党团,成立了独立的布尔什维克党团。

布尔什维克代表常在杜马里发表揭露专制制度的革命演说,并就工人遭到迫害和遭到资本家残酷剥削的事件向政府提出质询。

他们在杜马里还就土地问题发表演说,号召农民同农奴制地主作年争,揭露立宪民主党反对没收地主土地和把土地交给农民。

布尔什维克向国家杜马提出了八小时工作制法案,这个法案当然没有被黑帮杜马通过,但它起了很大的鼓动作用。

布尔什维克杜马党团同党中央、同列宁保持着密切的联系,经常从列宁那里得到指示。斯大林同志在彼得堡时直接领导过这个党团。

布尔什维克代表没有局限于在杜马内部进行工作,他们还在杜马外开展了大量活动。他们时常巡视各个工厂,到全国各工人中心去作报告,召集秘密会议来解释党的决议,成立新的党组织。代表们巧妙地把合法活动同秘密的地下工作结合起来。


\subsection[三\q 布尔什维克在合法组织中的胜利。革命运动的继续增长。帝国主义战争前夜]{三\\布尔什维克在合法组织中的胜利。\\革命运动的继续增长。\\帝国主义战争前夜}

在这个时期,布尔什维克党作出了领导无产阶级阶级斗争的各种形式和各种表现的范例。它建立地下组织。它印发秘密传单。它在群众中进行秘密的革命工作。同时,它还愈来愈多地夺得工人阶级的各种合法组织。党竭力争取工会、民众文化馆、夜大学、俱乐部和保险机关。这些合法组织向来是取消派分子的藏身处所。布尔什维克为把这些合法团体变成我们党的据点进行了坚决的斗争。布尔什维克巧妙地把秘密工作同合法工作结合起来,终于把两大首都的大多数工会组织争取到自己方面来了。布尔什维克在1913年选举彼得堡五金工会理事会时获得了特别辉煌的胜利:在二三千五金工人的大会上,只有一百五十人投取消派的票。

像第四届国家杜马中的社会民主党党团这样一个合法组织,它的情形也是如此。虽然孟什维克在杜马中有七名代表,布尔什维克只有六名代表,但孟什维克的七人小组主要是来自非工人地区,所代表的还不到工人阶级的五分之一,面布尔什维克的六人小组则是来自国内各个主要工业中心(彼得堡、莫斯科、伊万诺沃—沃兹涅先斯克、科斯特罗马、叶加特林诺斯拉夫、哈尔科夫),代表者全国工人阶级五分之四以上。工人认为自己的代表是六人小组(巴达也夫、彼得罗夫斯基等),而不是七人小组。

布尔什维克所以能争取到各种合法组织,是因为他们不管沙皇政府怎样野蛮迫害,不管取消派和托洛茨基派怎样造谣中伤,始终保持了秘密的党和自己队伍的坚强的纪律,坚定地捍卫工人阶级的利益,同群众保持着密切的联系,同工人运动的敌人进行了不调和的斗争。

这样,布尔什维克就在合法组织中获得了全面的胜利,而孟什维克在这些组织中遭到了全面的失败。无论在杜马讲坛进行鼓动方面,还是在工人报刊和其他合法组织中,孟什维克都被排挤到后而去了。卷入革命运动的工人阶级,确定不移地团结在布尔什维克周围而抛弃了孟什维克。

此外,孟什维克在民族问题上也遭到了破产。俄国各边沿地区发生的革命运动,要求有一个明确的民族问题纲领。但孟什维克除了崩得提出的那个谁也不会满意的“文化自治”外,提不出任何纲领。只有布尔什维克才提出了马克思主义的民族问题纲领,斯大林同志的文章《马克思主义和民族问题》以及列宁的文章《论民族自决权》和《关于民族问题的批评意见》阐述了这个纲领。

毫不奇怪,在孟什维主义遭到这样的失败之后,八月联盟便摇摇欲坠了。这个由各色各样分子组成的联盟,经不起布尔什维克一击,就开始土崩瓦解了。为了同布尔什维克作斗争而成立的八月联盟,很快就被布尔什维克打垮了。首先退出联盟的是前进派(波格丹诺夫、卢那察尔斯基等等),接着退出的是拉脱维亚人,然后剩下的人也散了伙。

取消派在同布尔什维克斗争中遭到失败后,就向第二国际求援。于是第二国际就来援助他们。第二国际借口要布尔什维克同取消派“和好”,借口建立“党内和平”,要求布尔什维克停止批评取消派的妥协主义政策。但布尔什维克决不调和:他们拒绝服从机会主义的笫二国际的决议,寸步也不让。

布尔什维克克在合法组织中获得胜利,不是也不可能是偶然的。其所以不是偶然的,不仅因为只有布尔什维克才有正确的马克思主义理论、明确的纲领和在战斗中锻炼出来的革命无产阶级政党。其所以不是偶然的,还因为布尔什维克的胜利反映的革命高潮正在到来。

革命工人运动一天天扩展,席卷了一批批新的城市和地区。1914年到来后,工人的罢工不仅没有逐渐平静下去。反而更加强烈地开展起来。罢工更加顽强持久,卷入罢工的工人也日益增多。1月9日有二十五万工人举行罢工,其中彼得堡有十四万。5月1口罢工人数超过了五十万。其中彼得堡有二十五万多。工人在这些罢工中表现得非常坚定。彼得堡的奥布霍夫工厂的罢工持续了两个多月,列斯涅尔工厂的罢工持续了三个月左右。彼得堡许多企业大批工人中毒的事件,激起了十一万五千工人举行罢工,接着又转为游行示威。运动继续发展着。1914年上半年(包括7月初),罢工的工人共有一百四十二万五千人。

5月间,巴库石油工业工人举行总罢工,引起了俄国全体无产阶级的密切注意。罢工进行得很有组织。6月20日,巴库有两万工人游行示威。警察用残酷手段对付巴库工人。为了抗议警察暴行并声援巴库工人,莫斯科开始举行罢工,接着罢工扩展到其他许多地区。

7月3日,彼得堡的普梯洛夫工厂为响应巴库罢工而举行了群众大会。警察向工人开枪。彼得堡无产阶级义愤填膺。7月4日,彼得堡九万工人响应党的彼得堡委员会的号召举行罢工以示抗议,7月7日罢工人数是十三万,7月8日达到了十五万,7月11日达到了二十万。

各个工厂都卷入了风潮,到处都在举行群众大会和游行示威。事情发展到构筑街垒。巴库和洛兹也构筑了街垒。在许多地方,警察向上人开枪。政府采取了“非常”手段来镇压运动,首都变成了军营,《真理报》被查封。

但这时出现了帝国主义战争这样一个国际性的新因素,结果把事变进程改变了。正当彼得堡七月革命事件发生的时候,法国总统彭加勒来到彼得堡,同沙皇谈判当前战争开始的问题。过了几天,德国就向俄国宣战了。沙皇政府利用战争来破坏布尔什维克组织和镇压工人运动。革命的高涨因世界大战爆发而中断,而沙皇政府正是想从这次战争中找到摆脱革命的出路。


\subsection{简短的结论}

在革命新高涨年代(1912—1914年),布尔什维克党领导了工人运动,并在布尔什维克口号下把它引向新的革命,党巧妙地把秘密工作同合法工作结合起来。党摧毁了取消派及其朋友托洛茨基派和召回派的反抗,因而掌握了各种形式的合法运动,并把合法组织变成了自己革命工作的据点。

党在同工人阶级的敌人及其在工人运动中的代理人作斗争中巩固了自己的队伍,扩人了自己同工人阶级的联系。党广泛地利用了杜马讲坛来进行革命鼓动,并创办了出色的群众性的工人报纸《真理报》,因而造就了新一代的革命工人,即真理派。这个工人阶层在帝国主义战争年代始终忠于国际主义和无产阶级革命的旗帜。他们后来成了布尔什维克党在1917年十月革命时期的核心。

在帝国主义战争前夜,党领导了工人阶级的革命运动。这是一种前卫战斗,它被帝国主义战争打断了(但三年后它又恢复起来,把沙皇制度推翻了)。布尔什维克党高举着飘扬招展的无产阶级国际主义旗帜进入了帝国主义战争这一艰苦的阶段。

