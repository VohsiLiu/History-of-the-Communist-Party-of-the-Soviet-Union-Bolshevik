\section[第十二章\q 布尔什维克党为完成社会主义社会建设和实施新宪法而斗争(1935—1937年)]{第十二章\\ 布尔什维克党为完成社会主义社会建设和\\实施新宪法而斗争 \\{\zihao{3}(1935—1937年)}}

\subsection[一\q 1935—1937年间的国际形势。经济危机的暂时缓和。新的经济危机的开始。意大利强占阿比西尼亚。德意两国武装干涉西班牙。日本侵入中国中部。第二次帝国主义战争的开始]{一\\ 1935—1937年间的国际形势。\\经济危机的暂时缓和。新的经济危机的开始。\\意大利强占阿比西尼亚。德意两国武装干涉西班牙。\\日本侵入中国中部。第二次帝国主义战争的开始}

1929年下半年在资本主义各国爆发的经济危机,一直继续到1933年底。然后工业低落暂时停止,危机转为停滞,接着工业开始略见活跃,略见上升。但这次上升之后并没有接着就出现在新的较高的基础上的工业繁荣。世界资本主义工业甚至不能上升到1929年的水平。到1937年年中只达到这一水平的百分之九十五至百分之九十六。而到1937年下半年又爆发新的经济危机,首先卷进去的是美国。到1937年底,美国失业人数又增加到了一千万。英国失业人数也开始迅速增加。

这样,各资本主义国家还没有从不久前的经济危机的打击下恢复过来就陷入了新的经济危机。

这种情况使各帝国主义国家间的矛盾以及资产阶级和无产阶级间的矛盾更加剧烈了。因此,各侵略国愈来愈趋向于以掠夺其他防御能力薄弱的国家来弥补国内经济危机造成的损失。并且这次除人家知道的德日两个侵略国外,还加上了第三个国家——意大利。

1935年,法西斯意大利侵犯阿比西尼亚,并征服了它。意大利侵犯阿比西尼亚,从“国际法”来看是没有任何根据或理由的,它是采取当时法西斯分子惯用的方式,不宣而战,偷偷地干的。这不但打击了阿比西尼亚,而且也打击了英国,打击了英国从欧洲到印度和亚洲的海上通道。英国想阻止意大利在阿比西尼亚站稳脚跟,没有得到什么结果。后来,意大利为了便于自由行动,就退出国际联盟,并加紧扩充军备。

于是,就在从欧洲到亚洲的最短的海上通道上结成了一个新的战争纽结。

法西斯德国用单方面的行动撕毁了凡尔赛和约,并决定实现它用暴力修改欧洲国家边界的计划。德国法西斯分子毫不隐晦地说,他们的目的是要征服邻国,至少是占领这些国家里德意志人居住的地区。这个计划预定首先占领奥地利,随后打击捷克斯洛伐克,然后也许打击波兰,那里也有整整一片与德国毗邻的德意志人居住区,然后……然后“再看分晓吧”。

1936年夏,德意两国开始对西班牙共和国实行武装干涉。意大利和德国借口援助西班牙法西斯分子,得到了可能悄悄地把自己的军队开进了西班牙境内,即开到法国背后,又把自己的舰队开进西班牙的水域——南面到巴利阿里群岛和直布罗陀一带,西面到大西洋一带,北面到比斯开湾一带。1938年初,德国法西斯分子强占了奥地利,侵入多瑙河中游,并扩展到欧洲南部,进到亚得利亚海附近。

德意法西斯分子在对西班牙进行武装干涉时向大家担保,他们在西班牙是进行反对“赤色分子”的斗争,并没有其他任何目的。但这是用来愚弄头脑简单的人的一种蠢笨拙劣的手腕。其实,他们是打击英国和法国,因为他们把英法两国通向其亚洲和非洲的广大殖民地领土的海上通道截断了。

至于强占奥地利,要说这是德国同凡尔赛条约作斗争,是德国为保护“民族”利益而力求收回因第一次帝国主义战争而丧失的领土,那已无论如何说不过去了。奥地利无论战前或战后都不是德国的领土。德国用暴力兼并奥地利,就是用蛮横的帝国主义手段侵占别国的领土。选种做法无疑暴露了法西斯德国想称霸西欧大陆的野心。

这首先是对法国和英国利益的打击。

于是就在欧洲南部,在奥地利和亚得利亚海一带。以及在欧洲最西部,在西班牙及其周围水域,结成了两个新的战争纽结。

1937年,日本法西斯军阀侵占北平,侵入中国中部,占领了上海。日本军阀侵入中国中部,也像几年前侵入东三省一样,用的是日本方式,即偷偷地干的,办法就是玩弄骗术,借日方自己制造的种种“地方事件”找岔寻衅,用实际行动破坏一切“国际的准则”、条约和协定等等。日本占领天津和上海,就抓到了同中国这一广阔市场通商的钥匙。这就是说,只要日本掌握着上海和天津,就随时都能把英美从它们有巨额投资的中国中部驱逐出去。

当然,中国人民及其军队抵抗日本侵略者的英勇斗争,中国规模巨大的民族运动的高涨,中国众多的人口和广阔的疆土,以及中国民族政府誓将中国解放斗争进行到底,直到完全把侵略者逐出中国国境的决心,——这一切都毫无疑问地说明,日本帝国主义者在中国是没有也不可能有前途的。

但另一方面,当日本还掌握着同中国通商的钥匙的时候,它对中国进行的战争实际上就严重地打击了英美的利益,这同样也是没有疑问的。于是,在太平洋,在中国一带,又形成了一个战争纽结。

所有这些事实表明,第二次帝国主义战争实际上已经开始。它是不宣而战.悄悄开始的。许多国家和民族都不知不觉地陷进了第二次帝国主义战争的漩涡。这次战争是由三个侵略国,即由德意日三国的法西斯统治集团在世界不同的角落挑起的。战争在从直布罗陀至上海这样广阔的地域内进行着。被卷入战争漩涡的已有五亿以上人口。这次战争归根到底是反对英法美的资本主义利益的,因为它的目的是要重新分割世界和势力范围,使侵略国得到利益而使这些所谓民主国家受到损害。

第二次帝国主义战争的目前特点在于,进行和开展这次战争的是几个侵略国,而其他国家,即战争锋芒所指向的“民主”国家,却装作这次战争与它们无关的样子,袖手旁观,节节退让,吹嘘自己爱好和平,责骂法西斯侵略者,并……一步一步把自己的阵地奉送给侵略者,同时却硬说它们在准备回击。

由此可见,这次战争是相当奇怪和带有单方面性质的战争。但它终究是一次残酷的和野蛮的侵略战争,是使防御能力薄弱的阿比西尼亚人民、西班牙人民和中国人民受到蹂躏的战争。

如果认为这次战争的这种单方面性质是由于各“民主”国家军事力量或经济力量薄弱,那是不正确了。各“民主”国家无疑要比法西斯国家强大。目前正在扩大的世界大战所以具有单方面的性质,是由于各“民主”国家没有结成反对法西斯国家的统一战线。这些所谓的“民主”国家当然不会赞同法西斯国家“走过了头”,并且害怕法西斯国家势力的加强。但他们更害怕欧洲工人运动和亚洲民族解放运动,认为法西斯主义是对付这一切“危险”运动的“良好消毒剂”。因此,各“民主”国家中的统治集团,特别是英国执政的保守党,只限于采取一种劝导猖狂的法西斯头目们“不要走过了头”的政策,同时还暗示,他们“完全谅解”和基本上同情法西斯头目们对工人运动和民族解放运动采取反动警察政策。英国统治集团在这里所采取的政策,大体上和俄国自由保皇派资产者在沙皇制度下所采取的政策相同,当时俄国自由保皇派资产者就是既害怕沙皇政策“走过了头”,但是更害怕人民,所以就采取劝导沙皇的政策,亦即勾结沙皇反对人民的政策。大家知道,俄国自由保皇派资产阶级因采取这种两面政策而吃了大亏。可以断定,英国统治集团及其法国和美国的朋友们也会得到历史的报应的。

苏联看到这样一种国际局势,当然不能把这种严重事变置于不顾。侵略者发动的任何一次战争,即使规模不大,也是对爱好和平国家的威胁。至于“不知不觉地”落到各国人民头上并且已包括了五亿多人口的第二次帝国主义战争,更不能不是对各国人民、首先是对苏联的极严重的威胁。德意日三国“反共联盟”的成立,就明显地说明了这一点。因此,我们国家一方面执行自己的和平政策,同时进一步加强我国边境地区的防御能力,加强红军和红海军的战斗准备。1934年底,苏联加入了国际联盟,因为苏联知道,国际联盟虽然软弱,终究可以成为一个揭露侵略者的场所,成为一种虽很软弱但多少总能阻碍战争爆发的和平工具。苏联认为,在目前这样的时候,即使是国际联盟这样一个软弱的国际组织,也不应当忽视。1935年5月,法苏两国签订了互助条约以防止侵略者可能的进攻。同时,同捷克斯洛伐克也签订了同样的条约。1936年3月,同蒙古人民共和国签订了互助条约。1937年8月,同中华民国签订了互不侵犯条约。


\subsection[二\q 苏联工农业的继续高涨。第二个五年计划的提前完成。农业的改造和集体化的完成。干部的意义。斯达汉诺夫运动。人民生活水平的提高。人民文化水平的提高。苏维埃革命的力量]{二\\ 苏联工农业的继续高涨。第二个五年计划的提前完成。\\农业的改造和集体化的完成。干部的意义。\\斯达汉诺夫运动。人民生活水平的提高。\\人民文化水平的提高。苏维埃革命的力量}

在各资本主义国家里,1930—1933年经济危机过去以后二年,又出现了新的经济危机,而苏联在整个这一时期,工业一直在继续高涨。整个世界资本主义工业到1937年年中只勉强达到1929年水平的百分之九十五至百分之九十六,并且到1937年下半年又进入了新的经济危机时期,而苏联工业却蓬勃发展。到1937年底达到了1929年水平的百分之四百二十八,比战前水平增长了六倍以上。

这些成就是党和政府坚定不移地执行改造政策的直接结果。

由于获得了这些成就,第二十五年计划在工业方面提前完成了。第二个五年计划到1937年4月1日,即在四年零三个月的时问内完成了。

这是社会主义最大的胜利。

农业方面也差不多达到同样的高涨。各种农作物的播种面积从1913年(战前)的一亿零五百万公顷增长到1937年的一亿三干五百万公顷。粮食产量从1913年的四十八亿普特增长到1937年的六十八亿普特;籽棉产量从四千四百万普特增长到一亿五千四百万普特;亚麻(纤维)产量从一千九百万普特增长到三千一百万普特;甜菜产量从六亿五千四百万普特增长到十三亿一千一百万普特;油料作物产量从一亿二千九百万普特增长到三亿零六百万普特。

应当指出,单是集体农庄(国营农场除外)在1937年供给国家的商品粮,就达到十七亿普特以上,即比地主、富农和农民在1913年所供给的总数至少多四亿普特。

农业中只有畜牧业一个部门仍然落后于战前水平,而且继续保持着缓慢的发展速度。

至于农业集体化,那么可以说已经完成了。集体农庄在1937年包括的农户达一千八百五十万户,即占农户总数的百分之九十三,而集体农庄的谷物播种面积刚已占农民的全部谷物播种面积总数的百分之九十九。

改造农业和大力供给农业以拖拉机和其他农业机器所产生的成果,已经历历在目。

由于工农业改造的完成,国民经济已装备了丰富的头等技术。工业和农业,运输业和军队,都已获得大量新的技术,大量新的机器和机床、拖拉机和农业机器、机车和轮船、大炮和坦克、飞机和军舰。必须配备几万以至几十万受过训练、能够驾驭所有这些技术并充分发挥这些技术的作用的干部。没有这一点,没有足够数量的掌握技术的人才,技术就有可能变成一堆废铁。这是严重的危险,它的造成是由于能够驾驭技术的干部增加的速度赶不上并且远远落后于技术增长的速度。而使事情复杂化的是,我们很大一部分工作人员没有意识到这种危险,认为有了技术,一切就都“迎刃而解”。从前人们是过低估计技术和鄙视技术,而现在刚是过高估计技术,把技术变成了偶像。人们不了解,没有掌握技术的人才,技术就是死的东西。人们不了解,只有有了掌握技术的人才,技术才能产生高度的生产率。

因此,掌握技术的干部问题,就有了头等重要的意义。

必须使我们的工作人员且过分迷信技术和低估干部的作用转变为注意掌握技术,精通技术,在培养大量的能够驾驭技术并充分发挥技术技能的干部方面狠下功夫。

从前,在改造时期的初期,在国内痛感缺乏技术的时候,党提出了“在改造时期,技术决定一切”的口号,现在,在技术已经很丰富;改造时期已基本结束、国内感到干部奇缺的时候,党应当提出新的口号,把大家从注意力从技术上转到人才上,转到能够充分利用技术的干部上。

在这方面,斯大林同志1935年5月在红军学院学生毕业典礼上的讲话起了巨大的作用。

\begin{quotation}
斯大林同志说“从前我们说:‘技术决定一切’。这个口号曾经帮助我们消灭了十分缺乏技术的现象,在一切工作部门里建立了极其广泛的技术基础,使我们能够用头等技术来武装我们的人才。这是很好的。但这还远远不够。为了把技术运用起来并得到充分利用,就得要有掌握技术的人才,就需要有能够精通并十分内行地运用这种技术的干部。没有掌握技术的人才,技术就是死的东西。有了掌握技术的人才,技术就能够而且一定会创造出奇迹来。如果在我们的头等工厂里,在我们的国营农场和集体农庄里,在我们的运输部门里,在我们的红军里,有足够数量的能够驾驭选种技术的干部,那么我们国家所得到的效果,就会比现有的要多两三倍。正因为如此,现在应当特别注意人才,特别注意干部,特别注意掌握技术的工作者。正因为如此。“技术决定一切”这个旧口号,反映了我们十分缺乏技术的过去的时期的口号,现在应当用新口号,用“干部决定一切”口号来代替了。这是现在的主要问题。……

毕竟应该了解:人才、干部才是世界上所有宝贵的资本中最宝贵的最有决定意义的资本。应该了解,在目前的条件下,‘干部决定一切’。如果我们在工业、农业、运输业和军队中拥有大量的优秀干部.那么我们的国家就将是不可战胜的。如果我们没有这样的干部,那我们就会寸步难移。”\footnote{见斯大林《列宁主义问题》第581页—583页。——译者注}
\end{quotation}

于是,加速培养技术干部和迅速掌握新技术来进一步提高劳动生产率,就成了头等重要的任务。

斯达汉诺夫运动是这种干部增长的最明显的例子,是我们的人才已掌握新技术和劳动生产率已进一步增长的最明显的例子。这个运动在顿巴斯,在煤炭工业中产生和开展起来,随即扩展到其他工业部门,推广到运输业,后来又普及到农业。这个运动之所以称为斯达汉诺夫运动,是因为它的发起者是“中央伊尔敏诺”矿井(顿巴斯)的采煤工人阿列克塞·斯达汉诺夫。还在斯达汉诺夫以前,尼基塔·伊点托夫已创造了前所未有的采煤纪录。1935年8月31日,斯达汉诺夫在一班工作时间内采煤一百零二吨,超过普通采煤定额十三倍。这一榜样促成了工人和集体农庄庄员提高生产定额,进一步提高劳动生产率的群众运动的诞生。汽车工业中的布塞根,制鞋工业中的斯美塔宁,运输业中的克里沃诺斯,森林工业中的穆辛斯基,纺织工业中的叶沃多基亚·维诺格拉多娃和玛丽亚·维诺格拉多娃,农业中的玛丽亚·杰姆铁科、玛丽娜·格娜田科、普·安格林娜、波拉古亭、科列索夫、科瓦尔达克和波林等,就是斯达汉诺夫运动第一批先驱者的名字。

继他们而起的还有其他先驱者,还有超过第一批先驱者的劳动生产率的整批整批先驱者。

1935年11月在克里姆林宫举行的全苏斯达汉诺夫工作者第一次会议,以及斯大林同志在这次会议上的讲话,对斯达汉诺夫运动的开展起了重大的作用。

\begin{quotation}
斯大林同志在这次讲话中说:“斯达汉诺夫运动表现了社会主义竞赛的新高涨,表现了社会主义竞赛的新的更高的阶段,……在过去,在三年以前,在社会主义竞赛的第一个阶段的时期内,社会主义竞赛并不一定和新技术相联系。而且在当时,我们本来也几乎没有什么新技术。相反,社会主义竞赛的现阶段,即斯达汉诺夫运动,却一定和新技术相联系。没有新的更高的技术,就不会有斯达汉诺夫运动。在座的这些人,斯达汉诺夫、布塞根、斯美塔宁、克里沃诺斯、昔罗宁、两位维诺格拉多娃以及其他许多同志。都是新的人才,都是完全掌握了本行的技术、驾驭并推动着这种技术前进的男女工人。三年以前,我们没有这样的人才。或者是几乎没有这样的人才……斯达汉诺夫运动的意义就在于:这一运动打破了不高的旧的技术定额。而且往往超过了先进资本主义国家的劳动生产率,这样就使我国在实际上有可能更加巩固社会主义,有可能把我国变成最富裕的国家。”\footnote{见斯大林《列宁主义问题》第584页—586页。——译者注}
\end{quotation}

然后,斯大林同志在说明斯达汉诺夫工作者的工作方法和阐明斯达汉诺夫运动对于我国前进的巨大意义时说:

\begin{quotation}
“请你们仔细看看这些斯达汉诺夫工作者同志吧。这是些什么人呢?他们大都是年轻的或中年的男女工人,是有文化素养、有技术素养的人才,他们作出了准确工作和认真工作的榜样;他们在工作中善于珍惜时间的因素,他们学会了不仅用分而且用秒来计算时间。在他们中间,大多教人都学过所谓基本技术知识,而且还在继续充实自己的技术知识。他们没有某些工程师、技师和经济工作人员的那种保守主义和顽固思想;他们勇敢地前进,打破旧的技术定额,创造新的更高的技术定额;他们对我国工业领导者制定的设计能力和经济计划提出修改,他们往往补充和修改工程师和技师的意见,他们时常教导工程师和技师,并推动工程师和技师前进,因为他们是完全掌握了本行技术并善于最大限度地利用技术的人才。今天斯达汉诺夫工作者还不很多,可是明天他们一定会增加十倍,——这一点有谁能怀疑呢?斯达汉诺夫工作者是我国工业的革新家,斯达汉诺夫运动代表着我国工业的未来,它包含着工人阶级未来文化技术高涨的种子,它为我们开辟了达到很高的劳动生产率指标的唯一途径,即从社会主义过渡到共产主义所必需的,为消灭脑力劳动和体力劳动的对立所必需的劳动生产率指标的唯一途径,——这一切难道不是都很明白吗?”\footnote{见斯大林《列宁主义问题》第584页—588页。——译者注}
\end{quotation}

斯达汉诺夫运动的开展和第二个五年计划的提前完成,为进一步提高劳动者的生活水平和文化水平创造了条件。

在第二十五年计划内,工人和职员的实际工资增加了一倍以上。工资基金在1933年为三百四十亿卢布,而到1937年则已增加到八百一十亿卢布。国家的社会保险基金在1933年为四十六亿卢布,而在1937年则已增加到五十六亿卢布。但仅在1937年这一年内,国家用在工人和职员的保险方而,用在生活条件的改善和文化需要方面,用在疗养院、天然疗养地、休养所和医疗方面的经费,就大约有一百亿卢布。

在农村,集体农庄制度已最终巩固。1935年2月举行的集体农庄突击队员第二次代表大会所通过的农业劳动组合章程,以及集体农庄的全部耕地归集体农庄永久使用的规定,在这方面起了很大的促进作用。由于集体农庄制度的巩固,农村中的贫穷和生活无保障的现象已经消灭。三年以前,每个劳动日只能分到一、两公斤粮食,而现在,产粮区的大多数集体农庄庄员每个劳动日已能分到五至十二公斤粮食,许多人甚至能分到二十公斤,此外还能分到其他产品和现金。已有几百万集体农庄农户,在产粮区每年分到五百至一千五百普特粮食,在棉花、甜菜、亚麻、牲畜,葡萄酒,柑橘和蔬菜产区每年分到几万卢布的收入。集体农庄富裕起来了。建筑新粮仓和仓库已成为集体农庄农户所关心的主要问题,因为原有的贮藏室的设计只考虑到每年分到少量产品,对于现在集体农庄庄员的新的需要是十分之一也不能满足的。

1936年,鉴于人民群众生活水平提高,政府颁布了禁止堕胎的法律。同时又拟定了大规模建筑产科医院、托儿所,乳品厨房和幼儿园的计划,1936年,用于这些设施的拨款达二十一亿七千四百万卢布,而1935年只有八亿七千五百万卢布。还有一项专门的法律规定给多子女的家庭大量补助。按照这个法律,1937年支出的补助金达十亿卢布以上。

由于普及义务教育的实施和新学校的建设,人民群众的文化水平有了很大提高。全国各地展开了大规模的学校建设工作。小学和中学学生人数1914年为八百万,而1936--1937年已增加到二千八百万。高等学校学生人数1914年为十一万二千,而1936—1937年已增加到五十四万二千。

这是一次文化革命。

人民群众物质生活状况和文化水平的提高,表明了我国苏维埃革命的强大有力和不可战胜。过去的革命所以遭到灭亡,是因为它们给了人身自由之后没有可能使人民的物质生活和文化生活状况得到切实的改善。它们的根本弱点就在这里。我国革命和其他一切革命不同的地方,就在于它不仅使人民摆脱沙皇制度、摆脱资本主义而获得了自由,并且根本改善了他们的物质生活和文化生活状况。我国革命所以有力量而且不可战胜,原因就在这里。

斯大林同志在全苏斯达汉诺夫工作者第一次会议上讲话时说:

\begin{quotation}
“在世界上,只有我国的无产阶级革命才不仅向人民显示了自己的政治成果,而且显示了自己的物质成果。在过去的一切工人革命中,我们知道,只有一次革命勉强得到过政权。这就是巴黎公社。但巴黎公社没有存在多久。固然,它也企图打破资本主义的枷锁,但是没有来得及打破,更没有来得及向人民显示革命的幸福生活的物质成果。只有我国革命才不仅打破了资本主义的枷锁,给了人民自由,而且给人民创造了富裕生活的物质条件。我国革命所以有力量而且不可战胜,原因就在这里。”\footnote{见斯大林《列宁主义问题》第590页—591页。——译者注}
\end{quotation}


\subsection[三\q 苏维埃第八次代表大会。苏联新宪法的通过]{三\\苏维埃第八次代表大会。\\苏联新宪法的通过}

1935年2月,苏维埃社会主义共和国联盟苏维埃第七次代表大会决定修改1924年通过的苏联宪法。苏联宪法必须修改,是因为从1924年以来,即从通过第一个苏联宪法以来,苏联生活中发生了许多重大的变化。在过去那些年代里,苏联阶级力量的对比已经完全改变:新的社会主义的工业已经建立起来,富农已被消灭,集体农庄制度已经获得胜利,生产资料社会主义所有制已在整个国民经济中确立起来,成为苏维埃社会的基础。由于社会主义的胜利,已有可能使选举制度更加民主化,已有可能实现普遍、平等,直接和无记名投票的选举制度。

由斯大林同志主持的专门的宪法委员会,拟定了苏联新宪法的草案。草案经过了五个半月之久的全民讨论。宪法草案被提交苏维埃第八次非常代表大会讨论。

1930年11月,召开了苏维埃第八次代表大会,大会的任务是批准或表决苏联新宪法草案。

斯大林同志在苏维埃第八次代表大会上作关于新宪法草案的报告,叙述了自1924年通过的宪法以来苏维埃国家内发生的基本变化。

1924年宪法是在新经济政策第一个时期制定的。当时,苏维埃政权还容许在发展社会主义的同时发展资本主义。当时,苏维埃政权打算在资本主义和社会主义两个体系竞赛过程中组织和保证社会主义在经济方面对资本主义的胜利。当时,“谁战胜准”的问题还没有解决。建立在旧的贫乏的技术基础上的工业,还没有达到战前水平。当时农业更不成样子。国营农场和集体农庄只不过是个体农户汪洋大海中的一些零星小岛。当时还不是要消灭富农,而只是限制富农。在商品流转方面,社会主义成分还只占百分之五十左右。

1936年苏联的情况已经不同了。到1936年,苏联经济已经完全改观了。到这时候,资本主义成分已完全消灭,社会主义体系已在国民经济一切部门获得了胜利。强大的社会主义工业的产量已超过战前六倍,并完全排挤了私人工业。在农业方面,世界上规模最大的,用新技术装备起来的机械化的社会主义生产——集体农庄和国营农场体系,已获得了胜利。到1936年,富农阶级已完全消灭了,而个体成分在国家的经济中已不起什么重大作用。全部商品流转已集中在国家和合作社手中。人剥削人的现象已被永远铲除。生产资料的社会主义公有制已在国民经济各部门确立起来,成为社会主义新制度的不可动摇的基础。在社会主义新社会中已永远消灭危机、贫困、失业和破产。已为苏维埃社会的全体成员过富裕而有文化的生活创造了条件。

斯大林同志在报告中说苏联居民的阶级成分也相应地发生了变化。地主阶级和旧时的帝国主义大资产阶级,在国内战争时期就被消灭了。在社会主义建设年代又消灭了所有的剥削分子——资本家,商人,富农和投机分子。现在只留下了已被消灭的剥削阶级的少数残余,而完全消灭这些残余不过是最近时期的问题。

在社会主义建设年代,苏联劳动者——工人、农民和知识分子,已经起了深刻的变化。

工人阶级已不再像资本主义制度下那样是被剥夺了生产资料的被剥削阶级。它已消来了资本主义,从资本家手中夺得了生产资料而把它变成了公有财产。它已不是原来的旧的意义上的无产阶级了。掌握国家政权的苏联无产阶级已变成一个崭新的阶级。它已变成摆脱了剥削、消灭了资本主义经济体系和确立了生产资料社会主义所有制的工人阶级,即人类历史上从来没有过的工人阶级。

苏联农民的情况也同样发生了深刻的变化。旧时代有两千多万分散的个体农户——小农户和中等农户,都是单独在自己的份地上辛辛苦苦地耕作。他们使用落后的技术,受地主,富农,商人,投机分子和高利贷者等等的剥削。现在,在苏联成长起来的是崭新的农民,因为那些剥削农民的地主、富农、商人和高利贷者已不存在了。绝大多数农户已加入集体农庄,集体农庄的基础不是生产资料私有制,而是在集体劳动基础上成长起来的集体所有制。这是摆脱了一切剥削的新型农民。这样的农民也是人类历史上从来没有过的。

苏联知识分子也发生了变化。苏联知识分子就其大多数来说,已经是崭新的知识分子。他们大多数是工农出身。他们不像旧知识分子那样为资本主义服务,而是为社会主义服务。知识分子已成为社会主义社会的平等的一员。这些知识分子同工人和农民一起建设着社会主义新社会。这是为人民服务的,摆脱了一切剥削的新型知识分子。这样的知识分子是人类历史上从来没有过的。

这样,苏联劳动者之间的阶级界限正在消除,旧的阶级特殊性正在消失。工人、农民,知识分子之间在经济上和政治上的矛盾正在缩小和消除。于是就造成了社会在道义上政治上一致的基础。

苏联生活中的这些深刻变化,社会主义在苏联获得的这些有决定意义的成就,都在苏联新宪法中得到了反映。

按照这个宪法,苏维埃社会是由工人和农民这两个友好的阶级组成,工人和农民之间还存在着阶级差别。苏维埃社会主义共和国联盟是工农社会主义国家。

苏联的政治基础是劳动者代表苏维埃,它由于推翻地主资本家政权和争得无产阶级专致而得以成长壮大。

苏联全部政权属于体现为劳动者代表苏维埃的城乡劳动者。

苏联的最高国家权力机关是苏联最高苏维埃。

苏联最高苏维埃由平等的两院——联盟院和民族院组成,由苏联公民按普遍、直接、平等和无记名投票的选举制度选出,任期四年。

苏联最高苏维埃的选举像各级劳动者代表苏维埃的选举一样,是普遍的。这就是说,所有年满十八岁的苏联公民,不分种族和民族,不分信仰、教育程度、居住期限、社会出身、财产状况以及过去的活动,都有选举代表和被选为代表的权利,只有精神病患者和由法庭判决褫夺选举权者除外。

代表的选举是平等的;这就是说,每个公民都有一票选举权。

所有公民都按平等原则参加选举。

代表的选举是直接的。这就是说,各级劳动者代表苏维埃,从村的和市的劳动者代表苏维埃到苏联最高苏维埃,都采用直接选举的办法由公民直接选出。

苏联苏维埃,苏维埃在两院联席会议上选出苏联最高苏维埃主席团和苏联人民委员会。

苏联的经济基础是社会主义经济体系和生产资料社会主义所有制。在苏联实行“各尽所能,按劳分配”的社会主义原则。

所有苏联公民都享有劳动权、休息权、受教育权,享有在年老以及患病和丧失劳动能力时获得物质保证的权利。

妇女在一切活动方面都享有同男子平等的权利。

苏联公民不分民族和种族一律平等,是确定不变的法律。

一切公民都有信仰自由和进行反宗教宣传的自由。

为了巩固社会主义社会,宪法保障言论、出版、集会和召开群众大会的自由,保障结成各种社会组织的权利,保障人身不受侵犯,住宅不受侵犯和通信秘密,保障因维护劳动群众利益或进行科学活动或进行民族解放斗争而受迫害的外国公民有居留权。

同时,新宪法责成苏联一切公民履行下列重要义务;遵守法律,遵守劳动纪律,忠实地履行社会义务,尊重社会主义公共生活规则,维护和巩同社会主义公有制,保卫社会主义祖国。

\begin{quotation}
“保卫祖国是每个苏联公民的神圣职责。”
\end{quotation}

关于公民组成各种团体的权利,有一条宪法条文说:

\begin{quotation}
“工人阶级和其他劳动阶层中最积极最觉悟的公民,则结成苏联共产党(布尔什维克),即劳动者为巩同和发展社会主义制度而斗争的先锋队,劳动者的所有社会组织和国家机关的领导核心。”
\end{quotation}

苏维埃第八次代表大会一致赞同和批准了苏联新宪法草案。

这样,苏维埃国家就有了新的宪法——社会主义和工农民主取得胜利的宪法。

这样,宪法就明文记载了一件具有全世界历史意义的事实,即苏联已进入新的建设时期。进入完成社会主义社会建设并逐渐过渡到以“各尽所能,按需分配”的共产主义原则为社会生活准则的共产主义社会的时期。


\subsection[四\q 布哈林派—托洛茨基派特务、暗害分子和叛国者余孽的被消灭。苏联最高苏维埃选举的任务。党的扩大。党内民主的方针。苏联最高苏维埃的选举]{四\\布哈林派—托洛茨基派特务、暗害分子\\和叛国者余孽的被消灭。\\苏联最高苏维埃选举的任务。\\党的扩大。党内民主的方针。\\苏联最高苏维埃的选举}

1937年,发现了属于布哈林派—托洛茨基派匪帮的一群恶棍的新材料。对皮达可夫、拉狄克等人案件的审判,对图哈切夫斯基、亚基尔等人案件的审判,以及对布哈林、李可夫、克列斯廷斯基、罗晋哥里茨等人案件的审判,都表明布哈林派和托洛茨基派早已结成一个与人民为敌的共同匪帮,即所谓“右倾分子—托洛茨基派联盟”。

审判表明,这些人类渣滓从十月社会主义革命最初几天起,就已和人民公敌托洛茨基、季诺维也夫和加米涅夫勾结起来阴谋反对列宁,反对党,反对苏维埃国家。在1918年初进行破坏布列斯特和约的挑衅尝试,在1918年春阴谋反对列宁,勾结“左派”社会革命党人图谋逮捕和杀害列宁、斯大林、斯维尔德洛夫;在1918年夏恶毒地向列宁开枪,使列宁受伤;在1918年夏发动了“左派”社会革命党人的叛乱;在1921年为了从内部动摇和推翻列宁的领导而故意使党内意见分歧尖锐化;在列宁患病时和列宁逝世后企图推翻党的领导,出卖国家机密,向外国间谍机关提供情报;凶杀基洛夫;进行暗害、破坏和爆炸;凶杀明仁斯基、古比雪夫和高尔基——所有这些以及诸如此类的罪行,二十年来原来都是在托洛茨基、季诺维也夫、加米涅夫、布哈林、李可夫及其走狗的参加或领导下,遵照外国资产阶级间谍机关的指示干出来的。

审判表明,托洛茨基派—布哈林派恶棍们遵照他们的主子外国资产阶级间谍机关的旨意,企图破坏党和苏维埃国家,破坏国防,帮助外国进行武装干涉,准备使红军遭受失败,让苏联被肢解,把苏联的沿海边区割让给日本,把苏联的白俄罗斯割让给波兰,把苏联的乌克兰割让给德国,消灭工人和集体农庄庄员所获得的成果,在苏联恢复资本生产奴隶制。

这些像虫子一样软弱无力的白卫小丑,俨然以国家主人自居(真可笑),竟以为他们真能把乌克兰,白俄罗斯和沿海边区割让和出卖给敌人。

这些白卫虫子忘记了,苏维埃国家的主人是苏联人民,而李可夫、布哈林、季诺维也夫、加米涅夫之流的老爷不过暂时窃据了国家职务,而这个国家是随时都可以把他们从办公室里当作废物扔出去的。

这些微不足道的法西斯分子奴仆忘记了,苏联人民只要动一动指头,就能把他们变成齑粉。

苏联法庭把布哈林派—托洛茨基派恶棍们判处了死刑。

内务人民委员部执行了这一判决。

苏联人民对消灭布哈林派―托洛茨基派匪帮表示赞许,接着转入当前的任务。

当前的任务就是准备苏联最高苏维埃的选举,并有组织地进行这次选举。

党用全力开展了这次选举的准备工作。党认为苏联新宪法的实施是国内政治生活中的一个转变。党认为这个转变就表现为选举制度完全民主化,从有限制的选举过渡到普遍的选举,从不完全平等的选举过渡到平等的选举,从多级的选举过渡到直接的选举,从公开投票的选举过渡到秘密投票的选举。

在新宪法实施以前,对僧侣、过去的白卫分子、过去的富农以及不从事有益于社会的劳动的人的选举权是有限制的,新宪法取消了对这几类公民的选举权的一切限制,使代表的选举成了普遍的选举。

从前代表的选举是不平等的,因为城市居民和农村居民的选举名额不同;现在选举的这种限制已没有必要,一切公民都有平等参加选举的权利。

从前苏维埃政权的中级和高级机关的选举是多级的;现在新宪法规定,各级苏维埃,从村苏维埃和市苏维埃到最高苏维埃,都采用直接选举的办法由公民直接选出。

从前选举苏维埃代表采取公开投票,并按整个名单投票,现在选举代表则应当采用无记名投票,并按选区提出的候选人逐个投票而不按整个名单投票。

这是国内政治生活中一个明显的转变。

新选举制度应当是而且确实是提高了群众的政治积极性,加强了群众对苏维埃政权机关的监督.加强了苏维埃政权机关对人民负责的精神。

为了有充分准备地迎接这个转变。党应当领导这个转变,完全保证自己在即将举行的选举中起领导作用。要做到这一点,就必须使各级党组织本身在自己的实际工作中成为彻底民主的组织,使它们在自己的党内生活中贯彻党章所要求的民主集中制原则,使各级党的机关都按选举产生,使批评和自我批评在党内充分开展起来,使党组织对党员群众完全负责,并使党员群众的积极性充分发挥出来。

1937年2月底,日丹诺夫同志在中央全会上就各级党组织作好准备以迎接苏联最高苏维埃选举的问题作了报告,其中指出,许多党组织在自己的实际工作中往往违背党章和民主集中制原则,用委派代替选举,用按整个名单投票代替按候选人逐个投票,用公开投票代替秘密投票等等。显然,采用这种做法的组织不可能在最高苏维埃选举中完成自己的任务。目此,必须首先消除党组织中这种反民主的做法,并在扩大民主的基础上改造党的工作。

因此,中央全会听取了日丹诺夫同志的报告后决定:

\begin{quotation}
(一)在无条件实现和彻底实现党章所规定的党内民主原则的基础上改造党的工作。

(二)取消委派党委会委员的做法,根据党章恢复各级组织领导机关按选举产生的制度。

(三)禁止在选举党机关时按整个名单投票,而要按候选人逐个投票,同时保证一切党员都有不选候选人和批评候选人的充分权利。

(四)党机关的选举采用秘密(无记名)投票的方式。

(五)各级组织都进行党机关(从基层组织的委员会起直到边区委员会、州委员会和民族共和国共产党中央委员会为止)的选举,最迟到5月20日结束。

(六)各级组织必须严格遵守党章所规定的党机关的选举期限,基层组织的党机关选举一年一次;区组织和市组织的一年一次;州组织、边区组织和共和国组织的一年半一次。

(七)基层组织必须严格遵守在全体党员大会上选举党委员会的制度,不得以代表会议代替全体党员大会。

(八)取消目前许多基层组织所通行的那种实际上是废止全体党员大会而代之以车间党员大会和代表会议的做法。\footnote{参看《苏联共产党决议汇编》第4分册第467页—468页。——译者注}
\end{quotation}

这样就开始了党对即将举行的选举的准备。

中央的这一决定具有重大的政治意义。它的意义不仅在于它揭开了党为选举苏联最高苏维埃而进行的选举运动的序幕。它的意义首先在于它帮助了各级党组织去进行自身的改造,去贯彻开展党内民主的方针,去有充分准备地迎接最高苏维埃选举。

党在开展选举运动时,决定把共产党员和非党群众结成选举同盟的思想作为自己选举政策的中心。于是党就同非党群众结成同盟,同非党群众结成联盟去进行选举,它决定同非党群众一起提出各个选区的共同候选人。这是资产阶级国家选举运动的实践中从来没有和完全不可能有的。而在我国,共产党员和非党群众的同盟却成了十分自然的现象,因为,在这里已没有了敌对的阶级,在这里各阶层人民道义上政治上的一致已是无可争辩的事实。

1937年12月7日,党中央发表了告全体选民书。其中说:

\begin{quotation}
“1937年12月12日,苏联劳动者将根据我国社会主义宪法选举苏联最高苏维埃代表。布尔什维克党同非党工人,农民、职员、知识分子结成同盟、结成联盟进行选举……布尔什维克党同非党群众并没有隔开,相反,它同非党群众结成同盟,结成联盟进行选举,同工人和职员的工会,同共青团和其他非党的组织和团体结成同盟进行选举。因此,将要提出的代表候选人对于共产党员和非党群众双方都是共同的,每个非党的代表也就是共产党员推举的代表,同样,每个党员代表也就是非党群众推举的代表。”
\end{quotation}

中央委员会告选民书在结束时向选民发出如下号召:

\begin{quotation}
“苏联共产党(布尔什维克)中央委员会号召全体共产党员和同情者像选举党员候选人那样一致投票选举非党的候选人。

苏联共产党(布尔什维克)中央委员台号召全体选民在1937年12月12日那天,都到选举箱前选举联盟院代表和民族院代表。

每个选民都应当行使其选举苏维埃国家最高机关代表的光荣权利。

每个积极公民都应当把促进所有一切选民参加最高苏维埃选举当作自己的公民职责,1937年12月12日应当成为苏联各族劳动者团结在列宁、斯大林胜利旗帜周围的伟大节日。”
\end{quotation}

1937年12月12日,即在选举前夕,斯大林同志在他被提名的选区发表演说,他在演说中讲到人民的使者,苏联最高苏维埃代表应当是怎样的活动家时说:

\begin{quotation}
“选民,人民,应当要求自己的代表始终胜任自己的任务;要求他们在自己的工作中不堕落为政治上的庸人;要求他们始终不愧为列宁式的政治活动家;要求他们成为像列宁那样时明朗和确定的活动家;要求他们像列宁那样在战斗中无所畏惧和对人民的敌人毫不留情;要求他们在事情开始复杂化在地平线上出现某种危险的时候,毫不惊慌失措;毫无任何类似惊慌失措的现象,要求他们也像列宁那样没有任何类似惊慌失措的迹象;要求他们在解决复杂问题、需要生面地确定方针、全面地考虑事情的正反方面的时候,也能够像列宁那样英明和从容;要求他们也像列宁那样诚实和正直,要求他们像列宁那样热爱自己的人民。”
\end{quotation}

12月12日,举行了苏联最高苏维埃选举。选举是在巨大的热潮中进行的。这不是平常的选举,而是伟大的节日,是苏联人民的盛典,是苏联各族人民伟大友谊的显示。

在九千四百万选民中,参加这次选举的有九千一百余万人,即占选民总数百分之九十六点八。其中有八千九百八十四万四千人,即百分之九十八点六的选民,都投票选举共产党员和非党群众的同盟提出的候选人。只有六十三万二千人,即不到百分之一的选民,投票反对共产党员和非党群众的同盟所提出的候选人。共产党员和非党群众的同盟所提出的候选人,无一例外地全部当选。

这样,九千万人用一致的投票证实了社会主义在苏联的胜利。

这是共产党员和非党群众的同盟获得的辉煌胜利。

这是布尔什维克党的光辉胜利。

莫洛托夫同志在纪念十月革命二十周年的有历史意义的演说中所说的苏联人民道义上政治上的一致,在这里得到了光辉的证实。


