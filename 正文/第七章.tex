\section[第七章\q 布尔什维克党在准备和进行十月社会主义革命时期(1917年4月—1918年)]{第七章\\ 布尔什维克党在准备和进行\\十月社会主义革命时期 \\{\zihao{3}(1917年4月—1918年)}}

\subsection[一\q 二月革命后的国内状况。党走出地下状态转向公开政治活动。列宁回到彼得格勒。列宁的四月提纲。党的向社会主义革命过渡的方针]{一\\二月革命后的国内状况。\\党走出地下状态转向公开政治活动。\\列宁回到彼得格勒。列宁的四月提纲。\\党的向社会主义革命过渡的方针}

事变和临时政府的行为,日益证明布尔什维克的路线正确。它们愈来愈清楚地表明:临时政府不支持人民而反对人民,不主张和平而主张战争;临时政府不愿意而且也不可能给予和平、土地和面包。布尔什维克的解释工作获得了适宜的土壤。

工人和士兵推翻了沙皇政府和消灭了君主制的根基,而临时政府则分明想把君主制保存下去。1917年3月2日,它密派古契柯夫和叔尔根去见沙皇。资产阶级想把政权转交给尼古拉·罗曼诺夫的弟弟米哈伊尔。但是当古契柯夫在铁路工人的集会上演说完毕高呼“米哈伊尔皇帝万岁”的时候,工人们要求立即逮捕和搜查古契柯夫,并忿然说道:“洋姜不比萝卜甜。”

很清楚,工人是决不容许恢复君主制的。

工人和农民干革命、洒热血,是盼望结束战争,想要获得面包和土地,要求采取坚决措施来消除经济破坏现象;而临时政府却对人民的这些切身要求置若罔闻。这个由资本家和地主的最有名的代表组成的政府,根本不想满足农民关于把土地转归他们的要求。同样,它也不可能给劳动者以面包,因为要做到这点,就得触犯大粮商的利益,就得用各种办法征收地主和富农的粮食,但政府是下不了决心这样做的,因为它本身就同这些阶级的利益分不开。同样,它也不可能给予和平。同英法帝国主义者勾结的临时政府,不仅不想停止战争,反而企图利用革命来使俄国更加积极地参加帝国主义战争,来实现其占领君士坦丁堡和两个海峡、占领加里西亚的帝国主义计划。

很清楚,人民群众对临时政府的政策所持的轻信态度很快就要完结了。

显然,二月革命后所形成的两个政权并存的局面不可能长久支持下去,因为事变进程要求政权集中到一方面:或是集中到临时政府宫墙内,或是集中到苏维埃手中。

诚然,孟什维克和社会革命党人的妥协主义政策,在人民群众中暂时还受到支持。当时还有不少工人,尤其是士兵和农民,相信“很快就会有立宪会议来把一切安排妥贴”,以为进行战争不是为了侵略,而是为了保卫国家的需要。列宁把这种人叫作诚心诚意误入迷途的护国派。当时这些人还把社会革命党人和孟什维克的许诺和劝说政策看作是正确的政策。但是很清楚,靠许诺和劝说是不可能支持多久的,因为事变进程和临时政府的行为日益暴露和表明,社会革命党人和孟什维克的妥协主义政策是拖延时日和欺骗轻信者的政策。

临时政府没有局限于暗中反对群众革命运动的政策,即用阴谋手段反对革命的政策。有时它也企图公开向民主自由进攻,企图“恢复纪律”(特别是在士兵中),企图“整顿秩序”,即把革命纳入资产阶级所需要的轨道。但不管它怎样朝这方面努力,总是不成功。人民群众照样起劲地实现着言论、出版、结社、集会、游行示威等民主自由。工人和士兵力求充分利用他们第一次争得的民主权利来积极参加国家的政治生活,以便认识和理解目前的局势,并决定今后如何行动。

二月革命后,曾在沙皇制度的极其困难条件下秘密活动的布尔什维克党组织走出了地下状态,开始进行公开的政治和组织活动。当时,布尔什维克组织的人数不过四万至四万五千。但这是在斗争中受过锻炼的干部。各级党委会已按民主集中原则实行了改组。确立了全党上下各级机关都按选举产生的制度。

党一转到合法状态,党内的意见分歧就暴露出来了。加米涅夫和莫斯科组织的某些工作人员,例如李可夫、布勃诺夫和诺根,采取半孟什维主义的立场,有条件地支持临时政府和护国派的政策。斯大林(他当时刚从流放地回来)和莫洛托夫等人,同党内多数同志一起.坚持不信任临时政府的政策,反对护国主义,号召积极地为争取和平、反对帝国主义战争而斗争。一部分党的工作人员态度动摇,反映出他们由于长期蹲监狱、被流放而造成的政治上的落后。

处处感觉到党的领袖列宁不在。

1917年4月3日(16日),列宁经过长期流亡以后,同到了俄国。

列宁的归来,对党、对革命有着巨大的意义。

列宁还在瑞士接到革命的最初消息时,就在从那里寄发的《远方来信》中向党和俄国工人阶级写道:

\begin{quotation}
“工人们!你们在反对沙皇制度的国内战争中,显示了无产阶级的人民的英雄主义的奇迹,现在你们应该显示出无产阶级的和全体人民的组织的奇迹,准备在革命的第二阶段上取得胜刺。”(《列宁全集》俄文第3版第20卷第19页)\footnote{见《列宁选集》第2版第3卷第11页。——译者注}
\end{quotation}

列宁于4月3日夜到达彼得格勒。当时在芬兰车站和车站前的广场上,聚集了成千上万的工人、士兵和水兵欢迎列宁。列宁一下车,群众就狂热地欢腾起来。他们把列宁举在手上,就这样把自己的领袖举到车站大厅。孟什维克齐赫泽和斯柯别列夫开始以彼得格勒苏维埃的名义致“欢迎”词,“表示希望”列宁会找到和他们“共同的语言”。但列宁没有理会他们,而绕过他们走向工人和士兵群众,并从装甲车上发表了他有名的演说,号召群众为社会主义革命的胜利而斗争。“社会主义革命万岁!”——列宁这样结束他经过多年流亡回来后的第一个演说。

列宁一回到俄国就全力以赴地投入革命工作。回国后的第二天,列宁在布尔什维克的会上作了关于战争和革命的报告,接着又在一个除布尔什维克外还有孟什维克参加的会上重述了自己报告的提纲。

这就是列宁著名的四月提纲,这个提纲向党和无产阶级提出了从资产阶级革命过渡到社会主义革命的明确的革命路线。

列宁的提纲对革命、对党后来的工作有巨大的意义。革命是全国生活中最大的转折,所以党在推翻沙皇制度后新的斗争条件下,必须有一个新的方针,以便能大胆而有信心地循着新的道路前进。列宁的提纲向党提供了这样的方针。

列宁的四月提纲向党提出了一个争取从资产阶级民主革命过渡到社会主义革命、从革命第一阶段过渡到第二阶段即社会主义革命阶段的天才计划。党自己所经历的全部历史,已使党准备好来执行这一伟大任务,早在1905年,列宁已在《社会民主党在民主革命中的两种策略》一书中说过,在推翻沙皇制度以后,无产阶级将进而实现社会主义革命。提纲中的新东西,就是提出了一个有理论根据的着手向社会主义革命过渡的具体计划。

在经济方面,过渡的办法是:在没收地主土地的情况下把全国所有土地收归国有,把所有的银行合并成一个国家银行并由工人代表苏维埃加以监督,对社会的产品生产和分配实行监督。

在政治方面,列宁提出由议会制共和国过渡到苏维埃共和国。这是在马克思主义的理论和实践方面迈出的重大的一步。在此以前,马克思主义的理论家们认为议会制共和国是向社会主义过渡的最好的政治形式。现在列宁提出用苏维埃共和国来代替议会制共和国,认为苏维埃共和国是从资本主义到社会主义的过渡时期对社会最适宜的政治组织形式。

\begin{quotation}
提纲中说:“目前俄国的特点是从革命的第一阶段过渡到革命的第二阶段,第一阶段由于无产阶级的觉悟性和组织性不够,政权落到了资产阶级手中,第二阶段则应当使政权转到无产阶级和贫苦农民阶层手中。”(《列宁全集》俄文第3版第20卷第88页)\footnote{见《列宁选集》第2版第3卷第14页。——译者注}
\end{quotation}

又说:

\begin{quotation}
“不要议会制共和国(从工人代表苏维埃回到议会制共和国,是倒退了一步),而要从下到上由全国的工人、雇农和农民代表苏雏埃组成的共和国。”(同上,第88页)\footnote{同上,第15页。——译者注}
\end{quotation}

列宁说,在新政府即临时政府的统治下,战争仍然是掠夺性的、帝国主义的战争。党的任务是要向群众说清这一点,并向他们指明,要用真正民主的非强制的和平来结束战争,就非推翻资产阶级不可。

关于对临时政府的态度,列宁所提出的口号是:“不给临时政府任何支持!”

其次,列宁在提纲中指出,我们党在苏维埃中暂时还占少数,在那里占统治地位的是向无产阶级传播资产阶级影响的孟什维克和社会革命党人的联盟。因此,党的任务是:

\begin{quotation}
“要向群众说明:工人代表苏维埃是革命政府唯一可能的形式,因此,当这个政府还受资产阶级影响时,我们的任务只能是耐心地、经常地、坚持不懈地、特别要根据群众的实际需要来说明他们的策略的错误。只要我们还是少数,我们就要进行批评,揭发错误,同时宣传全部政权归工人代表苏维埃的必要性……”(《列宁全集》俄文第3版第20卷第88页)\footnote{见《列宁选集》第2版第3卷第15页。——译者注}
\end{quotation}

这就是说,列宁并没有号召实行起义去反对当时得到苏维埃信任的临时政府,没有要求推翻它,而是力求用解释的和征集力量的工作来争得苏维埃中的多数,改变苏维埃的政策,通过苏维埃去改变政府的成分和政策。

这是和平地发展革命的方针。

其次,列宁要求抛弃“肌脏的对衫”,即放弃社会民主党这一名称。第二国际各党和俄国孟什维克也自称为社自民主党人。这个名称已被机会主义者和社会主义叛徒们玷污、糟蹋了。列宁建议像马克思和恩格斯称呼自己的党那样,称呼布尔什维克党为共产党。这个名称在科学上是正确的,因为布尔什维克党的最终目的是要达到共产主义。人类从资本主义只能直接过渡到社会主义,即过渡到生产资料公有和按各人的劳动分配产品。列宁说,我们党看得更远些。社会主义必然会逐渐成长为共产主义,而在共产主义的旗帜上写的是:“各尽所能,按需分配”。

最后,列宁在提纲中要求建立新的国际,建立没有沾染机会主义,没有沾染社会沙文主义的第三国际,即共产国际。

列宁的提纲引起了资产阶级、孟什维克和社会革命党平人的疯狂叫嚣。

孟什维克向工人发表了一篇宣言,开头就警告说:“革命在危险中”。所谓危险,据孟什维克的意见,就是布尔什维克提出了政权归工兵代表苏维埃的要求。

普列汉诺夫在自己的《统一报》上发表了一篇文章,把列宁的演说叫做“梦话”。普列汉诺夫引用孟什维克齐赫泽的活说:“只有列宁一人仍将处于革命之外,而我们将走我们自己的路。”

4月14日,召开了布尔什维克彼得格勒市代表会议。会议赞同列宁的提纲,并把它作为自己工作的基础。

此后不久,党的各个地方组织也赞同了列宁的提纲。

全党,除加米涅夫、李可夫、皮达可夫之流几个独夫外,都非常满意地接受了列宁的提纲。


\subsection[二\q 临时政府危机的开始。布尔什维克党四月代表会议]{二\\临时政府危机的开始。\\布尔什维克党四月代表会议}

当布尔什维克准备进一步开展革命时,临时政府却继续干着反人民的勾当。4月18日,临时政府外交部长米留可夫向盟国声明:“全体人民愿将世界大战进行到彻底胜利,临时政府决意完全遵守我们对盟国承担的义务。”

这样,临时政府发誓忠于沙皇条约,并且许诺说,帝国主义者为了达到“最后胜利”还需要人民流多少血,它就让流多少血。

4月19日,这个声明(“米留可夫照会”)已为工人和士兵们知道了。4月20日,布尔什维克党中央委员会号召群众抗议临时政府的帝国主义政策。1917年4月20—21日(5月3—4日),对“米留可夫照会”极表愤慨的工人和士兵群众,至少有十万人,举行了游行示威。旗帜上写的口号是:“公布秘密条约!”“打倒战争!”“全部政权归苏维埃!”工人和士兵自城郊走向市中心,走向临时政府所在地。在涅瓦大街和其他地方,他们和几群资产阶级分子发生了冲突。

科尔尼洛夫将军之流最露骨的反革命分子号召向示威者开枪,甚至下过这样的命令。但军队接到这种命令后拒绝执行。

党的彼得格勒委员会里的少数委员(巴格达齐也夫等)在游行示威时提出了立刻推翻临时政府的口号。布尔什维克党中央委员会严厉地谴责了这种“左的”冒险主义者的行为,认为这个口号不合适、不正确,妨碍党把苏维埃的多数争取到自己方面来,而且同党的和平地发展革命的方针相抵触。

4月20—21日事件意味着临时政府危机的开始。

这是孟什维克和社会革命党人的妥协主义政策中发生的第一个严重裂口。

1917年5月2日,在群众的压力下,从临时政府中撤销了米留可夫和古契柯夫。

成立了第一届联合临时政府,参加的除资产阶级代表外,还有孟什维克(斯柯别列夫和策烈铁里)和社会革命党人(切尔诺夫和克伦斯基等人)。

这样,在1905年否认社会民主党人代表可以参加革命的临时政府的孟什维克,现在却认为自己的代表可以参加反革命的临时政府了。

这表示孟什维克和社会革命党人转到反革命资产阶级营垒去了。

1917年4月24日,布尔什维克第七次(四月)代表会议开幕。这是有党以来第一次公开举行的布尔什维克代表会议,这次会议就其意义来说在党史上等于一次党代表大会。

全俄四月代表会议表明了党的蓬勃发展。出席会议的有一百三十三名有表决权的代表和十八名有发言权的代表,代表着八万名有组织的党员。

会议经过讨论,确定了党在战争和革命的一切基本问题(关于目前形势、关于战争、关于临时政府、关于苏维埃、关于土地问题、关于民族问题等等)上的路线。

列宁在报告中发挥了他先前在四月提纲中所阐述的原理。党的任务是要从革命的第一阶段过渡“到革命的第二阶段,第一阶段……政权落到了资产阶级手中,第二阶段则应当使政权转到无产阶级和贫苦农民阶层手中”(列宁)\footnote{见《列宁选集》第2版第3卷第14页。——译者注}。党应该采取准备社会主义革命的方针。列宁提出了“全部政权归苏维埃!”的口号作为党的当前任务。

“全部政权归苏维埃”的口号意味着必须结束两个政权并存的局面,即结束临时政府和苏维埃分掌政权的局面,必须使全部政权归苏维埃,而将地主资本家的代表驱逐出政权机关。

会议确认,党的最重要任务之一,是要不倦地向群众说明这样一个真相,即“临时政府按其性质来说,是地主资产阶级的统治机关”\footnote{见《苏联共产党决议汇编》第1分册第437页。——译者注},同时揭露社会革命党人和孟什维克的妥协主义政策的危害,指出他们是在用虚伪的诺言欺骗人民,使人民遭受帝国主义战争和反革命的打击。加米涅夫和李可夫在会上发言反对列宁。他们跟着孟什维克唱一个调子,说俄国还没有成熟到实现社会主义革命的程度,说俄国只能建立资产阶级共和国。他们建议党和工人阶级只限于“监督”临时政府。其实他们也和孟什维克一样,主张保存资本主义、保存资产阶级政权。

季诺维也夫也在会上发言反对列宁,他是在布尔什维克党应留在齐美尔瓦尔得联盟中还是同这个联盟决裂而建立新的国际这个问题上反对列宁的。正如战争年代所表明,这个联盟虽然进行和平宣传,但实际并没有同资产阶级护国派决裂。因此,列宁坚决主张立即退出这个联盟而组织新的国际,即共产国际。季诺维也夫建议仍旧同齐美尔瓦尔得派留在一起。列宁坚决驳斥了季诺维也夫的这个发言,称他的策略是“极端机会主义的和有害的”策略。

四月代表会议还讨论了土地问题和民族问题。

根据列宁关于土地问题的报告,会议通过了没收地主土地并把它交给农民委员会支配和把全国土地收归国有的决议。布尔什维克号召农民为争取土地而斗争,并向农民群众证明,布尔什维克党是真正帮助农民推翻地主的唯一的革命党。

斯大林同志关于民族问题的报告具有重大的意义。列宁和斯大林还在革命前,即在帝国主义战争前夜,就已规定了民族问题上布尔什维克党的政策的原则。列宁和斯大林说,无产阶级政党应当支持被压迫人民的反帝民族解放运动。因此,布尔什维克党坚持各民族有权自决直到分离并成立独立国家。中央的报告人斯大林同志在会上捍卫了这个观点。

发言反对列宁和斯大林的有皮达可夫,他和布哈林一起,还在战争年代就在民族问题上采取民族沙文主义的立场。皮达可夫和布哈林反对民族自决权。

党在民族问题上的坚定的和一贯的立场,党为实现民族的完全平等和为消灭一切形式的民族压迫和民族不平等而进行的斗争,保证党获得了被压迫民族的同情和支持。

下面就是四月代表会议所通过的关于民族问题决议的原文:

\begin{quotation}
“民族压迫政策是专制制度和君主制度的遗产,地主、资本家和小资产阶级支持这种政策,是为了维护其阶级特权,分化各民族的工人。现代帝国主义正在加紧征服弱小民族,它是加深民族压迫的新因素。

在资本主义社会里,要消除民族压迫,除非建立最民主的共和制度和国家管理制度,保证一切民族和语言完全平等。

必须承认俄国境内一切民族有自由分离和成立独立国家的权利。否认这种权利和不设法保证这种权利的实现,就等于拥护侵略政策或兼并政策。无产阶级只有承认民族分离权,才能保证各民族工人的充分团结,才能促进各民族真正民主的接近。……

决不允许把民族有权自由分离的问题和某一民族在某个时期实行分离是否适当的问题混为一谈。对于后面这个问题,无产阶级政党应当根据整个社会发展的利益和无产阶级争取社会主义的阶级斗争的利益,在各个不同的场合完全独立地加以解决。

党要求实行广泛的区域自治,取消自上而下的监督,废除带强制性的国语,并且根据当地居民自己对经济和生活条件、居民民族成分等等的估计,确定地方自治地区和区域自治地区的边界。

无产阶级政党坚决摈弃所谓‘民族文化自治’,就是说,反对把原来由国家管理的教育事宜等等交给本民族议会管理。民族文化自治人为地把那些在同一地方居住、甚至在同一企业做工的工人按其所属的‘民族文化’分开,就是说,使工人同本民族的资产阶级文化的联系加强起来;而社会民主党的任务是要加强全世界无产阶级的国际文化。

党要求把取消任何民族特权、不得侵犯少数民族权利作为一项基本法律包括在宪法里。

工人阶级的利益要求俄国各民族工人结成统一的无产阶级组织,如政治组织、工会组织、合作社—教育组织等等。只有各民族工人结成这种统一的组织,无产阶级才有可能胜利地进行反对国际资本、反对资产阶级民族主义的斗争。”(《联共(布)决议汇编》俄文版第1册第239—240页)\footnote{见《苏联共产党决议汇编》第1分册第446—447页。——译者注}
\end{quotation}

这样,四月代表会议就揭穿了加米涅夫、季诺维也夫、皮达可夫、布哈林、李可夫以及他们的一小撮志同道合者的机会主义的、反列宁主义的路线。

会议一致拥护列宁,在一切重大问题上采取了明确的立场,推行了争取社会主义革命胜利的路线。


\subsection[三\q 布尔什维克党在首都的成功。临时政府军队在前线进攻的失利。工人和士兵七月游行示威的被镇压]{三\\布尔什维克党在首都的成功。\\临时政府军队在前线进攻的失利。\\工人和士兵七月游行示威的被镇压}

党根据四月代表会议的决定展开了争取群众、用战斗精神教育和组织群众的巨大工作。党在这个时期的路线是,通过耐心解释布尔什维克的政策和揭穿孟什维克和社会革命党人的妥协主义,在群众中孤立这两个党,争取苏维埃中的多数。

除在苏维埃中工作外,布尔什维克还在工会和工厂委员会中进行了大量的工作。

特别是在军队中,布尔什维克进行了巨大的工作。到处都开始建立军事组织。在前线和后方,布尔什维克孜孜不倦地努力进行着士兵和水兵的组织工作。布尔什维克的前线报纸《战壕真理报》,在促使士兵革命化方面起了特别巨大的作用。

由于布尔什维克的这种宣传鼓动工作,在革命的头几个月,很多城市的工人就改选了苏维埃——特别是区苏维埃,驱逐了孟什维克和社会革命党人,选进了布尔什维克党的拥护者。

布尔什维克的工作产生了很好的效果,特别是在彼得格勒。

1917年5月30日—6月3日,召开了彼得格勒工厂委员会代表会议。在这次会上,拥护布尔什维克的代表已占四分之三。彼得格勒的无产阶级,差不多全体都拥护布尔什维克的“全部政权归苏维埃!”的口号。

1917年6月3日(16日),召开了全俄苏维埃第一次代表大会。布尔什维克当时在苏维埃中还占少数,他们在会上仅有一百多名代表,而孟什维克、社会革命党人等却有七八百名代表。

布尔什维克在苏维埃第一次代表大会上坚决地揭露了同资产阶级妥协的危害性,揭穿了战争的帝国主义性质。列宁在会上发表了演说,论证布尔什维克路线的正确,说只有苏维埃政权才能给劳动者面包、给农民土地,才能争得和平,才能使国家摆脱经济破坏状态。

这时彼得格勒各工人区正广泛酝酿着组织游行示威和向苏堆埃代表大会提出一些要求。彼得格勒苏维埃执行委员会为了防止工人自动游行示威,为了把群众的革命情绪引导来实现自己的目的,决定6月18日(7月1日)在彼得格勒举行一次游行示威。孟什维克和社会革命党人指望这次游行示威在反布尔什维克的口号下举行。布尔什维克党努力准备着这次游行示威。斯大林同志当时在《真理报》上写道:“……我们的任务是要使彼得格勒6月18日的游行示威在我们的革命口号下举行。”\footnote{见《斯大林全集》第3卷第83页。——译者注}

1917年6月18日在革命烈士墓旁举行的游行示威,成了布尔什维克党的力量的一次真正检阅。它表明群众的革命性日益高涨,对布尔什维克党的信任日益提高。孟什维克和社会革命党人提出的信任临时政府、必须继续战争的口号,淹没在布尔什维克口号的汪洋大海中了。四十万示威者举着旗帜,上面的口号是:“打倒战争!”“打倒十个资本家部长!”“全部政权归苏维埃!”

这是孟什维克和社会革命党人在首都的一次惨败,是临时政府在首都的一次惨败。

但是,临时政府因为得到了苏维埃第一次代表大会的支持,决定继续执行帝国主义政策。恰巧在6月18日这天,临时政府遵照英法帝国主义者的旨意,驱使前线士兵进攻。资产阶级认为这次进攻是结束革命的唯一机会。如果进攻胜利,资产阶级打算攫取全部政权,排挤苏维埃,摧毁布尔什维克。如果进攻失利,照样可以把全都罪过推给布尔什维克,就怪他们瓦解了军队。

进攻显然是要失败的,而它果真失败了。士兵的疲劳,他们对进攻目的的茫然不解,他们对异己的军官的不信任,炮弹和大炮的缺乏,——凡此种种,都决定了前线进攻的失败。

前线进攻以及接着进攻失败,消息传来,首都大哗。工人和士兵愤怒万分。原来,临时政府宣布和平政策是欺骗人民。原来,临时政府主张继续帝国主义战争。原来,全俄苏维埃中央执行委员会和彼得格勒苏维埃不想制止,或者说不可能制止临时政府的罪恶行动,并且自己做了临时政府的尾巴。

彼得格勒工人和士兵的革命义愤达到了极点。7月3日(16日),在彼得格勒的维波尔格区自发地开始了一次次的游行示威。这些游行示威持续了一整天。单个的游行示威发展成了总的大规模的武装游行示威,口号是政权归苏维埃。布尔什维克党本来反对在这个时候实行武装发动,因为它认为:革命危机还没有成熟;军队和外省还没有准备来支持首都的起义;在首都举行孤立的和为时过早的起义只会有助于反革命击溃革命先锋队。但是当看到阻止群众举行游行示威已不可能时,党决定参加游行示威,以便把它变成和平的和有组织的。布尔什维克党做到了这点,于是几十万示威者就向彼得格勒苏维埃和全俄苏维埃中央执行委员会进发,要求苏维埃把政权掌握在自己手里,同帝国主义资产阶级决裂而实行积极的和平政策。

虽然游行示威带有和平的性质,但还是调了反动部队(士官生和军官队伍)来对付示威者。彼得格勒街头洒满了工人和士兵的鲜血。为了击溃工人,从前线调回了最愚昧无知的反革命军队。

孟什维克和社会革命党人联合资产阶级和白卫将军镇压了工人和士兵的游行示威之后,就猛攻布尔什维克党。《真理报》编辑部被捣毁。《真理报》、《士兵真理报》和其他许多布尔什维克报纸被封闭。工人沃伊诺夫只是因为出售《真理小报》就在街上被士官生杀害。开始解除赤卫队的武装。彼得格勒卫戍部队中的革命部队被撤出首都而调往前线。在后方和前线进行逮捕。7月7日,颁布了逮捕列宁的命令。布尔什维克党的许多重要活动家遭逮捕。印刷布尔什维克出版物的“劳动”印刷所被捣毁。彼得格勒高等法院检察官发出布告,说列宁和其他许多布尔什维克因“叛国”和策划武装暴动,应受审判。加给列宁的罪名,是邓尼金将军司令部根据特务和奸细的证词伪造出来的。

这样,有策烈铁里和斯柯别列夫、克伦斯基和切尔诺夫这些孟什维克和社会革命党人的著名代表参加的联合临时政府,滚到公开的帝国主义和反革命的泥潭中去了。它不实行和平政策,而实行继续战争的政策。它不保护人民的民主权利,而实行取消这种权利并用武力摧残工人和士兵的政策。

资产阶级代表人物古契柯夫和米留可夫没有敢做的事情,“社会主义者”克伦斯基和策烈铁里、切尔诺夫和斯柯别列夫却做了。

两个政权并存的局面结束了。

结束得有利于资产阶级,因为全部政权转到了临时政府手中,而苏维埃及其社会革命党—孟什维克领导变成了临时政府的附属品。

革命和平发展时期宣告结束,因为刺刀已经提上日程。

鉴于形势政变,布尔什维克党决定改变自己的策略。党转入地下,把自己的领袖列宁深深地隐藏起来,开始准备起义,以便用武力推翻资产阶级政权和建立苏维埃政权。


\subsection[四\q 布尔什维克党准备武装起义的方针。党的第六次代表大会]{四\\布尔什维克党准备武装起义的方针。党的第六次代表大会}

在资产阶级和小资产阶级的报刊大肆攻击的形势下,布尔什维克党在彼得格勒召开了第六次代表大会。这次大会是在伦敦第五次代表大会后十年和布尔什维克布拉格代表会议后五年召开的。大会于1917年7月26日—8月3日秘密举行。报上只公布了开会的消息,没有指明开会的地点。头几次会在维波尔格区举行。后几次会在纳尔瓦门附近的一个学校举行,现在那里已经盖起了文化馆。资产阶级报纸要求逮捕大会参加者。密探疲于奔命,想找到开会的地点,但怎么也没有找到。

这样,在推翻了沙皇制度五个月以后,布尔什维克不得不秘密开会,而无产阶级政党的领袖列宁不得不在这个时候躲藏在拉兹里夫车站附近的一个草棚中。

列宁虽因受临时政府的密探追寻没能出席大会。但他通过自己在彼得格勒的战友和学生斯大林、斯维尔德洛夫、莫洛托夫和奥尔忠尼启泽,秘密地领导了大会。

出席大会的有一百五十七名有表决权的代表和一百二十八名有发言权的代表。当时党员人数在二十四万左右。截至7月3日,就是说在工人的游行示威被摧残之前,布尔什维克还在公开进行活动的时候,党拥有四十一个机关刊物,其中二十九个是俄文的,十二个是其他文字的。

在七月事变时对布尔什维克和工人阶级的迫害,不仅没有使我们党的影响削弱,反而使它更加扩大了。各地代表举出的大量事实说明:工人和士兵开始大批脱离孟什维克和社会革命党人,并轻蔑地称他们为“社会主义狱卒”。孟什维克党和社会革命党中的工人和士兵党员纷纷撕毁自己的党证,咒骂着离开他们的党,而请求布尔什维克接收他们加入自己的党。

大会上的主要问题是中央委员会的政治报告和政治形势问题,斯大林同志在关于这两个问题的报告中十分明确地指出,虽然资产阶级倾全力镇压革命,但是革命仍然在不断发展。他指出,革命提出了关于对产品的生产和分配实行工人监督、土地转归农民、政权从资产阶级手中转归工人阶级和贫苦农民的问题。他说,革命成为社会主义性质的了。

七月事变以后,国内的政治形势发生了急剧的变化。两个政权并存的局面结束了。苏维埃及其社会革命党孟什维克领导,不想掌握全部政权。因此,苏维埃已经没有权力了。政权已集中到资产阶级临时政府手中,而这个政府还在解除革命的武装,摧残革命的组织,摧残布尔什维克党。革命和平发展的可能性已经没有了。斯大林同志说,现在只有一条路:推翻临时政府,用暴力夺取政权。但能用暴力夺取政权的只有同农村贫民联盟的无产阶级。

仍然由孟什维克和社会革命党人领导着的苏维埃,已滚入资产阶级的营垒,在当时的情况下只能起临时政府走卒的作用。斯大林同志说:在七月事变以后,“全部政权归苏维埃”的口号应当撤回。但暂时撤回这个口号并不是说不再为苏维埃政权而斗争了。当时说的不是作为革命斗争机关的一般苏维埃,而仅仅是由孟什维克和社会革命党人领导的那些苏维埃。

\begin{quotation}
斯大林同志说;“革命的和平时期已经结束,不和平的时期,搏斗和爆发的时期已经来到。……”(《俄国社会民主工党(布)第六次代表大会记录》俄文版第111页)\footnote{参看《斯大林全集》第3卷第165页。——译者注}
\end{quotation}

党朝着武装起义前进。

台上有人反映资产阶级的影响,反对社会主义革命的方针。

托洛茨基分子普列奥布拉任斯基提议在关于夺取政权的决议上指出:只有在西方发生无产阶级革命时,才可把俄国引上社会主义道路。

斯大林同志反对这种托洛茨基主义的提议。

\begin{quotation}
斯大林同志说:“很有可能,俄国正是开辟社会主义道路的国家。……必须抛弃那种认为只有欧洲才能给我们指示道路的陈腐观念。有教条式的马克思主义,也有创造性的马克思主义。我是主张后一种马克思主义的。”(同上,第233—234页)\footnote{同上,第174页。——译者注}
\end{quotation}

布哈林采取托洛茨基主义立场,说农民怀有护国主义情绪,说农民同资产阶级结成了联盟而不会跟工人阶级走。

斯大林同志驳斥了布哈林。他证明说,有各种各样的农民:有富裕农民,也有贫苦农民。前者支持帝国主义资产阶级,而后者愿同工人阶级联盟,并一定会在争取革命胜利的斗争中支持工人阶级。

大会否决了普列奥布拉任斯基和布哈林提出的修正案,批准了斯大林同志提出的决议草案。

大会讨论并批准了布尔什维克的经济纲领。这个纲领的要点是:没收地主土地并把全国所有土地收归国有,把银行收归国有,把大工业收归国有,对生产和分配实行工人监督。

大会强调了为实现工人监督生产而斗争的意义(工人监督生产在把大工业收归国有时起了巨大的作用)。

第六次代表大会在它的所有决议中,特别强调了列宁关于无产阶级同贫苦农民结成联盟是社会主义革命胜利的条件的原理。

大会斥责了孟什维克的工会中立论。大会指出,只有在工会始终是承认布尔什维克党政治领导的战斗的阶级组织的情况下,俄国工人阶级所担负的重大任务才能实现。

大会通过了《关于青年团》的决议。当时青年团往往是自动成立的。经过后来的努力,党终于把这些青年组织作为党的后备军管起来了。

大会讨论了列宁应不应当到法庭受审的问题。加米涅夫,李可夫、托洛茨基等人还在大会召开以前就认为,列宁应当到反革命法庭去受审。斯大林同志坚决反对列宁到法庭受审。第六次代表大会也反对列宁到法庭受审,认为这不是审判,而是迫害。大会毫不怀疑,资产阶级的目的只有一个,就是把列宁当作最危险的敌人进行肉体摧残。大会抗议资产阶级对革命无产阶级领袖的警察式的迫害,并致函列宁表示慰问。

第六次代表大会通过了新的党章。党章指出,党的各级组织应当按民主集中制原则来建立。

这就是说:

(一)党的各级领导机关从上到下按选举产生;

(二)党的各级机关定期向自己的党组织报告工作;

(三)严格遵守党的纪律,少数服从多数;

(四)上级机关的决议,下级机关和全体党员必须绝对执行。

党章规定,申请入党的人,根据党员二人介绍,经党组织全体党员大会通过后,由地方党组织接收入党。

第六次代表大会接收了“区联派”及其首领托洛茨基入党。这是个人数不多的集团,它从1913年起就存在于彼得格勒,成员是托洛茨基派孟什维克和一部分从党内分裂出去的前布尔什维克。战争期间,“区联派”是个中派组织。他们反对过布尔什维克,但他们在许多方面也不赞同孟什维克,因此他们采取的是中间的、中派主义的、动摇的立场。党的第六次代表大会期间,“区联派”声明,他们在一切方面赞同布尔什维克,并请求接收他们入党。大会满足了他们的请求,指望他们将来能成为真正的布尔什维克。有些“区联派”分子,如沃洛达尔斯基和乌里茨基等,后来真正成了布尔什维克。至于托洛茨基及其某些亲密朋友,正如后来所证明的,他们入党不是为了进行有益于党的工作,而是为了动摇党和从内部炸毁党。

第六次代表大会的一切决议,都是为着准备无产阶级和贫苦农民去实行武装起义。第六次代表大会把党指向武装起义的目标,指向社会主义革命的目标。

大会发表了党的宣言,号召工人、士兵和农民准备力量同资产阶级进行决战。宣言结尾说:

\begin{quotation}
“我们的战斗同志们,准备迎接新的战斗!要坚定、勇敢、镇静,不受挑拨,积蓄力量,列成战斗的队伍!无产者和士兵们,站到党的旗帜下来!农村的被压迫者们,站到我们的旗帜下来!”\footnote{见《苏联共产党决议汇编》第1分册第507页。——译者注}
\end{quotation}


\subsection[五\q 科尔尼洛夫将军的反革命阴谋。阴谋被粉碎。彼得格勒和莫斯科苏维埃转到布尔什维克方面]{五\\科尔尼洛夫将军的反革命阴谋。阴谋被粉碎。\\彼得格勒和莫斯科苏维埃转到布尔什维克方面}

资产阶级夺得全部政权之后,就准备摧毁软弱无力的苏维埃,建立赤裸裸的反革命专政。百万富翁列布申斯基悍然声称他看到了摆脱现状的出路,说“饥饿的魔掌和人民的贫困定会扼杀冒充人民之友的民主的苏维埃和委员会”。在前线,为士兵设立的战地法庭和死刑猖獗一时。1917年8月3日,总司令科尔尼洛夫将军要求在后方也实行死刑。

8月12日,临时政府为动员资产阶级地主的力量,在莫斯科的大剧院召开了国务会议。参加会议的主要是地主、资产阶级、将军、军官和哥萨克的代表。代表苏维埃出席的是孟什维克和社会革命党人。

在国务会议开幕那天,布尔什维克在莫斯科组织了有大多数工人参加的总罢工作为抗议。在其他许多城市也同时举行了罢工。

社会革命党人克伦斯基在会上讲话时吹嘘自己的力量,扬言要用“铁和血”来镇压革命运动的任何尝试,包括农民擅自夺取地主土地的尝试在内。

反革命将军科尔尼洛夫公然要求“取缔各委员会和苏维埃”。

银行家、商人和工厂主,接踵到大本营(当时这样称呼总司令部)来见科尔尼洛夫将军,答应会给他钱和支持。

“盟国”即英国和法国的代表,也来见科尔尼洛夫将军,要求他立刻起事反对革命。

眼看科尔尼洛夫将军就要发动反对革命的阴谋了。

科尔尼洛夫的阴谋是公开准备的。为了转移视线,阴谋者们散布流言,说布尔什维克准备于8月27日革命半周年那天在彼得格勒举行起义。以克伦斯基为首的临时政府猛攻布尔什维克,对无产阶级政党加紧实行恐怖手段。同时,科尔尼洛夫将军调集军队,准备把它们开到彼得格勒来消灭苏维埃和建立军事独裁政府。

科尔尼洛夫准备这次反革命发动,事先是同克伦斯基商量好的。但是在科尔尼洛夫举行发动的时候,克伦斯基突然改变方针,和自己的同盟者划清了界限。克伦斯基害怕,如果他的资产阶级政府不立刻同科尔尼洛夫叛乱划清界限,那么人民群众起来反对和粉碎科尔尼洛夫叛乱时,也会把它一起扫除。

8月25日,科尔尼洛夫把克雷莫夫将军指挥的第三骑兵军开往彼得格勒,宣称他要“拯救祖国”。为了反击科尔尼洛夫的暴动,布尔什维克党中央号召工人和士兵对反革命势力进行积极的武装抵抗。工人迅速地武装起来准备抵抗。赤卫队在这些日子里增长了好几倍。工会动员了自己的会员。彼得格勒的革命部队也作好了战斗准备。在彼得格勒周围挖好了战壕,设置了铁丝网,拆毁了铁道。几千武装的喀琅施塔得水兵开来保卫彼得格勒。向进攻彼得格勒的“野蛮师”派出了一些代表,去对山民士兵说明科尔尼洛夫暴动的实质,于是“野蛮师”拒绝进攻彼得格勒了。对科尔尼洛夫的其他部队也派去了鼓动员。凡是有危险的地方,都成立了革命委员会和反科尔尼洛夫叛乱的指挥部。

在这些日子里,吓得要死的社会革命党—孟什维克领导,包括克伦斯基在内,都到布尔什维克这里来寻求保护,因为他们知道,在首都唯一能够击败科尔尼洛夫的实际力量是布尔什维克。

但是,布尔什维克在动员群众粉碎科尔尼洛夫叛乱的时候,也没有停止同克伦斯基政府作斗争。布尔什维克在群众面前揭露了克伦斯基政府、孟什维克和社会革命党人,指出他们的全部政策在客观上帮助了科尔尼洛夫的反革命阴谋。

由于采取了这一切措施,科尔尼洛夫叛乱被粉碎了。克雷莫夫将军开枪自杀。科尔尼洛夫及其同僚邓尼金和卢柯姆斯基被捕(不过克伦斯基不久就把他们释放了)。

科尔尼洛夫叛乱一下子被粉碎,揭示了和说明了革命和反革命的力量对比,表明整个反革命营垒,从将军们和立宪民主党到沦为资产阶级俘虏的孟什维克和社会革命党人,必然要灭亡。很清楚,把力不胜任的战争拖延下去的政策和由旷日持久的战争引起的经济破坏,彻底破坏了他们在人民群众中的影响。

其次,科尔尼洛夫叛乱被粉碎表明,布尔什维克党已成长为革命的决定力量,能够击败任何反革命阴谋。当时我们党还不是执政党,但它在科尔尼洛夫叛乱的日子里是作为真正执政的力量在那里起作用的,因为工人和士兵毫不犹豫地执行了它的指示。

最后,科尔尼洛夫叛乱被粉碎表明,仿佛已经死亡了的苏维埃实际上蕴藏着极大的革命抵抗力量。毫无疑义,正是苏维埃及其革命委员会挡住了科尔尼洛夫的军队,损伤了他们的力量。

反科尔尼洛夫叛乱的斗争使气息奄奄的工兵代表苏维埃复活起来,摆脱了妥协主义政策的束缚,走上了革命斗争的康庄大道,转到了布尔什维克党方面。

布尔什维克在苏维埃中的影响空前增长了。

布尔什维克在农村的影响也迅速增长起来。

科尔尼洛夫暴动向广大农民群众表明,地主和将军们摧残布尔什维克和苏维埃以后,接着就将进攻农民。因此,广大的贫苦农民群众愈来愈紧密地团结在布尔什维克周围。至于中农,如果说他们的动摇阻碍过1917年4—8月时期的革命发展,那么在科尔尼洛夫被击溃以后,他们已在向贫苦农民群众靠拢,开始肯定地转向布尔什维克党方面。广大的农民群众已经开始了解,只有布尔什维克党才能使他们摆脱战争,才有能力摧毁地主并准备把土地转交农民。1917年9月和10月,农民夺取地主土地的事件大大增加。自行耕种地主土地已成为普遍的现象。无论劝说或讨伐队,都已制止不住奋起革命的农民了。

革命高潮正在到来。

苏维埃活跃和革新的阶段即苏维埃布尔什维克化的阶段开始了。工厂和部队改选自己的代表,把布尔什维克党的代表选进苏维埃去代替孟什维克和社会革命党人。在战胜科尔尼洛夫叛乱的第二天,即8月31日,彼得格勒苏维埃表示拥护布尔什维克的政策,彼得格勒苏维埃原来以齐赫泽为首的孟什维克—社会革命党主席团宣布辞职,让位给布尔什维克。9月5日,莫斯科工人代表苏维埃转向布尔什维克方面。莫斯科苏维埃的社会革命党—孟什维克主席团也宣布辞职,让路给布尔什维克。

这就是说,成功的起义所必需的基本前提已经成熟。

“全部政权归苏维埃!”的口号重新提上了日程。

但这已不是把政权转到孟什维克—社会革命党苏维埃手中的那个旧口号。不,这是苏维埃起义反对临时政府以便使国内全部政权转归布尔什维克所领导的苏维埃的口号。

各妥协主义党派内部一片混乱。

社会革命党内部在怀有革命情绪的农民的压力下出现了一个左翼,即“左派”社会革命党人,他们开始表示不满意同资产阶级妥协的政策。

孟什维克中也出现了一个“左派”集团,即所谓“国际主义者”集团,他们开始倾向于布尔什维克。

至于无政府主义者,本来就是个没有多大影响的集团,现在已彻底瓦解为一些很小的集团。有的去同社会渣滓中的刑事犯、盗贼和奸细为伍,有的去作“有思想的”剥夺者,抢劫农民和小市民,掠夺工人俱乐部的房屋和储金;有的公开转入反革命分子的营垒,在资产阶级豢养下过日子。他们所有的人都反对一切政权,特别是反对工农的革命政权,因为他们深知,革命政权决不会让他们抢劫人民和侵吞人民的财产。

科尔尼洛夫叛乱被粉碎以后,孟什维克和社会革命党人还作过一次尝试,想削弱日益增长的革命高潮。为此他们在1917年9月12日召开了全俄民主会议,参加会议的有各社会主义政党、妥协派苏维埃、工会、地方自治局、工商界以及军队的代表。由这次会议产生了预备议会(共和国临时议会)。妥协派想利用预备议会来阻止革命,使俄国离开苏维埃革命道路而走上资产阶级宪制发展的道路,即走上资产阶级议会制道路。但这是已遭破产的政治家们想倒转革命车轮的绝望的尝试。它必然要遭到失败,而且果然遭到了失败。工人们嘲笑了妥协派的议会作业练习。他们挖苦预备议会是“澡堂预备间”。

布尔什维克党中央决定抵制预备议会。诚然,像加米涅夫和泰奥多罗维奇这样的人盘踞着的布尔什维克预备议会党团,不愿意退出预备议会。但党中央强迫他们退出了预备议会。

加米涅夫和季诺维也夫顽固地主张参加预备议会,企图诱使党不去准备武装起义。斯大林同志在全俄民主会议布尔什维克党团中坚决反对参加预备议会。他把预备议会叫作“科尔尼洛夫叛乱的流产儿”。

列宁和斯大林认为,甚至短时间参加预备议会也是严重的错误,因为这会在群众中造成一种幻想,似乎预备议会真能替劳动者做些事情。

同时,布尔什维克坚忍不拔地准备召开苏维埃第二次代表大会,打算在这次会上争取到多数。不管盘踞全俄中央执行委员会的孟什维克和社会革命党人怎样支吾搪塞,但在布尔什维克的苏维埃的压力下,终于定于1917年10月下半月召开全俄苏维埃第二次代表大会。


\subsection[六\q 彼得格勒的十月起义和临时政府人员的被捕。苏维埃第二次代表大会和苏维埃政府的成立。苏维埃第二次代表大会的和平法令和土地法令。社会主义革命的胜利。社会主义革命胜利的原因]{六\\彼得格勒的十月起义和临时政府人员的被捕。\\苏维埃第二次代表大会和苏维埃政府的成立。\\苏维埃第二次代表大会的和平法令和土地法令。\\社会主义革命的胜利。\\社会主义革命胜利的原因}

布尔什维克开始加紧准备起义。列宁指出:布尔什维克既然在两个首都(莫斯科和彼得格勒)的工兵代表苏维埃中取得多数,就能够而且必须夺取国家政权。列宁在总结已经走过的道路时着重指出:“多数人民是拥护我们的。”\footnote{见《列宁全集》第26卷第1页。——译者注}列宁在文章中和给中央委员会和各级布尔什维克组织的信中,提出了起义的具体计划:如何利用陆军、海军和赤卫队,彼得格勒哪些有决定意义的地点必须夺取以保证起义成功,等等。

10月7日,列宁从芬兰秘密回到彼得格勒。1917年10月10日,党中央举行了具有历史意义的会议,会议决定在最近期间开始武装起义。列宁给党中央起草的具有历史意义的决议说:

\begin{quotation}
“中央委员会认为,俄国革命的国际形势(德国海军中的起义,这是世界社会主义革命在全欧洲增长的最高表现;其次,帝国主义者为扼杀俄国革命而媾和的威胁)和军事形势(俄国资产阶级和克伦斯基之流已经明确地决定把彼得格勒让给德国人),无产阶级政党在苏维埃中获得了多数,再加上农民起义和人民转而信任我们党(莫斯科的选举),以及第二次科尔尼洛夫叛乱显然已在准备(军队撒出彼得格勒、哥萨克调往彼得格勒、哥萨克包围明斯克等等),——这一切把武装起义提到日程上来了。

因此中央委员会认为,武装起义是不可避免的,并且业已完全成熟。中央委员会建议各级党组织以此为指针,并从这一观点出发讨论和解决一切实际问题(北方区域苏维埃代表大会、军队撤出彼得格勒、莫斯科人和明斯克人的发动等等)。”(《列宁全集》俄文第3版第21卷第330页)\footnote{见《列宁选集》第2版第3卷第345页。——译者注}
\end{quotation}

发言和投票反对这个具有历史意义的决议的有两个中央委员,即加米涅夫和季诺维也夫。他们也和孟什维克一样梦想成立资产阶级议会制共和国,并诬蔑工人阶级,硬说它没有力量实现社会主义革命,硬说它还没有成长到夺取政权的程度。

托洛茨基在这次会上虽然没有直接对这一决议投反对票,但是他对决议案提出了一个必然会使起义化为乌有和遭到失败的修正案。他提议在苏维埃第二次代表大会开幕以前不要开始起义,这就等于拖延起义一事,事先泄露起义日期,把这事预先告诉临时政府。

布尔什维克党中央委派了全权代表分赴顿巴斯、乌拉尔、赫尔辛福斯、喀琅施塔得、西南战线等地去组织当地的起义。伏罗希洛夫、莫洛托夫、捷尔任斯基、奥尔忠尼启泽、基洛夫、卡冈诺维奇、古比雪夫、伏龙芝、雅罗斯拉夫斯基和其他同志,受了党的专门委托领导当地的起义。在乌拉尔的沙德林斯克军队中进行工作的是日丹诺夫同志。中央全权代表们向各地的布尔什维克组织的领导人传达了起义计划,组织他们做好动员准备,以便随时支援彼得格勒起义。

根据党中央委员会的指示,在彼得格勒苏维埃下面成立了革命军事委员会,它成了起义的公开的司令部

同时,反革命也赶紧纠集自己的力量。军官们成立了反革命的“军官联合会”。反革命分子到处都建立了突击营编建指挥部。到10月底,反革命已经拥有四十三个突击营。专门组织了几个乔治勋章军人营。

克伦斯基政府提出了把政府从彼得格勒迁往莫斯科的问题。由此可见,它准备把彼得格勒让给德国人,以防止彼得格勒发生起义。彼得格勒工人和士兵的反对迫使临时政府留在彼得格勒。

10月16日,党中央召开了扩大会议。会议选出了以斯大林同志为首的领导起义的党总部。这个党总部是彼得格勒苏维埃所属的革命军事委员会的领导核心,它实际上领导整个起义。

在这次中央会议上,投降主义者季诺维也夫和加米涅夫又发言反对起义。在遭到回击之后,他们竟在报刊上公开反对起义、反对党。10月18日,孟什维克的《新生活报》发表了加米涅夫和季诺维也夫的声明,说布尔什维克准备起义,而他们认为起义是冒险。这样,加米涅夫和季诺维也夫就向敌人泄漏了中央关于起义、关于在最近期间举行起义的决定。这是叛变。列宁关于此事写道:“加米涅夫和季诺维也夫向罗将柯和克伦斯基泄露了自己党中央关于武装起义……的决定。”\footnote{见《列宁全集》第26卷第205—206页。——译者注}列宁向中央提出了开除季诺维也夫和加米涅夫出党的问题。

革命的敌人接到叛徒的警告,立即采取措施来防止起义,来摧毁革命的领导司令部——布尔什维克党。临时政府召开了秘密会议,会议解决了关于同布尔什维克斗争的措施问题。10月19日,临时政府急忙从前线调军队来彼得格勒。街道上开始加岗巡逻。反革命在莫斯科纠集了特别大的力量。临时政府制定了一个计划,在苏维埃第二次代表大会开幕前一天进攻和占领布尔什维克中央所在地斯莫尔尼,粉碎布尔什维克的领导中心。为此,政府把它认为忠实可靠的部队集结到彼得格勒。

可是临时政府的末日已到。任何力量都不能阻止社会主义革命的胜利进军了。

10月21日,布尔什维克向所有的革命部队派去了革命军事委员会委员。在起义前的所有日子里,各部队、各工厂都在加紧进行战斗准备。两艘战舰——“阿芙乐尔”号巡洋舰和“自由曙光”号舰,也接到了明确的任务。

托洛茨基在彼得格勒苏维埃会议上通过吹嘘,向敌人泄露了起义的日期,泄露了布尔什维克预定开始起义的日子。为了不让克伦斯基政府有可能破坏武装起义,党中央决定在原定日期以前,即在苏维埃第二次代表大会开幕前一天开始和进行起义。

10月24日(11月6日)清晨,克伦斯基开始行动,下令查封布尔什维克党中央机关报《工人之路报》,并把一些装甲车开到《工人之路报》编辑部和布尔什维克印刷厂前。但到上午十点时,赤卫队和革命士兵遵照斯大林同志的指示把装甲车赶走了,并在印刷厂和《工人之路报》编辑部附近加强了防卫。上午十一点,《工人之路报》出版了,报上号召推翻临时政府。同时,根据领导起义的党总部的指示,革命士兵和赤卫队的队伍立即向斯莫尔尼集结。

起义开始了。

10月24日夜晚,列宁到了斯莫尔尼,直接领导起义。革命部队和赤卫队通宵不停地向斯莫尔尼开来。布尔什维克把它们派往首都中心去包围临时政府的老巢冬宫。

10月25日(11月7日),赤卫队和革命部队占领了火车站、邮局、电报局、政府各部和国家银行。

预备议会被解散了。

彼得格勒苏维埃和布尔什维克中央委员会所在的斯莫尔尼,成了革命的战斗司令部,从这里发出战斗的命令。

在这些日子里,彼得格勒的工人表明,他们在布尔什维克党领导下受到了很好的锻炼。经布尔什维克做过工作而作好了起义准备的革命部队,准确地执行了战斗命令,同赤卫队一起并肩战斗。海军也不比陆军落后。喀琅施塔得成了布尔什维克党的堡垒,这里早就不承认临时政府的政权了。10月25日,“阿芙乐尔”号巡洋舰向冬宫轰击的炮声,宣告了新纪元即伟大社会主义革命纪元的开始。

10月25日(11月7日),公布了布尔什维克《告俄国公民书》,宣布资产阶级临时政府已被推翻,国家政权已转到苏维埃手中。

临时政府躲在冬宫里面,由士官生和突击营警卫着。10月25日深夜,革命的工人、士兵和水兵发起冲锋,攻下了冬宫,逮捕了临时政府成员。

彼得格勒武装起义胜利了。

1917年10月25日(11月7日)晚十点四十五分,全俄苏维埃第二次代表大会在斯莫尔尼开幕,当时彼得格勒的胜利起义已达沸点,首都的政权实际上已经在彼得格勒苏维埃手中了。

布尔什维克在会上获得了压倒多数。孟什维克、崩得分子和右派社会革命党人眼看大势已去,声明不参加大会工作而离开了大会。他们在向苏维埃代表大会宣读的声明中,把十月革命称为“军事阴谋”。大会痛斥了孟什维克和社会革命党人,并且指出,对于他们的退出不仅不表示惋惜,反而表示欢迎,因为叛徒的退出使代表大会成了真正革命的工兵代表大会。

用大会名义宣布全部政权转到苏维埃手中。

苏维埃第二次代表大会宣言说:

\begin{quotation}
“根据绝大多数工人、士兵和农民的意志,依靠彼得格勒工人和卫戍部队所举行的胜利起义,代表大会已经把政权掌握在自己手里\footnote{见《列宁选集》第2版第3卷第352页。——译者注}。”
\end{quotation}

1917年10月26日(11月8日)夜,苏维埃第二次代表大会通过了和平法令。大会向各交战国建议,立即缔结至少三个月的停战协定,以便进行和平谈判。大会在向一切交战国的政府和人民呼吁的同时,还向“人类三个最先进的民族,这次战争中三个最大的参战国,即英法德三国的觉悟工人”呼吁。大会号召这些国家的工人帮助“把和平事业以及使被剥削劳动群众摆脱一切奴役和一切剥削的事业有成效地进行到底”\footnote{同上,第356页。——译者注}。

同夜,苏维埃第二次代表大会通过了土地法令,上面规定“立刻废除地主土地私有制,不付任何赎金”\footnote{同上,第363页。——译者注}。这个土地法令的基础就是根据二百四十二份地方农民委托书拟定的全国农民委托书。根据这个委托书,土地私有权永远废除,而代之以全民的、国家的土地所有制。地主、皇族和寺院的土地,一律无偿地交给全体劳动者使用。

农民根据这个法令从十月社会主义革命总共获得了一亿五千万俄亩以上的新土地,这些土地从前都掌握在地主、资产阶级、皇室、寺院和教堂手中。

农民免除了每年向地主交纳的五亿左右金卢布的地租。

所有地下矿藏(石油、煤炭、矿石等等)、森林和水流都转归人民所有。

最后,在全俄苏维埃第二次代表大会上成立了第一届苏维埃政府,即人民委员会。人民委员会完全由布尔什维党组成。列宁被选为第一届人民委员会主席。

具有历史意义的苏维埃第二次代表大会就这样结束了。

大会代表分赴各地,以传达彼得格勒苏维埃胜利的消息,并保证把苏维埃政权扩展到全国。

并不是所有的地方政权都是一下子转归苏维埃的。当彼得格勒已经有了苏维埃政权时,莫斯科街头还顽强激烈地进行了好几天战斗。为了不让政权转到莫斯科苏维埃手中,反革命党派孟什维克和社会革命党跟白卫分子和士官生一起,发动了反对工人和士兵的武装斗争。过了几天,叛乱者被击败,苏维埃政权才在莫斯科建立起来。

就是在彼得格勒本城及其几个区里,在革命取得胜利的最初几天,反革命分子还曾试图推翻苏维埃政权。1917年11月10日,克伦斯基(他是在起义期间从彼得格勒逃往北方战线地区的)纠集了一些哥萨克部队,让克拉斯诺夫将军率领着向彼得格勒进发。1917年11月11日,以社会革命党人为首的反革命组织“救国救革命委员会”,在彼得格勒发动士官生叛乱。但是没有费多大力气就把叛乱者打垮了。经过一个白天,到11月11日傍晚,水兵和赤卫队就平定了士官生的叛乱,而到11月13日,又在普尔科沃高地击溃了克拉斯诺夫将军。列宁也如在十月起义时一样,亲自领导粉碎反苏的叛乱。他的坚定不移的精神和确信胜利的镇静态度,鼓舞并团结了群众。敌人被打垮了。克拉斯诺夫被俘虏,他“保证”以后决不反对苏维埃政权。他凭这一“保证”被释放了,但是后来表明,克拉斯诺夫违背了自己的这种将军诺言。至于克伦斯基,已男扮女装藏得“不知去向”了。

在全军总司令的大本营所在地莫吉廖夫,杜鹤宁将军也企图发动叛乱。当苏维埃政府要杜鹤宁立刻着手同德军指挥部进行停战谈判时,他拒绝执行政府的指令。于是苏维埃政权下令撤掉杜鹤宁。反革命大本营被摧毁了,杜鹤宁本人则被起来反对他的士兵打死。

党内人所共知的机会主义者加米涅夫、季诺维也夫、李可夫、施喀普尼柯夫等人,也曾试图袭击苏维埃政权。他们要求成立“清一色的社会主义者政府”,让刚被十月革命推翻的孟什维克和社会革命党人参加。1917年11月15日,布尔什维克党中央通过决议,坚决反对同这些反革命党派达成协议,宣布加米涅夫和季诺维也夫是破坏革命的工贼。11月17日,加米涅夫、季诺维也夫、李可夫和米柳亭因不同意党的政策,声明退出中央委员会。同日,即11月17日,诺根用本人的名义和参加人民委员会的李可夫、弗·米柳亭、泰奥多罗维奇、亚·施喀普尼柯夫、达·梁赞诺夫、尤列涅夫和拉林等人的名义提出声明,说他们不同意党中央的政策,并退出人民委员会。一小撮懦夫的逃跑,使十月革命的敌人兴高采烈。整个资产阶级及其帮凶们都幸灾乐祸,叫嚷布尔什维主义在瓦解,预言布尔什维克党必然灭亡。但一小撮逃兵丝毫也没能动摇党。党中央轻蔑地斥责他们是革命的逃兵和资产阶级的帮凶之后,就转而进行当前的工作。

至于“左派”社会革命党人,他们想在明显地同情布尔什维克的农民群众中保持影响,决定不同布尔什维克争执,暂时同布尔什维克保持统一战线。1917年11月召开的农民苏维埃代表大会,承认十月社会主义革命的一切成果和苏维埃政权的一切法令。当时同“左派”社会革命党人达成了协议,几个“左派”社会革命党人(柯列加也夫.斯皮里多诺娃、普罗相和施泰因别尔格)参加了人民委员会。但这一协议只存在到签订布列斯特和约和成立贫农委员会时止,因为这时农民发生了深刻的分化,而愈来愈反映富农利益的“左派”社会革命党人发动了反布尔什维克的叛乱,被苏维埃政权打垮了。

在1917年10月至1918年1、2月这一时期,苏维埃革命扩展到了全国各地。苏维埃政权在广阔的国土上推进得如此迅速,列宁把这叫做苏维埃政权的“胜利进军”。

伟大十月社会主义革命胜利了。

从决定社会主义革命能在俄国这样较为容易获得胜利的种种原因中,应该指出如下几个主要原因:

(一)十月革命所遇到的是俄国资产阶级这样一个比较软弱、组织不好而又缺乏政治经验的敌人。在经济上还不强固而完全依赖政府定货的俄国资产阶级,既没有为找到摆脱现状的出路所必需的政治独立性,又没有为此所必需的充分的主动性。它既没有如法国资产阶级那种大规模搞政治权术和政治欺骗的经验,也没有如英国资产阶级那种大规模搞诡诈妥协的训练。它昨天还在谋求同被二月革命所推翻的沙皇达成协议,所以后来到执政时根本想不出什么高招,只能在一切基本问题上继续可恶的沙皇的政策。它和沙皇一样主张“战到最后胜利”,而不顾国家已经力不胜任、人民和军队已经疲惫不堪。它和沙皇一样主张基本上保存地主土地所有制,而不顾农民由于缺乏土地和遭受地主压迫而奄奄待毙。至于对工人阶级的政策,俄国资产阶级比沙皇更加仇视工人阶级,因为它不仅竭力保存和巩固工厂主的压迫,并且竭力用大规模的同盟歇业使这种压迫变得更加不堪忍受。

无怪乎人民认为沙皇的政策和资产阶级的政策没有重大区别,把对沙皇的仇恨转到了资产阶级临时政府身上。

当妥协主义的党派社会革命党和孟什维克在人民中间还有一定影响的时候,资产阶级还能靠它们掩护而保持住政权。但是在孟什维克和社会革命党人已暴露自己是帝国主义资产阶级的代理人、从而丧失了自己在人民中的影响之后,资产阶级及其临时政府就悬在空中了。

(二)领导十月革命的是俄国工人阶级这伴一个革命阶级;它是在战斗中锻炼出来的,在短短的时期内经历过两次革命,在第三次革命前夜在争取和平、土地、自由和社会主义的斗争中赢得了人民领袖的威信。没有俄国工人阶级这样一个受到人民信任的革命领袖,就不会有工农联盟;而没有这样一个联盟,十月革命就不能胜利。

(三)俄国工人阶级在革命中有占农村居民绝大多数的贫苦农民这样一个重要的同盟者。可以和几十年“常态”发展相比的八个月的革命经历,对于劳动农民群众并没有白白过去。在这个期间,他们有可能根据事实来检验俄国的一切政党,并且确信:无论立宪民主党,无论社会革命党和孟什维克,都不会为农民真正同地主翻脸和流血牺牲;俄国只有一个党同地主没有联系,并决心打倒地主,以满足农民的需要,——这就是布尔什维克党。这种情况成了无产阶级和贫苦农民联盟的现实基础。工人阶级和贫苦农民结成联盟,也决定了中农的态度,使他们在长期动摇之后,到十月起义前夜终于同贫苦农民站到一起,真正转到了革命方面。

根本用不着证明:如果没有这个联盟,十月革命就不能胜利。

(四)领导工人阶级的是布尔什维克党这样一个在政治斗争中考验出来的党。只有布尔什维克党这样的党,只有这样一个有充分的勇气、能引导人民进行坚决的冲击,又十分谨慎、能绕过前进道路上的一切暗礁的党,才能非常巧妙地将各种不同的革命运动——如争取和平的一般民主运动,夺取地主土地的农民民主运动,被压迫民族争取民族平等的民族解放运动和无产阶级推翻资产阶级、建立无产阶级专政的社会主义运动——汇合成一个总的革命洪流。

毫无疑义,这些不同的革命潮流汇合成一个总的强大的革命洪流,决定了俄国资本主义的命运。

(五)十月革命是在帝国主义战争正酣、主要资产阶级国家分裂成两个敌对营垒、它们正忙于互相交战和互相削弱而不可能认真干涉“俄国内政”和积极反对十月革命的时刻开始的。

毫无疑义,这种情形大大有助于十月社会主义革命的胜利。


\subsection[七\q 布尔什维克党为巩固苏维埃政权而斗争。布列斯特和约。党的第七次代表大会]{七\\ 布尔什维克党为巩固苏维埃政权而斗争。\\布列斯特和约。\\党的第七次代表大会}

为了巩固苏维埃政权,必须破坏和摧毁旧的资产阶级国家机构,代之以新的苏维埃国家机构。其次,必须消灭等级制度残余和民族压迫制度,废除教会特权,取缔反革命的出版物以及各种合法的和不合法的反革命组织,解散资产阶级立宪会议。最后,必须在土地国有化之后把全部大工业也收归国有,然后摆脱战争状态,结束那个最妨碍苏维埃政权巩固的战争。

这一切措施,在1917年底至1918年中这几个月内实现了。

社会革命党人和孟什维克组织的旧政府各部官吏的怠工被粉碎和消灭了。撤销了政府各部,代之以苏维埃管理机构和相应的人民委员部。成立了最高国民经济委员会来管理全国的工业。组织了以赞·捷尔任斯基为首的全俄肃反委员会来同反革命和怠工行为作斗争。颁布了关于建立红军和红海军的法令。基本上在十月革命前选出,拒绝批准苏维埃第二次代表大会的和平法令、土地法令和政权转归苏维埃法令的立宪会议被解散了。

为了彻底消灭封建残余、等级制和社会生活各方面的不平等现象,颁布了废除等级、取消民族限制和信教限制、教会同国家分离和学校同教会分离、妇女有同男子平等的投利、俄国各民族一律平等这样一些法令。

苏维埃政府的一项专门决议,即人所共知的《俄国各族人民权利宣言》,把俄国各族人民自由发展和完全平等定为法律。

为了摧毁资产阶级的经济实力和组织新的苏维埃国民经济(首先是组织新的苏维埃工业),把银行、铁路、对外贸易、商船以及大工业的所有部门——煤炭、冶金、台油、化学、机器制造、纺织和制糖等收归国有了。

为了使我国的财政独立和摆脱外国资本家的剥削,废除(取消)了沙皇和临时政府向外国借的国债。我国各族人民不愿偿付那些用来进行掠夺战争并使我国奴隶般地依赖于外国资本的债款。

所有这些以及诸如此类的措施,从根本上摧毁了资产阶级、地主、反动官吏和反革命政党的势力,使苏维埃政权在国内大为巩固。

但是,只要俄国还处于同德奥交战的状态,就不能认为苏维埃政权的地位是完全巩固的。要彻底巩固苏维埃政权,必须结束战争。因此,党从十月革命刚一胜利就展开了争取和平的斗争。

苏维埃政府“向一切交战国的人民及其政府”建议,“立即就公正的民主的和约开始谈判”\footnote{见《列宁选集》第2版第3卷第354页。——译者注}。但是,英法两个“盟国”不接受苏维埃政府的建议。由于英法两国拒绝和谈,苏维埃政府执行苏维埃的意志,决定同德奥两国进行谈判。

谈判于12月3日在布列斯特—里托夫斯克开始。12月5日,签订了停战协定,即暂时停止战争行动的协定。

谈判是在国民经济遭受严重破坏、人们普遍疲于战争和我国军队从前线撤退、前线陷于瓦解的局势下进行的。在谈判期间看出,德帝国主义者企图夺取前沙皇帝国的大块领土,而把波兰、乌克兰和波罗的海沿岸各国变为德国的附属国。

在这些条件下继续战争,无异于拿刚刚诞生的苏维埃共和国的生命作赌注。摆在工人阶级和农民面前的是,必须接受苛刻的和约条件,对当时最危险的强盗德帝国主义让步,以便取得喘息时机,巩固苏维埃政权,建立一支能保卫国家抵抗敌人进攻的新的军队即红军。

一切反革命分子,从孟什维克和社会革命党人到最坏的白卫分子,都疯狂地进行煽动,反对签订和约。他们的路线很明显,他们想破坏和谈,挑起德国人进攻。使尚未巩固的苏维埃政权遭受打击,使工农获得的成果受到威胁。

配合他们进行这一黑暗勾当的,是托洛茨基及其帮手布哈林——布哈林同拉狄克和皮达可夫起领导着个反党集团,这个集团为了掩饰自己,自称为“左派共产主义者”集团。托洛茨基和“左派共产主义者”集团在党内发动了反对列宁的激烈斗争,要求继续战争。这些人显然是在帮助德帝国主义者和国内的反革命分子,因为他们力图使年轻的、还没有自己军队的苏维埃共和国去受德帝国主义的打击。

这是用左的词句巧加掩饰的一种挑拨政策。

1918年2月10日,布列斯特—里托夫斯克的和谈中断了。虽然列宁和斯大林代表党中央坚决主张签订和约,但是托洛茨基作为驻布列斯特的苏维埃代表团首席代表,却叛卖性地违背了布尔什维克党的直接指示。他声明苏维埃共和国拒绝在德国提出的条件下签订和约,而同时他又通知德方,说苏维埃共和国将不进行战争,并在继续复员军队。

这真是骇人听闻。德帝国主义者所能要求于苏维埃国家利益叛卖者的,无过于此了。

德国政府中断停战,转入进攻。我国旧军队的残部抵挡不住德军攻势而开始溃散。德军迅速推进,夺取了大片领土,并威胁着彼得格勒。德帝国主义入侵苏维埃国家,目的是推翻苏维埃政权,把我们祖国变为它的殖民地。陷于瓦解的旧沙皇军队,抵挡不住德帝国主义的大批武装部队。它在德军打击下溃退了。

但是德帝国主义者的武装干涉激起了我国声势浩大的革命高潮。工人阶级响应党和苏维埃政府发出的“社会主义祖国在危急中”的召换,加紧组织红军部队。年轻的新军队(革命人民的军队)英勇地击退了武装到牙齿的德国强盗的攻击。德国占领者在纳尔瓦和普斯科夫附近遭到了坚决的回击。他们向彼得格勒的推进被阻止了。抗击德帝国主义军队的日子2月23日,成了年轻的红军的生日。

还在1918年2月18日,党中央就通过了列宁提出的致电德国政府立即缔结和约的建议。德方为了保证自己获得更有利的和约条件而继续进攻,直到2月22日德国政府才表示同意签订和约,但这时的和约条件比原先苛刻得多了。

为了争取通过关于缔结和约的决定,列宁,斯大林和斯维尔德洛夫不得不在中央对托洛茨基、布哈林和其他托洛茨基分子进行最顽强的斗争。列宁指出:布哈林和托洛茨基“实际上是帮助了德帝国主义者,阻碍了德国革命的壮大和发展”(《列宁全集》俄文第8版第22卷第307页)\footnote{见《列宁选集》第2版第3卷第417页。——译者注}。

2月23日。中央决定接受德军统帅部提出的条件并签订和约。托洛茨基和布哈林的叛卖行为使苏维埃共和国付出了高昂的代价。拉脱维亚、爱沙尼亚,不用说还有波兰,都割让给了德同。乌克兰脱离苏维埃共和国,变成德国的附庸国(附属国)。苏维埃共和国必须向德国人交纳赔款。

但“左派共产主义者”继续进行反对列宁的斗争,愈来愈深地陷入叛卖的泥潭。

暂时被“左派共产主义者”(布哈林,奥新斯基、雅柯夫列娃、斯土柯夫,曼策夫)篡夺的党的莫斯科区域局,通过了不信任中央的分裂主义决议,并说它认为“党在最近的将来恐难避免分裂”。在这个决议中,他们甚至作出了反苏的决定。“左派共产主义者”在这个决定中说:“为了国际革命的利益,我们认为,即使丧失目前完全流于形式的苏维埃政权,也是适当的。”

列宁称这个决定为“奇谈与怪论”。\footnote{见《列宁选集》第2版第3卷第438页。——译者注}

当时党还不清楚托洛茨基和“左派共产主义者”这种反党行为的真实原因。但是正如不久前对反苏的“右倾分子—托洛茨基派联盟”的审判(1938年初)所证实的:布哈林及其所领导的“左派共产主义者”集团同托洛茨基和“左派”社会革命党人一起,当时策划过反苏维埃政府的阴谋。布哈林、托洛茨基及其阴谋同伙曾打算破坏布列斯特和约,逮捕和杀害列宁、斯大林和斯维尔德洛夫,由布哈林派、托洛茨基派和“左派”社会革命党人共同组织一个新政府。

“左派共产生兑者”集团一面策划反革命阴谋,一面在托洛茨基的支持下对布尔什维克党进行公开的攻击,力图分裂党和瓦解党的队伍。但是,党在这个严重关头团结在列宁、斯大林和斯维尔德洛夫的周围,在和约问题上也和在其他一切问题上一样支持中央委员会。

“左派共产主义者”集团孤立了,被击败了。

为了彻底解决和约问题,召开了党的第七次代表大会。

党的第七次代表大会于1918年3月6日开幕。这是我们党取得政权以后召开的第一次代表大会。出席这次大会的有四十六个有表决权的代表和五十八个有发言权的代表。这次大会代表着十四万五千个党员。实际上当时党至少已有二十七万党员。所以会有这样一种差别,是因为这次大会带有非常的性质,相当一部分组织没有来得及选派代表,而暂时被德国人占领的地区的组织又没有可能选派代表。

列宁在这次会上做关于布列斯特和约的报告时说:“……我们的党由于党内出现左倾反对派而遇到的严重危机,是俄国革命所遇到的最大的危机之一。”(《列宁全集》俄文第3版第22卷第321页)\footnote{见《列宁选集》第2版第3卷第462页。——译者注}

列宁关于布列斯特和约问题的决议案以三十票赞成、十二票反对、四票弃权获得通过。

在决议通过的第二天,列宁在《不幸的和约》一文中写道:

\begin{quotation}
“和约条件的确苛刻得难以接受。但是历史终究会占上风……我们要从事组织,组织和组织。不管有怎样的考验,未来一定是我们的。”(同上,第288页)\footnote{见《列宁全集》第27卷第38页。——译者注}
\end{quotation}

大会的决议指出:帝国主义国家反苏维埃共和国的军事发动在今后也是不可避免的。因此大会认为,党的基本任务是采取最有力最坚决的措施来加强工人和农民的自觉纪律和纪律,发动群众作好奋勇保卫社会主义祖国的准备,组织红军,对人民进行普遍的军事训练。

大会确认列宁在布列斯特和约问题上的路线正确,谴责了托洛茨基和布哈林的立场,痛斥了已遭失败的“左派共产主义者”企图在这次会上继续进行分裂活动。

布列斯特和约的缔结,使党有可能赢得时间来巩固苏维埃政权,调整全国的经济。

和约的缔结,使得有可能利用帝国主义阵营内部的冲突(奥德两国同协约国在继续进行战争),瓦解敌人的力量,组织苏维埃经济,建立红军。

和约的缔结,使无产阶级有可能保持农民对自己的支持,为在国内战争时期击溃白卫将军们积蓄力量。

在十月革命时期,列宁教导布尔什维克党,当具备进攻所必需的条件时应怎样大胆坚决地进攻。在布列斯特和约时期,列宁教导党,当敌人力量显然超过我方力量时,应怎样有秩序地退却,以便用最大的努力准备对敌人实行新的进攻。

历史证明了列宁路线完全正确。

第七次代表大会通过了关于更改党的名称和修改党纲的决定。党开始称为俄国共产党(布尔什维克),简称俄共(布)。列宁提议把我们党称为共产党,是因为这个名称准确地符合党的目标——实现共产主义。

选出了由列宁、斯大林等人组成的专门委员会来起草新党纲,列宁拟定的草案被采纳为党纲的基础。

这样,第七次代表大会完成了具有历史意义的大事:击败了党内暗藏的敌人“左派共产主义者”和托洛茨基派,使俄国退出了帝国主义战争,争得了和平即喘息时机,使党赢得了时间来组织红军,向党提出了在国民经济中建立社会主义秩序的要求。


\subsection[八\q 列宁关于着手进行社会主义建设的计划。贫农委员会和制裁富农。苏维埃第五次代表大会和俄罗斯苏维埃联邦社会主义共和国宪法的通过“左派”社会革会党人的叛乱及其被镇压。]{八\\列宁关于着手进行社会主义建设的计划。\\贫农委员会和制裁富农。\\苏维埃第五次代表大会和俄罗斯苏维埃联邦社会主义共和国宪法的通过“左派”社会革会党人的叛乱及其被镇压。}

苏维埃政权在缔结了和约和取得了喘息时机后,就着手开展社会主义建设。从1917年11月至1918年2月这个时期,列宁称为“用赤卫队进攻资本”的时期。苏维埃政权在1918年上半年成功地摧毁了资产阶级的经济实力,掌握了国民经济的命脉(工厂、银行、铁路、对外贸易,商船等等),摧毁了资产阶级的国家政权机构,胜利地消灭了反革命势力企图推翻苏维埃政权的最初几次尝试。

但过一切还远远不够。要前进,还必须从破坏旧制度转到建设新制度。因此,在1918年春,就开始向社会主义建设的新阶段过渡——“从剥夺剥夺者”过渡到组织上巩固既得的胜利,即过渡到建设苏维埃国民经济。列宁认为必须最大限度地利用喘息时机来着手建设社会主义经济的基础。布尔什维克应该学会按新的方式来组织生产和管理生产。列宁写道:布尔什维克党已经说服了俄国,布尔什维克党已经夺取了俄国,从富人手中夺过来交给人民。列宁说,现在布尔什维克党应当学会管理俄国。

列宁认为这个阶段的生要任务是对国民经济中所生产的东西进行计算,对一切产品的消费实行监督。当时国内小资产阶级成分在经济中占优势。城乡千百万小业主是滋长资本主义的土壤——这些小业主既不承认劳动纪律,也不承认全国纪律;他们既不服从计算,也不服从监督。在过个困难时刻,特别危险的是小资产阶级投机谋利的自发势力和小业主、小商人利用人民困苦来发财的行为。

党同生产中的松懈现象、同工业中缺乏劳动纪律的现象进行了有力的斗争。群众养成新的劳动习惯很缓慢。因此,加强劳动纪律就成了这一时期的中心任务。

列宁指出,必须在工业中开展社会主义竞赛,实行计件工资制,反对平均主义,在采取说服教育办法的同时还要用强制手段对付那些想从国家手里多捞一些的、游手好闲的和投机倒把的分子。列宁认为新的纪律,即劳动纪律、同志关系纪律、苏维埃纪律,是由干百万劳动者在日常实际工作中培养起来的。他指出:“这件事情要占去整整一个历史时代。”(《列宁全集》俄文第3版第23卷第44页)\footnote{见《列宁选集》第2版第3卷第575页。——译者注}

所有这些社会主义建设的问题,这些建立新的社会主义的生产关系的问题,都由列宁在《苏维埃政权的当前任务》这一名著中阐明了。

“左派共产主义者”配合社会革命党人和孟什维克,在这些问题上也发动了反对列宁的斗争。布哈林和奥新斯基等人反对建立纪律,反对在企业中实行一长制,反对在工业中利用专家,反对实行经济核算。他们诬蔑列宁,硬说实行这样的政策就是回到资产阶级秩序。同时,“左派共产主义者”宣扬托洛茨基观点,认为社会主义建设和社会主义的胜利在俄国是不可能的。

“左派共产主义者”用“左的”词句作掩护,来维护那些反对纪律,敌视国家调节经济生活、敌视计算和监督的富农、懒汉和投机分子。

党在解决了组织新的即苏维埃的工业问题之后,就来解决农村问题。当时农村里贫农反对富农的斗争十分激烈。富农得了势,夺取了从地主那里剥夺来的土地。贫农急需帮助。富农同无产阶级国家作斗争,拒绝按固定价格把粮食卖给国家。他们想借助饥荒迫使苏维埃国家放弃实行社会主义措施。党提出了打垮反革命富农的任务。为了把贫农组织起来和顺利地反对拥有余粮的富农,组织了工人下乡运动。

\begin{quotation}
列宁写道:“工人同志们!你们要记着,革命情况危急。你们要记着,只有你们才能拯救革命;此外再没有别的人。我们需要几万名精悍、先进、忠实于社会主义的工人,他们决不会受贿行窃,而能组成钢铁一般坚强的力量去反对富农,反对投机者,反对抢劫者,反对贪赃受贿者,反对捣乱者。”(《列宁全集》俄文第3版第23卷第25页)\footnote{见《列宁选集》第27卷第364页。——译者注}
\end{quotation}

列宁说:“为粮食而斗争,就是为社会主义而斗争。”当时就是在这个口号下来组织工人下乡运动的。颁布了一系列法令,规定实行粮食专卖并赋予粮食人民委员部各机关按固定价格收购粮食的特别权力。

根据1918年6月11日法令成立了贫农委员会(贫委)。贫委在同富农斗争中,在重新分配没收的土地和分配农具方面,在收购富农的余粮方面,在为工人中心区和红军供应粮食方面,起了巨大的作用。富农的五千万公顷土地转到了贫农和中农手中。富农的相当一部分生产资料被没收来给了贫农。

成立贫农委员会是在农村中开展社会主义革命的又一个阶段。贫委是无产阶级专政在农村中的据点。当时用农村居民编建红军基干部队,在很大程度上是经过贫委来进行的。

无产者的下乡运动和贫农委员台的成立,巩固了农村的苏维埃政权,对于争取中农到苏维埃政权方面来具有巨大的政治意义。

1918年底,贫委完成了自己的任务,同农村的苏维埃合并而不再存在了。

1918年7月4口,苏维埃第五次代表大会开幕。在这次台上,“左派”社会革命党人展开了一场激战来反对列宁,维护富农。他们要求停止反富农的斗争,要求放弃派遣工人征粮队去农村。“左派”社会革命党人看到他们的路线受到大会多数的坚决回击,便在莫斯科发动了叛乱,占据了三仙巷,接着从那里向克里姆林宫开炮射击。但是经过几小时,“左派”社会革命党人的这次行动就被布尔什维克镇压下去了。在国内其他许多地点,“左派”社会革命党人的地方组织也企图暴动,但是这种冒险行动不管在哪里都很快被扑灭了。

正如现在对反苏的“右倾分子——托洛茨基派联盟”的审判所证实的,“左派”社会革命党人的叛乱足布哈林和托洛茨基知道和同意的,并且是布哈林派、托洛茨基派和“左派”社会革命党人反对苏维埃政权的反革命阴谋总计划的一部分。

同时,“左派”社会革命党人勃柳姆金(后来成了托洛茨基的代理人)钻进德国使馆刺死了德国驻莫斯科大使米尔巴赫,以挑起同德国的战争。但是,苏维埃政府成功地防止了战争,使反革命分子的挑拨来能得逞。

在苏维埃第五次代表大会上通过了俄罗斯苏堆埃联邦社会主义共和国宪法,即第一个苏维埃宪法。


\subsection{简短的结论}

布尔什维克党在1917年2月至10月的八个月中完成了极其困难的任务。争得了工人阶级和苏维埃中的多数,把千百万农民吸引到了社会主义革命方面来。它使这些群众摆脱了各小资产阶级党派(社会革命党、孟什维克、无政府主义者)的影响,它一步一步地揭露了这些政党的反对劳动者利益的政策。布尔什维克党在前线和后方开展了大规模的政治工作,组织群众作好了参加十月社会主义革命的准备。

在党的这一段历史上具有决定意义的因素是,列宁的回国,列宁的四月提纲,党的四月代表会议和第六次代表大会。工人阶级从党的决议中汲取了力量和胜利的信心,找到了对革命的极其重要的问题的答案。四月代表会议指引党争取从资产阶级民主革命过渡到社会主义革命。第六次代表大会把党指向实行武装起义去推翻资产阶级及其临时政府的目标。

妥协主义的党派社会革命党和孟什维克、无政府主义者和其他非共产主义党派,结束了自己的发展过程——还在十月革命前它们都已经成了保护资本主义制度完整无损的资产阶级党派。布尔什维克党独自领导了群众推翻资产阶级和建立苏维埃政权的斗争。

同时,布尔什维克粉碎丁党内投降主义者季诺维也夫、加米涅夫、李可夫、布哈林、托洛茨基和皮达可夫想使党离开社会主义革命道路的企图。

布尔什维克党所领导的工人阶级,联合贫苦农民,并在士兵和水兵的支持下,推翻了资产阶级政权,建立了苏维埃政权,创建了新型的国家即苏维埃社会主义国家,废除了地主土地所有制,把土地转交农民使用.把全国一切土地收归国有,剥夺了资本家,争取到退出战争而赢得了和平,获得了必要的喘息时机,从而创造了开展社会主义建设的条件。

十月社会主义革命击溃了资本主义,剥夺了资产阶级的生产资料,把工厂、土地、铁路和银行变成了全民财产,即公共财产。

十月社会主义革命建立了无产阶级专政,把一个巨大国家的领导权交给了工人阶级,从而使它成了统治阶级。

这样,十月社会主义革命就开辟了人类历史的新纪元——无产阶级革命的纪元。

