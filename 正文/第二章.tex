\section[第二章\q 俄国社会民主工党的成立。党内布尔什维克和孟什维克两派的出现(1901—1904年)]{第二章\\ 俄国社会民主工党的成立。\\ 党内布尔什维克和孟什维克两派的出现\\ {\zihao{3}(1901—1904年)}}

\subsection[一\q 1901—1904年间俄国革命运动的高涨]{一\\ 1901—1904年间俄国革命运动的高涨}

十九世纪末欧洲爆发了工业危机。这次危机很快就蔓延到了俄国。在危机年代(1900—1903年)倒闭的大小企业将近三千家。有十万多工人被解雇。在业工人的工资大大降低。工人先前通过顽强的经济罢工从资本家那里争得的一点让步,又被资本家夺回去了。

工业危机和失业并没能阻止和削弱工人运动。恰巧相反,工人的斗争愈来愈带有革命的性质了。工人开始从经济罢工转到政治罢工。最后,工人又转到游行示威,提出关于民主自由的政治要求,提出“打倒沙皇专制制度”的口号。

1901年,彼得堡奥布霍夫兵工厂发生的五一罢工变成了工人与军队间的流血冲突。工人们只能用石头和铁块去反抗武装的沙皇军队。工人们的顽强抵抗被击破了。接着就是残酷的镇压:约有八百个工人被捕,许多人被关进监狱或被流放服苦役。但英勇的“奥布霍夫防卫战”对俄国工人发生了很大的影响,在他们中间激起了同情的浪潮。

1902年3月,巴土姆工人举行了大规模的罢工和游行示威,这些行动都是社会民主党巴土姆委员会组织的。巴土姆的游行示威激发了南高加索全境的工农群众。

同年,在顿河岸罗斯托夫也发生了大规模的罢工。开始是铁路工人,接着有很多工厂的工人参加进来。罢工激发了全市工人。在城外开了几天的群众大会,到会工人达三万之多。在这些群众大会上宣读了社会民主党的宣言,一些人发表了演说。警察和哥萨克无法驱散这成千上万的工人集会。有几个工人被警察打死了,第二天送葬时举行了大规模的工人游行示威。沙皇政府只是从邻近城市调来大批军队,才把这次罢工镇压下去。罗斯托夫工人的斗争,是在俄国社会民主工党顿河区委员会领导下进行的。

1903年的罢工规模更大。这一年,在南俄各地,包括整个南高加索(巴库、梯弗里斯、巴土姆)和乌克兰各大城市(敖德萨、基辅、叶加特林诺斯拉夫),都发生了群众性政治罢工。罢工更持久更有组织了。同过去工人阶级的发动不同,现在工人的政治斗争几乎在每一个地方都是社会民主党委员会领导的。

俄国工人阶级已经奋起同沙皇政权作革命斗争了。

工人运动对农民发生了影响。1902年春夏两季,在乌克兰(波尔塔瓦省和哈尔科夫省)和伏尔加河流域,农民运动开展起来了。农民烧毁地主的庄园,夺取地主的土地,杀死他们痛恨的地方官和地主。政府派遣了军队来镇压起义的农民,向他们开枪射击,成百地逮捕他们,把许多领导者和组织者关进监狱,但革命的农民运动仍有增无减。

工农的革命发动表明俄国革命日益成熟、日益迫近了。

学生的反政府运动也在工人革命斗争影响下加强起来。政府为对付学生的游行示威和罢课封闭了学校,把成百的学生关进监狱,最后还想出了把不肯屈服的学生送去当兵的办法。对此,全国各大学学生于1901年底至1902年初这个冬天举行全国总罢课。参加这次罢课的达三万人。

工人和农民的革命运动,特别是政府对学生的迫害,使自由派资产者和那些把持着所谓地方自治局的自由派地主们也积极行动起来,坚决地“抗议”沙皇政府采取“极端行动”迫害他们在大学念书的子弟。

地方自治派的自由派分子的据点是地方自治局。所谓地方自治局,就是专管有关农村居民的纯粹地方性事宜(修筑道路,建造医院和学校)的地方管理机关。自由派地主在地方自治局里占有很大的势力。他们同自由派资产者有密切的联系,差不多同他们融为一体了。因为他们在自己的田庄上已开始从半农奴制经济过渡到更为有利的资本主义经济。这两部分自由派当然是拥护沙皇政府的,但他们反对沙皇政府的“极端行动”,担心正是这种“板端行动”会使革命运动加强起来。他们害怕沙皇政府的“极端行动”,但更害怕革命。自由派抗议沙皇政府的“极端行动”是想一举两得:第一,“开导”沙皇;第二,给自己戴上对沙皇制度“大为不满”的假面具,以博得人民的信任,使人民或一部分人民离开革命,从而削弱革命。

当然,地方自治派的自由主义运动对于沙皇制度的生存没有任何危险,但它毕竟表明沙皇制度的“永恒”基础情况不妙。

地方自治派的自由主义运动导致1902年成立了资产阶级的“解放社”,它就是后来俄国主要的资产阶级政党立宪民主党的核心。

沙皇政府眼见工农运动的洪流在全国日益汹涌澎湃,就拼命来制止革命运动。对工人的罢工和游行示威愈来愈多地采用武力,枪弹和皮鞭已成为沙皇政府回答工农发动的通常手段,监狱和流放地已有人满之患。

沙皇政府除加紧施行高压手段外,还企图用其他一些较为“灵活的”不带高压性的办法来引诱工人离开革命运动。曾经尝试建立一种受宪兵和警察监护的冒牌工人组织。这种组织当时被叫作“警察社会主义”的组织,或祖巴托夫组织(以建立这些警察的工人组织的宪兵上校祖巴托夫而得名)。沙皇的保安局通过自己的奸细力图欺骗工人,说沙皇政府自愿帮助工人来满足他们的经济要求。祖巴托夫分子向工人说:“既然沙皇自己站在工人方面,还何必搞什么政治,何必干什么革命呢。”祖巴托夫分子在几个城市里成立了自己的组织。1904年,加邦神父按照祖巴托夫组织的样式,为着同样的目的,建立了一个名为“彼得堡俄罗斯工厂工人大会”的组织。

但沙皇的保安局控制工人运动的尝试没有得逞。沙皇政府采取这种办法,对付不了日益高涨的工人运动。工人阶级的日益高涨的革命运动把这种警察组织一扫而光。

\subsection[二\q 列宁建立马克思主义政党的计划。“经济派”的机会主义。《火星报》为列宁计划而斗争。列宁的《怎么办?》一书。马克思主义政党的思想基础]{二\\ 列宁建立马克思主义政党的计划。\\ “经济派”的机会主义。\\ 《火星报》为列宁计划而斗争。\\ 列宁的《怎么办?》一书。\\ 马克思主义政党的思想基础}

虽然1898年举行的俄国社会民主党第一次代表大会宣告了党的成立,党还是没有建立起来。当时还没有党纲和党章。第一次代表大会选出的党中央委员会不久就被破获,再也没有恢复过,因为没有人去恢复它。不仅如此,第一次代表大会以后,党内思想上的混乱和组织上的涣散更加厉害了。

如果说1884—1894年是战胜民粹主义,为建立社会民主党而从思想上进行准备的时期,1894—1898年是试图(虽然没有成功)把各个马克思主义组织统一成社会民主党的时期,那末1898年以后的时期,就是党内思想上组织上混乱状态更加厉害的时期。马克思主义对民粹主义的胜利和工人阶级的革命发动,证明了马克思主义者的正确,增加了革命青年对马克思主义的同情。马克思主义已成为一种时髦了。于是知识分子中的大批革命青年涌进了马克思主义组织,他们在理论上很孱弱,在组织和政治上没有经验,只是从当时充满出版界的“合法马克思主义者”的机会主义作品中获得了一些关于马克思主义的模糊的、大部分是不正确的概念。这种情况,就使马克思主义组织的理论和政治水平降低,使其中掺进了“合法马克思主义的”机会主义情绪,使思想上的混乱、政治上的动摇和组织上的涣散更加厉害起来。

工人运动的日益高涨和革命时机的显然逼近,要求成立一个能领导革命运动的统一集中的工人阶级政党。但当时党的各个地方机关,各个地方委员会、团体和小组的情况非常不好,它们的组织上的涣散和思想上的混乱非常厉害,使建立这样一个政党成了极其困难的任务。

其所以困难,不仅是因为建党工作必须在沙皇政府残酷迫害的烈火中进行,沙皇政府经常把一些组织中的优秀工作人员抓走,把他们送去流放、关进监狱和判处服苦役。其所以困难,还因为很大一部分地方委员会及其工作人员,除了本地那种细小的实际工作外,什么事情也不愿过问,不懂得党内缺乏组织上和思想上的统一的害处,习惯于党的涣散状态,习惯于党内思想上的混乱,并认为没有统一集中的政党也可以过得去。

要建立集中的政党,就必须克服各个地方机关的这种落后性、守旧心理和狭隘的实际主义。

但这还不是全部。当时党内还有很大一批人拥有自己的机关刊物(在俄国有《工人思想报》,在国外有《工人事业》杂志),他们从理论上为党内组织上的涣散和思想上的混乱辩护,甚至往往赞美这种状况.认为建立统一集中的工人阶级政党是个不必要的和臆想出来的任务。

这些人就是“经济派”和他们的信徒。

要建立统一的无产阶级政党,首先必须击败“经济派”。

于是列宁就来执行这些任务和进行建立工人阶级政党的工作。

关于建立统一的工人阶级政党应该从何着手的问题,存在着不同的意见。有些人认为要建立党必须从召开党的第二次代表大会开始,认为召开代表大会就能把各个地方组织统一起来,把党建立成功。列宁反对这种意见。他认为在召开代表大会以前,首先必须把党的目的和任务问题弄清楚;必须知道我们所想建立的究竟是怎样一个党;必须在思想上同“经济派”划清界限;必须如实地公开地告诉党,在关于党的目的和任务问题上存在着两种不同的意见,即“经济派”的意见和革命社会民主派的意见;必须在报刊上广泛地宣传革命社会民主派的观点,正如“经济派”在他们的机关报刊上宣传自己的观点一样;必须让各个地方组织有在这两派之间作自觉选择的机会。只有做了这番必要的准备工作之后,才可以召开党代表大会。

列宁直截了当地说:

\begin{quotation}
“在统一以前,并且为了统一,首先必须坚决而明确地划清界限。”(《列宁全集》俄文第3版第4卷第378页)\footnote{见《列宁全集》第5卷第334页。——译者注}
\end{quotation}

因此列宁认为,建立工人阶级政党的工作应从创办一个为革命社会民主派的观点进行宣传鼓动的战斗的全俄政治报着手,创办这样的报纸应是建党工作的第一步。

列宁在《从何着手?》这篇著名的文章中,拟定了一个具体的建党计划(后来在著名的《怎么办?》一书中又加以发挥)。

\begin{quotation}
列宁在这篇文章中说:“我们认为,创办全俄政治报应当是行动的出发点,是建立我们所希望的组织\footnote{指建党。——译者注}的第一个实际步骤,并且是我们使这个组织不断向深广发展的纲。……没有报纸就不可能有系统地进行有坚定原则的和全面的宣传鼓动。进行这种宣传鼓动一般说来是社会民主党的经常的和主要的任务,而在目前,在广大居民阶层已经对政治、对社会主义问题产生兴趣时,这更是特别迫切的任务。”(《列宁全集》俄文第3版第4卷第110页)\footnote{见《列宁全集》第5卷第6—7页。——译者注}
\end{quotation}

列宁认为,这样的报纸不仅会成为从思想上把党凝为一体的工具,而且会成为从组织上把各个地方组织统一成为一个党的工具。这个报纸的代办员和通讯员网既然代表了各个地方组织,就会成为一个骨架而使党能够围绕着它在组织上集中起来。列宁说:因为“报纸不仅是集体的宣传员和集体的鼓动员,而且是集体的组织者”\footnote{见《列宁全集》第5卷第8页。——译者注}。

\begin{quotation}
列宁在同一篇文章中说:“这种代办员网将成为正是我们所需要的那种组织的骨干,——这种组织,其规模之大使它能够遍布全国各地;其广泛性和多样性使它能够实行精密而细致的分工;其坚定性使它在任何情况下,在任何‘转变关头’和意外情况下都能坚持不渝地进行自己的工作;其灵活性使它一方面在敌人把全部力量集中在一点的时候,善于避免同这个占绝对优势的敌人公开作战,另一方面又善于利用这个敌人的迟钝,在敌人最难料到的地方和时间突然攻其不备。”(《列宁全集》俄文第3版第4卷第112页)\footnote{见《列宁全集》第5卷第9页。——译者注}
\end{quotation}

《火星报》就应当成为这样的报纸。

而《火星报》也确实成了这样的全俄政治报,确实为党在思想上组织上凝为一体作好了准备。

讲到党本身的结构和成分时,列宁认为党应当由两部分组成:(一)人数不多、经常进行工作、作为骨干的领导工作人员,这里主要是职业革命家,即摆脱了其他任何职业而专搞党的工作的人员,这些人具有最必要的理论知识、政治经验、组织技能,并且善于同沙皇警察作斗争,善于避开警察的耳目;(二)广泛的地方党组织网,人数众多并且受到千百万劳动者同情和拥护的党员群众。

\begin{quotation}
列宁写道:“我认为:(1)任何革命运动,如果没有一种稳定的和能够保持继承性的领导者组织,便不能持久;(2)自发地卷入斗争……的群众越加广泛,这种组织也就越加迫切需要,也就应当越加巩固……(3)参加这种组织的主要应当是以革命活动为职业的人;(4)在专制制度的国家内,我们越缩小这种组织的成员的数量,缩小到只吸收那些以革命活动为职业并且在与政治警察作斗争的艺术方面受过专门训练的人参加,这种组织也就会越难‘捕捉’;(5)而工人阶级和其他社会阶级中能够参加这个运动并且在运动中间积极工作的人数也就会越多。”(同上,第456页)\footnote{见《列宁选集》第2版第1卷第334—335页。——译者注}
\end{quotation}

讲到当时应当建立的党的性质及其对于工人阶级的作用以及党的目的和任务时,列宁认为党应当是工人阶级的先进部队,应当是领导工人运动、统一并指导无产阶级阶级斗争的力量。党的最终目的是推翻资本主义和建立社会主义。最近目的是推翻沙皇制度和建立民主制度。要推翻资本主义,就必须先推翻沙皇制度,所以党在当时的主要任务是发动工人阶级,发动全体人民去同沙皇制度斗争,开展反对沙皇制度的人民革命运动并推翻沙皇制度,为走向社会主义扫除第一个大障碍。

\begin{quotation}
列宁说:“历史现在向我们提出的当前任务,是比其他任何一个国家的无产阶级的一切当前任务都更要革命的任务。实现这个任务,即摧毁这个不仅是欧洲的同时也是(我们现在可以这样说)亚洲的反动势力的最强大的堡垒,就会使俄国无产阶级成为国际革命无产阶级的先锋队。”(《列宁全集》俄文第3版笫4卷第382页)\footnote{见《列宁选集》第2版第1卷第245页。——译者注}
\end{quotation}

又说:

\begin{quotation}
“我们应当记住,为满足个别要求,为取得个别让步而同政府展开的斗争,不过是和敌人的小小接触,不过是小小的前哨战,决战还在后面。在我们面前矗立着一座强有力的敌人堡垒,从那里向我们发射出雨点般的炮弹和子弹,杀害我们的优秀战士。我们一定要夺取这座堡垒。只要我们能够把觉醒了的无产阶级的一切力量和俄国革命者的一切力量统一成一个党,并能使俄国一切生气勃勃和正直的人都倾向于这个党,我们就一定能够拿下这座堡垒。只有到那个时候,才能实现俄国工人革命家彼得·阿列克谢也夫的伟大预言:‘等到千百万工人群众举起筋肉条条的拳头,士兵刺刀保卫着的专制枷锁就会被粉碎’”(同上,第59页)\footnote{同上,第211—212页。——译者注}
\end{quotation}

这就是列宁在沙皇专制的俄国条件下建立工人阶级政党的计划。

“经济派”马上就来向列宁的计划开火了。

“经济派”认为反对沙皇制度的一般政治斗争是所有阶级的事情,首先是资产阶级的事情,所以这个斗争对于工人阶级无关紧要,因为工人主要是关心同厂主进行经济斗争,主要是为了增加工资、改善劳动条件等等。因此,社会民主党人主要的和当前的任务不是进行反对沙皇制度的政治斗争,不是推翻沙皇制度,而是组织“工人对厂主和政府作经济斗争”,并且他们所谓同政府作经济斗争,就是要求改善工厂立法。“经济派”硬说用这样的方式就能“赋予经济斗争本身以政治性质”。

“经济派”已经不敢公开说工人阶级不需要政党了。但他们认为党不应是工人运动的领导力量,不应去干预工人阶级的自发运动,尤其不应去领导它,而应跟着它走,研究它,从中吸取教训。

其次,“经济派”认为:自觉成分在工人运动中的作用,社会主义意识和社会主义理论的组织和指导作用,是微不足道的或者几乎是微不足道的;社会民主党不应该把工人提高到社会主义意识的水平,相反地应该去适应并降低到工人阶级中等阶层以至更落后的阶层的水平;社会民主党不应该向工人阶级灌输社会主义意识,而应该等待工人阶级的自发运动自己锻炼出社会主义意识。

讲到列宁提出的建党组织计划时,他们认为这个计划等于是在强制自发运动。

列宁在《火星报》上,特别是在著名的《怎么办?》一书中,给了“经济派”的这种机会主义哲学以迎头痛击,彻底粉碎了它。

(一)列宁指出,引诱工人阶级脱离反对沙皇制度的一般政治斗争,把它的任务局限于反对厂主和政府的经济斗争而毫不损伤厂主和政府,就是使工人永远陷于奴隶地位。工人对厂主和政府的经济斗争是争取改善向资本家出卖劳动力条件的工联主义斗争,但工人斗争的目的不仅是要改善自己向资本家出卖劳动力的条件,而且是要根本消灭那迫使工人向资本家出卖劳动力和遭受剥削的资本主义制度。但当工人运动道路上还站着沙皇制度这只资本主义的看门狗时,工人就无法开展反资本主义的斗争,无法开展争取社会主义的斗争。因此,党和工人阶级的当前任务是扫除沙皇制度,以便打通走向社会主义的道路。

(二)列宁指出,赞美工人运动的自发过程而否认党的领导作用,把党的作用归结为充当事变的登记者,这就是宣传“尾巴主义”,主张把党变成自发过程的尾巴,变成运动的消极力量,即只能观望自发过程和指靠自流趋势的力量。进行这种宣传,就是企图消灭党,也就是使工人阶级陷于没有政党的地位,使工人阶级陷于没有武器的地位。而当工人阶级面前站着这样的敌人,如拥有各种斗争手段的沙皇制度,以及按现代方式组织起来、在自己的政党领导下来同工人阶级作斗争的资产阶级的时候,使工人阶级陷于没有武器的地位就等于背叛工人阶级。

(三)列宁指出,崇拜工人运动的自发性而降低自觉性的作用,即降低社会主义意识、社会主义理论的作用,那就是:第一,侮辱向往自觉性象向往光明一样的工人;第二,使党蔑视理论,亦即使党蔑视其认识现在和预见将来的武器;第三,完全彻底地滚进机会主义泥潭。

\begin{quotation}
列宁说:“没有革命的理论,就不会有革命的运动。……只有以先进理论为指南的党,才能实现先进战士的作用。”(《列宁全集》俄文第3版第4卷第380页)\footnote{见《列宁选集》第2版第1卷第241—242页。——译者注}
\end{quotation}

(四)列宁指出,“经济派”硬说社会主义思想体系能从工人阶级自发运动中产生出来,这是欺骗工人阶级,因为实际上社会主义思想体系不是从自发运动中,而是从科学中产生出来的。“经济派”否认向工人阶级灌输社会主义意识的必要性,就是替资产阶级思想体系清扫道路,使其容易灌输和注入到工人阶级中去,因而就是葬送工人运动同社会主义相结合的思想,就是帮助资产阶级。

\begin{quotation}
列宁说:“对工人运动自发性的任何崇拜和对‘自觉成分’的作用即社会民主党的作用的任何轻视,完全不管轻视者自己愿意与否,都是加强资产阶级思想体系对于工人的影响。”(《列宁全集》俄文第3版第4卷第390页)\footnote{见《列宁选集》第2版第1卷第254页。——译者注}
\end{quotation}

又说:

\begin{quotation}
“问题只能是这样:或者是资产阶级的思想体系,或者是社会主义的思想体系。这里中间的东西是没有的……因此,对于社会主义思想体系的任何轻视和任何脱离,都意味着资产阶级思想体系的加强。”(同上,第391—392页)\footnote{同上,第256页。——译者注}
\end{quotation}

由于《怎么办?》一书的传播,在它出版一年后(它是1902年3月出版的),即到俄国社会民主党第二次代表大会的时候,“经济主义”的思想立场给人留下的已只是不愉快的回忆,而“经济派”这一称号在党内大多数工作人员心目中已经臭了。

这就是说,已经从思想上把“经济主义”彻底粉碎了,把机会主义即尾巴主义、自流主义的思想体系彻底粉碎了。

但列宁的《怎么办?》一书的意义还不限于此。

《怎么办?》一书的历史意义,在于列宁在这部有名的著作中:

(一)在马克思主义思想史上第一次彻底揭露了机会主义的思想根源,指出这种根源首先在于崇拜工人运动的自发性而降低社会主义意识在工人运动中的作用;

(二)把理论、自觉性和党的意义提到了应有的高度,指出它们是使自发的工人运动革命化的力量和领导力量;

(三)光辉地论证了马克思主义政党是工人运动同社会主义的结合这一马克思主义的根本原理;

(四)英明地制定了马克思主义政党的思想基础。

在《怎么办?》一书中所发挥的理论原理,后来成了布尔什维克党思想体系的基础。

《火星报》既拥有这样的理论财富,就能够为实现列宁的建党计划、聚集全党力量、召开党的第二次代表大会、捍卫革命的社会民主党、反对“经济派”、反对一切机会主义者、反对修正主义者而开展一个广泛的运动,并且确实开展起来了这样一个运动。

《火星报》最重要的任务是制定党纲草案。大家知道,工人党的纲领是对工人阶级的斗争目的和斗争任务的简要而科学的说明。党纲既要规定无产阶级革命运动的最终目的,也要规定党在实现最终目的的道路上所应争取的种种要求。因此,制定党纲草案不能不具有头等重要的意义。

拟制党纲草案时,在《火星报》编辑部内,列宁同普列汉诺夫以及其他编委之间发生了严重的意见分歧。这些意见分歧与争论几乎导致列宁和普列汉诺夫完全决裂。但决裂在当时还没有发生。由于列宁的坚持,终于在党纲草案中加进了无产阶级专政这一最重要的条文,并明确指出了工人阶级在革命中的领导作用。

党纲的土地问题部分,全部是列宁提出的。列宁当时已经主张土地国有,不过他认为在斗争第一阶段上必须提出一个要求:把“割地”即地主在“解放”农民时割去的那一部分农民土地归还给农民。普列汉诺夫反对土地国有。

列宁与普列汉诺夫在党纲问题上的争论,部分地决定了后来布尔什维克与孟什维克之间的意见分歧。


\subsection[三\q 俄国社会民主工党第二次代表大会。党纲党章的通过和统一的党的成立。大会上的意见分歧和党内两派即布尔什维克派与孟什维克派的出现]{三\\ 俄国社会民主工党第二次代表大会。\\ 党纲党章的通过和统一的党的成立。\\ 大会上的意见分歧和党内两派即布尔什维克派与\\孟什维克派的出现}

这样,列宁原则的胜利以及《火星报》为实现列宁组织计划所取得的成功,准备好了建立党或如当时所说建立真正的党所必需的一切基本条件。《火星报》的方针已在俄国各个社会民主主义组织中获得了胜利。现在已经可以召开党的第二次代表大会了。

1903年7月17日(7月30日)\footnote{本书中俄历和公历并用时,俄历在括号外,公历在括号内。——译者注},召开了俄国社会民主工党第二次代表大会。这次大会是在国外秘密举行的。大会最初几次会议在布鲁塞尔举行。后因比利时警察要求大会代表离开比境,于是大会移到伦敦举行。

参加这次大会的有四十三名代表,代表着二十六个组织。每个委员会有权选派两名代表参加大会,但有些委员会只派了一名代表。所以四十三名代表一共拥有五十一票表决权。

大会的主要任务是“在《火星报》所提出和制定的原则的和组织的基础上建立真正的政党”(《列宁全集》俄文第3版第6卷第164页)\footnote{见《列宁全集》第7卷第195页。——译者注}。

大会成分不是清一色的。大会上没有露骨的“经济派”代表出席,因为他们已经遭到失败。但他们在这个时期巧妙地改头换面了,以致能派出几个代表混进代表大会。此外,崩得的代表也只是口头上与“经济派”有所不同,实际上他们是拥护“经济派”的。

所以出席大会的不仅有《火星报》的拥护者,而且也有《火星报》的反对者。《火星报》的拥护者有三十三人,即占大多数。然而并非所有自命为火星派的人都是真正的列宁火星派。大会代表分成了几个集团。列宁的拥护者,即坚定的火星派,共有二十四票。有九个火星派分子是拥护马尔托夫的,他们是不稳定的火星派。有一部分代表动摇于《火星报》和《火星报》反对者之间,他们在大会上共有十票,这就是中派。公开的《火星报》反对者共有八票(三个“经济派”分子和五个崩得分子)。只要火星派内部一发生分裂,《火星报》的敌人就能占得上风。

由此可见,代表大会上的情形是异常复杂的。列宁费了很大力量,才保证《火星报》在大会上获得了胜利。

大会最重要的事情是通过党的纲领。讨论纲领时引起大会上机会主义分子反对的主要问题,就是无产阶级专政问题。机会主义者在纲领的其他许多问题上也不赞成大会上革命分子的意见。但他们决定主要是在无产阶级专政问题上开火,理由是说国外许多社会民主党的纲领上没有无产阶级专政这一条,因此也就可以不把这一条列入俄国社会民主党的纲领。

机会主义者还反对把农民问题上的要求写进党纲。这些人根本就不想革命,所以他们对工人阶级的同盟者农民采取歧视态度,对之表示厌恶。

崩得分子和波兰社会民主党人反对民族自决权。列宁总是教导说,工人阶级必须反对民族压迫。反对在纲领上规定这个要求,就等于主张抛弃无产阶级国际主义而助长民族压迫。

列宁对这一切反对意见都给予了致命的打击。

大会通过了《火星报》提出的纲领。

这个纲领分为最高纲领和最低纲领两部分。最高纲领说的是工人阶级政党的主要任务:进行社会主义革命,推翻资本家政权,建立无产阶级专政。最低纲领说的是党在推翻资本主义制度、建立无产阶级专政以前所应实现的当前任务:推翻沙皇专制制度,建立民主共和国,为工人实行八小时工作制,在农村中消灭一切农奴制残余,把地主夺去的农民土地(“割地”)归还给农民。

后来,布尔什维克用没收全部地主土地的要求代替了归还“割地”的要求。

第二次代表大会通过的纲领,是工人阶级政党的革命纲领。

这个纲领一直存在到党的第八次代表大会,这时,我们党在无产阶级革命胜利以后通过了新的纲领。

党的第二次代表大会通过了纲领之后,就来讨论党章草案。大会既已通过了党纲,为党在思想上的统一奠定了基础,当然也就要通过党章,以便彻底消除手工业方式和小组习气,消除组织涣散和党内缺乏坚强纪律的状况。

但是,如果说党纲的通过还比较顺利,那末党章问题却在大会上引起了激烈的争论。最尖锐的意见分歧,是在讨论党章第一条即党员资格这一条的条文时展开的。什么人可以当党员,党的成分应是怎样的,党在组织方面应是一个有组织的整体还是一种不定形的东西,这就是讨论党章第一条时产生的问题。当时有两个条文互相对立:一个是由列宁提出而为普列汉诺夫和坚定的火星派支持的条文;另一个是由马尔托夫提出而为阿克雪里罗得、查苏利奇、不稳定的火星派、托洛茨基以及代表大会上所有一切公开的机会主义分子支持的条文。

列宁的条文是说:凡承认党纲、在物质上支持党并参加党的一个组织的人,都可以成为党员。马尔托夫的条文虽认为承认党纲和在物质上支持党是做党员的必要条件,却不承认参加党的一个组织是做党员的条件,竟认为党员也可能不是党的一个组织中的一员。

列宁把党看作是有组织的部队,其中各个成员并不是自行列名入党,而是由党的一个组织接收入党,因此他们必须服从党的纪律;而马尔托夫却把党看作是一种组织上不定形的东西,其中各个成员都是自行列名入党,他们既不参加党的一个组织,因此也就不必服从党的纪律。

由此可见,马尔托夫的条文与列宁的条文不同,它为那些不稳定的非无产者分子大开入党之门。在资产阶级民主革命前夜,资产阶级知识分子中一些暂时同情革命的人有时甚至也能给党不大的帮助。但这些人决不会加入组织,不会服从党的纪律,不会执行党的委托,不会承担由此产生的危险。马尔托夫和其他孟什维克分子却主张承认这样的人为党员,主张给予他们影响党内事务的权利和机会。他们甚至主张让每个罢工者都有自行“列名”入党的权利,虽然参加罢工的也有非社会主义者、无政府主义者和社会革命党人。

因此,马尔托夫派所想有的并不是列宁和列宁派在大会上所力争的那种一元化的、战斗性的、组织严密的党,而是成分复杂、组织涣散和没有定形的党,这样一个党单只因为它成分复杂和不可能有坚强纪律,也就不可能成为一个战斗性的党。

不稳定的火星派脱离坚定的火星派而同中派结成联盟,再加上公开的机会主义者同他们联合,就使马尔托夫在这个问题上占得了优势。大会以一票弃权.二十八票对二十二票的多数通过了马尔托夫提出的党章第一条。

从火星派因为党章第一条而发生分裂以后,大会上的斗争更加剧烈了。大会已临近结束,即将选举党的领导机关——党中央机关报(《火星报》)编辑部和中央委员会了。可是,在代表大会进行选举以前出了几件事,使大会上的力量对比发生了变化。

由于党章的关系,大会讨论到了崩得问题。崩得想在党内获得一种特殊地位。它要求承认它是俄国犹太工人的唯一代表。接受崩得这一要求,就等于在党组织里把工人分成不同的民族类别而放弃工人阶级统一的阶级的地区组织。大会否决了崩得在组织问题上的民族主义。于是崩得分子退出了大会。接着,两个“经济派”分子也退出了大会,因为大会拒绝承认他们那个国外联合会为党的国外代表机关。

七个机会主义者退出大会,使力量对比发生了有利于列宁派的变化。

对于党中央机关的人选问题,列宁一开始就特别注意。列宁认为必须把坚定彻底的革命者选进中央委员会。马尔托夫派却竭力想使不坚定的、机会主义的分子在中央委员会里占优势。大会多数在这个同题上赞成列宁的意见。选进中央委员会的都是拥护列宁的人。

根据列宁的提议,选进《火星报》编辑部的是列宁、普列汉诺夫和马尔托夫。马尔托夫在大会上要求把《火星报》原先六个编委都选进《火星报》编辑部,因为他们大多数是拥护马尔托夫的分子。大会以多数否决了这个提议。列宁提出的三人小组当选了。于是马尔托夫声明,他不参加中央机关报编辑部。

这样,大会通过对党中央机关问题的表决确定了马尔托夫派的失败和列宁派的胜利。

从这时起,拥护列宁的人因在大会选举时获得多数票,就被称为布尔什维克;而反对列宁的人因为获得少数票,就被称为孟什维克\footnote{布尔什维克是俄语的音译,意即多数派;孟什维克也是俄语音译,意即少数派。——译者注}。

把第二次代表大会的工作总结一下,可以得出如下的结论:

(一)大会巩固了马克思主义对于“经济主义”,即对于公开的机会主义的胜利;

(二)大会通过了党纲和党章,建立了社会民主党,因而建好了统一的党的骨架;

(三)大会揭示出在组织问题上存在着严重的意见分歧,这种分歧把党一分为二,即分成了布尔什维克和孟什维克,前者坚持革命社会民主党的组织原则,后者则滚进了组织涣散的泥潭,滚进了机会主义的泥潭;

(四)大会表明,旧的机会主义者“经济派”虽已被党击溃,新的机会主义者孟什维克又在党内代之而起;

(五)大会在组织问题上没能胜任,表现过动摇,有时甚至让孟什维克占了优势,虽然大会到结束时已有所改正,但它毕竟不仅没能揭穿孟什维克在组织问题上的机会主义,使他们在党内陷于孤立,甚至没能向党提出这样的任务。

后一情况,就成了布尔什维克和孟什维克之间的斗争在代表大会后不仅没有熄灭,反而更加尖锐的一个主要原因。


\subsection[四\q 孟什维克首领们的分裂行动和第二次代表大会后党内斗争的尖锐化。孟什维克的机会主义。列宁的《进一步,退两步》一书。马克思主义政党的组织基础]{四\\ 孟什维克首领们的分裂行动和\\第二次代表大会后党内斗争的尖锐化。\\ 孟什维克的机会主义。\\ 列宁的《进一步,退两步》一书。\\ 马克思主义政党的组织基础}

第二次代表大会后,党内斗争更加尖锐了。孟什维克竭力破坏党的第二次代表大会的决议,夺取党的中央机关。孟什维克要求让他们的代表加入《火星报》编辑部和中央委员会,而且他们的人数要在编辑部里占多数,在中央委员会里同布尔什维克相等。由于这同第二次代表大会决议根本抵触,布尔什维克拒绝了孟什维克的要求。于是孟什维克就瞒着党暗中成立了他们以马尔托夫、托洛茨基和阿克雪里罗得为首的反党派别组织,并如马尔托夫所说,“发动了反列宁主义的起义”。他们所采取的反党斗争手段,就是“破坏全部党的工作,败坏事业,阻挠一切”(列宁语)\footnote{参看《列宁全集》第7卷第352页。——译者注}。他们盘踞在十分之九是由那些脱离了俄国实际工作的旅外知识分子组成的俄国社会民主党人“国外同盟”里,从那里向党、向列宁和列宁派开火射击。

普列汉诺夫给孟什维克帮了大忙。在第二次代表大会上,普列汉诺夫本来是同列宁一道走的。但在第二次代表大会以后,普列汉诺夫被孟什维克实行分裂的要挟吓倒了。他决定无论如何要同孟什维克“和解”。把普列汉诺夫拉到孟什维克方面去的,是他过去那些机会主义错误的货色。普列汉诺夫本人很快就从一个主张同机会主义者孟什维克调和的人变成了孟什维克。普列汉诺夫要求把代表大会否决了的所有原来的孟什维克编委都加进《火星报》编辑部。列宁当然不能同意这种做法,于是退出了《火星报》编辑部,以便在党中央委员会里站定脚跟,从这个阵地上去打击机会主义者。普列汉诺夫违反代表大会的意志,独自一人把先前的孟什维克编委补进了《火星报》编辑部。从这时起,即从《火星报》第52号起,孟什维克就把《火星报》变成了自己的机关报,并经过《火星报》来宣传自己的机会主义观点。

从这时起,党内开始有了旧《火星报》和新《火星报》的叫法,前者是指列宁的布尔什维克的《火星报》,而后者是指孟什维克的机会主义的《火星报》。

自从《火星报》转入孟什维克手里,它就成了反对列宁、反对布尔什维克的机关报,成了宣传孟什维克机会主义,首先是宣传组织问题上的孟什维克机会主义的机关报。孟什维克同“经济派”和崩得分子结合起来后,就在《火星报》上,如他们所说,开始向列宁主义大举进攻。普列汉诺夫没能坚持调和立场,不久也加入了这种进攻。按事物的逻辑,这也是势所必然的:谁主张同机会主义者调和,谁就必然要滚到机会主义那里去。在新《火星报》上出现了雪片似的声明和论文,说党不应当成为一个有组织的整体;说党内必须容许自由的团体和个人存在,他们不必服从党机关的决议;说必须让每个同情党的知识分子以及“每个罢工者”和“每个示威者”都有自行宣布为党员的权利;说要求党员服从党的一切决议是一种“形式主义的官僚主义的”态度;说要求少数服从多数就是“硬性压制”党员意志;说要求全体党员,无论领导人或普通党员,都同样服从党的纪律,就是在党内建立“农奴制度”;说“我们”在党内不是需要集中制,而是需要无政府主义的“自治制”,使各个人和各个党组织都有权不执行党的决议。

这是放肆宣传组织上的松懈,宣传破坏党性和党的纪律,赞美知识分子个人主义,为不守纪律的无政府主义行为辩护。

孟什维克显然是在拖党后退,从党的第二次代表大会退到组织涣散状态上去,退到小组习气上去,退到手工业方式上去。

必须给予孟什维克一个坚决的回击。

于是,列宁在《进一步,退两步》这本有名的著作(1904年5月出版)中给了他们这样一个回击。

以下就是列宁在这本书里所发挥而后来成了布尔什维克党组织基础的基本组织原理:

(一)马克思主义政党是工人阶级的一部分,是它的一支部队。但工人阶级有很多部队,所以并不是工人阶级的任何一支部队都可以称为工人阶级的党。党与工人阶级其他部队不同的地方首先就在于,党是工人阶级的先进的部队、觉悟的部队、马克思主义的部队而不是普通的部队,它以社会生活的知识、社会生活发展规律的知识、阶级斗争规律的知识为武装,所以它善于引导工人阶级,善于领导工人阶级的斗争。因此,决不能把党和工人阶级混淆起来,就像不能把部分和整体混淆起来、不能让每个罢工者都能自行宣布为党员一样,因为谁要是把党和阶级混淆起来,谁就会把党的觉悟水平降低到“每个罢工者”的水平,谁就会把党是工人阶级的先进觉悟部队的这种作用取消掉。党的任务不是把自己的水平降低到“每个罢工者”的水平,而是把工人群众、把“每个罢工者”提高到党的水平。

\begin{quotation}
列宁写道:“我们是阶级的党,因此,几乎整个阶级(而在战争时期,在国内战争年代,甚至完全是整个阶级)都应当在我们党的领导下行动,都应当尽量紧密地靠近我们党;但是,如果以为在资本主义制度下,不论在什么时候,几乎整个阶级或者整个阶级都能提高到自己的先进部队即自己的社会民主党的觉悟程度和积极程度,那就是马尼洛夫精神\footnote{马尼洛夫是果戈里小说《死魂灵》中的一个地主,他具有多愁善感、痴心妄想的性格。这里,马尼洛夫精神是想入非非的意思。——译者注}和‘尾巴主义’。还没有一个明白事理的社会民主党人怀疑过,在资本主义制度下,连工会组织(比较原始的、比较容易为落后阶层的觉悟程度接受的组织)也不能包括几乎整个工人阶级或者整个工人阶级。忘记先进部队和倾向于它的所有群众之间的区别,忘记先进部队的经常责任是把愈益广大的阶层提高到这个先进的水平,那只是欺骗自己,漠视我们的巨大任务,缩小这些任务。”(《列宁全集》俄文第3版第5卷第205—206页)\footnote{见《列宁选集》第2版第1卷第457—458页。——译者注}
\end{quotation}

(二)党不仅是工人阶级先进的觉悟的部队,而且同时又是工人阶级有组织的部队,它有为党员必须遵守的纪律。因此,党员一定要参加党的一个组织。如果党不是本阶级有组织的部队,不是一个组织体系,而是一些自行宣布为党员,但不参加党的一个组织、因而没有组织起来、亦即不必服从党的决议的人所构成的简单的总和,那它永远不会有统一的意志,永远不能实现自己党员行动上的统一,因而就会无法领导工人阶级的斗争。党只有当它所有的党员都组织成一个由统一意志、统一行动、统一纪律团结起来的统一部队时,才能实际地领导工人阶级的斗争,把它引向一个目标。

孟什维克提出反驳,说这样一来,许多知识分子,譬如大学教授、大学生、中学生等等,就会留在党外,因为他们不愿意加入党的某个组织,这或是因为他们经受不起党的纪律,或是因为他们如普列汉诺夫在第二次代表大会上所说的那样,觉得“加入某个地方组织是有失体面的事情”。孟什维克的这种反驳正好打了孟什维克自己,因为党并不需要经受不起党的纪律和害怕加入党的组织的党员。工人并不害怕纪律和组织,所以他们一决定来做党员,就乐意加入组织。只有怀着个人主义心理的知识分子才害怕纪律和组织,因此他们确实是会留在党外的。但这正是一种好现象,因为这样一来,党就能避免不稳定分子涌进党内来的现象,而这种现象在当时,即在资产阶级民主革命开始高涨的时期,是特别厉害的。

\begin{quotation}
列宁写道“如果我说,党应当是组织的总和(并且不是什么简单的算术式的总和,而是一个整体),那末……我只是以此来十分明确地表示自己的愿望,自己的要求,使作为阶级的先进部队的党成为尽量有组织的,使党只容纳至少能接受最低限度组织性的分子……”(《列宁全集》俄文第3版第6卷第203页)\footnote{见《列宁选集》第2版第1卷第454—455页。——译者注}
\end{quotation}

又说:

\begin{quotation}
“马尔托夫的条文在口头上是拥护无产阶级广大阶层的利益的,但是事实上却是为那些害怕无产阶级纪律性和组织性的资产阶级知识分子的利益效劳。谁也不敢否认:作为现代资本主义社会中特殊阶层的知识分子,他们的特点,一般和整个说来,正是个人主义和不能接受纪律性和组织性。”(同上,第212页)\footnote{同上,第466页。——译者注}
\end{quotation}

又说:

\begin{quotation}
“……无产阶级是不怕组织和纪律的!无产阶级是不会因为那些不愿加入组织的大学教授先生和中学学生先生在党组织监督下工作,就急于要承认他们是党员的。……并不是无产阶级,而是我们党内某些知识分子,在组织和纪律方面缺乏自我教育。”(同上,第307页)\footnote{同上,第484页。——译者注}
\end{quotation}

(三)党不只是一个有组织的部队,而且是工人阶级一切组织中的“最高组织形式”,其使命是领导工人阶级其他一切组织。党既是本阶级中以先进理论、阶级斗争规律的知识和革命运动的经验为武装的优秀分子所组成的最高组织形式,就完全能够领导、而且也应该领导工人阶级的其余一切组织。孟什维克力图缩小和降低党的领导作用,结果就会削弱党所领导的无产阶级的其他一切组织,亦即削弱无产阶级并解除它的武装,因为“无产阶级在争取政权的斗争中,除了组织而外,没有别的武器”(《列宁全集》俄文第3版第6卷第328页)\footnote{同上,第510页。——译者注}。

(四)党是工人阶级先进部队与工人阶级千百万群众联系的体现。党若不与非党群众发生联系,不扩大这种联系,不巩固这种联系,那末,无论它是怎样优秀的先进部队,无论它组织得怎样好,也是不能生存和得到发展的。党如果闭关自守而与群众隔绝,丧失或哪怕是削弱同本阶级的联系,那它一定会失去群众的信任和支持,因而不可避免要陷于灭亡。党要想有旺盛的生命力和得到发展,就应当扩大同群众的联系,获得本阶级千百万群众的信任。

\begin{quotation}
列宁说:“要成为社会民主党,就必须得到本阶级的支持。”(《列宁全集》俄文第3版第6卷第208页)\footnote{见《列宁选集》第2版第1卷第160页。——译者注}
\end{quotation}

(五)党要正确地发挥作用和有计划地领导群众,就必须按集中制原则组织起来,就需要有统一的党章,需要有统一的党的纪律,需要有由党代表大会所体现、在党代表大会闭会期间由党中央委员会所体现的统一的全党最高领导机关,需要少数服从多数、各个组织服从中央、下级组织服从上级组织。没有这些条件,工人阶级的党就不能成为真正的党,就不能实现领导本阶级的任务。

当然,由于党在沙皇专制制度条件下处于秘密存在的地位,党的组织当时不可能建立在自下而上的选举制基础上,因此党不得不具有极秘密的性质。但列宁认为这是我们党的生活中一种暂时的现象,这种现象在推翻沙皇制度以后立刻就会消失,那时党就会成为公开的合法的党,而党的组织就会建立在民主选举的原则上,建立在民主集中制的原则上。

\begin{quotation}
列宁写道:“从前,我们党还不是正式的有组织的整体,而只是各个集团的总和,所以在这些集团间除了思想影响以外,别的关系是不可能有的。现在,我们已经成为有组织的政党,这也就是说造成了一种权力,思想威信变成了权力威信,党的下级机关应该服从党的上级机关。”(《列宁全集》俄文第3版第6卷第291页)\footnote{见《列宁全集》第7卷第360页。——译者注}
\end{quotation}

列宁责备孟什维克是组织上的虚无主义和老爷式的无政府主义,因为他们不容许党的权力和党的纪律加在他们头上,列宁写道:

\begin{quotation}
“这种老爷式的无政府主义是俄国虚无主义者所特有的。党的组织在他们看来是凶恶可怕的‘工厂’;部分服从整体和少数服从多数的原则在他们看来是‘农奴制’……他们一听见在中央领导下实行分工,就发出可怜又可笑的嚎叫,反对把人们变成‘小轮子和小螺丝钉’(并且他们认为特别吓死人的,就是把编辑变成工作人员),他们一听见有人提起党的组织章程,就作出瞧不起人的样子,发表鄙视的(对“形式主义者”)意见,说完全不要章程也可以。”(同上,第310页)\footnote{见《列宁选集》第2版第1卷第487页。——译者注}
\end{quotation}

(六)党要保持自己队伍的统一,就应当在实践上实行统一的无产阶级纪律,即全体党员,无论是领导人或普通党员,都必须同样遵守的纪律。因此,党内不应有什么不必服从纪律的“上等人”和必须服从纪律的“平凡人”之分。没有这个条件,就不能保持党的完整和党的队伍的统一。

\begin{quotation}
列宁写道:“马尔托夫及其伙伴们根本没有正当的理由反对代表大会所任命的编辑部,这最好是用他们自己所说的‘我们不是农奴!’……一语来说明。这就非常明显地暴露出资产阶级知识分子的心理,他们把自己看成高于群众组织和群众纪律的‘上等人物’。……知识分子个人主义者,总觉得任何一种无产阶级组织和纪律,都好象是农奴制。”(《列宁全集》俄文第3版第6卷第282页)\footnote{见《列宁全集》第7卷第348—349页。——译者注}
\end{quotation}

又说:

\begin{quotation}
“随着我们这个真正的政党的形成,觉悟的工人应当学会辨别无产阶级军队的战士的心理和爱说无政府主义空话的资产阶级知识分子的心理,应当学会不仅要求普通党员,而且要求‘上层人物’履行党员的义务。”(同上,第312页)\footnote{见《列宁选集》第2版第1卷第490页。——译者注}
\end{quotation}

列宁总结对于意见分歧的分析、指出孟什维克的立场是“组织问题上的机会主义”时,认为孟什维主义的基本罪过之一就是低估了党组织的意义,不承认党组织是无产阶级在争取解放的斗争中的武器。孟什维克认为无产阶级的党组织对于革命的胜利没有重大的意义。列宁反对孟什维克的看法,认为单靠无产阶级的思想统一还不足以获得胜利,——为要获得胜利,还必须用无产阶级“组织的物质统一”来“巩固”思想上的统一。列宁认为无产阶级只有在这个条件下,才能成为不可战胜的力量。

\begin{quotation}
列宁写道,“无产阶级在争取政权的斗争中,除了组织而外,没有别的武器。无产阶级既然被资产阶级世界中居于统治地位的无政府的竞争所分散,既然被那种为资本的强迫劳动所压抑,既然经常被抛到赤贫、粗野和退化的‘底层’,无产阶级所以能够成为而且必然会成为不可战胜的力量,就是因为它根据马克思主义原则形成的思想统一是用组织的物质统一来巩固的,这个组织把千百万劳动者团结成工人阶级的大军。在这支大军面前,无论是已经衰败的俄国专制政权或正在衰败的国际资本政权,都是支持不住的。”(《列宁全集》俄文第3版第6卷第328页)\footnote{见《列宁选集》第2版第1卷第510页。——译者注}
\end{quotation}

列宁就是用这样一段预言来结束他这本书的。

列宁在《进一步,退两步》这本有名的著作中所发挥的基本的组织原理,就是如此。

这本书的意义首先在于,它反对小组习气而捍卫了党性,反对捣乱派而捍卫了党,粉碎了孟什维克在组织问题上的机会主义,奠定了布尔什维克党的组织基础。

但这本书的意义还不止于此。它的历史意义就在于,列宁在这本书中在马克思主义历史上第一次制定了关于党的学说,确定党是无产阶级的领导组织,是无产阶级手中的基本武器,没有这个武器,就无法在为无产阶级专政进行的斗争中获得胜利。

由于列宁的《进一步,退两步》一书在党的工作者中间的传播,大多数地方组织团结到了列宁的周围。

可是,各地方组织愈是紧密地团结在布尔什维克周围,孟什维克首领们的行为也愈加狠毒。

1904年夏,孟什维克由于普列汉诺夫的帮助和两个变质的布尔什维克——克拉辛和诺斯科夫的叛变而夺得了中央委员会的多数。当时看得很清楚,孟什维克是力图造成分裂的。布尔什维克失去《火星报》和中央委员会之后,处于很困难的境地。必须创办布尔什维克自己的报纸。必须筹备新的代表大会,即党的第三次代表大会,以便建立新的党中央和清算孟什维克的破坏行为。

于是列宁、布尔什维克就来执行这一任务。

布尔什维克积极准备召开党的第三次代表大会。1904年8月,在瑞士境内由列宁领导举行了有二十二名布尔什维克出席的会议。会议通过了《告全党书》,它成了布尔什维克为召开第三次代表大会而斗争的纲领。

在三个区域(南方区域、高加索区域、北方区域)举行的布尔什维克委员会代表会议上选出了多数派委员会常务局,它负责进行了实际筹备党的第三次代表大会的工作。

1905年1月4日,布尔什维克主办的《前进报》创刊号出版了。

这样,在党内就形成了两个彼此独立的派别,即布尔什维克和孟什维克,各有自己的中央,各有自己的机关报。


\subsection{简短的结论}

在1901—1904年时期,在革命工人运动增长的基础上,俄国各地马克思主义的社会民主主义组织成长和巩固起来了。在反对“经济派”的原则的、顽强的斗争中,列宁的《火星报》的革命路线获得了胜利,克服了思想上的涣散和“手工业方式”。

《火星报》把零散的社会民主主义小组和团体联成一体,并为召开党的第二次代表大会进行准备。在1903年召开的第二次代表大会上,成立的俄国社会民主工党,通过了党纲和党章,建立了党的中央领导机关。

在第二次代表大会上为火星派方针的最终胜利进行的斗争中,俄国社会民主工党内部出现了两个集团,即布尔什维克集团和孟什维克集团。

在第二次代表大会以后,布尔什维克和孟什维克间的主要意见分歧是在组织问题上展开的。

孟什维克同“经济派”接近起来,并在党内代替了他们的地位。孟什维克的机会主义暂时还只表现在组织问题上。孟什维克反对建立列宁式的、战斗的、革命的党,而主张建立涣散的、无组织的、尾巴主义的党。他们在党内推行分裂路线。他们在普列汉诺夫帮助下夺得了《火星报》和中央委员会,并利用这两个中央机关去实现他们的分裂主义的目的。

布尔什维克眼看孟什维克要造成分裂,就采取措施来约束分裂派,动员各个地方组织准备召开第三次代表大会,并出版了自己的报纸《前进报》。

这样,在俄国第一次革命前夜,在日俄战争已经开始的时候,布尔什维克和孟什维克已表现为两个彼此独立的政治派别。

