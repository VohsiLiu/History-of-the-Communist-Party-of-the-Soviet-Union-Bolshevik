\section[第四章\q 孟什维克和布尔什维克在斯托雷平反动时期。布尔什维克正式形成为独立的马克思主义政党(1908—1912年)]{第四章\\ 孟什维克和布尔什维克在斯托雷平反动时期。\\布尔什维克正式形成为独立的马克思主义政党 \\{\zihao{3}(1908—1912年)}}

\subsection[一\q 斯托雷平实行反动。反政府阶层知识分子的蜕化。颓废情绪。党内一部分知识分子转入马克思主义敌人营垒而企图修正马克思主义理论。列宁在《唯物主义和经验批判主义》一书中驳斥修正主义而捍卫马克思主义政党的理论基础]{一\\ 斯托雷平实行反动。\\ 反政府阶层知识分子的蜕化。颓废情绪。\\党内一部分知识分子转入马克思主义敌人营垒而企图\\修正马克思主义理论。\\列宁在《唯物主义和经验批判主义》一书中驳斥修正主义\\而捍卫马克思主义政党的理论基础}

1907年6月3日,沙皇政府解散了第二届国家杜马,于是这天在历史上就称为六三政变日。沙皇政府颁布了新的法令,即第三届国家杜马选举法,从而违背了它自己在1905年10月17日发表的宣言,因为这个宣言上说,沙皇政府必须经过杜马同意才能颁布新的法令。

第二届杜马中的社会民主党党团被交付法庭审判,工人阶级的代表们被进去服苦役和终身流放。

新选举法使地主和工商业资产阶级在杜马中的代表人数大大增加,而使本来就很少的工农代表人数减少了好几成。

第三届杜马其成分是黑帮和立宪民主党的杜马。在总共四百四十二个杜马代表席位中,右派(黑帮)占一百七十一席,十月党人和同他们相近的团体占一百一十三席,立宪民主党人和同他们相近的团体一百零一席,劳动派十三席,社会民主党十八席。

右派(所以如此称呼,是因为他们在杜马开会时坐在右边议席上)所代表的是工农最凶恶的敌人,即黑帮农奴制地主(他们常常在镇压农民运动时鞭笞和枪杀大批农民,他们是蹂躏犹太人、殴打示威工人、在革命期间野蛮地焚烧群众集会场所的组织者)。右派主张用最残暴的手段镇压劳动群众,拥护沙皇的无限权力,反对1905年10月17日颁布的沙皇宣言。

在杜马里同右派接近的是十月党,或称“十月十七日同盟”。十月党人所代表的是大工业资本和采用资本主义经营方式的大地主的利益(1905年革命开始时,立宪民主党人中很大一部分大地主转到了十月党人方面)。十月党人同右派的区别,只在于他们承认——也只是口头上承认——10月17日宣言、十月党人完全拥护沙皇政府的对内对外政策。

“立宪民主”党在第三届杜马中所占的议席,要比它在第一、第二两届杜马中所占的少,因为一部分立宪民主党人地主的票已转到十月党人方面去了。

第三届杜马中有一个人数不多的小资产阶级民主派集团,即所谓劳动派。劳动派在杜马中动摇于立宪民主党人和工人民主派(布尔什维克)之间。列宁指出,虽然劳动派在杜马中个分软弱,但是他们代表着群众,代表着农民群众。劳动派动摇于立宪民主党人和工人民主派之间,这是小业主的阶级地位必然产生的结果。列宁向布尔什维克代表即工人民主派提出了一个任务:“……帮助软弱的小资产阶级民主派,使他们摆脱自由派的影响,团结民主派阵营去反对反革命的立宪民主党人,而不仅仅反对右派分子。”(《列宁全集》俄文第3版第15卷第486页)\footnote{见《列宁全集》第18卷第38页。——译者注}

在1905年革命的过程中,特别是在革命失败以后,立宪民主党人愈来愈暴露出他们是一种反革命的力量。他们愈来愈抛弃自己的“民主”假而具,而表现为十足的保皇派,沙皇制度的维护者。1909年,一群著名的立宪民主党人著作家出版了一部《路标》文集,立宪民主党人在这部文集中代表资产阶级感谢沙皇镇压了革命。立宪民主党人向沙皇的皮鞭绞架政府匍匐跪拜,直言不讳地写道:应该“为这个政权祝福,因为只有它才用刺刀和牢狱为我们(即自由资产阶级)挡开人民的凶焰”。

沙皇政府在解散了第一届国家杜马和镇压了社会民主党杜马党团之后,就来大力摧毁无产阶级的政治组织和经济组织。苦役牢房、大狱和流放地关满了革命者。革命者在监狱里遭到毒打,受尽各种刑罚和折磨。黑帮的恐怖猖獗到极点。沙皇大臣斯托雷平在全国各地布满绞架。有几千个革命者惨遭杀害。当时一般人把绞架叫做“斯托雷平的领带”。

沙皇政府在镇压工农革命运动的时候,不可能局限于高压手段,局限于讨伐队、枪毙,监禁和苦役。沙皇政府眼看农轻信“沙皇老爹”的心理消失下去而惶恐不安,于是它就大耍手腕,打算培植一个人数众多的农村资产阶级即富农,作为自己在农村中的坚强支柱。

1906年11月9口,斯托雷平颁布了一道新的土地法令,允许农民退出公社而另立农庄。根据斯托雷平的土地法令,公社土地使用制可以破坏。每个农民可以把自己的份地变成私产,可以退出公社。农民可以出卖自己的份地,而他们从前是没有权利这样做的。公社必须给每个退社农民分地,分的地必须在一个地方(所谓独立农庄,独立田庄)。

富裕的农民——富农现在有了可能用贱价向力量单薄的农民收买土地。在该法令颁布后的几年间,有一百多万力量单薄的农民完全失去土地而陷于破产,富农独立农庄和独立田庄的数目靠掠夺力量单薄的农民的土地而增加起来。有时它们简直成了十足的大地产,在那里广泛采用雇佣劳动,即雇农劳动。政府强迫农民把公社中最好的土地分给富农庄主。

如果说从前在“解放”农民时掠夺农民土地的是地主,那么现在掠夺公社土地的便是富农,他们得到最好的地段,用贱价向贫农收买份地。

沙皇政府给富农发放大量贷款来收买土地和成立独立农庄。斯托雷平想把富农变成小地主,变成沙皇专制制度的忠实卫士。

在九年(1906--1915年)内退出公社的农户,总共在二百万户以上。

斯托雷平政策使少地的农民和农村贫民的状况更恶化了。农民中的分化加剧。农民开始同富农庄主发生冲突。

同时农民已经开始理解,只要存在着沙皇政府和地主与立宪民主党的国家杜马,他们就无法得到地主的土地。

起初,在另立独立农庄的现象盛行时期(1907—1909年),农民运动有过低落,但是不久,到1910—1911年间以及较晚的时候,农民反对地主和富农庄主的运动就在公社社员同庄主发生冲突的基础上加强起来了。

工业方面的革命以后也起了很大的变化。工业集中的情况,即工业规模扩大和集中于愈来愈大的资本家集团手中的情况,大大加剧了。还在1905年革命以前,资本家已开始联合起来成立同盟,以图在国内提高商品价格,把赚来的超额利润用作鼓励商品输出的基金,使商品可以输出到国外市场去贱价销售,夺取国外市场。资本家的这种同盟,这种联合组织(垄断组织),就叫做托拉斯和辛迪加。革命以后,资产阶级托拉斯和辛迪加的数量更多了。大银行数目也日渐增加、它们在工业中的作用愈益增长。流入俄国的外国资本有增无已。

于是俄国资本主义就愈益变成垄断的、帝国主义的资本主义了。

工业经过几年停滞以后又重新活跃起来,煤炭、金属和石油的产量提高了,纺织品和食糖的生产增长了。粮食的出口大大增加了。

虽然俄国当时在工业方面已有若干进步,但它同西欧相比仍然是个落后的国家,并且是个依赖于外国资本家的国家。当时俄国还不能生产机器和机床,它们都是从国外进口。当时俄国还没有汽车工业,还没有化学工业,还不能生产矿质肥料。在武器制造方面,俄国也比其他资本主义国家落后。

列宁指出俄国金属消费很低是俄国落后的标志时写道:

\begin{quotation}
“俄国自农奴获得解放后的半世纪内,铁的消费增加了四倍,但是俄国依然是一个难以置信的空前落后、贫穷和半野蛮的国家,它所装备的现代生产工具比英国少四分之三,比德国少五分之四,比美国少十分之九。”(《列宁全集》俄文第3版第16卷第543页)\footnote{见《列宁全集》第19卷第287页。——译者注}
\end{quotation}

俄国经济落后的直接结果,就是俄国资本主义和沙皇制度本身都依赖于西欧资本主义。

这表现于俄国国民经济中如煤炭、石油、电器工业和冶金业等最重要的部门都操在外资手中,沙俄所用的机器和设备几乎全部要从国外输入。

这表现于重利盘剥的外债,沙皇政府每年要从人民身上榨取几万万卢布来交付外债的利息。

这表现于与“盟国”缔结的许多秘密条约,根据这些条约,沙皇政府在战争爆发时,必须提供几百万俄国士兵到帝国主义战线上去支援“盟国”,以保证英法资本家获得骇人听闻的利润。

在斯托雷平反动年代,宪兵和警察,沙皇奸细和黑帮分子,用盗匪手段袭击工人阶级的现象特别流行。当时用高压手段来迫害工人的不仅有沙皇的鹰犬。工厂主在这方面也不落后,他们在工业停滞和失业人数增加的年代特别加紧向工人阶级进攻。工厂主宣布大批开除工人(同盟歇业),把积极参加罢工的觉悟工人列入“黑名册”。凡加入了本工业部门厂主同盟的企业,都拒绝雇用列入这种“黑名册”或“黑名单”的工人。计件工资标准在1908年降低了百分之十至十五。工作日普遍延长到十至十二小时。抢劫式的罚款制度又盛行起来了。

1905年革命的失败,使革命同路人开始瓦解和蜕化。这种蜕化和颓废情绪在知识分子中间特别厉害。同路人是在革命汹涌高涨时期从资产阶级方面跑进革命队伍的,他们一到反动时期就离开党了。其中一部分跑进了公开与革命为敌的阵营、另一部分则盘踞在保全下来的工人阶级合法团体中,竭力引诱无产阶级离开革命道路,拼命破坏无产阶级革命政党的威信。同路人离开革命竭力去迁就反动势力,想同沙皇制度和睦相处。

沙皇政府利用革命的失败,把那些贪生怕死和钻营私利的革命同路人招去替它当走狗,即替它当奸细。沙皇的保安局派遣到工人组织和党组织中去充当内奸的许多无耻叛徒,在内部进行特务活动,出卖革命者。

反革命势力在思想战线上也大举进攻。一大群时髦作家涌现出来,他们“批评”和“谴责”马克思主义,辱骂革命,讥笑革命,赞美叛卖行为,借口“崇拜个性”而鼓吹淫乱。

在哲学方面“批评”、修正马克思主义的现象加剧了,还出现了各种各样用冒牌“科学”论据作掩饰的宗教流派。

对马克思主义进行“批评”已成时髦。

这班老爷虽然牌号不同,但目的都一样:引诱群众离开革命。

一部分党内知识分子也浸染了颓废情绪和怀疑心理,他们自命为马克思主义者,但从来也没有在马克思主义立场上站稳过。其中有波格丹诺夫、巴札罗夫、卢那察尔斯基(他们是1905年归附布尔什维克的)、尤什凯维奇、瓦连廷诺夫(两人都是孟什维克)一类的著作家。他们从两方面同时展开“批评”,既反对马克思主义的哲学理论基础,即反对辩证唯物主义,又反对马克思主义的科学历史基础,即反对历史唯物主义。这种批评与一般批评不同的地方,就在于这种批评不是采取公开的直率的方式,而是以“维护”马克思主义基本立场为幌子,采用了暧昧的骗人的手法。他们说:我们基本上是马克思主义者。不过我们想把马克思主义“改善”一下,想使它摆脱某些基本原理的束缚。实际上,他们是仇视马克思主义的,因为他们在竭力破坏马克思主义的理论基础,虽然他们口头上伪善地否认他们对马克思主义的仇视,并继续用两面派手法自称为马克思主义者。这种伪善的批评非常危险,因为它是要欺骗党内普通工作人员,而且确实能把他们引入迷途。这种破坏马克思主义理论基础的批评愈伪善,它对党也就愈危险,因为它同反动势力对党对革命的总进攻会愈加紧密地结合起来。一部分离开了马克思主义的知识分子甚至鼓吹必须创造一种新的宗教(即所谓“寻神派”和“造神派”)。

在马克思主义者面前摆着一个极迫切的任务:必须给予这些背叛马克思主义理论的蜕化变节分子以应有的驳斥,撕破他们的假面具,把他们彻底揭穿,以捍卫马克思主义政党的理论基础。

本来可以期望普列汉诺夫和他那些自命为“著名马克思主义理论家”的孟什维克朋友们把这一任务担负起来的。但他们却宁愿撰写几篇无足轻重的批评性小品文来敷衍一下,接着就溜之大吉了。

这个任务。是由列宁在他那本1909年出版的有名著作《唯物主义和经验批判主义》中完成的。

\begin{quotation}
列宁在该书中写道:“不到半年就出版了四本书,这四本书主要是并且几乎完全是攻击辩证唯物主义的。其中,第一本是1908年在圣被得堡出版的巴札罗夫,波格丹诺夫、卢那察尔斯基,别尔曼、格尔方德、尤什凯维奇、苏沃洛夫的论文集《关于<?应当说是:反对>\footnote{尖括号内的话和标点符号是列宁加的。——译者注}马克思主义哲学的概论》,其次是尤什凯维奇的《唯物主义和批判实在论》,别尔曼的《从现代认识论来看辩证法》和瓦连廷诺夫的《马克思主义的哲学体系》。……所有这些因敌视辩证唯物主义而联合起来的人(尽管政治观点截然不同)在哲学上又自命为马克思主义者!别尔曼说;恩格斯的辩证法是‘神秘主义’。恩格斯的观点‘过时了’——巴札罗夫随口说了这么一句、好像这是不言而喻的。唯物主义看来被我们勇敢的战士驳倒了,他们自豪地引证着‘现代认识论’,引证着‘最新哲学’(或“最新实证论”),引证着‘现代自然科学的哲学’或者‘二十世纪的自然科学的哲学’。”(《列宁全集》俄文第3版第13卷第11页)\footnote{见《列宁选集》第2版第2卷第12—13页。——译者注}
\end{quotation}

当卢那察尔斯基为他那些哲学上的修正主义者朋友们辩护,说“也许我们错了,但我们是在探索”的时候,列宁回答道:

\begin{quotation}
“至于我自己,也是哲学上的一个‘探索者’。这就是说,我在本书中(指《唯物主义和经验批判生义》一书。——编者注)给自己提出的任务是:探索那些在马克思主义的幌子下发表一种非常混乱、含糊而有反动的言论的人们是在什么地方失足的”(《列宁全集》俄文第3版第13卷第12页)\footnote{见《列宁选集》第2版第2卷第14页。——译者注}
\end{quotation}

但事实上列宁这本书远远超出了这个谦虚的任务的范围。实际上,列宁这本书不仅对波格丹诺夫、尤什凯维奇、巴札罗夫、瓦连廷诺夫及其哲学老师阿芬那留斯和马赫进行了批评,批评他们在自己的著作里企图用精巧圆滑的唯心主义来对抗马克思主义的唯物主义。同时,列宁这本书还捍卫了作为马克思主义理论基础的辩证唯物主义和历史唯物主义,并用唯物主义观点总结了从恩格斯逝世到列宁《唯物主义和经验批判主义》一书问世的整个历史时期内,在科学方面、首先在自然科学方面所获得的一切最重大的和最主要的成果。

列宁在他这本书中狠狠地批评了俄国经验批判主义及其外国老师们之后,对哲学上理论上的修正主义做出了如下几个批判性的结论:

\begin{quotation}
(一)“日益巧妙地伪造马克思主义,日益巧妙地把各种反唯物主义的学说装扮成马克思主义,这就是现代修正主义在政治经济学上、策略问题上和一般哲学……上表现出来的特征”(《列宁全集》俄文第3版第13卷第270页)\footnote{见《列宁选集》第2版第2卷第337页,——译者注}

(二)“马赫和阿芬那留斯的整个学派……走向唯心主义”(同上,第291页)\footnote{同上,第304页。——译者注}

(三)“我们的马赫主义者全都落到了唯心主义……的网里去了”(同上,第282页)\footnote{同上,第353页。——译者注}

(四)“在经验批判主义认识论的繁琐语句后面,不能不看到哲学上的党派斗争,这种斗争归根到底表现着现代社会中敌对阶级的倾向和思想体系”(同上,第292页)\footnote{同上,第365页。——译者注}

(五)“经验批判主义的客观的、阶级的作用完全是在于替信仰主义者(印排斥科学而崇尚信仰的反动分子。——编者注)服役,帮助他们反对一般唯物主义、特别是历史唯物主义”(同上)\footnote{同上。——译者注}

(六)“哲学唯心主义是……通向僧侣主义的道路”(同上,第304页)\footnote{同上,第715页。——译者注}
\end{quotation}

为了评价列宁这本书在我党历史上的重大意义,为了了解列宁在反对斯托雷平反动时期各色各样的修正主义分子和蜕化变节分子时捍卫了多么巨大的理论财富,必须哪怕是简略地介绍一下辩证唯物主义和历史唯物主义的原理。

其所以必须这样做,尤其是因为辩证唯物主义和历史唯物主义是共产主义的理论基础,是马克思主义政党的理论基础,而了解这个基础,就是说,掌握这个基础,是我们党的每个积极活动家应尽的义务。

那么:

(一)什么是辩证唯物主义呢?

(二)什么是历史唯物主义呢?


\subsection[二\q 论辩证唯物主义和历史唯物主义]{二\\ 论辩证唯物主义和历史唯物主义}

辩证唯物主义是马克思列宁主义党的世界观。它所以叫作辩证唯物主义,是因为它对自然界现象的看法、它研究自然界现象的方法、它认识这些现象的方法是辩证的,而它对自然界现象的解释、它对自然界现象的了解、它的理论是唯物主义的。

历史唯物主义就是把辩证唯物主义的原理推广去研究社会生活,把辩证唯物主义的原理应用于社会生活现象,应用于研究社会,应用于研究社会历史。

马克思和恩格斯在说明自己的辩证方法的时候,通常援引黑格尔,认为他是表述了辩证法基本特征的哲学家。但这并不是说,马克思和恩格斯的辩证法同黑格尔的辩证法是一样的。其实,马克思和恩格斯从黑格尔的辩证法中采取的仅仅是它的“合理的内核”,而摈弃了黑格尔的唯心主义的外壳,并且向前发展了辩证法,赋予辩证法以现代的、科学的形态。

\begin{quotation}
马克思说:“我的辩证方法,从根本上来说,不仅和黑格尔的辩证方法不同,而且和它截然相反。在黑格尔看来,思维过程,即他称为观念而甚至把它变成独立主体的思维过程。是现实事物的创造主,而现实事物只是思维过程的外部表现。我的看法则相反,观念的东西不外是移入人的头脑并在人的头脑中改造过的物质的东西而已,”(马克思《〈资本论〉第一卷德文第二版跋》)\footnote{见《马克思恩格斯选集》第2卷第217页。——译者注}
\end{quotation}

马克思和恩格斯在说明自己的唯物主义的时候,通常援引费尔巴哈,认为他是恢复了唯物主义应有权威的哲学家。但这并不是说,马克思和恩格斯的唯物主义和费尔巴哈的唯物主义是一样的,其实,马克思和恩格斯是从费尔巴哈唯物主义中采取了它的“基本的内核”,把它进一步发展成为科学的哲学唯物主义理论,而摒摈弃了它那些唯心主义的和宗教伦理的杂质。大家知道,费尔巴哈虽然在基本上是唯物主义者,但是他竭力反对唯物主义这个名称。恩格斯屡次说过:费尔巴哈“虽然有唯物主义的基础,但是在这里还没有摆脱传统的唯心主义束缚”,“我们一接触到费尔巴哈的宗教哲学和伦理学,他的真正的唯心主义就显露出来了”(《马克思恩格斯全集》俄文第1版第14卷第652—654页)\footnote{见《马克思恩格斯选集》第4卷第226-227页和第229页。——译者注}

辩证法来源于希腊文``dialego"一词,意思就是进行谈话,进行论战。在古代,所谓辩证法,指的是以揭露对方论断中的矛盾并克服这些矛盾来求得真理的艺术。古代有些哲学家认为,思维矛盾的揭露以及对立意见的冲突,是发现真理的最好方法。这种辩证的思维方式后来推广到自然界现象中去,就变成了认识自然界的辩证方法,这种方法把自然界现象看作是永恒地运动着、变化着的现象,把自然界的发展看作是自然界中各种矛盾发展的结果,看作是自然界中对立力量互相影响的结果。

辩证法从根本上说来,是同形而上学截然相反的。

(一)马克思主义的辩证方法的基本特征是:

(1)同形而上学相反,辩证法不是把自然界看作彼此隔离、彼此孤立、彼此不依赖的各个对象或现象的偶然堆积,而是把它看作有联系的统一的整体,其中各个对象或现象互相有机地联系着,互相依赖着,互相制约着。

因此,辩证方法认为,自然界的任何一种现象,如果被孤立地、同周围现象没有联系地拿来看,那就无法理解,因为自然界的任何领域中的任何现象,如果把它看作是同周围条件没有联系、与它们隔离的现象,那就会成为毫无意义的东西;反之,任何一种现象,如果把它看作是同周围现象有着不可分割的联系、是受周围现象所制约的现象,那就可以理解、可以论证了。

(2)同形而上学相反,辩证法不是把自然界看作静止不动、停滞不变的状态,而是看作不断运动和变化、不断更新和发展的状态,其中始终有某种东西在产生,在发展;有某种东西在破坏,在衰颓。

因此,辩证方法要求我们观察现象时不仅要从各个现象的相互联系和相互制约的角度去观察,而且要从它们的运动、它们的变化、它们的发展的角度,从它们的产生和衰亡的角度去观察。

在辩证方法看来,最重要的不是现时似乎坚固,但已经开始衰亡的东西,而是正在产生、正在发展的东西,哪怕它现时似乎还不坚固,因为在辩证方法看来,只有正在产生、正在发展的东西,才是不可战胜的。

\begin{quotation}
恩格斯说:“整个自然界,从最小的东西最大的东西,从沙粒到太阳,从原生生物(原始的细胞。——编者注)到人,都处于永恒的产生和消灭中,处于不断的流动中,处于无休止的运动和变化中。“(《马克思恩格斯全集》俄文第1版第14卷第484页)\footnote{见《马克思恩格斯选集》第3卷第454页。——译者注}
\end{quotation}

恩格斯说,因此,辩证法“在考察事务及其在头脑中的反映时,本质上是从它们的联系、它们的连结、它们的运动、它们的产生和消失方面去考察的”(同上,第23页)\footnote{同上,第419-420页。——译者注}。

(3)同形而上学相反,辩证法不是把发展过程看作简单的增长过程,看作量变不引起质变的过程,而是看作从不显著的、潜在的量的变化到显露的变化,到根本的变化,到质的变化的发展,在这种发展过程中,质变不是逐渐地发生,而是迅速地、突然地发生的,表现为从一种状态飞跃式地进到另一种状态,并且不是偶然发生的,而是有规律地发生的,是由许多不明显的逐渐的量变积累而成的。

因此、辩证方法认为,不应该把发展过程了解为循环式的运动,了解为过去事物的简单重复,而应该把它了解为前进的动力,了解为上升的运动,了解为从旧质态到新质态的转化,了解为从简单到复杂、从低级到高级的发展。

\begin{quotation}
恩格斯说:“自然界是检验辩证法的试金石,而且我们必须说,现代自然科学为这种检验提供了极其丰富的、与日俱增的材料,并从而证明了,自然界的一切归根到底是辩证地而不是形而上学地发生的;自然界不是循着一个永远一样的不断重复的圆圈运动,而是经历着实在的历史。这里首先就应当指出达尔文,他极其有力地打击了形而上学的自然观,因为他证明了今天的整个有机界,植物和动物,因而也包括人类在内,都是延续了几百万年的发展过程的产物。”(《马克思恩格斯全集》俄文第1版第14卷第23页)\footnote{见《马克思恩格斯选集》第3卷第420页。——译者注}
\end{quotation}

恩格斯所在说明辩证的发展就是从量变到质变的转化时写道:

\begin{quotation}
“在物理学中……每种变化都是量到质的转化,是物体所固有或所承受的某一形式的运动的量在数量上发生变化的结果。例如,水的温度最初对它的液体状态是无足轻重的;但是由于液体水的温度的增加或减少,便会达到这样的一点,在这一点上这种聚集状态就会发生变化,水就会变为蒸汽或冰。……例如,必须有一定的最低强度的电流才能使电灯泡中的白金丝发光,每种金属都有自己的白热点和融解点,每种液体在一定的压力下都有其特定的冰点和沸点,——只要我们有办法造成相应的湿度;最后,例如,每种气体都有其临界点,在这一点上相当的压力和冷却能使气体变成液体。……物理学的所谓常数(从一种状态到另一种状态的转变点。——编者注),大部分不外是这样一些关节点的名称,在这些关节点上,运动的量的(变化)增加或减少会引起该物体的状态的质的变化、所以在这些关节点上,量转化为质。”(《马克思恩格斯全集》俄文第1版第14卷第527—528页)\footnote{见《马克思恩格斯选集》第3卷第487页。——译者注}
\end{quotation}

接着,恩格斯讲到化学时又说;

\begin{quotation}
“化学可以称为研究物体由于量的构成的变化而发生的质变的科学。黑格尔本人已经知道这一点……拿氧来说。如果结合在一个分子中的有三个原子,而不是象普通那样只有两个原子,那末我们就得到臭氧,一种在气味和作用上与普通氧很不相同的物体。更不待说,如果把氧同氮或硫按不同的比例化合起来,那么其中每一种化合都会产生出一种在质的方面和其他一切物体不同的物体!”(同上,第528页)\footnote{同上。第487-488页。——译者注}
\end{quotation}

最后,恩格斯在批评杜林,批评这位大骂黑格尔而暗中又剽窃黑格尔关于从无感觉世界王国进到感觉的王国,从无机界王国进到有机生命王国,是向新状态的飞跃这一著名原理的杜林时写道:

\begin{quotation}
“这完全是黑格尔的度量关系的关节线,在过里纯粹量的增多或减少,在一定的关节点上就引起质的飞跃,例如在把水加热或冷却的时候,沸点和冰点就是选种关节点,在这种关节点上——在标准压力下——完成了进入新的聚集状态的飞跃,因此,在这里量就转变为质。”(同上,第45—46页)\footnote{同上,第81页,——译者注}
\end{quotation}

(4)同形而上学相反,辩证法的出发点是:自然界的对象、自然界的现象含有内在的矛盾,因为它们都有其反面和正面,都有其过去和将来,都有其衰颓着的东西和发展着的东西,而这种对立而的斗争,旧东西和新东西之间、衰亡着的东西和产生着的东西之间、衰颓着的东西和发展着的东西之间的斗争,就是发展过程的内在内容,就是量变转化为质变的内在内容。

因此,辩证方法认为,从低级到高级的发展过程不是通过现象和谐的开展,而是通过对象、现象本身固有矛盾的揭露,通过在这些矛盾基础上活动的对立趋势的“斗争”进行的。

\begin{quotation}
列宁说:“就本来的意义说,辩证法是研究对象的本质自身中的矛盾。”(列宁《哲学笔记》俄文版第263页)\footnote{见《列宁全集》第38卷第278页。——译者注}
\end{quotation}

其次:

\begin{quotation}
“发展是对立面的‘斗争’。”(《列宁全集》俄文第3版第13卷第301页)\footnote{见《列宁全集》第2版第2卷第712页。——译者注}
\end{quotation}

简略说来,马克思主义的辩证方法的基本特征就是这样。

不难了解,把辨证方法的原理推广去研究社会生活和社会历史,该有多么巨大的意义;把这些原理应用到社会历史上去,应用到无产阶级党的实际活动上去,该有多么巨大的意义。

既然世界上没有孤立的现象,既然一切现象都是彼此联系、互相制约的,那就很明显,在估计历史上每一种社会制度、每一个社会运动的时候,不应当象历史学家常做的那样,从“永恒正义”或其他某种成见出发,而应当从产生这种制度、这个社会运动的条件和同它们有联系的条件出发。

奴隶占有制度,从现代的条件看来,是不可思议的现象,是反常的荒谬事情。然而在原始公社制度瓦解的条件下,奴隶占有制度却是完全可以理解的、合乎规律的现象,因为它同原始公社制度相比是前进了一步。

资产阶级民主共和国的要求,在沙皇制度和资产阶级社会存在的条件下,譬如说在1905年的俄国,是一种完全可以理解的、正确的和革命的要求,因为资产阶级共和国在当时意味着前进一步。资产阶级民族共和国的要求,从我们苏联现在的条件看来,却是一种不可思议的和反革命的要求,因为资产阶级共和国同苏维埃共和国相比是后退了一步。

一切以条件、地点和时间为转移。

显然,没有这种现察社会现象的历史观点,历史科学就会无法存在和发展,因为只有这样的观点才能使历史科学不致变成偶然现象的糊涂账,不致变成一堆荒谬绝伦的错误。

其次。既然世界是处在不断的运动和发展中,既然旧东西衰亡和新东西生长是发展的规律,那就很明显,没有什么“不可动摇的”社会秩序,没有什么私有制和剥削的“永恒原则”,没有什么农民服从地主、工人服从资本家的“永恒观念”。

这就是说,资本主义制度可以用社会主义制度来替代,正如资本主义制度在当时代替了封建制度一样。

这就是说,不要指靠已经不再发展的社会阶层,即使这些阶层在现时还是占优势的力量,而要指靠正在发展的、有前途的阶层,即使这些阶层在现时还不是占优势的力量。

在十九世纪八十年代,在马克思主义者和民粹派斗争的时期,俄国无产阶级同当时占人口绝大多数的个体农民比较起来,还是占极少数。但是无产阶级是一个发展着的阶级,农民则是一个日趋瓦解的阶级。正因为无产阶级是一个发展着的阶级,所以马克思主义者就指靠无产阶级。他们没有错,因为大家知道,无产阶级后来从一个不大的力量发展成了历史上和政治上的头等力量。

这就是说,要在政治上不犯错误,就要向前看,而不要向后看。

其次,既然从缓慢的量变进到迅速的、突然的质变是发展的规律,那就很明显,被压迫阶级进行的革命变革,是完全自然的和必不可免的现象。

这就是说,从资本主义过渡到社会主义,工人阶级摆脱资本主义压迫,不可能通过缓慢的变化,通过改良来实现,而只能通过资本主义制度的质变,通过革命来实现。

这就是说,要在政治上不犯错误,就要做革命者,而不要做改良主义者。

其次,既然发展是通过内在矛盾的揭露,通过基于这些矛盾的对立势力的冲突来克服这些矛盾而进行的,那就很明显,无产阶级的阶级斗争是完全自然的和必不可免的现象。

这就是说、不要掩饰资本主义制度的各种矛盾,而要暴露和揭开这些矛盾,不要熄灭阶级斗争,而要把阶级斗争进行到底。

这就是说,要在政治上不犯错误,就要执行无产阶级的不调和的阶级政策、而不要执行使无产阶级利益同资产阶级利益相协调的改良主义政策。不要执行使资本主义“长入”社会主义的妥协政策。

以上就是应用马克思主义的辩证方法观察社会生活,观察社会历史的情形。

至于马克思主义的哲学唯物主义,那从根本上说来,它是同哲学唯心主义截然相反的。

(二)马克思主义哲学唯物主义的基本特征是:

(1)唯心主义认为世界是“绝对观念”、“宇宙精神”、“意识”的体现,而马克思的哲学唯物主义却与此相反,它认为,世界按其本质说来是物质的;世界上形形色色的现象是运动着的物质的不同形态;辩证方法所判明的现象的相互联系和相互制约,是运动着的物质的发展规律;世界是按物质运动规律发展的,并不需要什么“宇宙精神”。

\begin{quotation}
恩格斯说:“唯物主义的自然现不过是对自然界本来而目的朴素的了解,不附加以任何外来的成分。”(《马克思恩格斯全集》俄文第1版第14卷第651页)\footnote{见《马克思恩格斯选集》第3卷第527页。——译者注}
\end{quotation}

古代哲学家赫拉克利特持着唯物主义的现点,认为“世界是包括一切的整体,它不是由任何神或任何人所创造的,它过去、现在和将来都是按规律燃烧着,按规律熄灭着的永恒的活火”。列宁在谈到这个唯物主义观点时说:“这是对辩证唯物主义原则的绝妙的说明。”(列宁《哲学笔记》俄文版第318页)\footnote{见《列宁全集》第38卷第395页。}

(2)唯心主义硬说,只有我们的意识才是真实存在的,物质世界、存在,自然界只是在我们的意识中,在我们的感觉、表象、概念中存在,而马克思主义的哲学唯物主义却与此相反,它认为,物质、自然界、存在,是在意识以外,不依赖意识而存在的客观实在;物质是第一性的,因为它是感觉、表象,意识的来源;而意识是第二性的,是派生的,因为它是物质的反映,存在的反映;思维是发展到高度完善的物质的产物,即人脑的产物,而人脑是思维的器官;因此,如果不愿意大错特错,那就不能把思维和物质分开。

\begin{quotation}
恩格斯说;“思维对存在、精神对自然界的关系问题,全部哲学的最高问题……哲学家依照他们如何回答选个问题而分成了两大阵营。凡是断定精神对自然界说来是本原的……组成唯心主义阵营。凡是认为自然界是本原的,则属于唯物主义的各种学派。”(《马克思选集》俄文版第1卷第329页)\footnote{见《马克思恩格斯选集》第4卷第220页。——译者注}
\end{quotation}

其次:

\begin{quotation}
“我们自己所属的物质的、可以感知的世界,是唯一现实的;而我们的意识和思维,不论它看起来是多么超感觉的,总是物质的、肉体的器官即人脑的产物。物质不是精神的产物,而精神却只是物质的最高产物。”(《马克思选集》俄文版第1卷第332页)\footnote{见《马克思恩格斯选集》第4卷第223页。——译者注}
\end{quotation}

马克思谈到物质和思维问题时说道:

\begin{quotation}
“不可能把思维同思维着的物质分开。物质是世界上所发生的一切变化的基础。”(同上,第302页)\footnote{见《马克思恩格斯选集》第3卷第384页。——译者注}
\end{quotation}

列宁在说明马克思主义的哲学唯物主义时写道:

\begin{quotation}
“一般唯物主义认为客观真实的存在(物质)不依赖于……意识、感觉、经验……意识都不过是存在的反映,至多也只是存在的近似正确的(恰当的、十分确切的)反映。”(《列宁全集》俄文第3版第13卷第265—267页)\footnote{见《列宁全集》第2版第2卷第382页。——译者注}
\end{quotation}

其次:

\begin{quotation}
“物质是作用于我们的感官而引起感觉的东西;物质是我们通过感觉感知的客观实在……物质、自然界,存在、物理的东西是第一性的,而精神、意识、感觉,心理的东西是第二性的。”(同上,第119--120页)\footnote{同上,第146—147页。——译者注}

“世界图景就是物质运动和‘物质思维’的图景。”(同上,第288页)\footnote{同上,第361页。——译者注}

“脑是思想的器官。”(同上,第125页)\footnote{同上,第153页。——译者注}
\end{quotation}

唯心主义否认认识世界及其规律的可能性,不相信我们知识的可靠性,不相信客观真理,并且认为世界上永远充满着科学永远不能认识的“自在之物”,而马克思主义的唯物主义却与此相反,它认为,世界及其规律完全可以认识,我们关于自然界规律的知识,经过经验和实践检验过的知识,是具有客观真理意义的、可靠的知识,世界上没有不可认识的东西,而只有还没有被认识、而将来科学和实践的力量会加以揭示和认识的东西。

恩格斯在批判康德和其他唯心主义者所谓世界不可认识和“自在之物”不可认识的论点,坚持唯物主义关于我们的知识是可靠知识这一著名原理时写道:

\begin{quotation}
“对这些以及其他一切哲学上的怪论的最令人信服的驳斥是实践,即实验和工业。既然我们自己能够制造出某一自然过程。使它按照它的条件产生出来,并使它为我们的目的服务,从而证明我们对这一进程的理解是正确的,那末康德的不可捉摸的‘自在之物’就完结了。动植物体内所产生的化学物质,在有机化学把它们一一制造出来以前,一直是这种‘自在之物’;当有机化学开始把它们制造出来时。‘自在之物’就变成为我之物了,倒如茜草的色素——茜素,我们已经不再从田地里的茜草根中取得,而是用便宜得多、简单得多的方法从煤焦油里提炼出来了。哥白尼的太阳系学说有三百年之久一直是一种假说,这个假说尽管有百分之九十九、百分之九十九点九、百分之九十九点九九的可靠性,但毕竟是一种假说;而当勒维烈从这个太阳系学说所提供的数据,不仅推算出一定还存在一个尚未知道的行星,而且还推算出选个行星在太空中的位置的时候,当后来加勒确实发现了这个行星的时候,哥白尼的学说就被证实了。”(《马克思选集》俄文版第1卷第330页)\footnote{见《马克思恩格斯选集》第4卷第221-222页}
\end{quotation}

列宁指责波格丹诺夫、巴扎罗夫,尤什凯维奇以及马赫的其他信徒堕入信仰主义,列宁坚持唯物主义的著名原理,即我们关于自然界规律的科学知识是可靠的;科学的规律是客观真理、列宁写道:

\begin{quotation}
“现代信仰主义决不否认科学;它只否认科学的‘过分的奢望’,即科学想达到客观真理的奢望。如果客观真理存在着(如唯物主义者所认为的那样),如果只有那在人类‘经验’中反映外部世界的自然科学才能给我们提供客观真理,那末一切信仰主义就被完全否定了。”(《列宁全集》俄文第3版第13卷第102页)\footnote{见《列宁选集》第2版第2卷第124页。——译者注}
\end{quotation}

简略说来,马克思主义的哲学唯物主义的特征就是这样。

显而易见、把哲学唯物主义原理推广去研究社会生活和社会历史,该有多么巨大的意义;把这些原理应用到社会历史上去,应用到无产阶级党的实际活动上去,该有多么巨大的意义。

既然自然现象的联系和相互制约是自然界发展的规律,那末由此可见,社会生活现象的联系和相互制约也同样不是偶然的事情,而是社会发展的规律。

这就是说,社会生活、社会历史不再是一堆“偶然现象”,因为社会历史成为社会有规律的发展,对社会历史的研究成为一种科学。

这就是说,无产阶级党的实际活动不应该以“卓越人物”的善良愿望为基础,不应该以“理性”、“普遍道德”等等的要求为基础,而应该以社会发展的规律为基础,以研究这些规律为基础。

其次。既然世界可以认识,既然我们关于自然界发展规律的知识是具有客观真理意义的、可靠的知识,那末由此应该得出结论;社会生活、社会发展也同样可以认识,关于社会发展规律的科学成果是具有客观真理意义的、可靠的成果。

这就是说,尽管社会生活现象错综复杂,但是社会历史科学能够成为例如同生物学一样准确的科学,能够拿社会发展规律来实际应用。

这就是说,无产阶级党在它的实际活动中,不应该以任何偶然动机为指南,而应该以社会发展规律、以这些规律中得出的实际结论为指南。

这就是说,社会主义从关于人类美好未来的空想变成了科学。

这就是说,科学和实际活动的联系,理论和实践的联系、它们的统一,应当成为无产阶级党的指路明星。

其次。既然自然界、存在、物质世界是第一性的,而意识,思维是第二性的,是派生的;既然物质世界是不依赖于人们意识而存在的客观实在,而意识是这一客观实在的反映,那么由此应该得出结论:社会的物质生活、社会的存在,也是第一性的,而社会的精神生活是第二性的,是派生的;社会的物质生活是不依赖于人们意志而存在的客观实在,而社会的精神生活是这一客观实在的反映,是存在的反映。

这就是说,形成社会的精神生活的源泉,产生社会思想、社会理论、政治观点和政治设施\footnote{原文为“учреждение”,系指和一定的理论观点相适应的组织和机构。——译者注}的源泉,不应当到思想、理论,观点和政治设施本身中去寻求,而要到社会的物质生活条件,社会存在中去寻求,因为这些思想、理论和观点等等是社会存在的反映。

这就是说,社会历史的不同时期所以有不同的社会思想,理论、观点和政治设施,——在奴隶占有制度下是一种社会思想、理论、观点和政治设施,在封建制度下是另一种,在资本主义制度下又是一种,——那不能用思想、理论、观点和政治设施本身的“本性”和“属性”来解释,而要用不同的社会发展时期的不同的社会物质生活条件来解释。

社会存在怎样,社会物质生活条件怎样,社会思想、理论、政治观点和政治设施也就怎样。

因此马克思说:

\begin{quotation}
“不是人们的意识决定人们的存在,相反,是人们的社会存在决定人们的意识。”(《马克思选集》俄文版第1卷第269页)\footnote{见《马克思恩格斯选集》第2卷第82页。——译者注}
\end{quotation}

这就是说,要在政治上不犯错误,要不陷入空想家的地位,无产阶级党在自己的活动中就不应当从抽象的“人类理性原则”出发,而应当从具体的社会物质生活条件,即从社会发展的决定力量出发;不应当从“伟大人物”的善良愿望出发,而应当从社会物质生活发展的现实需要出发。

包括民粹主义者、无政府主义者、社会革命党人在内的空想派之所以垮台,其原因之一,就是他们不承认社会物质生活条件在社会发展过程中的首要作用,他们陷入唯心主义,不是把自己的实际活动建筑在社会物质生活发展的需要上,而是不顾这种需要并且违反这种需要,把自己的实际活动建筑在脱离社会现实生活的“理想计划”和“包罗万象的方案”上。

马克思列宁主义的力量和生命力在于,它在自己的实际活动中正是以社会物质生活发展的需要为依据,任何时候也不脱离社会的现实生活。

但是,从马克思的话中不能引出这样的结论:社会思想、理论、政治观点和政治设施在社会生活中没有意义,它们不反过来影响社会存在,影响社会生活物质条件的发展。我们在这里暂且只是说到社会思想、理论、观点和政治设施的起源,只是说到它们的产生,只是说到社会精神生活是社会物质生活条件的反映。至于社会思想、理论、观点和政治设施的意义,至于它们在历史上的作用,那末历史唯物主义不仅不否认,相反,正是着重指出它们在社会生活和社会历史中的重大作用和意义。

有各种各样的社会思想和理论。有旧的思想和理论,它们是衰颓的、为社会上衰颓的势力服务的。它们的作用就是阻碍社会发展,阻碍社会前进。也有新的先进的思想和理论,它们是为社会上先进的势力服务的。它们的作用就是促进社会发展,促进社会前进,而且它们愈是确切地反映社会物质生活发展的需要,它们的意义就愈大。

新的社会思想和理论,只有在社会物质生活的发展向社会提出新的任务以后,才会产生。可是,一经产生,它们就会成为促进解决社会物质生活的发展所提出的新任务,促进社会前进的最重大的力量。正是在这里表现出新思想、新理论、新政治观点和新政治设施的那种极其伟大的组织作用、动员作用和改造作用。新的社会思想和理论所以产生,正是因为它们是社会所必需的,因为没有它们那种组织工作、动员工作和改造工作,就不可能解决社会物质生活发展中的已经成熟的任务。新的社会思想和理论在社会物质生活的发展所提出的新任务的基础上一经产生,就为自己开拓道路,成为人民群众的财富,它们动员人民群众,组织人民群众去反对社会上衰颓的势力,从而有助于推翻社会上衰颓的、阻碍社会物质生活发展的势力。

可见,社会思想、理论和政治设施,在社会物质生活的发展即社会存在的发展所提出的已经成熟的任务的基础上一经产生,便反过来影响社会存在,影响社会物质生活,为彻底解决社会物质生活的已经成熟的任务,为社会物质生活能进一步发展,创造必要的条件。

因此马克思说:

\begin{quotation}
“理论一经群众掌握,也会变成物质力量。”(《马克思恩格斯全集》俄文第1版第1卷第406页)\footnote{见《马克思恩格斯选集》第1卷第9页。——译者注}
\end{quotation}

这就是说,无产阶级党为着有可能去影响社会物质生活条件,加速这些条件的发展,加速这些条件的改善,就应当依据这样一种社会理论和社会思想,这种理论和思想正确反映社会物质生活发展的需要,因而能发动广大人民群众,能动员他们,把他们组织成一直决心粉碎社会反动势力、为社会先进势力开辟道路的无产阶级党的大军。

“经济派”和孟什维克所以垮台,其原因之一,就是他们不承认先进理论、先进思想有动员作用、组织作用和改造作用,他们陷入庸俗唯物主义,把先进理论和先进思想的作用看成几乎等于零,从而要党消极起来,无所作为。

马克思列宁主义的力量和生命力在于,它以正确反映社会物质生活发展需要的先进理论为依据,把这种理论提到它应有的高度,并且把充分利用选种理论的动员力量、组织力量和改造力量,看作自己的职责。

历史唯物主义就是这样来解决社会存在和社会意识之间、社会物质生活发展条件和社会精神生活发展之同的关系问题的。

(三)历史唯物生义。

现在还要说明一个同题:从历史唯物主义观点来看,对于归根到底决定社会面貌、社会思想、观点和政治设施等等的“社会物质生活条件”,应该作怎样的了解?

“社会物质生活条件”究竟是什么,它们的特征究竟怎样?

首先,“社会物质生活条件”这一概念无疑包括社会所处的自然环境,即地理环境,因为地理环境是社会物质生活的必要的和经常的条件之一,它当然影响到社会的发展。地理环境在社会发展中的作用怎样呢?地理环境是不是决定社会面貌,决定人们的社会制度的性质,决定从一种制度过渡到另一种制度的主要力量呢?

历史唯物主义对这个问题的答复是否定的。

地理环境无疑是社会发展的经常的和必要的条件上一,它当然影响到社会的发展,——加速或者延缓社会发展进程。但是它的影响并不是决定的影响,因为社会的变化和发展比地理环境的变化和发展快得不可比拟。欧洲载三千年内已经更换过三种不同的社会制度;原始公社制度、奴隶占有制度、封建制度;而在欧洲东部,即在苏联,甚至更换了四种社会制度。可是,在同一时期内,欧洲的地理条件不是完全没有变化,便是变化极小,连地理学也不愿提到它。这是很明显的。地理环境的稍微重大一些的变化都需要几百万年,而人们的社会制度的变化,甚至是极其重大的变化,只需要几百年或一两千年也就够了。

由此应该得出结论:地理环境不可能成为社会发展的主要的原因,决定的原因,因为在几万年间几乎保持不变的现象,决不能成为在几百年间就发生根本变化的现象发展的主要原因。

其次,人口的增长,人口密度的大小,无疑也包括在“社会物质生活条件”这一概念中,因为人是社会物质生活条件的必要因素,没有一定的最低限度的人口,就不可能有任何社会物质生活。人口的增长是不是决定人们社会制度性质的主要力量呢?

历史唯物主义对于过个问题的答复也是否定的。

当然,人口的增长对社会的发展有影响,它促进或者延缓社会的发展,但是它不可能是社会发展的主要力量,它对社会发展的影响不可能是决定的影响,因为人口的增长本身并不能说明为什么某种社会制度恰恰被一定的新制度所代替,而不是被其他某种制度所代替;为什么原始公社制度恰恰被奴隶占有制度所代替,奴隶占有制度被封建制度所代替,封建制度被资产阶级制度所代替,而不是被其他某种制度所代替。

如果人口的增长是社会发展的决定力量,那么较高的人口密度就必定会产生出相应的较高类型的社会制度。可是,事实上没有这样的情形。中国的人口密度比美国高三倍,但是从社会发展来看,美国高于中国,因为在中国仍然是半封建制度占统治、而美国早已达到资本主义发展的最高阶段。比利时的人口密度比美国高十八倍,比苏联高二十五倍,但是从社会发展来看,美国高于比利时,同苏联相比,比利时更是落后整整一个历史时代,因为在比利时占统治的是资本主义制度,而苏联已经消灭了资本主义,在国内确立了社会主义制度。

由此应该得出结论:人口的增长不是而且不可能是决定社会制度性质、决定社会面貌的社会发展的主要力量。

(1)既然如此,那末在社会物质生活条件体系中,究竟什么是决定社会面貌、决定社会制度性质、决定社会从这一制度发展到另一制度的主要力量呢?

历史唯物主义认为,这种力量就是人们生存所必需的生活资料的谋得方式,就是社会生存和发展所必需的食品、农服、鞋子、住房、燃料和生产工具等等物质资料的生产方式。

要生活,就要有食品、衣服、鞋子、住房和燃料等等,要有这些物质资料,就必须生产它们,要生产它们,就需要有人们用来生产食品、衣服、鞋子、住房和燃料等等的生产工具,就需要善于生产这些工具,善于使用这些工具。

用来生产物质资料的生产工具,以及有一定的生产经验和劳动技能来使用生产工具,实现物质资料生产的人,——所有这些因素共同构成社会的生产力。

但是生产力还只是生产的一个方面,生产方式的一个方面,它所表现的是人们对于那些用来生产物质资料的自然对象和力量的关系。生产的另一个方面,生产方式的另一个方面,就是人们在生产过程中的相互关系,即人们的生产关系。人们同自然界作斗争以及利用自然界来生产物质资料,并不是彼此孤立、彼此隔绝、各人单独进行的,而是以团体为单位、以社会为单位共同进行的。因此,生产在任何时候和任何条件下都是社会的生产。人们在实现物质资料生产的时候,在生产内部彼此建立这种或那种相互关系,即这种或那种生产关系。这些关系可能是不受剥削的人们彼此间的合作和互助关系,可能是统治和服从的关系,最后,也可能是从一种生产关系形式过渡到另一种生产关系形式的过渡关系。可是,不管生产关系带有怎样的性质,它们在任何时候和任何制度下,都同社会的生产力一样,是生产的必要因素。

\begin{quotation}
马克思说。“人们在生产中不仅仅影响自然界,而且也互相影响。他们如果不以一定方式结合起来共同活动和互相交换其活动,便不能进行生产。为了进行生产,人们便发生一定的联系和关系,只有在这些社会联系和社会关系的范围内,才会有他们对自然界的关系,才会有生产”(《马克思恩格斯全集》俄文第1版第5卷第429页)\footnote{见《马克思恩格斯选集》第1卷第362页。——译者注}
\end{quotation}

可见,生产、生产方式既包括社会生产力,也包括人们的生产关系,而体现着两者在物质资料生产过程中的统一。

(2)生产的第一个特点就是它永远也不会长久停留在一点上,而是始终处在变化和发展的状态中,同时,生产方式的变化又必然引起全部社会制度、社会思想、政治观点和政治设施的变化,即引起全部社会结构和政治结构的改造。在不同的发展阶段上,人们利用不同的生产方式,或者说得粗浅一些,过着不同方式的生活。在原始公社制度下有一种生产方式,在奴隶制度有另一种生产方式,在封建制度下又有一种生产方式,如此等等。与此相适应、人们的社会制度、他们的精神生活、他们的观点、他们的政治设施也是各不相同的。

社会的生产方式怎样,社会本身基本上也就怎样,社会的思想和理论、政治观点和政治设施也就怎样。

或者说得粗浅一些:人们的生活方式怎样,人们的思想方式也就怎样。

这就是说,社会发展史首先是生产的发展史,是许多世纪以来依次更迭的生产方式的发展史,是生产力和人们生产关系的发展史。

这就是说,社会发展史同时也是物质资料生产者本身的历史,即作为生产过程的基本力量、生产社会生存所必需的物质资料的劳动群众的历史。

这就是说。历史科学要想成为真正的科学。就不能再把社会发展史归结为帝王将相的行动,归结为国家“侵略者”和“征服者”的行动,而首先应当研究物质资料生产者的历史,劳动群众的历史,各国人民的历史。

这就是说,研究社会历史规律的关键,不应该到人们的头脑中,到社会的观点和思想中去寻求,而要到社会的每个特定历史时期所采取的生产方式中,即到社会的经济中去寻求。

这就是说,历史科学的首要任务是研究和揭示生产的规律,生产力和生产关系发展的规律,社会经济发展的规律。

这就是说、无产阶级党要想成为真正的党,首先应当掌握生产发展规律的知识,社会经济发展规律的知识、

这就是说,要在政治上不犯错误,无产阶级党在制定自己的党纲以及进行实际活动的时候,首先应当从生产发展的规律出发,从社会经济发展的规律出发。

(3)生产的第二个特点就是生产的变化和发展始终是从生产力的变化和发腱,首先是从生产工具的变化和发展开始的。所以生产力是生产中最活动、最革命的因素。先是社会生产力变化和发展,然后,人们的生产关系、人们的经济关系依赖这些变化、与这些变化相适应地发生变化。但这并不是说,生产关系不影响生产力的发展,生产力不依赖于生产关系。生产关系依赖于生产力的发展而发展,同时又反过来影响生产力,加速或者延缓它的发展。而且必须指出:生产关系不能过分长久地落后于生产力的增长并和这一增长相矛盾,因为只有当生产关系适合于生产力的性质、状况。并且给生产力以发展余地的时候,生产力才能充分地发展。因此,无论生产关系怎样落后于生产力的发展,但是它们迟早必须适合——也确实在适合——生产力的发展水平,适合生产力的性质。不然,生产体系中的生产力和生产关系的统一就会根本破坏,整个生产就会破裂,生产就会发生危机,生产力就会遭到破坏。

资本主义国家中的经济危机就是生产关系不适合生产力性质的例子,是两者冲突的例子,在那里,生产资料的私人资本主义所有制同生产过程的社会性,同生产力的性质极不适合。这种不适合的结果,就是破坏生产力的经济危机,而这种不适合的情况本身是以破坏现存生产关系、建立适合于生产力性质的新生产关系为使命的社会革命的经济基础。

反之,苏联的社会主义国民经济是生产关系完全适合生产力性质[26]的例子,这里的生产资料的公有制同生产过程的社会性完全适合,因此在苏联没有经济危机,也没有生产力破坏的情形。

所以,生产力不仅是生产中最活动、最革命的因素,而且是生产发展的决定因素。

生产力怎样,生产关系就必须怎样。

生产力的状况所回答的问题是人们用怎样的生产工具生产他们所必需的物质资料,生产关系的状况所回答的则是另一个问题;生产资料(土地、森林、水流,矿源、原料、生产工具、生产建筑物、交通工具,通信工具等等)归谁所有,生产资料由谁支配——由全社会支配,还是由个人、集团和阶级支配并且被用来剥削其他的个人、集团和阶级。

下面就是从古代到今天的生产力发展的一般情景。从粗笨的石器过渡到弓箭,与此相联系,从狩猎生活过渡到驯养动物和原始畜牧;从石器过渡到金属工具(铁斧、铁铧犁等等),与此相适应,过渡到种植植物和农业;加工材料的金属工具进一步改良,渡过到冶铁风箱,过渡到陶器生产,与此相适应,手工业得到发展,手工业脱离农业,独立手工业生产以及后来的工场手工业生产得到发展;从手工业生产工具过渡到机器。手工业——工场手工业生产转变为机器工业;进而过渡到机器制。出现现代大机器工业,——这就是人类史上社会生产力发展的一个大致的、远不完备的情景。这里很明显,生产工具的发展和改善是由参加生产的人来实现的,而不是与人无关的,所以,生产工具变化和发展了,生产力的最重要的因素——人也随着变化和发展,人的生产经验、劳动技能以及运用生产工具的本领也随着变化和发展。

随着社会生产力在历史上的变化和发展,人们的生产关系、人们的经济关系也相应地变化和发展。

历史上有五种基本类型的生产关系:原始公社制的、奴隶占有制的、封建制的、资本主义的、社会主义的。

在原始公社制度下,生产关系的基础是生产资料的公有制。这在基本上适合当时的生产力性质。石器以及后来出现的弓箭,使人无法单身去同自然力量和猛兽斗争。为了在森林中采集果实,在水里捕鱼,建筑某种住所,人们不得不共同工作,否则就会饿死,就会成为猛兽或邻近部落的牺牲品。公共的劳动导致生产资料和产品的公有制。这里还不知道什么是生产资料私有制,不过有些同时用来防御猛兽的生产工具归个人所有。这里没有剥削,也没有阶级。

在奴隶占有制度下,生产关系的基础是奴隶主占有生产资料和占有生产工作者,这些生产工作者就是奴隶主可以把他当作牲畜来买卖屠杀的奴隶。这样的生产关系基本上适合当时的生产力状况。这时人们拥有的已经不是石器,而是金属工具;这时,不知道畜牧业、也不知道农业的那种贫乏原始的狩猎经济,已经被畜牧业、农业、手工业以及这些生产部门之间的分工所代替;这时已经有可能在各个人之同和各部落之同交换产品,有可能把对富积累在少数人手中,而生产资料确实积量在少教人手中,这时已经有可能迫使大多数人服从少数人并且把大多教人变为奴隶。这里社会一切成员在生产过程中的那种共同的自由的劳动没有了,占主要地位的是受不劳动的奴隶主剥削的奴隶的强迫劳动。因此生产资料和产品的公有制也没有了。代替它的是私有制。这里,奴隶主是第一个和基本的十足的私有者。

富人和穷人,剥削者和被剥削者,享有完全权利的人和毫无权利的人,他们彼此间的残酷的阶级斗争,——这就是奴隶占有制度的情景。

在封建制度下,生产关系的基础是封建主占有生产资料和不完全地占有生产工作者——农奴,封建主已经不能屠杀农奴,但是可以买卖农奴。除了封建所有制以外,还存在农民和手工业者以本身劳动为基础的个体所有制,他们占有生产工具和自己的私有经济。这样的生产关系基本上适合当时的生产力状况。铁的冶炼和加工更进一步的改善,铁犁和织布机的推广,农业、种菜业、酿酒业和榨油业的继续发展,除手工业作坊以外工场手工业企业的出现,——这就是当时生产力状况的特征。

新的生产力要求生产者在生产中能表现出某种主动性,愿意劳动,对劳动感兴趣。于是,封建主就抛弃奴隶,抛弃这种对劳动不感并趣、完全没有主动性的工作者,宁愿利用农奴,因为农奴有自己的经济、自己的生产工具,具有为耕种土地并从自己收成中拿出部分实物缴给封建主所必需的某种劳动兴趣。

私有制在这里得进一步的发展。剥削几乎同奴隶制度下的剥削一样残酷、不过是稍许减轻一些罢了。剥削者和被剥削者之间的阶级斗争,就是封建制度的基本特征。

在资本主义制度下,生产关系的基础是生产资料的资本主义所有制,同时这里已经没有了私自占有生产工作者的情形,这时的生产工作者,即雇佣工人,是资本家既不能屠杀,也不能出卖的,因为雇佣工人摆脱了人身依附,但是他们没有生产资料,所以为了不致饿死、他们不得不出卖自己的劳动力给资本家,套上剥削的枷锁。除生产资料的资本主义所有制以外,还存在着摆脱了农奴制依附关系的农民和手工业者以本身劳动为基础的、生产资料的私有制,而且这种私有制在初期是很流行的。手工业作坊和工场手工业企业被用机器装备起来大工厂所代替。用农民简陋的生产工具耕作的贵族庄园,被根据农艺学经营的、使用农业机器的资本主义大农场所代替。

新的生产力要求生产工作者比闭塞无知的农奴更有文化、更加伶俐,能够懂得机器和正确地使用机器。因此,资本家宁愿利用摆脱农奴制羁绊、有相当文化程度来正确使用机器的雇佣工人。

可是,资本主义把生产力发展到巨大的规模以后,使陷入它解决不了的矛盾中。资本主义生产出日益增多的商品并且减低商品的价格,这样就使竞争尖锐化,使大批中小私有者破产,把他们变成无产者,缩小他们的购买力,因而使生产出来的商品无法销售出去。资本主义扩大生产并把千百万工人集合在大工厂内,这样就使生产过程具有社会性,因而破坏本身的基础,因为生产过程的社会性要求有生产资料的公有制,而生产资料的所有制却仍然是同生产过程的社会性不相容的私人资本主义所有制。

生产力性质和生产关系之间的这种不可调和的矛盾,通过周期性的生产过剩的危机暴露出来,在危机时期,资本家由于自己使居民群众遭受破产而找不到有支付能力的需求者,不得不烧掉产品,销毁制成的商品,停止生产,破坏生产力;千百万居民则被迫失业挨饿。而这并不是由于商品不够,却是因为商品生产太多。

这就是说,资本主义的生产关系已不再适合社会生产力状况,它同社会生产力发生了不可调和的矛盾。

这就是说,资本主义孕育着革命,这个革命的使命就是要用社会主义的生产资料所有制来代替现存的资本主义的生产资料所有制。

这就是说,剥削和被剥削者之间的最尖锐的阶级斗争,是资本主义制度的基本特征。

在社会主义制度下,在目前还只有在苏联实现的这种制度下,生产资料的公有制是生产关系的基础。这里已经没有剥削者,也没有被剥削者。生产出来的产品是根据“不劳动者不得食”的原则按劳动分配的。这里,人们在生产过程中的相互关系的特征,是不受剥削的工作者之间的同志合作和社会主义互助。这里生产关系同生产力状况完全适合,因为生产过程的社会性是由生产资料的公有制所巩固的。

因此,苏联的社会主义生产没有周期性的生产过剩的危机,没有同危机相联系的荒谬现象。

因此,生产力在这里以加快的速度发展着,因为适合于生产力的生产关系使生产力有这样发展的充分广阔的天地。

这就是人类史上人们生产关系发展的情景。

这就是生产关系的发展依赖于社会生产力的发展,首先是依赖于生产工具的发展的情况,因为有这种依赖关系,所以生产力的变化和发展迟早要引起生产关系相应的变化和发展。

\begin{quotation}
马克思说:“劳动资料\footnote{马克思所说的“劳动资料”,主要是指生产工具。——编者注}的使用和创造,虽然就其萌芽状态来说已为某几种动物所固有,但是这毕竟是人类劳动过程独有的特征,所以富兰克林给人下的定义是制造工具的动物。动物遗骸的结构对于认识已经绝迹的动物的机体有重要的意义,劳动资料的遗骸对于判断已经消亡的社会经济形态也有同样重要的意义。各种经济时代的区别,不在于生产什么,而在于怎样生产…劳动资料不仅是人类劳动力发展的测量器,而且是劳动借以进行的社会关系的指示器。”(马克思《资本论》1935年俄文版第1卷第121页)\footnote{见《马克思恩格斯全集》第23卷第204页。——编者注}
\end{quotation}

其次:

\begin{quotation}
“社会关系和生产力密切相联。随着新生产力的获得,人们改变自己的生产方式,随着生产方式即保证自己生活的方式的改变,人们也就会改变自己的一切社会关系。手推磨产生的是封建主为首的社会,蒸汽磨产生的是工业资本家为首的社会。”(《马克思恩格斯全集》俄文第1版第5卷第364页)\footnote{见《马克思恩格斯选集》第1卷第108页。——编者注}

“生产力的增长、社会关系的破坏、观念的产生都是不断变动的,只有运动的抽象……才是停滞不动的。”(《马克思恩格斯全集》俄文第1版第5卷第364页)\footnote{见《马克思恩格斯选集》第1卷第109页。——译者注}
\end{quotation}

恩格斯在说明《共产党宣言》所表述的历史唯物主义时说道:

\begin{quotation}
“每一历史时代的经济生产以及必然由此产生的社会结构,是该时代政治的和精种的历史的基础;因此(从原始土地公有制解体以来)全部历史都是阶级斗争的历史,即社会发展各个阶段上被剥削阶级和剥削阶级之间、被统治阶级和统治阶级之间斗争的历史,而这个斗争现在已经达到这样一个阶段,即被剥削被压迫的阶级(无产阶级),如果不同时使整个社会永远摆脱压迫和阶级斗争,就不再能使自己从剥削它压迫它的那个阶级(资产阶级)下解放出来……”(恩格斯为《宣言》德文版所作的序言)\footnote{同上,见232页。——译者注}
\end{quotation}

(4)生产的第三个特点就是新的生产力以及同它相适合的生产关系的产生过程不是离开旧制度而单独发生,不是在旧制度消灭以后,而是在旧制度内部发生的;不是人们有意的、自觉的活动的结果,而是自发地、不自觉地、不以人们意志为转移地发生的。其所以是自发地、不以人们意志为转移地发生,有以下两个原因。

第一个原因,就是人们不能自由选择这种或那种生产方式。因为每一辈新人开始生活时,他们就遇到现成的生产力和生产关系,即前辈人工作的结果,因此新的一辈在最初必须接受他们在方面所遇到的一切现成东西,必须适应这些东西,以便有可能生产物质资料。

第二个原因,就是人们在改进这种或那种生产工具、这种或那种生产力因素时,不会意识到,不会了解到,也不会想到,这些改进将会引起怎样的社会结果,而只是想到自己的日常利益,只是想要减轻自己的劳动,谋得某种直接的、可以感触到的益处。

原始公社社会的某些成员在逐渐地摸索着石器过渡到铁器的时候,当然不知道,也没有想到这种革新会引起怎样的社会结果;他们没有了解到,也没有意识到,从石器过渡到金属工具是意味着生产中的变革,结果一定会引起奴隶占有制度,——当时他们只是想要减轻自己的劳动和谋得眼前的感觉得到的益处,——他们当时的自觉活动只局限于这种日常个人利益的狭隘范围。

欧洲年轻的资产阶级在封建制度时期开始建大工场手工业企业,同行会小作坊并存,从而推进了社会生产力,它当然不知道,也没有想到,这种革新会引起怎样的社会后果;它没有意识到,也没有了解到,这种“细微的”革新会引起社会力量的重新配置,结果会发生革命,这个革命不但会打倒它所十分感恩的国王政权,而且会打倒它的优秀人物往往梦想厕身其间的贵族,——当时资产阶级只是想要减低商品生产的费用,把更多的商品投到亚洲市场和刚发现的美洲市场,以便获得更多的利润、——它当时的自觉活动只局限于这种日常实践的狭隘范围。

俄国资本家和外国资本家一起加紧在俄国培植现代机器化大工业,丝毫也不触动沙皇制度,听凭地主们宰割农民,当时,他们当然不知道,也没有想到,生产力的这种严重增长会引起怎样的社会后果;他们没有意识到,也没有了解到,社会生产力方面的这种重大的飞跃会引起社会力量的重新配置,结果会使无产阶级有可能和农民联合起来实现胜利的社会主义革命,——当时他们只是想要极度地扩大工业生产,掌握巨大的国内市场,变成垄断资本家,并且从国民经济中汲取更多的利润,——他们当时的自觉活动并没有超出他们日常的狭隘实践的利益。

因此马克思说:

\begin{quotation}
“人们在自己生话的社会生产中(即在人们生活所必需的物质资料的生产中。——编者注)发生一定的、必然的、不以他们的意志为\dotemph{转移}\footnote{着重号是编者加的。}的关系,即同他们的物质生产力的一定的发展阶段相适合的生产关系。”(《马克思选集》俄文版第1卷第269页)\footnote{见《马克思恩格斯选集》第2卷第82页。——译者注}
\end{quotation}

但这并不是说,生产关系的变化以及从旧生产关系到新生产关系的过渡是一帆风顺、不经过冲突、不经过震荡的。相反地,这种过渡通常是用革命手段推翻旧生产关系、树立新生产关系的办法实现的。到一定时期为止,生产力的发展以及生产关系方面的变化,是自发地、不以人们意志为转移地进行的。但这只是到一定时候为止,只是到已经产生和正在发展的生产力还没有充分成熟的时候为止。而在新生产力成熟以后,现存的生产关系以及体现这种生产关系的统治阶级就变成“不可克服的”障碍,这只有通过新兴阶级的自觉活动,只有通过这些阶级的暴力行动,只有通过革命才能被扫除。在这方面特别明显地表现出新社会思想、新政治设施和新政权的巨大作用,它们的使命就是用暴力消灭旧生产关系,在新生产力同旧生产关系冲突的基础上。在社会新的经济需要的基础上产生出新的社会思想,新思想组织和动员群众,群众团结成为新的政治大军,建立起新的革命政权,并且运用这个政权,以便用暴力消灭生产关系方面的旧秩序,建立新秩序。于是,自发的发展过程让位给人们自觉的话动,和平的发展让位给暴力的变革,进化让位给革命。

\begin{quotation}
马克思说:“无产阶级在反对资产阶级的斗争中一定要联合为阶级…它通过革命使自己成为统治阶级,并以统治阶级的资格用暴力消灭旧的生产关系。”(《共产党宣言》1938年俄文版第52页)\footnote{见《马克思恩格斯选集》第1卷第273页。——译者注}
\end{quotation}

其次:

\begin{quotation}
“无产阶级将利用自己的政治统治,一步一步地夺取资产阶级的全部资本,把一切生产工具集中在国家即组织成为统治阶级的无产阶级手里,并且尽可能快地增加生产力的总量。”(同上,第50页)\footnote{同上,第272页。——译者注}

“暴力是每一个孕育着新社会的旧社会的助产婆。”(马克思《资本论》1935年俄文版第1卷第603页)\footnote{见《马克思恩格斯选集》第2卷第256页。——译者注}
\end{quotation}

以下就是马克思在1859年为他的名著《政治经济学批判》所写的有历史意义的《序言》中,对历史唯物主义的实质所作的天才的表述:

\begin{quotation}
“人们在自己生活的社会生产中发生一定的、必然的、不以他们的意志为转移的关系,即同他们的物质生产力的一定发展阶段相适合的生产关系。这些生产关系的总和构成社会的经济结构,即有法律的和政治的上层建筑竖立其上并有一定的社会意识形式与之相适应的现实基础。物质生活的生产方式制约着整个社会生活、政治生活和精神生活的过程。不是人们的意识决定人们的存在,相反,是人们的社会存在决定人们的意识。社会的物质生产力发展到一定阶殷,便同它们一直在其中活动的现存生产关系或财产关系(这只是生产关系的法律用语)发生矛盾。于是这些关系便由生产力的发展形式变成生产力的桎梏。那时社会革命的时代就到来了。随着经济基础的变更,全部庞大的上层建筑也或慢或快地发生变革。在考察这些变革时,必须时刻把下面两者区别开来:一种是生产的经济条件方面所发生的物质的,可以用自然科学的精确性指明的变革,一种是人们借以意识到这个冲突并力求把它克服的那些社会的、法律的、政治的、宗教的、艺术的或哲学的,简言之,意识形态的形式。我们判断一个人不能以他对自己的看法为根据,同样,我们判断这样一个变革时代也不能以它的意识为根据;相反,这个意识必须从物质生活的矛盾中,从社会生产由和生产关系之间的现存冲突中去解释。无论哪一个社会形态,在它们所能容纳的全部生产力发挥出来以前,是决不会灭亡的:而新的更高的生产关系,在它存在的物质条件在旧社会的胎胞里成熟以前,是决不会出现的。所以人类始终只提出自己能够解决的任务,因为只要仔细考察就可以发现,任务本身,只有在解决它的物质条件已经存在或者至少是在形成过程中的时候。才会产生。”(《马克思选集》俄文版第1卷第269—270页)\footnote{见《马克思恩格斯选集》第2卷第82—83页。——译者注}
\end{quotation}

这就是把马克思主义的唯物主义应用于社会生活和社会历史的情形。

这就是辩证唯物主义和历史唯物主义的基本特征。

由此就可看出,列宁打退修正主义分子和蜕化变节分子的谋害尝试而为党捍卫了多么巨大的理论财富,列宁的《唯物主义和经验批判主义》一书的出现对我党的发展有多么重要的意义。


\subsection[三\q 布尔什维克和孟什维克在斯托雷平反动年代。布尔什维克反对取消派和召回派的斗争]{三\\布尔什维克和孟什维克在斯托雷平反动年代。\\布尔什维克反对取消派和召回派的斗争}

党组织在反动年代进行工作,要比过去革命开展时期困难得多。党员人数锐减。党内许多小资产阶级同路人,特别是知识分子,都因害怕沙皇政府迫害而离开了党。

列宁指出,革命政党在这样的时候应当补习一下。在革命高涨时期,它们学习了怎样进攻;在反动时期,它们应当学习怎样正确地退却,怎样转入地下,怎样保存和巩同秘密党,怎样利用合法的机会,利用各种合法组织特别是群众组织来巩固自己同群众的联系。

孟什维克不相信革命有重新高涨的可能而仓皇退却,他们可耻地背弃了党纲上的革命要求和党的革命口号,想要取消、消灭无产阶级的革命的秘密党。因此这样的孟什维克就被称为取消派。

布尔什维克和孟什维克不周,他们深信最近几年革命会有新的高涨,认为党必须使群众作好准备去迎接这个新高涨。革命的基本任务还没有解决。农民还没有获得地主土地,工人还没有获得八小时工作制,人民深恶痛绝的沙皇专制制度还没有推翻,而且这个沙皇专制制度现在又把人民在1905年从它手里争得的一点点政治自由毁灭掉了。所以,产生1905年革命的种种原因仍然有效。因此,布尔什维克确信革命运动定会重新高涨,并为迎接它进行着准备,聚集着工人阶级的力量。

布尔什维克之所以确信革命必然会重新高涨,还因为1905年革命使工人阶级学会了在群众性的革命斗争中争取自己的权利。在资本实行进攻的反动年代,工人决不会忘记他们在1905年所得到的这些教训。列宁当时引证了工人的来信,信上讲到厂主又在虐待和侮辱工人时说道;“你们等着吧、1905年又会到来的!”

当时布尔什维克的基本政治目标,仍然同1905年一样,是要推翻沙皇制度。把资产阶级民主革命进行到底,过渡到社会主义革命。布尔什维克一分钟也没有忘记这个目标,继续向群众提出那几个基本的革命口号:建立民主共和国,没收地主土地,实行八小时工作制。

可是,党已不能采取1905年革命高涨时期那样的策略了。例如,党决不能在最近时期号召群众举行政治总罢工或武装起义,因为这时革命运动已经低落,工人阶级已非常疲惫,反动阶级已大大加强,党不能不考虑到新的形势。进攻的策略必须转变为防御的策略,即聚集力量的策略,使干部转入地下,在地下进行党的工作的策略,把秘密工作同合法工人组织的工作结合起来的策略。

而布尔什维克也就巧妙地执行了这个任务。

\begin{quotation}
列宁写道。“我们在革命以前就进行了多年的工作。难怪人们把我们叫作坚如磐石的人。社会民主党人已建立起无产阶级的党,这个党决不会因第一次军事进攻遭到失败而颓丧,决不会张皇失措,决不会醉心于冒险行动。”(《列宁全集》俄文第3版第12卷第126页)\footnote{见《列宁全集》第13卷第422页。——译者注}
\end{quotation}

布尔什维克为保持和巩固秘密的党组织而进行着斗争。但同时布尔什维克又认为必须利用一切合法机会,利用一切合法借口来维持和保存党同群众的联系,借以加强党的力量。

\begin{quotation}
“这是我们党从对沙皇制度进行公开的革命斗争转到采取迂回的斗争方法,转到利用从保险基金会起到杜马讲坛止的所有一切合法机会的时期。这是我们在1905年革命中遭到失败后实行退却的时期。这个转变要求我们善于运用新的斗争方法,以便聚集力量再去对沙皇制度进行公开的革命斗争。”(斯大林语,见《第十五次代表大会速记记录》1935年俄文版第366-367页)\footnote{见《斯大林全集》第10卷第319页。——译者注}
\end{quotation}

保全下来的合法组织可以说是地下党组织的掩护物和联系群众的工具。布尔什维克为保持同群众的联系而利用了工会和其他各种合法的社会组织:疾病救济会、工人合作社、俱乐部,文化团体以及民众文化馆等。布尔什维克利用国家杜马讲坛来揭露沙皇政府的政策,揭露立宪民主党人,并把农民吸引到无产阶级方面来、由于保存了秘密的党组织,并且通过它领导了其他各种类型的政治工作,也就保证了党能贯彻执行党的正确路线,即准备力量去迎接革命的新高涨。

布尔什维克在执行革命路线的时候,进行了两条路线的斗争,即同时反对党内两种机会主义:既反对公开反党的取消派,又反对暗中反党的所谓召回派。

自从取消派这一机会主义派别开始出现时起,列宁和布尔什维克就同它进行了毫不调和的斗争。列宁指出,取消派是自由资产阶级在党内的代理人。

1908年12月,在巴黎召开了俄国社会民主工党第五次(全俄)代表会议。这次会议根据列宁提议斥责了取消主义,即党内一部分知识分子(孟什维克)企图“取消现有的俄国社会民主工党组织,而代之以一种不定形的团体,这种团体无论如何要在合法范围内活动,甚至不惜以公然放弃党的纲领、策略和传统为代价来换取合法性”(《联共(布)决议汇编》俄文版第1册第128页)\footnote{见《苏联共产党那个决议汇编》第1分册第246页。——译者注}。

会议号召全党所有的组织坚决反对取消派的企图。

可是,孟什维克没有服从代表会议的这一决议,却日益滑到背叛革命而同立宪民主党人接近的取消主义的道路上去。孟什维克日益公开地抛弃无产阶级党的革命纲领,抛弃成立民主共和国,实行八小时工作制、没收地主土地的要求。孟什维克想以抛弃党的纲领和策略为代价,换取沙皇政府准许公开的合法的冒牌“工人”政党存在。孟什维克决定同斯托雷平制度妥协,迁就斯托雷平制度。所以取消派又被称为“斯托雷平工党”。

布尔什维克一方面同革命的公开敌人,即由唐恩、阿克雪里罗得和波特列索夫等人领导并获得马尔托夫、托洛茨基同和其他孟什维克分子帮助的取消派进行斗争时,又同暗藏的取消派,即用“左的”词句掩盖自己机会主义面目的召回派进行不调和的斗争。所谓召回派,是指一部分先前的布尔什维克,他们要求召回国家杜马中的工人代表,主张根本停止在合法组织中进行工作。

1908年。有一部分布尔什维克要求从国家杜马中召回社会民主党的代表。由此就有“召回派”这一名称。召回派组织了自己单独的团体(波格丹诺夫、卢那察尔斯基、阿列克辛斯基、波克罗夫斯基和布勃诺夫等人),来开始进行反对列宁和列宁路线的斗争。召回派坚决拒绝在工会和其他合法团体中进行工作。这样,他们就使工人事业受到严重的损害。召回派使党脱离工人阶级,使党失去同非党群众的联系,想在地下组织里闭关自守,并使地下组织失去利用合法掩护物的可能而受到打击。召回派不了解布尔什维克在国家杜马里和通过国家杜马可以影响农民,可以揭露沙皇政府的政策,揭露立宪民主党人用欺骗手段引诱农民的政策。召回派阻碍党聚集力量去迎接革命的新高涨。所以,召回派是“改头换面的取消派”,因为他们力图取消利用合法团体的机会,并且在实际上放弃了对广大非党群众实行无产阶级的领导,放弃了革命工作。

1909年为讨论召回派的行为而召集的布尔什维克《无产者报》编辑部扩大会议斥责了召回派。布尔什维克声明他们和召回派毫无共同之处,并把召回派从布尔什维克组织中开除出去。

取消派和召回派都不过是无产阶级及其政党的小资产阶级同路人。在无产阶级遇到困难的时刻,取消派和召回派就特别明显地暴露了自己的真面目。


\subsection[四\q 布尔什维克反对托洛茨基主义的斗争。反党的八月联盟]{四\\ 布尔什维克反对托洛茨基主义的斗争。\\ 反党的八月联盟}

当布尔什维克进行两条战线的斗争,坚决反对取消派和召回派,捍卫无产阶级政党坚定路线的时候,托洛茨基却支持孟什维克取消派。正是在这些年代,列宁称他为“犹大什克·托洛茨基”\footnote{犹大什克是出卖耶稣的犹大的卑称,这里列宁借用它来称呼托洛茨基叛徒。见《列宁全集》第17卷第28页。——译者注}。托洛茨基在维也纳(奥地利)组织了一个著作家集团,创办了一个“非派别性的”而其实是孟什维克派的报纸。关于托洛茨基,列宁当时选样写道;“托洛茨基的行为表明他是一个最卑鄙的野心家和派别活动者……嘴上滔滔不绝的谈党,而行动却比所有其他的派别活动者还坏。”\footnote{见《列宁全集》第34卷第411页。——译者注}

后来,在1912年,托洛茨基组织了八月联盟,即所有一切反布尔什维克的集团和派别所结成的反对列宁、反对布尔什维克党的联盟。取消派和召回派也在这个敌视布尔什维主义的联盟中联合起来了,这就证明他们是一路货。托洛茨基和托洛茨基派在一切基本问题上都采取了取消主义立场。可是,托洛茨基却用中派主义,即用调和主义把自己的取消主义掩盖起来,硬说他是站在布尔什维克和孟什维克之外,他想使他们双方达到和解。列宁谈到这一点时说道。托洛茨基比公开的取消派更卑鄙,更有害,因为他欺骗工人说他是站在“派别之外”,其实他是完全支持孟什维克取消派的。托洛茨基主义是培植中派主义的主要集团。

\begin{quotation}
斯大林同志写道:“中派主义是一个政治概念。它的思想体系是迁就的思想体系,是在一个共同的党内使无产阶级利益服从小资产阶级利益的思想体系。这种思想体系是和列宁主义相违背的,相对立的。”(斯大林《列宁主义问题》俄文第9版第379页)\footnote{见《斯大林全集》第11卷第242页。——译者注}
\end{quotation}

在这个时期,加米涅夫、季诺维也夫和李可夫事实上是暗藏的托洛茨基代理人,因为他们常常帮助他反对列宁。1910年1月,在季诺维也夫、加米涅夫、李可夫和其他暗藏的托洛茨基同盟者的帮助下,违反列宁意旨而召集了中央全会。当时中央委员会的成分因有许多布尔什维克被捕而发生了变化,所以动摇分子有可能通过反列宁的决议。例如这次全会竟决定停办布尔什维克的《无产者报》,并拨款帮助托洛茨基在维也纳出版的《真理报》。加米涅夫参加了托洛茨基报纸的编辑部,并和季诺维也夫一起力图把托洛茨基的报纸变成中央机关报。

只是由于列宁的坚持,中央一月全会才通过了谴责取消派和召回派的决议。但就在这里季诺维也夫和加米涅夫也坚持了托洛茨基的主张,不把取消派明确点出来。

结果正如列宁所预见和警告的那样:只有布尔什维克才遵照中央全会的决议停办了自己的机关报《无产者报》,而孟什维克则继续出版他们那一派的取消主义的《社会民主党人呼声报》

斯大林同志完全拥护列宁的立场,他在《社金民主党人报》第11号上发表了一篇专文,这篇文章谴责了托洛茨基主义帮凶们的行为,指出必须消除由于加米涅夫、季诺维也夫和李可夫的叛卖行为而在布尔什维克派组织中所造成的不正常状态。该文还提出了后来由党的布拉格代表会议实现了的迫切任务,召开全党代表会议,在俄国出版合法的党报和成立党的秘密的实际中心。斯大林同志的文章是根据完全拥护列宁主张的巴库委员会的决议写成的。

为了同从取消派和托洛茨基派起到召回派和造神派止完全由反党分子组成的托洛茨基反党八月联盟相对抗,主张保存和巩固无产阶级秘密党的分子组成了一个护党联盟。参加这个联盟的有列宁所领导的布尔什维克和普列汉诺夫所领导的人数不多的孟什维克护党派。普列汉诺夫及其孟什维克护党派虽然在许多问题上坚持孟什维克立场。却坚决同八月联盟和取消派划清界限,并力求同布尔什维克达成协议。列宁接受了普列汉诺夫的提议,同普列汉诺夫成立了共同反对反党分子的暂时联盟,因为他认为这种联盟对党有利,而对取消派却有致命的危险。

斯大林同志完全拥护这个联盟。他当时在流放地。斯大林同志从流放地写信给列宁说:

\begin{quotation}
“照我的意见,联盟(列宁—普列汉诺夫)这条路线是唯一正确的:(一)这条路线,也只有这条路线,才真正符合俄国国内工作的利益,即把一切真正有党性的分子团结起来的利益。(二)这条路线,也只有这条路线,才能在孟克\footnote{孟克是孟什维克的简称。——编者注}工人和取消派之间挖掘一道鸿沟,驱散并歼灭取消派,使合法组织赶快从取消派的挟持之下解放出来。”(《列宁斯大林文集》俄文版第1卷第529—530页)\footnote{见《斯大林全集》第2卷第193页。——编者注}
\end{quotation}

由于把地下工作和合法工作巧妙地结合起来。布尔什维克终于成了公开的工人组织中的重大力量。例如,布尔什维克在当时四个合法的代表大会,即民众大学代表大会、妇女代表大会、工厂医生代表大会和禁酒代表大会的工人代表团中就起过重大的作用。布尔什维克在这些合法的代表大会上的演说具有重大的政治意义,获得了全国各地的响应。例如,布尔什维克工人代表团在民众大学代表大会上发言时,揭露了沙皇制度摧残一切文化活动的政策,并指出,不消灭沙皇制度,俄国就绝对不可能有真正的文化高涨。工人代表团在工厂医生代表大会上发言时,讲述了工人在极不卫生的条件下工作和生活的情形,并得出结论说,不推翻沙皇制度,就不可能真正举办工厂医疗事业。

布尔什维克把取消从那些保全下来的合法组织中逐渐排挤出去。由于同普列汉诺夫护党派成立统一战线这种特殊的策略,布尔什维克夺得了许多工人孟什维克组织(如在维波尔格区,叶加特林诺斯拉夫城等处)。

在这个困难时期,布尔什维克通过自己的工作做出了把合法工作和秘密工作结合起来的典范。


\subsection[五\q 1912年召开的党的布拉格代表会议。布尔什维克正式形成为独立的马克思主义政党]{五\\ 1912年召开的党的布拉格代表会议。\\ 布尔什维克正式形成为独立的马克思主义政党}

同取消派和召回派的斗争,以及同托洛茨基派的斗争,向布尔什维克提出了一项迫切的任务——必须把全体布尔什维克团结成为一个整体,使他们正式形成为一个独立的布尔什维克党。其所以绝对必须这样做,不仅是因为必须铲除党内那些分裂工人阶级的机会主义派别,而且还因为必须把聚集工人阶级力量的工作进行到底,必须准备工人阶级去迎接革命的新高涨。

可是。要实现过个任务,首先就必须把机会主义分子即孟什维克从党内清除出去。

现在在布尔什维克中间,谁都认为布尔什维克继续同孟什维克留在一个党内是绝对不可能了。由于孟什维克在斯托雷平反动年代的叛卖行为,由于他们企图取消无产阶级政党而组织一个改良主义新党,布尔什维克同他们的决裂成了必不可免的事情。当布尔什维克还同孟什维克留在一个党里时,他们对孟什维克的行为总要担负一种道义上的责任。可是,布尔什维克自己如果不愿背叛党和工人阶级,就绝对不能再替孟什维克的公开叛卖行为担负道义上的责任。所以,在一个党的范围内同孟什维克保持统一,就变成了对工人阶级及其政党的背叛。

因此,必须把同孟什维克的实际上的决裂贯彻到底,直到在组织上同他们正式决裂,把孟什维克驱逐出党,只有这样做,才能重新建立一个具有统一纲领、统一策略、统一阶级组织的无产阶级的革命政党。只有这样做,才能把被孟什维克破坏了的党的真正的(而不只是形式上的)统一建立起来。

这个任务应由布尔什维克筹备召开的第六次全党代表会议来完成。

但这个任务还只是事情的一面。同孟什维克正式决裂,使布尔什维克正式形成为独立的党,这当然是很重要的政治任务。但在布尔什维克而前还摆着另一个更重要的任务。当时的任务不仅在于同孟什维克决裂而正式形成为独立的党,而且首先在于同孟什维克决裂后建立一个新的党,即建立一个与通常那种西方社会民主党不同的、清除了机会主义分子的、能够引导无产阶级去夺取政权的新型的党。

在反对布尔什维克的时候,所有一切孟什维克,不分色彩,从阿克雪里罗得和马尔丁诺夫起,到马尔托夫和托洛茨基止,都始终不渝地使用着他们从西欧社会民主党人武库中搬来的武器。他们希望在俄国也有一个象德国或法国社会民主党那样的党。他们之所以反对布尔什维克,就是因为他们感觉到布尔什维克是一种新的、不寻常的、异于西方社会民主党的力量。当时西方各国社会民主党究竟是一种什么东西呢?它们是一种混合物,是一种大杂烩,其中既有马克思主义分子,也有机会主义分子;既有革命的朋友,也有革命的敌人;既有拥护党性的人,也有反对党性的人、并且前者在思想上逐渐同后者调和,前者在实际上逐渐向后者屈服。为什么要同机会主义分子,同革命叛徒调和呢?——布尔什维克问西欧社会民主党人。为了“党内和平”,为了“统一”,——他们这样回答布尔什维克。同谁统一呢,同机会主义分子统一吗?是的,是同机会主义分子“统一”,——他们这样回答。很明显,这样的党决不能成为革命的党,布尔什维克不能不看见,从恩格斯逝世以后、西欧各国社会民主党已开始由主张社会革命的党蜕化成为主张“社会改良”的党,其中每一个党,作为一个组织来说,都已由领导力量变成了自己议会党团的附属品。

布尔什维克不能不知道,这样的党对无产阶级不会有好处,这样的党决不能引导工人阶级去进行革命。

布尔什维克不能不知道,无产阶级所需要的不是这样的党,而是另一种党,即新的、真正马克思主义的党,它对机会主义分子采取不调和态度和对资产阶级采取革命态度,它团结紧密而坚如磐石;它是主张社会革命的党,是主张无产阶级专政的党。

布尔什维克想在俄国建立的正是这样的新的党。而布尔什维克当时正是在建立和准备这样的党。布尔什维克同“经济派”、孟什维克,托洛茨基派,召回派,各色各样唯心主义者直至经验批判主义者作斗争的全部历史,就是准备建立这样一个党的历史。布尔什维克想要建立一个新的、布尔什维主义的党,可供一切想要建立真正革命马克思主义党的人们效法的党。布尔什维克从旧《火星报》时期起就在准备建立这样的党了。他们坚持到底、坚韧不拔、勇往直前地准备着。在这一准备上起了基本和决定作用的,是列宁的《怎么办?》《两种策略》等这样一些著作。列宁的《怎么办?》一书,是这样一个党在思想上的准备。列宁的《进一步,退两步》一书,是这样一个党在组织上的准备。列宁的《社会民主党在民主革命中的两种策略》一书,是这样一个党在政治上的准备。最后,列宁的《唯物主义和经验批判主义》一书,是这样一个党在理论上的准备。

可以确有把握地说,历史上还从来没有过一个政治集团是象布尔什维克集团这样经过认真准备才形成为一个党的。

在这样的情况下,布尔什维克正式形成为一个党是一件完全准备好了的、完全成熟了的事情。

党的第六次代表会议的任务,就是要以驱逐孟什维克和宣告新的党即布尔什维克党正式形成的手续来完成这件已经准备好了的事情。

党的第六次全俄代表会议于1912年1月在布拉格举行。参加这次会议的有二十多个党组织的代表。因此,它在形式上具有全党代表大会的意义。

会议的通报报道了原被破坏的党中央机关已经恢复、党中央委员会已经成立的消息,指出反动年代是党从俄国社会民主党形成为一定的组织以来最艰难的时期。尽管受到种种迫害,尽管受到来自外部的沉重打击,尽管经历了党内机会主义分子的背叛和动摇,无产阶级的党终于保持了自己的旗帜和自己的组织。

会议的通报说:“保全下来的不仅有俄国社会民主党的旗帜、纲领和革命传统,而且还有它的组织。迫害手段虽能破坏和削弱这个组织,但是任何迫害都不能把它彻底消灭。”\footnote{见《苏联共产党决议汇编》第1分册第342页。——译者注}

会议指出了俄国工人运动重新高涨的征兆和党的工作活跃起来的事实。

会议听取了各地的报告之后确认:“在各地,社会民主党工人都在为巩固当地社会民主党的秘密组织和小组积极进行工作。”\footnote{见《苏联共产党决议汇编》第1分册第384页。——译者注}

会议指出,各地都肯定了布尔什维克在退却时期的策略的主要原则——把秘密工作同党在各种合法工人社团中的合法工作结合起来。

布拉格代表会议选出了党的布尔什维克中央委员会,当选为中央委员的有列宁、斯大林、奥尔忠尼启泽、斯维尔德洛夫、斯潘达梁等人。斯大林和斯维尔德洛夫两同志是缺席选进中央的,因为他们当时还在流放地。选为中央候补委员的有加里宁同志。

当时成立了领导俄国革命工作的实际中心(中央委员会俄国局),由斯大林同志负责主持。参加中央委员会俄国局的,除斯大林同志外,还有雅·斯维尔德洛夫、苏·斯潘达梁、谢·奥尔忠尼启泽和米·加里宁等同志。

布拉格代表会议把布尔什维克过去反对机会主义的全部斗争作了一个总结,并决定把孟什维克驱逐出党。

布拉格代表会议把孟什维克驱逐出党,就正式宣告了布尔什维克党的独立存在。

布尔什维克既已在思想上打垮了孟什维克,把他们驱逐出党,就给自己保存了俄国社会民主工党一面旧有的党旗。因此布尔什维克党一直到1918年都叫作俄国社会民主工党,后面加括号标明:“布尔什维克”。

1912年初,列宁在给高尔基的信中谈到布拉格代表会议的结果时写道:

\begin{quotation}
“不管取消派混蛋们怎样捣乱,我们终于把党和它的中央委员会恢复起来了。我想您会和我们一起为这件事情高兴的。”(《列宁全集》俄文第3版第29卷第19页)\footnote{见《列宁全集》第35卷第1页。——译者注}
\end{quotation}

斯大林同志在评价布拉格代表会议的意义时说道:

\begin{quotation}
“这次代表会议在我们党的历史上有极大的意义,因为它划清了布尔什维克和孟什维克之间的界限,把全同各地的布尔什维克组织联合成了统一的布尔什维克党。”(《联共(布)第十五次代表大会速记记录》俄文版第361—362页)\footnote{见《斯大林全集》第10卷第310页。——译者注}
\end{quotation}

从孟什维克被驱逐而布尔什维克正式形成为一个独立的政党以后,布尔什维克党变得更坚强更有力了。党是靠清洗自己队伍中的机会主义分子而巩固起来的。——这就是与第二国际各社会民主党根本不同的新型政党布尔什维克党的口号之一。第二国际各国党口头上自称为马克思主义的党,实际上却容忍马克思主义的敌人——公开的机会主义分子留在自己队伍中间,让他们瓦解、断送第二国际。与此相反,布尔什维克同机会主义分子进行不调和的斗争,从无产阶级党身上不断清除机会主义的污泥浊水,结果创立了新型的党,列宁的党,即后来取得了无产阶级专政的党。

如果在无产阶级党队伍里留下了机会主义分子,布尔什维克党就不可能走上康庄大道并引导无产阶级前进,就不可能取得政权和建立无产阶级专政,就不可能成为国内战争的胜利者,就不可能建成社会主义。

布拉格代表会议在它的决议中提出了如下的最低纲领作为党在当前的主要政治口号:成立民主共和国,实行八小时工作制,没收一切地主土地。

在这些革命口号下,布尔什维克进行了选举第四届国家杜马的选举运动。

在这些口号下,工人群众的革命运动在1912—1914年重新高涨起来。


\subsection{简短的结论}

1908—1912年,是革命工作最困难的时期。在革命失败以后,在革命运动低落和群众感到疲惫的情况下,布尔什维克改变了自己的策略,从直接进行反对沙皇制度的斗争转到用迂回方法进行这一斗争。在斯托雷平反动年代的艰苦条件下,布尔什维克为保持自己同群众的联系而利用了甚至极小的合法机会(从保险基金会和工会起到杜马讲坛止)。布尔什维克始终不懈地聚集着力量去迎接革命运动的新高涨。

在革命遭到失败,反政府派别分崩离析、脱离了党的知识分子(波格丹诺夫,巴札罗夫等人)对革命失望并对党的理论基础加紧进行修正主义袭击的艰苦环境中,布尔什维克表明,党内只有它这支力量没有卷起党的旗帜,而仍然忠于党的纲领,并打退了马克思主义理论的“批评者”的攻击(列宁的《唯物主义和经验批判主义》一书)。马克思列宁主义的思想锻炼和对革命前途的明确认识,帮助了团结在列宁周围的布尔什维克的基本核心来捍卫党和党的革命原则。列宁这样评论布尔什维克:“难怪人们把我们叫作坚如磐石的人。”\footnote{见《列宁全集》第13卷第422页。——译者注}

孟什维克在这个时期愈来愈离开革命。他们演变成为取消派,要求取消、消灭无产阶级的革命的秘密党,他们愈来愈公开地抛弃党的纲领,抛弃党的革命任务和口号,企图自己另组织一个改良主义的党,即工人所称的“斯托雷平工党”。托洛茨基支持取消派,但他假装维护“党的统一”,实际上是同取消派统一。

另一方面,有一部分布尔什维克不了解当时必须采取新的迂回的斗争方法去反对沙皇制度,却要求党拒绝利用合法机会,要求党召回参加国家杜马的工人代表。召回派促使党脱离群众,阻碍党聚集力量去迎接革命的新高涨。召回派用“左的”词句来掩饰自己,其实他们也和取消派一样拒绝进行革命斗争。

取消派和召回派结合成为一个反对列宁的总联盟,即由托洛茨基组织起来的八月联盟。

布尔什维克在反对取消派和召回派的斗争中,在反对八月联盟的斗争中获得了胜利,成功地捍卫了无产阶级的秘密党。

这个时期最重要的事件,就是俄国社会民主工党在布拉格举行的代表会议(1912年1月)。这次会议把孟什维克驱逐出党,永远结束了布尔什维克同孟什维克形式上联合在一个党内的局面。布尔什维克由一个政治集团正式形成为独立的俄国社会民主工党(布尔什维克)。布拉格代表会议创立了新型的党,列宁主义的党,即布尔什维克党。

布拉格代表会议把机会主义分子即孟什维克从无产阶级党里清除出去,对于党和革命后来的发展具有重大的决定的意义。如果布尔什维克当时没有把背叛工人事业的妥协派即孟什维克驱逐出党,无产阶级党在1917年就不可能发动群众去夺取无产阶级专政。

