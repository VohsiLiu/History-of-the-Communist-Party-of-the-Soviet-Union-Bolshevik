\section{结束语}

对布尔什维克党所经过的历史道路可以作出什么样的基本总结呢?

联共(布)的历史教导我们的是什么呢?

(一)首先,党的历史教导说,没有一个革命的无产阶级政党,没有一个消除了机会主义、对妥协主义者和投降主义者毫不调和、对资产阶级及其国家政权采取革命态度的党,无产阶级革命的胜利、无产阶级专政的胜利是不可能的。

党的历史教导说,把无产阶级弄到没有这样一个党,就是把它弄到没有革命的领导,而把它弄到没有革命的领导,就是使无产阶级革命事业遭到失败。

党的历史教导说,能够成为这样的党的,决不是通常那种西欧类型的社会民主党,那种在国内和平条件下熏陶出来、被机会主义分子牵着走、幻想“和平改良”而害怕社会革命的党。

党的历史教导说,能够成为这样的党的,只能是新型的党,马克思列宁主义的党,主张社会革命的党,能够训练无产阶级去同资产阶级决战并组织无产阶级革命的胜利的党。

这样的党在苏联就是布尔什维克党。

\begin{quotation}
斯大林同志说:“在革命前的时期,在比较和平发展的时期,第二国际各党是工人运动中的统治力量,议会斗争形式是基本的斗争形式,——在这种条件下,党没有而且不可能有它后来在公开的革命搏斗的条件下所具有的那种重大的和决定的意义。考茨基在第二国际遭受攻击时替它辩护说:第二国际各党是和平的工具,不是战争的工具,正因为如此,它们在战争时期,在无产阶级的革命发动时期,就没有力量采取什么重大措施。这是完全对的。但是这说明什么呢?说明第二国际各党对于无产阶级的革命斗争是不适用的,它们不是引导工人去夺取政权的无产阶级的战斗的党,而是迁就议会选举和议会斗争的选举机关。正因为如此,在第二国际机会主义者统治的时期,无产阶级的基本政治组织并不是党,而是议会党团。大家知道,事实上党在这个时期只是议会党团的附属品和服役者。几乎用不着证明,在这样的条件下,在这样的党的领导下,是谈不到训练无产阶级去进行革命的。

可是,新时期一到来,情形就根本改变了。新时期是公开的阶级冲突的时期,是无产阶级革命发动的时期,是无产阶级革命的时期,是直接准备力量去推翻帝国主义而由无产阶级夺取政权的时期。这个时期在无产阶级面前提出种种新的任务,要按新的革命规范去改造党的全部工作,要用夺取政权的革命斗争精神去教育工人,要准备后备军并使他们跟上来,要和邻国的无产者结成联盟,要同殖民地和附属地的解放运动建立巩固的联系,如此等等。如果以为这些新任务可以用那些在议会制度的和平条件下教育出来的旧社会民主党的力量来解决,那就是使自己陷于绝望的境地,遭到必不可免的失败。肩负着这样的任务而仍然以旧的党为领导,那就是完全解除武装。几乎用不着证明,无产阶级是不能容忍这种情形的。

因此,必须有新的党,战斗的党,革命的党。这个党要有充分的勇气,能够引导无产者去夺取政权。这个党要有充分的经验,能认清革命环境的复杂条件,这个党要有充分的机智,能够绕过横在前进道路上的一切暗礁。

没有这样的党,就休想推翻帝国主义,就休想争得无产阶级专政。

这个新的党就是列宁主义的党。”(斯大林《列宁主义问题》俄文第10版第62—63页)\footnote{见(斯大林《列宁主义问题》第72—74页)——译者注}
\end{quotation}


(二)其次,党的历史教导说,工人阶级的党如果不掌握工人运动的先进理论,不掌握马克思列宁主义理论,就当不了本阶级的领导者,就当不了无产阶级革命的组织者和领导者。

马克思列宁主义理论的力量,就在于它使党能判明局势,了解周围事变的内在联系,预察事变的进程,不但洞察事变在目前怎样发展和向何处发展,而且洞察事变在将来怎样发展和向何处发展。

只有掌握了马克思列宁主义理论的党,才能信心百倍地前进,并引导工人阶级前进。

相反,没有掌握马克思列宁主义理论的党,却不得不徘徊摸索,对自己的行动失去信心,没有能力引导工人所级前进。

有人也许以为,掌握马克思列宁主义理论,且要做到用心熟读马克思、恩格斯和列宁著作中的某些结论和原理,学会及时引证这些结论和原理,就算不错了。他们是指望把熟读的结论和原理用于各种环境,用于实际生活的一切场合。但这样对待马克思列宁主义理论是完全不正确的。决不能把马克思列宁主义看成是教条汇编、看成是教条问答、看成是信条,而把马克思主义者本身看成咬文嚼字的人和书呆子。马克思列宁主义的理论是关于社会发展的科学,关于工人运动的科学,关于无产阶级革命的科学,关于共产主义社会建设的科学。它既是一种科学,就不会也不可能停留不前,而会不断发展和不断完善。显然它在自己的发展进程中不能不用新的经验和新的知识来丰富自己;它的某些原理和结论不能不随着时间的推移而改变,不能不用适合于新的历史条件的新结论和新原理来代替。

掌握马克思列宁主义理论,决不是说要熟读它的一切公式和结论,抱住这些公式和结论的每一个字句不放。要掌握马克思列宁主义理论,首先必须学会把它的字句和实质区别开来。

掌握马克思列宁主义理论,是说要领会这个理论的实质,学会在无产阶级阶级斗争的各种条件下运用这个理论来解决革命运动的实际问题。

掌握马克思列宁主义理论,是说要善于用革命运动的新经验来丰富这个理论,善于用新原理和新结论来丰富这个理论,善于发展和推进这个理论.不怕根据这个理论的实质去用适合于新的历史形势的新原理和新结论来代替它的某些已经陈旧的原理和结论。

马克思列宁主义理论不是教条,而是行动的指南。

在俄国第二次革命(1917年2月)以前,各国的马克思主义者都认为,议会制民主共和国是从资本主义到社会主义的过渡时期对社会最适宜的政治组织形式。固然,马克思在十九世纪七十年代曾指出,无产阶级专政最适宜的形式不是议会制共和国,而是巴黎公社式的政治组织。但可惜马克思的这一指示没有在著作里得到进一步的发挥,于是就被人遗忘了。此外,恩格斯在1891年对爱尔福特纲领草案的批判中所作的权威性声明,即“民主共和国……是无产阶级专政的特殊形式”\footnote{见《马克思恩格斯全集》第22卷第274页。——译者注},也无容置疑地表明马克思主义者继续认为民主共和国是无产阶级专政的政治形式。恩格斯的这个原理后来成了所有马克思主义者——包括列宁在内——奉行的准则。但是,俄国1905年的革命,特别是1917年2月的革命,提出了工农代表苏维埃,即一个社会的新的政治组织形式。列宁从马克思主义理论出发,根据他对俄国两次革命经验的研究,得出结论说,无产阶级专政最好的政治形式不是议会制民主共和国,而是苏维埃共和国。根据这一点,列宁在1917年4月,即在从资产阶级革命向社会主义革命过渡的时期,提出了成立苏维埃共和国作为无产阶级专政最好的政治形式的口号。当时各国机会主义者都抱住议会制共和国不放,责备列宁离开了马克思主义,破坏了民主。但掌握了马克思主义理论的真正的马克思主义者当然是列宁,而不是机会主义者,因为列宁向前推进了马克思主义理论,用新的经验丰富了这个理论,而机会主义者则把这个理论向后拉,把它的一个原理变成了教条。

如果列宁为马克思主义的字句所束缚,不敢用苏维埃共和国这一适合新的历史形势的新原理来代替恩格斯所表述的马克思主义的旧原理。那末党、我国革命和马克思主义会是什么情况呢?党就会在黑暗中徘徊,苏维埃就会瓦解,我们就不会有苏维埃政权,马克思主义理论就会受到严重的损害,无产阶级就会遭到失败,无产阶级的敌人就会获得胜利。

恩格斯和马克思在研究帝国主义以前的资本主义时得出结论说,社会主义革命不可能在单独一个国家内获得胜利,它只有在一切或大多数文明国家同时举行进攻的条件下才能获得胜利。这是在十九世纪中叶说的,这个结论后来成了所有马克思主义者奉行的准则。但是,到二十世纪初,帝国主义以前的资本主义已经转变成帝国主义阶段的资本主义,上升的资本主义已经变成垂死的资本主义。列宁从马克思主义理论出发,根据他对帝国主义阶段的资本主义研究,得出结论说,恩格斯和马克思的旧公式已经不适合于新的历史形势,社会主义革命完全可能在单独一个国家内获得胜利。当时各国机会主义者都抱住恩格斯和马克思的旧公式不放,责备列宁离开了马克思主义。但掌握了马克思主义理论的真正的马克思主义者当然是列宁,而不是机会主义者,因为列宁向前推进了马克思主义理论,用新的经验丰富了这个理论,而机会主义者则把这个理论向后拉,把它变成木乃伊。

如果列宁为马克思主义的字句所束缚,如果他在理论上没有足够的勇气抛开马克思主义的旧结论,而代之以关于社会主义能够在单独一个国家内获得胜利这一适合于新的历史形势的新结论,那末党、我国革命和马克思主义会是什么情况呢?党就会在黑暗中徘徊,无产阶级革命就会失去领导,马克思主义理论就会开始衰退,无产阶级就会遭到失败,无产阶级的敌人就会获得胜利。

机会主义并不总是意味着直接否定马克思主义理论或它的某些原理和结论。机会主义有时还表现在企图抱住马克思主义中某些已经过时的原理不放,把它们变成教条,以便阻碍马克思主义向前发展,从而也阻碍无产阶级革命运动的发展。

可以毫不夸大地说,恩格斯逝世后,最伟大的理论家列宁,以及继列宁之后的斯大林和列宁的其他学生,是唯一向前推进了马克思主义理论、用无产阶级阶级斗争新条件下的新经验丰富了这个理论的马克思主义者。

正因为列宁和列宁主义者向前推进了马克思主义理论,所以列宁主义是马克思主义的进一步发展,是无产阶级阶级斗争新条件下的马克思主义。是帝国主义和无产阶级革命时代的马克思主义,是社会主义在全世界六分之一的土地上获得胜利的时代的马克思主义。

如果布尔什维克党的先进干部没有掌握马克思主义理论,如果他们没有学会把这个理论看作行动的指南,如果他们没有学会向前推进马克思主义理论,用无产阶级阶级斗争的新经验来丰富这个理论,那末布尔什维克党就不会有1917年十月革命的胜利。

恩格斯在批评那些担负着美国工人运动领导工作的侨居美国的德国马克思主义者时写道:

\begin{quotation}
“德国人一点不懂得把他们的理论变成能推动美国群众的杠杆,他们大部分连自己也不懂得这种埋论,而用学理主义和教条主义的态度去对待它,认为只要把它背得烂熟,就足以应付一切。对他们来说,这是教条,而不是行动的指南。”(《马克思思格斯全集》俄文第1版第27卷第606页)\footnote{见《马克思恩格斯选集》第4卷第456页。——译者注}
\end{quotation}

列宁批评加米涅夫和某些老布尔什维克在1917年4月,即在革命运动已经向前发展而要求过渡到社会主义革命的时候仍抱住工农革命民主专政的旧公式不放时写道:

\begin{quotation}
“马克思和恩格斯总是说,我们的学说不是教条,而是行动的指南,他们公正地讥笑了只会背诵和简单重复‘公式’的人们,因为‘公式’至多只能指出一般的任务,而这些任务随着历史过程中每个特殊阶段的具体的经济和政治环境必然有所改变。……现在必须弄清一个不容置辩的真理,就是马克思主义者必须考虑生动的实际生活,必须考虑现实的确切事实,而不应当抱住昨天的理论不放……”(《列宁全集》俄文第3版第20卷第100—101页)\footnote{见《列宁选集》第2版第3卷第24—26页。——译者注}
\end{quotation}

(三)其次,党的历史教导说,如果不打垮那些在工人阶级队伍中进行活动、把工人阶级的落后阶层推进资产阶级怀抱,从而破坏工人阶级的统一的小资产阶级党派,那么无产阶级革命就不能获得胜利。

我们党的历史是同各小资产阶级党派——社会革命党、孟什维克、无政府主义者和民族主义者作斗争并把它们打垮的历史。不战胜这些党派、不把它们从工人阶级队伍中驱逐出去,就不能达到工人阶级的统一,而没有工人阶级的统一,就不能实现无产阶级革命的胜利。

如果不打垮这些起初主张保存资本主义,到十月革命后又企图恢复资本主义的党派,那就不能保持无产阶级专政,不能战胜外国武装干涉和建成社会主义。

一切为了欺骗人民而自称“革命”的和“社会主义”的党派的小资产阶级党派——社会革命党、孟什维克、无政府主义者和民族主义者,早在十月社会主义革命前就已经成了反革命的政党,后来又变成了外国资产阶级间谍机关的代理人,变成了一帮特务、暗害分子、破坏分子、凶手和叛国者,这决不是偶然的。

\begin{quotation}
列宁说,“在社会革命时代,只有依靠最革命的马克思主义政党,只有同其他一切党派进行无情的斗争,才能实现无产阶级的团结。”(《列宁全集》俄文第3版第26卷第50页)\footnote{见《列宁全集》第31卷第472页。——译者注}
\end{quotation}

(四)其次,党的历史教导说,工人阶级的党不同自己队伍中的机会主义者作不调和的斗争,不打垮自己队伍中的投降主义者,就不能保持自己队伍的统一和纪律,就担当不了无产阶级革命的组织者和领导者,就担当不了社会主义新社会的建设者。

我们党内生活发展的历史,是同党内机会主义集团——“经济派”、孟什维克、托洛茨基派、布哈林派和民族主义倾向分子作斗争并把他们打垮的历史。

党的历史教导说,所有这些投降主义集团实质上都是孟什维克主义在我们党内的代理人,是孟什维克主义的仆从,是孟什维克主义的继续。它们也同孟什维克主义一样,起着在工人阶级中和党内传播资产阶级影响的作用。因此,消灭党内这些集团的斗争,就是消灭孟什维主义的斗争的继续。

如果不打败“经济派”和孟什维克,我们就不能把党建立起来并引导工人阶级去进行无产阶级革命。

如果不打败托洛茨基派和布哈林派,我们就不能为建成社会主义准备好必要的条件。

如果不打败形形色色的民族主义倾向分子,我们就不能用国际主义精神来教育人民,就不能保住苏联各族人民伟大友谊的旗帜,就不能把苏维埃社会主义共和国联盟建立起来。

也许有人以为,布尔什维克用了过多的时间去同党内机会主义分子作斗争,过高估计了党内机会主义分子的作用。但这是完全不正确的。决不能容忍自己队伍中有机会主义,正如不能容忍健全的机体上有脓疮一样。党是工人阶级的领导部队,是它的先头堡垒,是它的战斗司令部。在工人阶级的领导司令部中,决不能容许信念不坚定者、机台主义者、投降主义者和有叛徒立足的余地。在自己的司令部中、在自己的堡垒中有投降主义者和叛徒而同资产阶级作殊死斗争,就会陷于腹背受击的地位。不难理解,这样的斗争只会遭到失败的结局。堡垒是最容易从内部攻破的。为要达到胜利,首先必须在工人阶级的党内,在工人所有的领导司令部内,在工人阶级的先头堡垒内,把投降主义者、逃兵、工贼和叛徒清除出去。

托洛茨基派、布哈林派和民族主义倾向分子反对列宁反对党,遭到了与孟什维克党和社会革命党同样的结局,即变成了法西斯间谍机关的代理人。变成了特务、暗害分子、凶手、破坏分子和叛国者,这决不是偶然的。

\begin{quotation}
列宁说:“在自己的队伍里,有改良主义者,有孟什维克,就不能在无产阶级革命中取得胜利,就不能捍卫住无产阶级革命。这在原则上是很明显的。这是已经由俄国和匈牙利的经验具体证实了的。……在俄国……曾经有很多次处于困难的境地。当时如果让孟什维克、改良主义者、小资产阶级民主派留存我们党内……那么苏维埃制度就一定会被推翻的。”(《列宁全集》俄文第3版第25卷第462—463页)\footnote{见《列宁全集》第31卷第345—346页。——译者注}
\end{quotation}

\begin{quotation}
斯大林同志说:“如果说我们党已经建立了自己内部的统一和自己队伍的空前的团结,那末这首先是因为它及时清洗了机会主义的肮脏东西。从党内驱逐了取消派和孟什维克,无产阶级政党发展和巩固的道路就是把机会主义者和改良主义者、社会帝国主义者和社会沙文主义者、社会爱国主义者和社会和平主义者从党内清洗出去的道路。党是靠清洗自己队伍中的机会主义分于而巩固起来的……(斯大林《列宁主义问题》俄文第10版第72页)\footnote{见斯大林《列宁主义问题》第84页。——译者注}
\end{quotation}

(五)其次,党的历史教导说,如果党陶醉于胜利而开始骄傲起来,如果它不再注意自己工作中的缺点,如果它害怕承认自己的错误,害怕公开地老实地及时地改正这些错误,那它就当不了工人阶级的领导者。

如果党不害怕批评和自我批评,如果它不掩盖自己工作中的错误和缺点,如果它用自己工作中的错误来教导和教育干部,如果它善于及时改正自己的错误,那它就会是不可战胜的。

如果党隐瞒自己的错误,掩饰老大难的问题,用虚假的表面上的一切满意来掩盖自己的缺点,不能容忍批评和自我批评浸透自满情绪,一味妄自尊大,躺倒在功劳簿上,那它就会遭到灭亡。

\begin{quotation}
列宁说:“一个政党对自己的错误所抱的态度,就是衡量这个党是否郑重,是否真正履行它对本阶级和劳动群众所负义务的一个最重要最可靠的尺度。公开承认错误,揭露错误的原因,分析产生错误的环境,仔细耐心改正错误的方法,——这才是一个郑重的党的标志,这就是党履行自己的义务,这才是教育和训练阶级,以至于群众。”(《列宁全集》俄文第3版第25卷第200页)\footnote{见《列宁选集》第2版第4卷第213页。——译者注}
\end{quotation}

又说:

\begin{quotation}
“过去一切灭亡了的革命政党所以灭亡,就是因为它们骄傲自大,不善于看到自己力量的所在,怕说出自己的弱点。而我们是不会灭亡的,因为我们不怕说出自己的弱点,并且能够学会克服弱点。”(《列宁选集》俄文第3版第27卷第260—261页)\footnote{见《列宁全集》第33卷第275页。——译者注}
\end{quotation}

(六)最后,党的历史教导说,工人阶级的党如果不同群众保持广泛的联系,不经常巩固这种联系,不善于倾听群众的呼声和了解他们的疾苦,没有不仅教导群众而且向群众学习的决心,那它就不能成为能够领导千百万工人阶级群众和全体劳动群众的真正群众性的党。

如果党善于像列宁所说那样“同最广大的劳动群众,首先是同无产阶级劳动群众,但同样也同非无产阶级劳动群众联系、接近,甚至可以说在某种程度上同他们打成一片”(《列宁全集》俄文第3版第25卷第174页)\footnote{见《列宁选集》第2版第4卷第182页。——译者注},那它就会是不可战胜的。

如果党在自己的党的狭小圈子里闭关有守,如果它脱离群众,如果它蒙上了官僚主义的灰尘,那它就会遭到灭亡。

\begin{quotation}
斯大林同志说:“当布尔什维克保持同广大人民群众的联系时,他们将是不可战胜的,——这可以认为是一个规律。相反地,布尔什维克只要一脱离群众和失去同群众的联系,只要染上官僚主义的毛病,他们就会丧失任何力量,而变成空架子。

在古代希腊人的神话中,有一个著名的英雄名叫安泰,据神话说,他是海神波赛东和地神盖娅的儿子,他对生育、抚养和教导他成人的母亲是非常依恋的,没有哪一个英雄能同这个安泰抗衡,大家公认他是无敌的英雄。他的力量存什么地方呢?他的力量就在于,每当他同敌人决斗而遇到困难时,便往地上一靠,就是说,往生育和抚养他成人的母亲身上一靠。就取得了新的力量。可是他毕竟有一个弱点,就是怕别人用什么方法使他离开地面。敌人注意到他的这个弱点,于是时刻暗中窥伺他。后来有一个敌人利用了他的弱点,就战胜了他。这个敌人名叫海格立斯。可是,他是怎样战胜安泰的呢?原来这个敌人使安泰离开了地面,把他举到空中,使他无法再靠近地面,这样就在空中把他扼死了。

我认为,布尔什维克很像希腊神话中的英雄安泰。布尔什维克也同安泰一样,其所以强大,就是因为他们同自己的母亲,即同那生育、抚养和教导他们成人的群众保持联系。只要他们同自己的母亲、同人民保持联系,他们就有一切把握,始终是不可战胜的。

这就是布尔什维克领导不可战胜的关键。”(斯大林《论党的工作缺点》)
\end{quotation}

以上就是布尔什维克党所经过的历史道路的基本教训。

