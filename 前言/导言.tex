\section{导言}

苏联共产党(布尔什维克)经历了漫长而光荣的途程,从十九世纪八十年代在俄国出现的人数不多的第一批马克思主义小组和团体,发展为现在领导着世界上第一个社会主义工农国家的伟大的布尔什维克党。

联共(布)是在革命前俄国工人运动基础上,由那些同工人运动相联系并把社会主义意识灌输到工人运动中去的马克思主义小组和团体成长起来的。联共(布)过去和现在都是以马克思列宁主义的革命学说为指针。联共(布)的领袖们在帝国主义,帝国主义战争和无产阶级革命时代的新条件下,发展了马克思恩格斯的学说,把它提高到了一个新的阶段。

联共(布)在工人运动内部是同小资产阶级的党派,即同社会革命党(更早是同他们的前辈——民粹派)、孟什维克、无政府主义者和形形色色的资产阶级民族主义者作原则斗争中,在党内则是同孟什维主义的、机会主义的派别,即同托洛茨基派、布哈林派、民族主义倾向分子和其他反列宁主义集团做原则性斗争中成长壮大起来的。

联共(布)是在同工人阶级的一切敌人、劳动群众的一切敌人即地主、资本家、富农、暗害分子和特务作革命斗争中,在同资本主义包围势力的一切雇佣走狗作革命斗争中,得到巩固和锻炼的。

联共(布)的历史是三次革命,即1905年的资产阶级民主革命、1917年2月的资产阶级民生革命和1917年10月的社会主义革命的历史。

联共(布)的历史是推翻沙皇制度、推翻地主资本家政权的历史,是在国内战争时期粉碎外国武装干涉的历史,是在我国建成苏维埃国家和社会主义社会的历史。

研究联共(布)的历史,就是用我国工农为社会主义斗争的经验丰富自己。

研究联共(布)的历史,研究我党同马克思列宁主义的一切敌人作斗争、同劳动群众的一切敌人作斗争的历史,有助于掌握布尔什维主义,能提高政治警惕性。

研究布尔什维克党的英勇历史,就是用社会发展和政治斗争规律的知识武装自己,用革命动力的知识武装自己。

研究联共(布)的历史,就能增强信心,确信列宁斯大林党的伟大事业必将最后胜利,确信共产主义必将在全世界胜利。

本书是对苏联共产党(布尔什维克)历史的简要叙述。


